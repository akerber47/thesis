%%%%%%%%%%%%%%%%%%%%%%%%%%%%%%%%%%%%%%%%%%%%%%%%%
\section{Decomposition of pared 3-manifolds}
%%%%%%%%%%%%%%%%%%%%%%%%%%%%%%%%%%%%%%%%%%%%%%%%%

In the geometrically finite case, we've reduced the quasi-Fuchsian surface
subgroup question to a topological problem about pi1-injective maps of
a surface into a pared 3-manifold. However, infinite volume hyperbolic
3-manifolds may have quite complex topology, so it is not immediately clear how
to find such surfaces in the resulting pared 3-manifolds. It is not clear how
to find (QF) surfaces directly. Instead, we decompose the pared 3-manifold
along disks and annuli. While the precise statements and proofs that follow are
new, these are known facts and standard arguments.

% See Jaco, Jaco-Shalen, Johannson, Canary-McCullough

% Also see Canary-McCullough for JSJ for pared 3-manifolds!!

% NOTE NOTE these are the same arguments (or at least very very similar ones)
% as I'm currently using in section 6 for specific book of I-bundles topology
% facts!!  Need to straighten that out, avoid duplication, etc. Cite backwards
% to this section where needed.
%
% Keep writing this, then fix section 6 to cite it...

First, a simple proposition.

prop

Let M be a compact orientable irreducible 3-manifold. Let f colon S to M be
a pi1-injective map, where S is a closed surface.  Let D cin M be a compression
disk for bd M.  That is, D is properly embedded, and bd D does not bound a disk
in bd M.  Then f is homotopic to a map such that f(S) cap D = emptyset. Given
any finite collection of disjoint compression disks we can homotope f away from
all of them.

end prop

proof

Intuitively, f is pi1-injective, so any loop where S crosses D is contractible
in D, hence contractible in S. Then use irreducibility and an innermost disk
argument.

More precisely, we proceed as follows. First suppose f is not necessarily an
embedding. Homotope f locally so it's transverse to D. D is compact and
properly embedded, and S is compact, so f-1(D) is a properly embedded compact
submanifold of S. That is, it's a union of finitely many disjoint simple closed
curves alpha1,dots,alphak. For each alphai, f(alphai) cin D, so f(alphai) is
homotopically trivial in M.  Since f is pi1-injective, each alphai must be
homotopically trivial in S. Since S is a closed surface, this means each alphai
bounds a disk in S. Choose an innermost such disk D' cin S, that is, D' is
bounded by some alphai and doesn't contain any other alphaj in its interior.

Now construct a map phi colon S2 to M as follows. The closed upper hemisphere
is homeomorphic to D' cin S. Map it in along D'. Then the equator maps in along
f(alpha), which is contractible, as it's in D. Choose the lower hemisphere of
phi to correspond to such a contraction of f(alpha) within D. M is irreducible,
hence aspherical, so phi must be homotopically trivial. Homotoping phi to
a point and re-expanding, we can see that f|D' is homotopic to a map into D (ie
corresponding to a contracting homotopy of alpha). Pushing f slightly away from
D using a tubular neighborhood of D in M, we can eliminate the intersection
curve f(alpha) between f(S) and D. Since this homotopy only affects
a neighborhood of D' cin S, it cannot add any additional intersections or
affect any of the other intersection curves. Repeating this process, we can
eliminate all intersection curves by homotopy.

end proof

We have a similar fact for the pared locus of a pared 3-manifold.

prop

Let (M,P) be a pared 3-manifold where M is compact orientable irreducible, and
D cin M a compression disk for bd M. Then we can isotope P within bd M such
that P has minimal intersection with bd D. In this minimal intersection, torus
components of P do not intersect D, and if we cut M along D, all the resulting
rectangles obtained by cutting annulus components of P are essential. We can
perform a similar operation for any finite collection of disjoint compression
disks.

end prop

proof

This is straightforward. Torus components of P cannot intersect bd D, as each
torus component of P is precisely a torus component of bd M. This is because
a torus cannot map pi1-injectively to any other genus surface, as it'd
correspond to a rank 2 abelian subgroup. So if bd D intersects a torus
component of P, it'd have to lie entirely inside it. Then D would be
a compression disk for this torus component, violating pi1-injectivity.

Now consider annulus components of P that lie on the boundary component F of
M that contains bd D. It is sufficient to consider the core curves of these
annuli, which are a disjoint set of simple closed curves in F. Our statement
now follows from the elementary fact (about curves on surfaces) that we can put
these in minimal position with respect to another simple closed curve, bd D,
such that the intersection contains only essential arcs. Briefly, if there are
inessential arcs, they cobound disks, so begin with the innermost disks and
isotope to remove any inessential arcs. Again, we can repeat this process for
any disjoint collection of compression disks. This completes the proof.

end proof

cor

Let (M,P) be a pared 3-manifold, and f colon S to M a pi1-injective map of
a closed surface. Suppose P0 is a component of P which has an essential
intersection with a compression disk D. By abuse of notation, we'll refer to
its core curve as P0 as well. Then either:

(1) P0 is homotopic in M to a different curve in bd M that does not intersect
D. That is, there are sufficiently many compression disks cobounded by arcs in
P0 with endpoints on bd D that "transfer P0 across D to a different part of bd
M".

or

(2) fstar(pi1S) does not intersect any conjugates of pi1P0, that is, a multiple
of P0 is not freely homotopic in M into any image of a loop in S.

end cor

proof

Note that if the algebraic intersection number between P0 and D is nonzero, it
immediately follows that P0 cannot be homotoped in M into the image of a loop
in S (which would have to be disjoint from D). The algebraic intersection
number is a homotopy, indeed a homology invariant in M. So we need only
consider the case of zero algebraic intersection.

Deformation retract D to a point, and choose this as our basepoint for pi1M.
This yields an expression for pi1M as either a free product pi1M1 star pi1M2
(if M setminus D is disconnected) or trivial HNN extension pi1M0 star
= <pi1M0,t> (if M setminus D is connected). Since S can be homotped to be
disjoint from D, f star (pi1S) is contained in one of these subgroups
pi1M1,pi1M2,pi1M0.  Cutting P0 along D breaks it into a union of arcs, and
after the deformation retract this corresponds to a word in the free product
/ HNN with each arc inducing a single subgroup letter in the word (or pair of
letter, in the HNN case when the arc connects between the two copies of D in
M0).  In the HNN case, note that the zero algebraic intersection implies that
the number of times the stable generator t appears in such a word is the same
as the number of times t-1 appears.

But in the HNN case, t does not appear in fstar(pi1S).  Similarly, in the free
product case, words from only one of pi1M1 or pi1M2 appear in fstar(pi1S).
Without loss of generality say it's pi1M1. That is, in order for (2) to be
false, that is, for a multiple of P0 to appear in fstar(pi1S), we must have
sufficient cancellation in the element P0 to remove all powers of t, or all
letters from pi1M2. We claim that (1) holds in this case.

To see this, observe that pi1M is a \emph{trivial} free product or HNN, so
there are no added relations. The only way to produce an element in a subgroup
is if all the in-between letters are in fact trivial. For each letter in the
word that is trivial in its subgroup (pi1M1, pi1M2, or pi1M0), we have
a basepointed closed curve in M, which corresponds to a contractible boundary
arc in the original M (before deformation retract) with both endpoints on D.
Connect the endpoints of this arc with a segment along bd D to produce a closed
curve in bd M. This is homotopically trivial, by the above word argument.
Therefore, by Dehn's lemma, it bounds a disk in M. Pushing P0 along this disk
and slightly outside it moves it to a new position in bd M with fewer
intersections with bd D. Using an innermost disk argument, we can do this
isotopically to obtain (1).

end proof

Together these demonstrate that we can reduce our main problem when M admits
boundary compressions. All closed pi1-injective surfaces can be made to avoid
the compression disks, and all components of the pared structure can either be
made to avoid the compression disks (case (1)) or are irrelevant because they
cannot be homotoped into any closed pi1-injective surface (case (2)). We push
the case (1) pared structure components away from the compression disks and
discard the case (2) components. The resulting (possibly disconnected)
incompressible pared manifold contains a (QF) surface if and only if the
original does.

Therefore we can reduce to the case where M is a pared 3-manifold (compact
orientable irreducible) with incompressible boundary. However, in order to
understand the problem better, we want another type of decomposition, along
disjoint properly embedded essential annuli.


