%%%%%%%%%%%%%%%%%%%%%%%%%%%%%%%%%%%%%%%%%%%%%%%%%
\section{Decomposition of pared $3$-manifolds}
%%%%%%%%%%%%%%%%%%%%%%%%%%%%%%%%%%%%%%%%%%%%%%%%%

In the geometrically finite case, we've reduced the quasi-Fuchsian surface
subgroup question to a topological problem about $\pi_1$-injective maps of
a surface into a pared $3$-manifold. However, infinite volume hyperbolic
$3$-manifolds may have quite complex topology, so it is not immediately clear
how to find such surfaces in the resulting pared $3$-manifolds. It is not clear
how to find (QF) surfaces directly. Instead, we decompose the pared
$3$-manifold along disks and annuli. While the precise statements and proofs
that follow are new, these are standard techniques.

See \cite{Ja}, \cite{JacoShalen}, and \cite{Johannson} for an overview of the
JSJ theory we will use.  See \cite{CMc} for some discussion of JSJ theory for
pared $3$-manifolds.  Note also that our choice of decomposition was inspired
by the decomposition used by Thurston to prove geometrization of Haken
manifolds.  This is a decomposition into acylindrical pieces, $I$-bundles, and
solid tori.  See \cite{Mo} or \cite{ThurstonIII} for details.

First, a few simple propositions.

\begin{prop}

Let $M$ be a compact orientable irreducible $3$-manifold. Let $f \colon S \to
M$ be a $\pi_1$-injective map, where $S$ is a closed surface.  Let $D \cin M$
be a compression disk for $\bd M$. That is, $D$ is properly embedded, and $\bd
D$ does not bound a disk in $\bd M$. Then $f$ is homotopic to a map such that
$f(S) \cap D = \emptyset$. Given any finite collection of disjoint compression
disks we can homotope f away from all of them.

\end{prop}

\begin{proof}

Intuitively, $f$ is $\pi_1$-injective, so any loop where $S$ intersects $D$ is
contractible in $D$, hence contractible in $S$. Then use irreducibility and an
innermost disk argument.

More precisely, we proceed as follows. Homotope $f$ locally so it's transverse
to $D$. $D$ is compact and properly embedded, and $S$ is compact, so
$f^{-1}(D)$ is a properly embedded compact submanifold of $S$. That is, it's
a union of finitely many disjoint simple closed curves
$\alpha_1,\dots,\alpha_k$.  For each $\alpha_i$, $f(\alpha_i) \cin D$, so
$f(\alpha_i)$ is homotopically trivial in $M$.  Since $f$ is $\pi_1$-injective,
each $\alpha_i$ must be homotopically trivial in $S$.  Since $S$ is a closed
surface, this means each $\alpha_i$ bounds a disk in $S$.  Choose an innermost
such disk $D' \cin S$, that is, $D'$ is bounded by some $\alpha_i$ and doesn't
contain any other $\alpha_j$ in its interior.

Now construct a map $\phi \colon S^2 \to M$ as follows. The closed upper
hemisphere is homeomorphic to $D' \cin S$. Map it in along $D'$. Then the
equator maps in along $f(\alpha)$, which is contractible, as it's in $D$.
Choose the lower hemisphere of $\phi$ to correspond to such a contraction of
$f(\alpha)$ within $D$.  $M$ is irreducible, hence aspherical, so $\phi$ must
be homotopically trivial.  Homotoping $\phi$ to a point and re-expanding, we
can see that $f|_{D'}$ is homotopic to a map into $D$ (ie corresponding to
a contracting homotopy of $\alpha$). Pushing $f$ slightly away from $D$ using
a tubular neighborhood of $D$ in $M$, we can eliminate the intersection curve
$f(\alpha)$ between $f(S)$ and $D$.  Since this homotopy only affects
a neighborhood of $D' \cin S$, it cannot add any additional intersections or
affect any of the other intersection curves.  Repeating this process, we can
eliminate all intersection curves by homotopy.

\end{proof}

We have a similar fact for the pared locus of a pared $3$-manifold.

\begin{prop}\label{P:pared1}

Let $(M,P)$ be a pared $3$-manifold where $M$ is compact orientable
irreducible, and $D \cin M$ a compression disk for $\bd M$. Then we can isotope
$P$ within $\bd M$ such that $P$ has minimal intersection with $\bd D$. In this
minimal intersection, torus components of $P$ do not intersect $D$, and if we
cut $M$ along $D$, all the resulting rectangles obtained by cutting annulus
components of $P$ are essential.  We can perform a similar operation for any
finite collection of disjoint compression disks.

\end{prop}

\begin{proof}

This is straightforward. Torus components of $P$ cannot intersect $\bd D$, as
each torus component of $P$ is precisely a torus component of $\bd M$. This is
because a torus cannot map $\pi_1$-injectively to any other genus surface, as
it'd correspond to a rank 2 abelian subgroup. So if $\bd D$ intersects a torus
component of $P$, it'd have to lie entirely inside it. Then $D$ would be
a compression disk for this torus component, violating $\pi_1$-injectivity.

Now consider annulus components of $P$ that lie on the boundary component $F$
of $M$ that contains $\bd D$. It is sufficient to consider the core curves of
these annuli, which are a disjoint set of simple closed curves in $F$. Our
statement now follows from the elementary fact (about curves on surfaces) that
we can put these in minimal position with respect to another simple closed
curve, $\bd D$, such that the intersection contains only essential arcs.
Briefly, if there are inessential arcs, they cobound disks, so begin with the
innermost disks and isotope to remove any inessential arcs. Again, we can
repeat this process for any disjoint collection of compression disks. This
completes the proof.

\end{proof}

\begin{cor}

Let $(M,P)$ be a pared $3$-manifold, and $f \colon S \to M$ a $\pi_1$-injective
map of a closed surface. Suppose $P_0$ is a component of $P$ which has an
essential intersection with a compression disk $D$. By abuse of notation, we'll
refer to its core curve as $P_0$ as well. Then one of the following holds:

\begin{enumerate}

\item $P_0$ is homotopic in $M$ to a different curve in $\bd M$ that does not
intersect $D$. That is, there are sufficiently many compression disks cobounded
by arcs in $P_0$ with endpoints on $\bd D$ that ``transfer $P_0$ across $D$ to
a different part of $\bd M$.''

\item $f_*(\pi_1S)$ does not intersect any conjugates of $\pi_1 P_0$, that is,
a multiple of $P_0$ is not freely homotopic in $M$ into any image of a loop in
$S$.

\end{enumerate}

\end{cor}

\begin{proof}

Note that if the algebraic intersection number between $P_0$ and $D$ is
nonzero, it immediately follows that $P_0$ cannot be homotoped in $M$ into the
image of a loop in $S$ (which would have to be disjoint from $D$). The
algebraic intersection number is a homotopy, indeed a homology invariant in
$M$.  So we need only consider the case of zero algebraic intersection.

Deformation retract $D$ to a point, and choose this as our basepoint for
$\pi_1M$.  This yields an expression for $\pi_1M$ as either a free product
$\pi_1M_1 * \pi_1M_2$ (if $M \setminus D$ is disconnected) or trivial HNN
extension $\pi_1M_0 * = \langle\pi_1M_0,t\rangle$ (if $M \setminus D$ is
connected).  Since $S$ can be homotped to be disjoint from $D$, $f_* (\pi_1S)$
is contained in one of these subgroups $\pi_1M_1$,$\pi_1M_2$,$\pi_1M_0$.
Cutting $P_0$ along $D$ breaks it into a union of arcs, and after the
deformation retract this corresponds to a word in the free product / HNN with
each arc inducing a single subgroup letter in the word (or pair of letter, in
the HNN case when the arc connects between the two copies of $D$ in $M_0$).  In
the HNN case, note that the zero algebraic intersection implies that the number
of times the stable generator $t$ appears in such a word is the same as the
number of times $t-1$ appears.

But in the HNN case, $t$ does not appear in $f_*(\pi_1S)$.  Similarly, in the
free product case, words from only one of $\pi_1M_1$ or $\pi_1M_2$ appear in
$f_*(\pi_1S)$.  Without loss of generality say it's $\pi_1M_1$. That is, in
order for (2) to be false, that is, for a multiple of $P_0$ to appear in
$f_*(\pi_1S)$, we must have sufficient cancellation in the element $P_0$ to
remove all powers of $t$, or all letters from $\pi_1M_2$. We claim that (1)
holds in this case.

To see this, observe that $\pi_1M$ is a \emph{trivial} free product or HNN, so
there are no added relations. The only way to produce an element in a subgroup
is if all the in-between letters are in fact trivial. For each letter in the
word that is trivial in its subgroup ($\pi_1M_1$, $\pi_1M_2$, or $\pi_1M_0$),
we have a basepointed closed curve in $M$, which corresponds to a contractible
boundary arc in the original $M$ (before deformation retract) with both
endpoints on $D$.  Connect the endpoints of this arc with a segment along $\bd
D$ to produce a closed curve in $\bd M$. This is homotopically trivial, by the
above word argument.  Therefore, by Dehn's lemma, it bounds a disk in $M$.
Pushing $P_0$ along this disk and slightly outside it moves it to a new
position in $\bd M$ with fewer intersections with $\bd D$. Using an innermost
disk argument, we can do this isotopically to obtain (1).

\end{proof}

Together these demonstrate that we can reduce our main problem when $M$ admits
boundary compressions. All closed $\pi_1$-injective surfaces can be made to
avoid the compression disks, and all components of the pared structure can
either be made to avoid the compression disks (case (1)) or are irrelevant
because they cannot be homotoped into any closed $\pi_1$-injective surface
(case (2)). We push the case (1) pared structure components away from the
compression disks and discard the case (2) components. Note of course that no
case (1) component can fail to be $\pi_1$-injective, as it'd already fail to be
$\pi_1$-injective before the compression (by homotoping across the compression
disk). In fact the result is a new (possibly disconnected) pared $3$-manifold.
The resulting (possibly disconnected) incompressible pared manifold contains
a (QF) surface if and only if the original does.

Therefore we can reduce to the case where $M$ is a pared $3$-manifold (compact
orientable irreducible) with incompressible boundary. However, in order to
understand the problem better, we want another type of decomposition, along
disjoint properly embedded essential annuli. We obtain the following similar
results for a decomposition along annuli.

\begin{prop}

Let $M$ is compact orientable irreducible with incompressible boundary, and $f
\colon S \to M$ be a $\pi_1$-injective map of a closed surface. Let $A \cin M$
be a properly embedded annulus which is essential --- that is, incompressible,
boundary incompressible, and not parallel to an annulus in $\bd M$. Then $f$ is
homotopic to a map such that $f(S) \cap A$ is a union of multiples of the core
curve of $A$, each of which pulls back to an essential curve in $S$. For
a disjoint union of annuli, we have the same result.

\end{prop}

\begin{proof}

Homotope $f$ to be transverse to $A$. As earlier, $f^{-1}(A)$ is a union of
simple closed curves in $S$. Each curve that bounds a disk in $S$ is
contractible in $M$, hence its image is contractible in $A$ since $A$ is
essential.  These two contractions, each of which is a singular map of a disk
into $M$, together induce a map $S^2 \to M$. Since $M$ is irreducible, it is
aspherical. So this sphere is contractible, that is, there is a singular map
$B^3 \to X$ filling it. Homotoping across this sphere, we can remove this
intersection between $f(S)$ and $A$. Doing this repeatedly ensures that all
intersection curves in $f(S) \cap A$ pull back to essential curves in $S$.

Similarly, any component of $f(S) \cap A$ is either homotopic to a multiple of
the core curve or to a contractible loop. We can locally homotope $f$ such that
$f(S)$ intersects $A$ in curves of this form. Then each contractible curve must
correspond to a curve that's contractible in $S$ as well, by
$\pi_1$-injectivity.  By the same ball argument as above, we can homotope to
eliminate any such intersection. These homotopies are all local, so don't
affect any other intersections between $f(S)$ and $A$, or between $f(S)$ and
other annuli. Therefore we can do this for all intersections between $f(S)$ and
for a disjoint union of annuli. This completes the proof.

\end{proof}

We also have a similar fact about the pared locus, but for annuli.

\begin{prop}

Let $(M,P)$ be a pared $3$-manifold where $M$ is compact oriented irreducible
and $A \cin M$ is a properly embedded essential annulus. Then we can isotope
$P$ within $\bd M$ such that $P$ has minimal intersection with $\bd A$.  In
this minimal intersection, if we cut $M$ along $A$, all the resulting
rectangles obtained by cutting annulus components of $P$ are essential.  We can
perform a similar operation for any finite collection of disjoint properly
embedded essential annuli.

\end{prop}

\begin{proof}

This is proven identically to Proposition~\ref{P:pared1} above.

\end{proof}

We will need the following definition.

\begin{defn}

Let $M$ be a compact $3$-manifold with boundary which is not a 3-ball. $M$ is
\emph{acylindrical} if every proper $\pi_1$-injective map $A^2 \to M$ is
homotopic into $\bd M$.

\end{defn}

Note that strictly speaking, balls have incompressible boundary and satisfy the
acylindrical condition, but we will want to consider them as a separate case
later in this chapter. We have the following key fact from the theory of JSJ
decompositions.  Analogously to the Loop and Sphere Theorems, it allows us to
go from singular surfaces to embedded surfaces.

\begin{thm}[Annulus theorem]

Let $M$ be a compact orientable irreducible $3$-manifold with incompressible
boundary. If $M$ is not acylindrical, then $M$ admits a properly embedded
essential annulus.

\end{thm}

\begin{proof}

See \cite{JacoShalen} or \cite{Johannson}. See \cite{Scottannulus} for an
algebraic proof.

\end{proof}

We construct a hierarchy for a compact orientable irreducible $3$-manifold
M with nonempty boundary.  Let $M_0 = M$.  Construct $M_{i+1}$ from $M_i$ as
follows:

\begin{enumerate}

\item If $i$ is even, choose a maximal set of disjoint non-parallel compression
disks $D_{ik}$ for the boundary of $M_i$. Cut $M_i$ along these disks to obtain
$M_{i+1}$.

\item If $i$ is odd, choose a maximal set of disjoint non-parallel properly
embedded essential annuli $A_{ik}$. Cut $M_i$ along these annuli to obtain
$M_{i+1}$.

\end{enumerate}

In both cases, if $M_i$ is not connected, we make these choices in all
components.

\begin{defn}

We refer to this as an \emph{alternating hierarchy} for $M$.

\end{defn}

\begin{lemma}[Alternating hierarchy lemma]

Let $M$ be a compact orientable irreducible $3$-manifold with nonempty
boundary.  Then the alternating hierarchy for $M$ has finite length $n$. Each
component of $M_n$ is either a ball or an acylindrical $3$-manifold with
incompressible boundary.

\end{lemma}

\begin{proof}

It is immediately clear that the hierarchy terminates in a union of balls and
acylindrical $3$-manifolds with incompressible boundary. If any component did
not have incompressible boundary, we could cut along a compression disk.
Similarly, if any component was not acylindrical, we could apply the annulus
theorem to find an annulus to cut along. It remains to show that the hierarchy
is finite length.

By cutting out tubular neighborhoods of the disks and annuli we're removing, we
can embed each level of the hierarchy: $M_n \hookrightarrow M_{n-1}
\hookrightarrow \dots \hookrightarrow M_0$. Furthermore, by isotoping along
a small tubular neighborhood of the boundary, we can in fact embed each $M_i$
into the interior of $M_{i-1}$. The hierarchy induces a hierarchy of nested
disjoint embedded surfaces $\bd M_n, \bd M_{n-1}, dots, \bd M_0$. For odd $i$,
each $\bd M_i$ is incompressible in $M_{i-1}$, since we constructed $M_i$ from
a maximal set of compression disks for $\bd M_{i-1}$.  We claim that each odd
$\bd M_i$ is in fact incompressible in $M_{i-2}$.  Suppose not, and let $D \cin
M_{i-2}$ be a compression disk for $\bd M_i$. Let $A \cin M_{i-2}$ be one of
the essential annuli that we cut along to obtain $M_{i-1}$. Both $A$ and $D$
are properly embedded in $M_{i-2}$, so after making them transverse, $A \cap D$
is a union of disjoint properly embedded arcs and curves in $D$.

First, suppose the intersection contains at least one closed curve. Take an
innermost such curve $\alpha \cin A \cap D$ --- that is, one such that its
interior disk $D' \cin D$ does not intersect any other curves in $A \cap D$.
$\alpha$ cannot be essential in $A$, as otherwise $D'$ would induce
a compression of $A$, contradicting the assumption that $A$ was essential in
$M_{i-2}$.  But $A$ is an annulus, so this implies $\alpha$ bounds a disk $D''$
in $A$.  Since $M$ is irreducible, the $M_i$ are irreducible as well.  So $D'
cup D''$ bounds a ball in $M_{i-2}$, and we can isotope $D$ across this ball to
remove the intersection.  Repeating this process, we can remove all closed
curve intersections.

We now claim that $A$ and $D$ cannot intersect in any arcs. If they do, choose
an outermost such arc --- that is an arc $\alpha \cin A \cap D$ such that, for
some $\beta \cin \bd D$ connecting the endpoints of $\alpha$, $\alpha \cup
\beta$ bounds a disk $D'$ that does not intersect any other arcs in $A \cap D$.
But $\bd D \cin \bd M_{i-2}$, so this $D'$ is a boundary compression disk for
$A$, contradicting the assumption that $A$ was essential. Therefore after this
isotopy we can guarantee that $D$ is disjoint from $A$.

Repeating this for all such cutting annuli $A$, $D$ passes to a properly
embedded disk in $M_{i-1}$. But $\bd M_i$ is incompressible in $M_{i-1}$, so
$\bd D$ must bound a disk $D''$ in $\bd M_{i-1}$. The cutting annuli in
$M_{i-2}$ induce boundary annuli in $M_{i-1}$. As proven above, none of these
annuli intersect $D$.  No annuli can be contained in $D''$ as they wouldn't be
essential (as $D''$ is a boundary disk). So no annuli can intersect $D''$.
Therefore the disk $D''$ induces a boundary disk in $M_{i-2}$. This proves that
$\bd M_i$ is incompressible in $M_{i-2}$.

Furthermore, we claim that for odd $i$, unless the sequence has terminated
(that is, $M_i=M_{i-1}=M_{i-2}$), $\bd M_i$ is not boundary parallel in
$M_{i-2}$.  That is, $\bd M_i$ and $\bd M_{i-2}$ are not parallel in $M_{i-2}$.
Suppose that they are.  Observe that there must be at least one properly
embedded essential annulus $A \cin M_{i-2}$ that we cut along. $A$ corresponds
to two embedded annuli $A_0, A_1 \cin \bd M_{i-1}$ after cutting.  Note that
the core curves of both $A_0$ and $A_1$ are essential in $\bd M_{i-1}$, as
otherwise $A$ would not be essential in $M_{i-2}$.  Look at the compression
disks we cut along to reduce $M_{i-1}$ to $M_i$.  Suppose that none of these
disks essentially intersect $A_0$ (that is, we can isotope them so they don't
intersect $A_0$). Now simply observe that $A_0$ is itself boundary parallel in
$M_{i-1}$ (it's part of the boundary), and therefore boundary parallel in $M_i$
as well. But $\bd M_i$ is parallel to $\bd M_{i-2}$ by assumption, so $A_0$ is
boundary parallel in $M_{i-2}$, contradicting the assumption that $A$ was
essential in $M_{i-2}$.

Now suppose there are cutting disks that essentially intersect $A_0$. Then it
is immediately clear that $\bd M_i$ cannot be parallel to $\bd M_i$, as a curve
following along one of the boundary components of $A$ (which induces one of the
boundary components of $A_0$) cannot possibly be parallel to a closed curve in
$\bd M_{i-2}$, as it'd have to cross the compression. This proves that $\bd
M_i$ is not boundary parallel in $\bd M_{i-2}$.

This means that considering only the odd terms in this alternating
decomposition, we have a Haken hierarchy. Finite depth immediately follows from
Haken finiteness. This completes the proof.

\end{proof}

We now combine the earlier propositions with this lemma. Given an alternating
hierarchy for a pared $3$-manifold, at each stage, we can push the pared
structure to avoid any decomposition disks and have essential intersections
with the decomposition annuli. Note that our propositions above are not
sufficient to push the pared structures repeatedly down the hierarchy, as after
the first annular step in the decomposition the pared locus includes rectangles
(intersecting the annulus cuts essentially) in addition to annuli and tori.
This requires the notion of some kind of boundary pattern to keep track of
where in the boundary we cut along annuli and disks at each inductive step.
Therefore, to make this argument precise, we'll need to generalize the above
propositions. This ought to be straightforward.

At the bottom of the hierarchy is a union of acylindrical $3$-manifolds and
balls. We make the following conjecture:

\begin{conj}

Let $(M,P)$ be a pared acylindrical $3$-manifold. Then $M$ admits a (QF) surface.

\end{conj}

Intuitively, this is because any acylindrical $3$-manifold $M$ contains a great
many proper essential $\pi_1$-injective surfaces $S$. If we combine $S$ with
$\bd M$ where they meet, this induces a map from a book of $I$-bundles $M'$
into $M$. We believe that applying a combination theorem, for instance that of
Baker--Cooper \cite{BCcombination} or Agol--Groves--Manning
\cite{AGMcombination}, we can guarantee (under the right conditions) that this
map is $\pi_1$-injective. This allows us to pull the problem back to $M'$,
pulling the pared structure back and constructing a (QF) surface there. By
$\pi_1$-injectivity, This will induce a (QF) surface in $M$.

We also believe that in \emph{most} cases with no acylindrical pieces (that is,
manifolds where the bottom of the hierarchy is a union of balls), there should
still be a (QF) surfaces. We hope the same book of $I$-bundlegs mapping
argument should work in most cases. More work is needed.

In the remainder of this paper, we address the case of books of $I$-bundlegs.
