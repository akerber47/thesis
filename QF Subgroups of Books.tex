\documentclass[12pt]{amsart}
\usepackage{amsmath,amscd,amssymb,amsthm,amsfonts}

\setlength{\topmargin}{0.5cm}
\setlength{\oddsidemargin}{-0.2cm}
\setlength{\evensidemargin}{-0.2cm}
\textheight = 22cm
\textwidth = 16.2cm

% For proof reading
%\renewcommand{\baselinestretch}{2.5}

\newtheorem{theorem}{Theorem}[section]
\newtheorem{thm}[theorem]{Theorem}
\newtheorem{lemma}[theorem]{Lemma}
\newtheorem{prop}[theorem]{Proposition}
\newtheorem{cor}[theorem]{Corollary}

\theoremstyle{definition}
\newtheorem{defn}[theorem]{Definition}
\newtheorem{conj}[theorem]{Conjecture}

\theoremstyle{remark}
\newtheorem{example}[theorem]{Example}
\newtheorem*{rmk}{Remark}
\newtheorem*{claim}{Claim}
\newtheorem*{ntn}{Notation}

\newcommand{\x}{\times}
\newcommand{\bd}{\partial}
\newcommand{\Om}{\Omega}
\newcommand{\Si}{\Sigma}
\newcommand{\cin}{\subseteq}

\newcommand{\cC}{\mathcal{C}}
\newcommand{\cM}{\mathcal{M}}
\newcommand{\cN}{\mathcal{N}}

\begin{document}

\title{Quasi-Fuchsian surface subgroups of books of I-bundles}

\author{Alvin Kerber}
%\address{University of California, Berkeley\\
%970 Evans Hall \\
%Berkeley, CA, 94720}
    \email{alvin@math.berkeley.edu}
%\thanks{Agol supported by something or other}
%\date{%
%\today}

\begin{abstract}

Given a Kleinian group $\Gamma$, one can ask whether the group contains any
quasi-Fuchsian surface subgroups. Equivalently, given a pared 3-manifold
$(M,P)$, one can ask whether there exists a closed immersed $\pi_1$-injective
surface in $M$ that avoids the peripheral subgroups associated to $P$.  This is
known to be true for closed hyperbolic 3-manifolds, and more generally for
finite volume hyperbolic 3-manifolds. We outline a strategy to solve the case
of infinite volume hyperbolic 3-manifolds. We prove explicit results in the
special case where $M$ is a book of $I$-bundles.

\end{abstract}


\maketitle

%%%%%%%%%%%%%%%%%%%%%%%%%%%%%%%%%%%%%%%%%%%%%%%%%
\section{Introduction}
%%%%%%%%%%%%%%%%%%%%%%%%%%%%%%%%%%%%%%%%%%%%%%%%%

Let $N$ be a complete hyperbolic 3-manifold. $N$ can be realized as the
quotient manifold of a Kleinian group $\Gamma=\pi_1N$. This paper is devoted to
the following general question:

\begin{prob}

Does $\pi_1N$ contains any quasi-Fuchsian surface subgroups?

\end{prob}

In this paper, we answer this question in the special case where $N$ is a book
of $I$-bundles. We tackle this problem topologically, representing $N$ with
a pared 3-manifold $(M,P)$.  The pared structure $P\cin\bd M$ specifies the
locus corresponding to cusps of $N$. Finding a quasi-Fuchsian surface subgroup
of $\pi_1N$ amounts to asking whether we can find a closed immersed
$\pi_1$-injective surface in $M$ which avoids $P$ up to homotopy.  That is,
we're trying to choose a surface subgroup of $\pi_1M$ which trivially intersect
any of the peripherial subgroups associated to components of $P$. We prove full
results on which $(M,P)$ admit a surface subgroup in the book of $I$-bundles
case.

To motivate this choice of problem and special case to solve, we briefly
discuss the history of this problem and previous work on it. We begin with
a straightforward, purely topological question:

\begin{prob}

Which 3-manifold groups contain surface subgroups? Equivalently, which
3-manifolds contain closed $\pi_1$-injective surfaces?

\end{prob}

This problem arises quite directly when considering 3-manifold groups. The
first explicit answer was proposed by Waldhausen (cite: Kirby problem list) in
the Surface Subgroup Conjecture: every closed irreducible 3-manifold with
infinite $\pi_1$ contains a surface subgroup. As originally stated, the
Conjecture only applies to closed 3-manifolds. It was mostly solved by
geometrization, as the question is straightforward for Seifert fibered spaces.
After some preliminary efforts in the hyperbolic case by Cooper-Long
\cite{CooperLong} and Li \cite{Li}, it was finally
resolved in general by Kahn and Markovic in 2012 \cite{KM}.  In fact, the
surface subgroups they find are quasi-Fuchsian, and they also obtained
a density result stating that there are many such surfaces \cite{KM2}.

We can naturally ask this question for manifolds with boundary as well.
Cooper, Long, and Reid showed that a compact connected irreducible 3-manifold
with non-empty incompressible boundary  must contain an essential closed
surface, unless it's covered by a product $F\times I$ \cite{CLR}. In
particular, this shows that every non-closed hyperbolic 3-manifold group
contains a surface subgroup.

However, since the result of Kahn and Markovic provided us with surface
subgroups which are quasi-Fuchsian, we might ask whether we can do this in the
non-closed case as well. Note that the Cooper-Long-Reid result is insufficient,
as the surfaces obtained may contain accidental parabolics, that is, overlap in
$\pi_1$ with the pared locus $P$. If this occurs they will not be
quasi-Fuchsian. This does commonly occur in practice --- see \cite{CLRbundles}
and \cite{MenascoReid} for examples. Of course, in the closed case there cannot
be any parabolics to worry about, as the boundary is empty.

% FIXME something is wrong with Masters-Zhang citation / bib entry - I can only
% find their older paper (hyp knot complements) on mathscinet, not the full new
% one with link complements

More recently, there have been efforts to find quasi-Fuchsian surface subgroups
specifically. Masters and Zhang \cite{MZ} showed that every hyperbolic knot
complement contains a quasi-Fuchsian surface subgroup, and later extended this
result to link complements \cite{MZ2}. This fully addresses the case of finite
volume cusped hyperbolic 3-manifolds. An alternate proof was provided by
Baker-Cooper \cite{BC}.

The remaining case is infinite volume hyperbolic 3-manifolds. We So far we've only
addressed a particular example of this case --- books of $I$-bundles. We
conjecture that all books of $I$-bundles admit quasifuchsian surface subgroups,
except for a few obvious negative cases where the pared locus is too large. We
prove this in a few special cases, so far.


%%%%%%%%%%%%%%%%%%%%%%%%%%%%%%%%%%%%%%%%%%%%%%%%%
\section{Quasi-Fuchsian surface subgroups}
%%%%%%%%%%%%%%%%%%%%%%%%%%%%%%%%%%%%%%%%%%%%%%%%%

We assume familiarity with standard terminology and results of 3-manifold
topology and Kleinian groups.  For 3-manifold topology, see \cite{He},
\cite{Ja}, \cite{Sch}, \cite{Thurstonbook}, \cite{Thurstonnotes},
\cite{Lackenbynotes}, \cite{Hatchernotes}, and \cite{Calegarinotes2}.  For
general hyperbolic geometry, see \cite{Ratcliffe} or \cite{BenedettiPetronio}.
And for Kleinian groups, see \cite{Maskit}, \cite{Marden},
\cite{MatsuzakiTaniguchi}, \cite{Thurstonnotes}, \cite{Calegarinotes},
\cite{Kapovich}, \cite{Hubbard}.

All Kleinian groups in this paper are assumed to be finitely generated and
torsion-free. Unless stated otherwise, all hyperbolic manifolds are complete.

We first recall the following basic definitions.

\begin{defn}

A \emph{surface group} is a group isomorphic to the fundamental group of
a closed surface of negative Euler characteristic. Such a group can be written
with the finite presentation
%
$\pi_1 \Si_g = \langle a_1,b_1,\dots,a_g,b_g | a_1b_1a_1^{-1}b_1^{-1} \dots
a_gb_ga_g^{-1}b_g^{-1} = 1\rangle$
%
where $g \geq 2$.

\end{defn}

Note that surface groups are not free. The fundamental group of a compact
surface with boundary of negative Euler characteristic is $\Si_{g,b}$ is the
free group on $2g + b - 1$ generators. We wish to consider surface subgroups of
Kleinian groups which are quasi-Fuchsian.

\begin{defn}

A {quasi-Fuchsian group} is a Kleinian group $\Ga < PSL2C$ for which there
exists a quasi-conformal homeomorphism $f \colon S^2_\infty \to S^2_\infty$
which conjugates $\Ga$ to a Fuchsian group, that is, a discrete subgroup of
$PSL2R$.

\end{defn}

As discussed in the introduction, our general problem statement is quite
simple.

\begin{prob}

Let $\Ga$ be a Kleinian group of infinite covolume. Does $\Ga$ contain
a surface subgroup which is quasi-Fuchsian?

\end{prob}

We know that being quasi-Fuchsian is equivalent to a few other conditions.
These will be the conditions we actually use in this paper.

\begin{prop}

Let $S$ be a compact surface (possibly with boundary), and $\phi \colon \pi_1S
\to PSL2C$ be a discrete faithful representation. The following are equivalent.

\begin{enumerate}

\item $\phi(\pi_1S)$ is quasi-Fuchsian.

\item $\phi(\pi_1S)$ is geometrically finite, and the conjugates of peripheral
subgroups of $\pi_1S$ are precisely those elements of $\pi_1S$ which are
parabolic after applying $\phi$.

\item $\bH^3/\phi(\pi_1S)$ has convex core homeomorphic to $S \times I$.

$\phi(\pi_1S)$ has limit set a Jordan curve, and no element of $\phi(\pi_1S)$
interchanges the complementary components.

\end{enumerate}

\end{prop}

\begin{proof}

See \cite{Mo}.% Proposition 9.2

\end{proof}

As discussed in the introduction, we wish to identify quasi-Fuchsian surface
groups --- that is, quasi-Fuchsian groups as above where $S$ is a closed
surface.  Condition (2) above tells us that such a subgroup must be
geometrically finite.  Furthermore, it must not contain any parabolic elements
at all.

We now make another observation. By condition (3), any quasi-Fuchsian surface
group $\phi(\pi_1S)$ will contain an embedded copy of the surface $S$ inside
its convex core.  This embedding $i \colon S \to \bH^3/\phi(\pi_1S)$ is an
isomorphism on $\pi_1$.  Given a Kleinian group $\Ga$, a quasi-Fuchsian surface
subgroup $\phi(\pi_1S) < \Ga$ corresponds to a covering map $p \colon
\bH^3/\phi(\pi_1S) \to \bH^3/\Ga$. This map is $\pi_1$-injective.  Projecting
$i(S)$ downward into $\bH^3/\Ga$, we obtain a closed surface $p(i(S))$. $p
\circ i \colon S to \bH^3/\Ga$ is $\pi_1$-injective. In fact, $(p \circ
i)_\star(\pi_1S))) = \phi(\pi_1S)$.

We can therefore re-state our general problem as follows.

\begin{prob}

Let $N$ be an infinite volume hyperbolic 3-manifold. That is, $\pi_1N$ is
a Kleinian group of infinite covolume. Does there exist a $\pi_1$-injective map
$f \colon S \to N$, where $S$ is a closed surface, such that $f_\star(\pi_1S)$
is geometrically finite and contains no parabolic elements?

\end{prob}

In the remainder of this paper, we'll refer to surface maps $(S,f)$ which
satisfy the problem condition as \emph{quasi-Fuchsian surfaces} or \emph{(QF)
surfaces}. By abuse of notation, we'll often simply refer to the image $f(S)$
as a (QF) surface if the choice of $S$ and $f$ is clear.

However, even though we are now looking for a topological surface map, solving
this question as written still requires knowledge of the geometry of $N$. We
need to know which elements of $\pi_1N$ are parabolic, and whether or not
a given subgroup is geometrically finite. We now discuss how to reduce these
conditions to a purely topological conditions on $S$ and $f$.

%%%%%%%%%%%%%%%%%%%%%%%%%%%%%%%%%%%%%%%%%%%%%%%%%
\section{Pared 3-manifolds and hyperbolization}
%%%%%%%%%%%%%%%%%%%%%%%%%%%%%%%%%%%%%%%%%%%%%%%%%

%Definition and important facts about pared manifolds. After Morgan, The Smith
%Conjecture, V ("Uniformization Theorem for Three-Dimensional Manifolds"),
%p 58-60. Or Canary-McCullough, Homotopy Equivalences of 3-Manifolds and
%Deformation Theory of Kleinian Groups, Ch 5 p. 87-92. Also Ch 7 p.105-107.
%Note that Canary-McCullough is much more recent.
%
%(Morgan p58) (Canary-McCullough p87)
\begin{defn}

A \emph{pared 3-manifold} $(M,P)$ is a compact orientable irreducible
3-manifold $M$, together with a submanifold $P\cin\bd M$, such that the
following conditions hold:

\begin{enumerate}
\item Every component of $P$ is a torus or annulus, incompressible in $M$.

\item Every noncyclic abelian subgroup of $\pi_1M$ is peripheral with respect
to $P$ -- ie, conjugate to the fundamental group of a component of $P$.

\item $(M,P)$ is ``$A^2$-incompressible'': every $\pi_1$-injective map $(A^2,
\bd A^2) \to (M,P)$ is homotopic (as a map of pairs) to a map into $P$.

\end{enumerate}

We call $P$ the \emph{pared locus} or \emph{parabolic locus} of the pared
3-manifold $(M,P)$.

\end{defn}

Note that there are a few pared manifolds that are special cases, like with
elementary Kleinian groups. In fact, these are precisely the pared manifolds
that correspond to elementary Kleinian groups when we construct pared
3-manifolds from Kleinian groups below.

%(Canary-McCullough p88)
\begin{defn}

A pared manifold $(M,P)$ is \emph{elementary} if it is homeomorphic (as a pair)
to one of the following: $(T^2\x I,T^2\x 0)$, $(A^2\x I,A^2\x 0)$, or $(A^2\x
I,\emptyset)$, or $(S3,\emptyset)$.

\end{defn}

Pared manifolds arise naturally in the study of Kleinian groups. Intuitively,
given a geometrically finite Kleinian group $\Ga$, the pared 3-manifold is
a compact core, that is a compact submanifold $M \cin \bH^3/\Ga$ with $\pi_1M
= \Ga$.  The pared locus $P$ is the union of boundary components corresponding
to cusps of $\bH^3/\Ga$, that is, parabolic subgroups of $\Ga$ (up to
conjugacy).  Reversing this correspondence hyperbolizes a pared 3-manifold,
giving it a geometrically finite geometric structure where the desired boundary
regions become cusps.

To be precise, given a geometrically finite Kleinian group $\Ga$, we construct
a pared 3-manifold $(M,P)=(\cM(\Ga),\cP(\Ga))$ as follows.

We know that the convex core $C(\Ga) \cin \bH^3/\Ga$ has finite volume. For
epsilon sufficiently small, The thick part $C(\Ga)[\epsilon,\infty)$ is
compact, and the thin part $C(\Ga)(0,\epsilon]$ is a union of small
neighborhoods of the cusps.  As long as epsilon is sufficiently small, the
exact choice of $\epsilon$ is arbitrary, as the resulting thick and thin parts
are homeomorphic.  Now let $\cM(\Ga) = C(\Ga)[\epsilon,\infty)$, and $\cP(\Ga)
= C(\Ga)[\epsilon] = \bd C(\Ga)[\epsilon,\infty) \cap \bd C(\Ga)(0,\epsilon]$.
That is, $\cP(\Ga)$ is the boundary locus along which we truncated to remove
the thin part. We can rebuild the convex core from $(\cM(\Ga),\cP(\Ga))$ by
gluing cusp neighborhoods onto each component of $\cP(\Ga)$.

\begin{prop}

$(\cM(\Ga),\cP(\Ga))$ is a pared 3-manifold. $\cM(\Ga)- \cP(\Ga)$ is
homeomorphic to the convex core of $\Ga$, $C(\Ga)$. The parabolic elements of
$\Ga$ precisely correspond to conjugates of the peripheral subgroups
$\pi_1\cP(\Ga) \cin \pi_1\cM(\Ga)$ (really, the conjugates of the peripheral
subgroups associated to each component of $\cP(\Ga)$).

\end{prop}

\begin{proof}

See \cite{Mo}. % Corollary 6.10

\end{proof}

We refer to $(\cM(\Ga),\cP(\Ga))$ as the \emph{pared 3-manifold associated to}
$\Ga$.  Similarly, if $N$ is a geometrically finite hyperbolic 3-manifold, we
refer to $(\cM(\pi_1N),\cP(\pi_1N))$ as the \emph{pared 3-manifold associated
to} $N$.

Conversely, there is Thurston's famous hyperbolization theorem:

\begin{thm}

Let $(M,P)$ be a pared 3-manifold with $M$ Haken. Then there exists
a geometrically finite hyperbolic 3-manifold $N$ such that
$\cM(\pi_1N),\cP(\pi_1N)$ is homeomorphic to $(M,P)$.

\end{thm}

\begin{proof}

See \cite{Mo}, \cite{ThurstonviaScott}, or \cite{Kapovich} for a broad overview
of the proof. Thurston first announced the theorem in \cite{Thurston0}. Key
details of the argument are presented in \cite{ThurstonI}, \cite{ThurstonII},
and \cite{ThurstonIII}.

\end{proof}

For more details and examples of pared 3-manifolds, see \cite{CMc}, \cite{Mo},
or \cite{ThurstonI}.

We now return to our original problem. We can convert a given geometrically
finite Kleinian group $\Ga$ to a pared 3-manifold $(M,P)$. Intuitively, since
$P$ corresponds to cusps, ie, parabolics, a surface subgroup of $\Ga$ avoids
parabolic elements if and only if it avoids the peripheral subgroups associated
to components of $P$.

First, suppose we have a geometrically finite Kleinian group.  We use the
following theorem of Thurston.

\begin{thm}

Let $N$ be a geometrically finite hyperbolic manifold with convex core $C(N)$.
Suppose that $N$ has infinite volume. Then every covering space $N'$ of $N$
with finitely generated nonelementary fundamental group is also geometrically
finite. In fact, it suffices to assume $N$ is geometrically finite and $\bd
C(N)$ is nonempty.

%In our case, we have a geometrically finite hyperbolic manifold of infinite
%volume, so letting E be an (infinite volume) relative end, C(N) cap E must
%yield a nonempty piece of bd C(N). This suffices to use the theorem.

\end{thm}

\begin{proof}

See \cite{Mo}. % Proposition 7.1.

\end{proof}

% This argument (the next few paragraphs) follows Morgan. Proposition 7.3

Let $(M,P)$ be the pared 3-manifold structure associated to the hyperbolic
3-manifold $N$. The only elements of $\pi_1M$ which are parabolic are the
conjugates of elements in $\pi_1P_0$, where $P_0$ is a component of $P$.

Let $\Ga' < \pi_1N$ be a surface subgroup which is quasi-Fuchsian. Let $N'$ be
the cover of $N$ associated to $\Ga'$. $N'$ and hence its convex core $C(N')$
have the fundamental group of a closed surface. In fact, because $\Ga'$ is
quasi-Fuchsian, we know that the convex core $C(N')$ is homeomorphic to $S
\times I$, where $S$ is a closed surface such that $\pi_1S = \Ga'$.  Map $S \to
S \times \{1/2\} \cin S \times I$. This map is an isomorphism on $\pi_1$.
Composing this with the covering map $N' \to N$, we obtain a $\pi_1$-injective
map $\phi \colon S \to N$.  This corresponds to a $\pi_1$-injective map to
$(M,P)$ by retracting.  Applying the theorem above, no element of
$\Ga'=\phi(\pi_1S)$ is parabolic.  Because $(M,P)$ is a pared structure for
$N$, this implies that no loop in $S$ is freely homotopic to a loop in $P$.

Conversely, suppose we are given a $\pi_1$-injective map $S \to M$ where $S$ is
a closed surface.  Suppose further that no loop in $S$ is freely homotopic to
a loop in $P$.  This implies that the induced subgroup $\pi_1S < \pi_1M$ is
disjoint from all conjugates of $\pi_1P$.  Because $(M,P)$ is a pared structure
for the hyperbolic 3-manifold $N$, this implies that no elements of $\pi_1S$
are parabolic under the representation $\rho \colon \pi_1S \to PSL2C$ induced
by $\pi_1S < \pi_1M=\pi_1N$.  Furthermore $\rho(\pi_1S)$ is geometrically
finite (by the first theorem), and $S$ is a surface without cusps.  Applying
the theorem, $\pi_1S$ induces a quasi-Fuchsian surface subgroup of $\pi_1N$.

We summarize our discussion of the geometrically finite (infinite volume) case
as follows.

\begin{prop}

Let $N$ be a geometrically finite hyperbolic 3-manifold of infinite volume, and
$(M,P)$ a pared 3-manifold structure for $N$. Then quasi-Fuchsian surface
subgroups of $\pi_1N$ precisely correspond to $\pi_1$-injective maps of
a closed surface $S \to M$ such that no loops in $P$ are freely homotopic to
images of loops in $S$.

\end{prop}

This allows us to reduce the main problem to the following:

\begin{prob}

Let $(M,P)$ be a pared 3-manifold (which has an infinite volume
hyperbolization).  Does there exist a closed surface $S$ and
a $\pi_1$-injective map $f \colon S \to M$ such that no loops in $P$ are freely
homotopic to images of loops in $S$?

\end{prob}

Similarly, we will refer to surfaces $(S,f)$ which satisfy this condition for
a pared 3-manifold $(M,P)$ as \emph{quasi-Fuchsian surfaces} or \emph{(QF)
surfaces}.  Again, by abuse of notation. We will simply refer to $f(S)$ as (QF)
if $f$ and $S$ are clear from context.

We now briefly discuss the geometrically infinite case. Because the arguments
that follow in later sections are purely topological arguments applied to the
pared structure, we can always use Thurston's uniformization theorem to produce
a geometrically finite Kleinian group to which our arguments apply. However, we
would like to be able to address the geometrically infinite case also.

First note that any finitely generated Kleinian group, whether geometrically
finite or geometrically infinite, is \emph{tame}, that is, homeomorphic to the
interior of a compact 3-manifold. This is a deep result due to Agol and
independently, Calegari and Gabai. See \cite{Agoltameness} and
\cite{CalegariGabai}.

We now begin with the following theorem of Canary.

\begin{thm}

Let $N = \bH^3/\Ga$ be an infinite volume tame hyperbolic 3-manifold, and let
$\Ga' < \Ga$ be a finitely generated subgroup. Then either

\begin{enumerate}

\item $N' = \bH^3 / \Ga'$ is geometrically finite, or

\item $\Ga'$ contains a (conjugate of a) finite index subgroup of
a geometrically infinite peripheral subgroup of $\Ga$.

\end{enumerate}

Above, a \emph{geometrically infinite peripheral subgroup} is a subgroup
corresponding to the subsurface of a relative compact core boundary which cuts
off a geometrically infinite relative end of N. Equivalently, since all
geometrically infinite ends of tame manifolds are simply degenerate,
a geometrically infinite peripheral subgroup is the subgroup associated to the
boundary in the interior of $N$ of a geometrically infinite end.

\end{thm}

\begin{proof}

See \cite{Canary}. Also see \cite{Canary2}, \cite{Bonahon}, and
\cite{Thurstonnotes} for previous related work.

\end{proof}

The above proposition in the geometrically finite case still works in the
forward direction. The reverse direction will carry through if we can show that
$N'$ (ie $\Ga'$) is geometrically finite. That is, we want to avoid condition
(2) in Canary's theorem above. This requires an additional condition, which we
incorporate in the following proposition. (2) holds precisely when our surface
is freely homotopic to a finite-sheeted cover of the corresponding
geometrically infinite end boundary surface. Since such an end is simply
degenerate, that is, homeomorphic to a thickened surface, we can also apply the
covering lemma (see \ref{L:covering}) to obtain the following proposition.

\begin{prop}

Let $N$ be a hyperbolic 3-manifold of infinite volume, possibly geometrically
infinite, and $(M,P)$ a pared 3-manifold structure for $N$. Then quasifuchsian
surface subgroups of $N$ precisely correspond to $\pi_1$-injective maps of
a closed surface $S \to M$ such that the following conditions hold.

\begin{enumerate}

\item No loops in $P$ are freely homotopic to images of loops in $S$.

\item $S$ is not freely homotopic into a geometrically infinite end of $M$, or,
equivalently, not freely homotopic to a finite-sheeted cover of a geometrically
infinite end boundary surface.

\end{enumerate}

\end{prop}

We will not elaborate on the geometrically infinite ends case in this paper.
However, analogous constructions should provide a proof of results similar to
our main result.


%%%%%%%%%%%%%%%%%%%%%%%%%%%%%%%%%%%%%%%%%%%%%%%%%
\section{Topology of books of I-bundles}
%%%%%%%%%%%%%%%%%%%%%%%%%%%%%%%%%%%%%%%%%%%%%%%%%

In this section we establish basic definitions and facts about books of
I-bundles and quasi-Fuchsian surfaces contained inside them.

Recall that we are considering possible pared structures $P$ on $M$ a convex
hyperbolic 3-manifold. Our goal is to find a surface $S \cin M$ such that

\begin{equation}\label{E:qf}
S \text{ is closed immersed $\pi_1$-injective, and $\pi_1P_k \cap \pi_1S = 1$
for each component $P_k$ of $P$} \tag{QF}
\end{equation}

Note that since we haven't fixed a basepoint, these subgroups are really only
defined up to conjugacy (that is, they're sets of free homotopy classes) - what
we're saying is they fail to intersect for an arbitrary choice of conjugacy
class for each subgroup. This corresponds to multiples of the parabolic curves
being freely homotopic into the immersed image of the given surface

\begin{defn}

An \emph{$I$-bundle} is a fiber bundle with fiber $I=[0,1]$, and base space
a compact surface with boundary. We'll refer to this as its \emph{base
surface}.  In this paper we adopt the convention that all $I$-bundles are
oriented.  The boundary of an $I$-bundle $I to B to Si$ decomposes into the
\emph{side boundary} $\pi^{-1}(\bd Si)$ and the \emph{binding boundary} $cup
\bd \pi^{-1}(x)$. Note that the binding boundary is a collection of disjoint
annuli. The reason for these terms is the following definition.

A \emph{book of $I$-bundles} is a 3-manifold obtained from the following
construction. Let sB be a collection of I-bundles, and sC a collection of solid
tori. Attach the I-bundles to the solid tori by attaching some (or all) of the
binding boundary components to disjoint annuli in the boundaries of the solid
tori.  If we let sA be the union of glued annuli in the resulting manifold, our
manifold can be written as $M=sB cup_sA sC$. Note that by construction each
component of sA is properly embedded, 2-sided, and incompressible in M.

\end{defn}

% cite/ref Morgan or Thurston or Culler-Shalen (or Ian's work?) for this def

Intuitively, if we take a physical book with thick pages and imagine bending it
in a circle so its top and bottom are identified, the spine becomes a solid
torus.  If we also attach each page along its top and bottom, the pages become
thickened annuli, attached to the spine at one end. This manifold can be built
from the above construction. Take $sC$ to be a single solid torus, and $sB$ to
be a union of thickened annuli, one for each page of the book. Glue one gluing
annulus of each component of $sB$, and glue them all to parallel longitudinal
annuli in $\bd sC$.

Therefore, in keeping with the "book" terminology, we will generally refer to
the components of sB as \emph{pages}, the components of sC as \emph{spines},
and the components of sA as \emph{bindings} or \emph{binding annuli}.
Furthermore, we'll say that M is \emph{fully bound} if \emph{all} the binding
boundary components are glued. For reasons we'll see later, the books of
I-bundles that we want to consider will satisfy this criterion.

Algebraically, a book of I-bundles corresponds to a particularly simple graph
of groups. The graph is bipartitie, and one side of the partition (the spines)
are just copies of $Z$.

In this paper we impose the following standard conditions on a book of
$I$-bundles, to avoid elementary or degenerate cases.

\begin{enumerate}

\item The underlying manifold is a compact connected orientable irreducible
hyperbolic 3-manifold with incompressible boundary.

\item Each page is an I-bundle over a surface with boundary of negative Euler
characteristic.  That is, no base surface is a sphere, projective plane, disk,
Mobius strip, or annulus.

\end{enumerate}

Let's briefly explain why. Obviously we want M to be compact connected
orientable hyperbolic, as this is the general case we're studying (Kleinian
manifolds). If M were nonorientable we could reduce to the orientable double
cover. Hyperbolic 3-manifolds must be irreducible.

If M had compressible boundary, let D be a compressing disk for the boundary.
Any closed pi1-injective surface S cin M that intersects D must do so in
a union of closed curves, since S is properly embedded. Every such closed curve
$\alpha$ is homotopically trivial in M, because it's a closed curve in the disk
D.  But S is pi1-injective, so $\alpha$ must be homotopically trivial in S as
well. Since S is a surface, contracting the loop $\alpha$ yields a disk $D'\cin
S$. Note that D' may be immersed, but we can use it to produce a smaller disk
D'' which is embedded and still bounded by $\alpha$. Combining D and D' yields
a sphere S2 cin M. By irreducibility of M, this bounds a ball. Using a standard
innermost disk argument, we can homotope S across these balls (starting from
the innermost and working our way out) until S and D are disjoint.  This proves
that any quasi-Fuchsian surfaces are essentially disjoint from compression
disks. Therefore, since we're interested in quasi-Fuchsian surfaces, we might
as well compress as much as possible before looking for surfaces.

Each page's base surface must have at least one boundary component to glue to
in order for M to be connected. We don't want to consider base surfaces which
are disks.  Observe that each page which is a copy of $S0,1$ means that that
page and its attached spine form a 3-manifold with a finite-sheeted cover by
a ball (we can arrange things so that in the cover, the $S0,1$ attaching map
only traverses the longitude once, and we get a thickened disk).  This means
that we'll have finite order summands in our group, which correspond to
elliptic pieces, for instance lens spaces, in the JSJ decomposition. These
cases are not hyperbolic.

%TODO
\textbf{TODO explain Mobius strips / annuli with "big bad loops" idea which
breaks our book of I-bundles reduction theorem.}

We first make the following simple observation.

prop

Let $M$ be a book of $I$-bundles, and $M'$ to $M$ be a finite-sheeted covering
space.  Then $M'$ is also a book of $I$-bundles.

proof

This is straightforward. Lift the spines and pages of $M$ to obtain spines and
pages of $M'$. Lift the gluing annuli of $M$ to gluing annuli of $M'$.

end proof

These facts are well-known basic results in 3-manifold topology.  The following
result follows immediately from the classification of I-bundles over surfaces.

prop

Let M be a book of I-bundles. Recall we require that M is oriented. To be
orientable, every I-bundle in M is either a trivial bundle over an oriented
surface, or a Z/2-twisted bundle over a nonorientable surface. In the latter
case, the homotopy classes with nontrivial bundle twisting are precisely those
with orientation-reversing monodromy (word choice?).

Furthermore, every twisted I-bundle has a double cover which is a trivial
I-bundle over a base surface which is an oriented double cover of the original
nonorientable base surface.

proof

See Hempel I-bundle facts. {\bf TODO}

prop

Let X be a page or spine in a book of I-bundles M. Then X is irreducible, and
each binding annulus in bd X (ie component of sA cap X) is incompressible in X.

proof

We first claim pi2X = 0. A trivial I-bundle is a thickened surface of negative
Euler characteristic, hence is aspherical. A twisted I-bundle has a double
cover which is a trivial I-bundle over the oriented double cover of the base
surface. Solid tori are aspherical as well. It follows from the sphere theorem
that X is irreducible.

Let A cin bd X be a binding annulus. To prove A is incompressible in X, we'll
show that it is pi1-injective. Let X be a solid torus. Then since A intersects
a meridian disk at least once (as shown above), it must contain a multiple of
a generator of pi1X. So A is pi1-injective. Let X be an I-bundle. Again, we can
pass to a double cover which is a trivial I-bundle. Now A cin X is
pi1-injective because it's a thickening of a boundary component of a compact
surface. This completes the proof.

end proof

To analyze the boundary components of a book of $I$-bundles, we'll need
a couple more simple definitions.

defn

Let M be a book of I-bundles, and C a spine in M.  The \emph{valence} v(C) is
the number of binding annuli that intersect C or, equivalently, that are
contained in $\bd C$.

Let A be a binding annulus in M. Note that A lies in the boundary of exactly
one spine $C_A$. The \emph{degree} of A is the geometric intersection number
i(A,D), where D is a meridian disk of $C_A$. That is, it's the minimal number
of components of A cap D as we properly isotope $D cin C_A$ and $A cin bd C_A$.

In fact, note that all binding annuli in a given spine C must have the same
degree. This is because their core curves are curves on a torus, and any curves
with different slopes must intersect. Since they have the same slope, they
intersect a meridian curve (slope $\infty$) the same number of times.
Therefore, we refer to this as the \emph{degree} d(C) of the spine C.

enddefn

prop

Let M be a book of I-bundles. Recall that we require that M has incompressible
boundary. Then M cannot have any spines C such that $v(C)=0$, $d(C)=0$, or
$v(C)=d(C)=1$.

proof

Any spine C with $v(C)=0$ forces C to be a solid torus component. This does not
have incompressible boundary (it's also not hyperbolic).

Any spine C with $d(C)=0$ has a meridian disk that properly embeds in the
resulting book of I-bundles, since it's disjoint from all the binding annuli.
This is a boundary compressing disk.

% The following case is my old nemesis, the "one-point intersection lemma"

Let C be a spine with $v(C)=d(C)=1$. Let A be the sole binding annulus, B the
attached page, and Si its base surface. Let beta be the projection of A down to
Si. beta is a boundary component of Si. Let alpha be an essential arc in Si
with both (distinct) endpoints in beta. Note that such an alpha must exist as
Si has negative Euler characteristic. The preimage of alpha under the
projection is a rectangle R cin B with two edges gamma1,gamma2 in the the side
boundary of B, and two edges delta1,delta2 that are parallel proper essential
arcs in A.  Since A intersects each meridian disk of C exactly once, and does
so in a single essential arc, we can construct two meridian disks D1, D2 such
that D1 cap R = delta1, and D2 cap R = delta2. Let D = D1 cup delta1 R cup
delta2 D2.

We claim that D is a compression disk for bd M. It is properly embedded: D1 and
D2 are properly embedded in C, R is properly embedded in B, and the union lines
up the boundary components along A. It suffices to show that bd D does not
bound a disk in bd M. The options for such a disk D' are very limited. C cap bd
M is an annulus, and bd D divides it into 2 disk regions. D' cap C must be one
of these two regions, which implies it intersects A in two parallel boundary
arcs, each with one endpoint in delta1 and one in delta2. Call these arcs
epsilon1 and epsilon2. But now we can see that the only way to bound a disk in
bd M is if gamma1 cup epsilon1 or gamma2 cup epsilon2 bounded a disk. But if
this disk were inside B, either of these would project down to Si to contradict
the assumption that alpha was essential. Note that the disk cannot intersect
any spines or pages outside of B and C, as its boundary is entirely contained
within B cup C so we can use standard innermost disk/irreducibility arguments
to push it out of anywhere else. This proves that D is a compression disk.

endproof

We make a few observations about the boundary of M. Each page contributes two
(if it's a trivial bundle) or one (if it's a twisted bundle) side boundary to
the boundary of M. It also contributes any leftover binding annuli that are not
glued. Each spine contributes a number of disjoint annuli equal to its valence,
that is, the number of attached pages. These are the annuli in its boundary
which lie in between the binding annuli. As these glue up to form the boundary
components of M, each spine annulus connects two side boundaries of "adjacently
glued" pages. This intuitive picture will be very important later as we
construct quasi-Fuchsian surfaces.

% TODO proof that boundary is incompressible goes here!

We now discuss the possible closed pi1-injective surfaces inside a book of
I-bundles. These are the surfaces we'll need to consider in order to find
a closed quasi-Fuchsian surface - that is, a quasi-Fuchsian surface subgroup.

Note that since every hyperbolic 3-manifold is aspherical, by elementary
obstruction theory any injective map pi1S to pi1M is induced by a pi1-injective
map S to M.
% TODO cite Long-Reid, p11

Furthermore, by minimal surface theory, we can guarantee that any pi1-injective
surface in a hyperbolic 3-manifold is homotopic to an immersed surface.
A surface which is not immersed will contradict minimality.
% TODO cite Neumann p1

In what follows we'll speak only of pi1-injective surfaces, or possibly surface
subgroups. But it is important to note that in this situation those are the
same thing, and can be chosen to be immersed as well.

lemma (The Covering Lemma, as I actually use it)

Let phi: S to S' be a proper pi1-injective map of compact connected oriented
surfaces with boundary, and suppose that S is not a 2-sphere.  Then phi is
homotopic rel boundary to a finite-sheeted covering map.  Note that the
converse holds more generally - that is, any finite-sheeted covering is
a proper pi1-injective map.

proof

For the forward direction, we can apply minimal surface theory to perturb phi
via homotopy rel boundary to an immersion. It's straightforward to ensure that
phi is a local homeomorphism on boundary curves. Now think of int(S') as being
embedded as a totally geodesic core subsurface of a Fuchsian group quotient of
hyperbolic 3-space.  Then as we pull phi(int(S)) cin int(S')  to minimality its
image will necessarily remain in int(S'), as it's totally geodesic.  Since it
is connected and not a sphere, it can't shrink down to nothing. Since it's
properly embedded, the boundary components of S correspond to local maps of
cusps into cusp tubes of the thickened int(S') and must remain fixed on the
same cusp as we pull phi(int(S)) to a minimal surface. Hence the boundary
components of S do not move under this homotopy. The result is an immersion
produced via a homotopy rel boundary.

Then since phi is a proper immersion between manifolds of the same dimension,
it's a local homeomorphism on boundary and interior neighborhoods, and
therefore a covering map. It must be finite-sheeted because preimages of points
are discrete sets because it's a covering, and S is compact, so it follows that
point fibers are finite sets.

The converse follows from elementary covering space theory. As in the above
note, we can use minimal surface theory to make that proper pi1-injective map
into an immersion.

lemma (surface decomposition lemma)

Let M be a book of I-bundles, and S immerse M be a pi1-injective map, where
S is a (connected) closed orientable surface.  Then, we can place S in minimal
position with respect to M such that:

(1) for each page B, S cap B is a finite-sheeted cover of the page's base
surface (that is, if B to Si, S is a finite-sheeted cover of Si).

(2) for each spine C, S cap C is a union of essential annuli.

(3) the page covers and spine annuli are attached along curves parallel to the
multiples of the bnding annuli at each spine. That is, for each binding annulus
A, S cap A is a union of multiples of the core curve of A. All boundary
components of the page and spine intersections are attached in this way to
yield a closed surface.

proof

Homotope S to be transverse to sA, the union of all binding annuli. Cutting
S along sA decomposes it into a union of properly immersed surfaces in each
page or spine of M. We first move S into minimal position with respect to sA.
For each page or spine X cin M, S cap X is a union of properly immersed
surfaces. Consider an arbitrary component S' of S cap X.

If pi1(S')=1, then S' must be a disk as S is a connected surface and is not
a sphere. First observe that bd S' is contained in one of the binding annuli
A cin bd X. Since A is incompressible, bd S' bounds a disk in A, that is, S'
cobounds a sphere. Since X is irreducible, this sphere bounds a ball, giving
a homotopy of S' into A. To keep S transverse to sA, we will actually homotope
it slightly past A into whatever page or spine is glued to the other side of A.

Suppose we have eliminated every piece of S with trivial pi1 in this way, by
homotoping it across the binding annuli. We may need to repeat inductively, but
at every stage we're reducing the number of intersections with sA so this is
guaranteed to terminate. We now claim that after no more such moves can be
made, every piece of S is pi1-injective.

Again consider an arbitrary page or spine X, and an arbitrary component S' of
S cap X.  We know pi1(S')neq1 by the above argument. We claim it is
pi1-injective as a map S' to X (abusing notation slightly - really this is
asking about the preimage of S' on the left-hand abstract surface S). Suppose
not. Then we have a nontrivial loop alpha in S', which is homotopically trivial
in X.  Therefore alpha is homotopically trivial in M, using that same homotopy.
Since S to X is pi1-injective, alpha must be homotopically trivial in S. Since
S is an oriented surface, the only way alpha can be homotopically trivial in
S but not S' is if one of the boundary components of S' cuts off a disk D cin
S. If it cuts off a surface of any higher genus, we do not obtain any relations
among elements of pi1(S'). But if we cut D along sA, we can see that at least
one innermost piece of D must be a disk, contradicting our minimality
assumption. Hence each piece S' is pi1-injective in X.

Let B be a page. We know that each piece S' of S cap B is pi1-injective and
proper. Projecting B to its base surface Si, we know this is a deformation
retract (B is an I-bundle) so the composition S' to Si is also pi1-injective
and proper. Applying the covering lemma, we see that each piece S' is homotopic
rel boundary to a finite-sheeted covering of Si. This proves (1).

Let C be a spine. Again we know that each piece S' of S cap C is pi1-injective
and proper. Since C has cyclic fundamental group, and no piece S' is a disk,
every piece S' must be an annulus. These annuli are pi1-injective, hence
essential. This proves (2).

To prove (3), look at the boundary curves of these annuli in the spines. Each
boundary curve must be contained in a binding annulus A cin bd C. Therefore it
must be a multiple of the cure curve of A. This proves the first statement in
(3). The remainder of (3) follows directly from the fact that S is a closed
surface and that M is a union of pages and spines along binding annuli.

end proof

We now consider the possible pared structures (M,P), where M is a book of
I-bundles. We have a straightforward fact.

\begin{prop}

Let $M$ be a book of $I$-bundles, and $P$ a pared structure on $M$. $P$ cannot
contain any tori.  That is, $P$ consists entirely of annuli.

\end{prop}

\begin{proof}

Each page of $M$ has a base surface of negative Euler characteristic, so each
component of a page boundary must also have negative Euler characteristic (it's
either a copy or an orientable double cover of the base surface, as discussed
earlier). Since each boundary component of M consists of page boundaries glued
together along annuli in spines, which contribute nothing to Euler
characteristic, it immediately follows that each boundary component of M has
negative Euler characteristic itself. So no boundary component admits
a pi1-injective torus, and all pared locus components must be annuli.

\end{proof}

lemma (minimal position for pared structures / pared decomposition lemma)

Let (M,P) be a pared 3-manifold, where M is a book of I-bundles. Then we can
isotope P within bd M such that:

(1) For each page B, P cap B is a union of disjoint rectangles and annuli, each
of which projects to a thickened arc or curve in the base surface Si of B. Any
such arc is essential in Si. Any such curve is either essential or parallel to
a boundary component of Si which does not lift to a binding annulus (that is,
parallel to a "free binding boundary" of B).

(2) For each spine C, P cap C is a union of disjoint rectangles and annuli.
Each rectangle connects two adjacent binding annuli in bd C along a thickened
arc that is essential in that component of bd M cap C (the "intermediate
annulus" between these two binding annuli). Each annulus is parallel to the
binding annuli in bd C.

(3) For each binding annulus A, P cap A is a union of disjoint arcs in bd A.

proof

We can isotope P locally, keeping all its components disjoint and embedded in
bd M, so that it's transverse to sA. This immediately guarantees (3).

Let X be a page or spine in M. P cap X is a union of disjoint rectangles and
annuli, which are thickenings of arcs and curves in bd M cap X. Suppose one of
these is an arc alpha which is not essential in bd M cap X. Then alpha cobounds
a disk with an arc in bd (bd M cap X). There may be other arcs in the interior
of this disk, but with an innermost disk argument we can inductively push all
such arcs out of bd M cap X while keeping them disjoint. Note that the interior
of this disk cannot contain any closed curves coming from P because they would
not be pi1-injective. This is an isotopy of P inside bd M.  Repeat this
inductively for all pages and spines until no such arcs remain at all.  Since
at each step we're decreasing the number of intersections of P with sA, this
process is guaranteed to terminate.

Let B be a page. Let alpha be a closed curve in bd M cap B which thickens to an
annulus in P cap B. alpha cannot bound a disk in bd M cap B, or that component
of P would fail to be pi1-injective in bd M cap B, hence in M. So either alpha
is essential in bd M cap B, or alpha is parallel to a component of bd(bd M cap
B). In the latter case, if this component corresponds to a binding annulus A,
push the thickened alpha annulus across A into the attached spine. There cannot
be any annuli or rectangles in between, because components of P can't be
parallel and we already eliminated all the non-essential rectangles.

Suppose we've performed the above isotopy on P. Since we can preserve
transversality, (3) still holds. We claim the result satisfies (1) and (2).

To prove (1), let B be a page. Observe that the rectangles and annuli in P cap
B are thickenings of arcs and curves which are essential (or
free-binding-boundary parallel in a special case) in bd M cap B. Let Si be the
base surface of B. As discussed above, bd M cap B is either two copies of Si
(if B is a trivial I-bundle) or an orientable double cover of Si (if B is
a twisted I-bundle). In either case the projection of each arc and curve must
be essential in Si, as otherwise we could lift a contracting or
boundary-exiting isotopy to bd M cap B. This proves (1).

To prove (2), let C be a spine. Observe that bd M cap C is a union of parallel
annuli, each of which is parallel to the binding annuli in bd C. Rectangles in
C are essential, as guaranteed earlier. Up to isotopy, the only embedded closed
curve in an annulus is the core curve. So the only annuli in P cap C are
parallel to the binding annuli. This completes the proof.

end proof

lemma (parabolic lifting criterion)

Let M be a book of I-bundles, and S immerse M be a pi1-injective map, where
S is a connected closed surface.  After placing S and P in minimal position as
above, S is a (QF) surface if and only if the following criterion holds. Let
B be a page in M, and C a spine in M. P cap B is a union of disjoint arcs and
curves in B. We have canonical covering maps S cap B to Si and S cap C to A2 by
the surface decomposition lemma. Let tildePS be the union of the preimages of
P cap B. % TODO finish statement here

proof

% TODO

% TODO delete the following lemma and its proof
\begin{lemma}\label{L:sc}

Let $S$ be a surface satisfying \eqref{E:qf}. Homotope $S$ to have minimal
intersection with each $A_i$, that is, so there are no ``bumps''.  Then for
each $M_i$, each component of $S \cap M_i$ is a finite-sheeted covering of the
core surface.  That is, given such a component $S' \cin M_i$, the map $S' \to
M_i = \Si_{1,1}\x I \to \Si_{1,1}\x{1/2}$ is homotopic to a finite-sheeted
covering map. Conversely, given any finite-sheeted covering $\widetilde{S} \to
\Si_{1,1}$, there exists a corresponding proper immersed $\pi_1$-injective
surface $S \cin \Si_{1,1}\x I$.

\end{lemma}
\begin{proof}

$S$ is compact, so every such component $S'$ is a compact surface with
boundary, properly immersed in $M_i$. Since $S$ is $\pi_1$-injective in $M$,
$S'$ must be $\pi_1$-injective in $M_i$. Otherwise, we'd have a nontrivial
element of $\pi_1S'$ which is trivial in $\pi_1M_i$, hence in $\pi_1M$, hence
in $\pi_1S$, contradicting the minimal position homotopy above. So the map
$\phi : S'\to\Si_{1,1}\x{1,2}$ is also $\pi_1$-injective, since $M_i$
deformation retracts to its core. Let $H = \phi_*(\pi_1S')$, and let
$\Si_{1,1}^H$ be the cover of $\Si_{1,1}$ associated to $H<\pi_1\Si_{1,1}$.
$\phi$ lifts to $\widetilde{\phi}\colon S'\to \Si_{1,1}^H$. This is a proper
map of compact surfaces which is an isomorphism on $\pi_1$.  By the
classification of surfaces, it must be homotopic (as a proper map) to
a homeomorphism. So $\phi$ is homotopic to a covering map. It must be
finite-sheeted as $S'$ is compact (by classification of surfaces again).

Conversely, given a finite-sheeted cover $\widetilde{S}\to \Si_{1,1}$, compose
with the embedding $\Si_{1,1} = \Si_{1,1}\x{1/2} \cin \Si_{1,1}\x I$. This is
proper and $\pi_1$-injective.  Perturb locally to obtain an immersion.

\end{proof}


%%%%%%%%%%%%%%%%%%%%%%%%%%%%%%%%%%%%%%%%%%%%%%%%%
\section{Separability properties of groups and spaces}
%%%%%%%%%%%%%%%%%%%%%%%%%%%%%%%%%%%%%%%%%%%%%%%%%

We discuss some algebraic properties that will allow us to perform nice
topological constructions. These are standard definitions. See
\cite{Agolsurvey}, \cite{LR}, or \cite{AFW} for an overview and survey of
recent work.

\begin{defn}

Let $G$ be a group. $G$ is \emph{residually finite}, or \emph{RF}, if for any
$g \in G$, there exists a finite index subgroup $H<G$ such that $g \notin H$.

Let $G$ be a group, and $H$ a subgroup of $G$. $H$ is \emph{separable} in $G$
if for any $g \in G$, $g \notin H$, there exists a finite index subgroup $H'<G$
such that $H' \geq H$, and $g \notin H'$.

$G$ is \emph{subgroup separable}, or \emph{LERF}, if all of its finitely
generated subgroups are separable.

\end{defn}

The following equivalences are well-known.  We'll use these facts later in our
proofs of topological properties.

\begin{prop}\label{P:lerfmap}

$G$ is RF if and only if for any $g \in G$ there exists a map $\phi \colon
G \to F$, where $F$ is a finite group, such that $\phi(g) \neq id$.

$G$ is LERF if and only if for any $g \in G$, $H < G$, $g \notin H$, there
exists a map $\phi \colon G \to F$ such that $\phi(g) \notin \phi(H)$.

\end{prop}

\begin{proof}

See \cite{LR}.

\end{proof}

\begin{cor}\label{C:lerfmap'}

$G$ is LERF if and only if for any $g_1,\dots,g_n \in G$, $H<G$, $g_1,\dots,g_n
\notin H$ there exists a map $\phi \colon G \to F$ to a finite group such that
$\phi(g_i) \notin \phi(H)$ for all $i$.

\end{cor}

\begin{proof}

Let $G$ be LERF. For each $g_i$, $g_i \notin H$, so there exists a map $\phi_i
\colon G \to F_i$, a finite group, such that $\phi_i(g_i) \notin \phi_i(H)$.
Now let $F = F_1 \times \dots \times F_n$, and $\phi = \phi_1 \times \dots
\times \phi_n$.  For each $i$, $\phi(g_i)_{(i)} \notin \phi(H)_{(i)}$, so
$\phi(g_i) \notin \phi(H)$.  This proves the forward direction.  The converse
is trivial.

\end{proof}

A priori it is not obvious that any well-known groups are LERF. It is
a classical theorem of Hall \cite{Hall} that free groups are LERF. Peter Scott
\cite{Scott} showed that surface groups are LERF as well. However, we will need
stronger results in this paper. Deep work of Wise \cite{Wise}, building on work
of Haglund--Wise \cite{HaglundWise} and Hsu--Wise \cite{HsuWise} shows that
every non-closed hyperbolic $3$-manifold has LERF fundamental group.  We note
that the closed case has also been settled by Agol \cite{Agol}, incorporating
work of Bergeron--Wise \cite{BergeronWise} and Kahn--Markovic \cite{KM}. Since
we're studying books of $I$-bundles, we will use the former result in this
paper.

Note that Wise's theorem is actually an extremely deep fact. Wise's proof
actually shows that hyperbolic $3$-manifold with boundary groups satisfy
a technical condition, namely that they are virtually compact special.
Additional work by Haglund--Wise \cite{HaglundWise1} demonstrates that
a virtual compact special group is virtually a quasi-convex subgroup of
a right-angled Artin group. Therefore later work by Haglund \cite{Haglund}
applies in this case, showing that these groups are in fact LERF. See Agol's
survey \cite{Agolsurvey} for a more detailed overview.

We now describe some elementary topological consequences of LERF. These exact
statements are new, but most of the following propositions are simple
translations of the conclusions of LERF into topological statements about loops
and covers. In all of the following we assume $X$ is connected, path connected,
and semi-locally simply connected. That is, we make the assumptions needed to
apply elementary covering space theory. \emph{We also assume that $\pi_1X$ is
LERF.}


\begin{prop}\label{P:lerf1}

Fix a basepoint $x_0 \in X$.  Suppose that $Y \cin X$ is a homotopically
nontrivial connected path-connected semi-locally simply connected subspace
containing $x_0$, $\pi_1Y$ is finitely generated, and $\alpha$ is a loop at
$x_0$ which is not homotopic into $Y$ (relative to the basepoint $x_0$). Then
there is a finite-sheeted regular cover $p \colon (X',x_0') \to (X,x_0)$ with
the property that the lift $\alpha'$ of $\alpha$ to $x_0'$ connects two
different connected components of $p^{-1}(Y)$.

\end{prop}

\begin{proof}

We apply Proposition~\ref{P:lerfmap}. Let $g = \alpha$ and $H = \pi_1Y \leq
\pi_1X$.  By assumption, $H$ is finitely generated and $g \notin H$. Therefore
there exists a map $\phi \colon G \to F$ such that $\phi(g) \notin \phi(H)$,
where $F$ is a finite group.  Without loss of generality we can assume this map
is surjective (take its image). Let $H' = ker \phi$. Now $H'$ is a finite index
normal subgroup, so it induces a finite sheeted regular cover $p \colon X' \to
X$.  Let $x_0'$ be an arbitrary lift of $x_0$ to $X'$. We claim that
$(X',x_0')$ has the desired property.  $X'$ is a regular cover, so it has
covering transformation group $F$.  $F$ acts on the fiber $p^{-1}(x_0)$. We
know that any loop $\gamma \cin X$ based at $x_0$ lifts to a loop $\gamma'$
which starts at $x_0'$ and ends at $\phi(\gamma') \cdot x_0'$.  Since
$\phi(\alpha) \notin \phi(\pi_1Y)$, $\alpha$ cannot end at any point that is
connected to $x_0'$ by a lift of a loop in $\pi_1Y$.  Therefore its right
endpoint must lie in a different connected component of $p^{-1}(Y)$.

\end{proof}

\begin{prop}\label{P:lerf2}

Fix a basepoint $x_0 \in X$.  Suppose that $\alpha$ and $\beta$ are
homotopically nontrivial loops at $x_0$ such that $\alpha$ is not homotopic
(relative to the basepoint $x_0$) to a multiple of $\beta$.  Then there is
a finite-sheeted regular cover $p \colon (X',x_0') \to (X,x_0)$ with the
property that, when we consider the lift $\alpha'$ of $\alpha$ to $x_0'$ has
its right endpoint at a point not reachable by lifting multiples of $\beta$ to
$x_0'$.

\end{prop}

\begin{proof}

Let $g = \alpha$ and $H = \langle\beta\rangle$. Since $\alpha$ is not homotopic
to a multiple of $\beta$, $\alpha \notin H$. Apply Proposition~\ref{P:lerfmap}
to find a map $\phi \colon G \to F$ such that $\phi(g) \notin \phi(H)$, where
$F$ is a finite group.  Let $H' = ker \phi$, and $X'$ be the corresponding
finite-sheeted regular cover of $X$. By the same argument as above, a lift
$\alpha'$ of $\alpha$ to a lifted basepoint $x_0'$ cannot end at any point that
is connected to $x_0'$ by a lift of a multiple of $\beta$. This completes the
proof.

\end{proof}

\begin{prop}\label{P:lerf3}

Fix a basepoint $x_0 \in X$.  Suppose that we have a homotopically nontrivial
loop $\alpha$ at $x_0$ of infinite order in $\pi_1X$.  Then given an integer
$k>0$, there is a finite-sheeted regular cover $p \colon (X',x_0') \to (X,x_0)$
with the following property.  Consider the subset $C$ of $p^{-1}(\alpha)$
obtained by lifting multiples of $\alpha$ to $x_0'$. Let $d$ be the degree of
the restricted covering map $p|C \colon C \to \alpha$. Then $k \mid d$.

\end{prop}

\begin{proof}

Let $H = \langle\alpha^k\rangle$. Since $\alpha$ has infinite order,
$\alpha,\alpha^2,\dots,\alpha^{k-1} \notin H$. Apply Corollary~\ref{C:lerfmap'}
to find a map $\phi \colon G \to F$ such that
$\phi(\alpha),\dots,\phi(\alpha^{k-1}) \notin \phi(H)$. Let $H' = ker \phi$,
and $p \colon X' \to X$ be the associated finite-sheeted regular cover.  We
claim that $X'$ has the desired property. To see this, observe that $d$ is the
smallest integer such that $\alpha^d$ lifts to a closed curve in $X'$. $C$ is
a finite-sheeted cover of the loop $\alpha$, so it must be cyclic. Because $X'$
is regular, this means $d$ is the smallest integer such that $\phi(\alpha)^d$
is trivial, that is, the deck transformation induced by $\alpha$ has order $d$.

We cannot have $d<k$, as $\phi(\alpha^d)=\phi(\alpha)^d=1 \in \phi(H)$
contradicts the LERF assumption on $\phi$. Write $d = m_1k + m_2$, for some
$m_1,m_2 \in Z, 0 \leq m_2 < k$.
$\phi(\alpha^d)=1=\phi(\alpha^{m_1k})\phi(\alpha^{m_2})$.  That is,
$\phi(\alpha^{m_2})=\phi(\alpha^k)^{-m_1}$. The right-hand side is in
$\phi(H)$.  By the LERF assumption on $\phi$, this forces $m_2=0$. So $d \mid
k$.

\end{proof}

We also have the following propositions, which generalize the above
propositions to multiple simultaneous basepoints and loops/subspaces at those
basepoints. Note that the following three propositions are equivalent to saying
that the three properties above are preserved under taking covers.

\begin{prop}\label{P:lerf1'}

Consider a finite collection of basepoints $x_1,\dots,x_n$ with corresponding
loops $\alpha_i$ and subspaces $Y_i$ at each basepoint $x_i$. Assume all the
$Y_i$ are homotopically nontrivial connected path-connected semi-locally simply
connected subspaces with finitely generated fundamental group. Then there is
a finite-sheeted regular cover $X' \to X$ such that each pair $\alpha_i$,$Y_i$
satisfies the conclusion of Proposition~\ref{P:lerf1}.  Note that $X'$ is
regular, so the choice of lifted basepoints is arbitrary.

\end{prop}

\begin{prop}\label{P:lerf2'}

Consider a finite collection of basepoints $x_1,\dots,x_n$ with corresponding
loops $\alpha_i$,$\beta_i$ at each basepoint $x_i$. Then there is
a finite-sheeted regular cover $X' \to X$ such that each pair
$\alpha_i$,$\beta_i$ satisfies the conclusion of Proposition~\ref{P:lerf2}.

\end{prop}

\begin{prop}\label{P:lerf3'}

Consider a finite collection of basepoints $x_1,\dots,x_n$ with corresponding
loops $\alpha_i$ at each basepoint $x_i$. Then given integers
$k_1,\dots,k_n>0$, there is a finite-sheeted regular cover $p \colon X' \to X$
such that each pair $\alpha_i$,$k_i$ satisfies the conclusion of
Proposition~\ref{P:lerf3}.

\end{prop}

\begin{proof}

All these lemmas are proved the same way. Fix a basepoint $x_0$ once and for
all.  Since $X$ is connected, choose a path from $x_0$ to $x_i$ for each $x_i$,
and use this to fix an isomorphism $\sigma_i \colon \pi_1(X,x_0) \cong
\pi_1(X,x_i)$.  For each $x_i$, we apply the appropriate proposition above
(\ref{P:lerf1}, \ref{P:lerf2}, or \ref{P:lerf3}) to construct a map $\phi_i
\colon \pi_1(X,x_i) \to F_i$ with the appropriate property. Now let $F = F_1
\times \dots \times F_n$ and $\phi = (\phi_1 \circ \sigma_1) \times \dots
\times (\phi_n \circ \sigma_n)$.  By the same argument as
Corollary~\ref{C:lerfmap'}, this cover will have the appropriate property for
all $n$ conditions.

\end{proof}

%%%%%%%%%%%%%%%%%%%%%%%%%%%%%%%%%%%%%%%%%%%%%%%%%
\section{Our first example}
%%%%%%%%%%%%%%%%%%%%%%%%%%%%%%%%%%%%%%%%%%%%%%%%%

We construct a book of $I$-bundles $M$ as follows. In the following we take
$\Si_{1,1}$ to be the compact surface of genus 1 with a single boundary
component. Let $B_1$,$B_2$,$B_3$ be 3 trivial $I$-bundles over $\Si_{1,1}$.
$B_1=B_2=B_3=\Si_{1,1}\x I$. Each $B_i$ has a single binding boundary and two
side boundaries. Denote the side boundaries by $\bd_+B_i$ and $\bd_-B_i$. Each
side boundary component of a trivial $I$-bundle is a homeomorphic copy of the
base surface $\Si_{1,1}$.

Let $C = S^1\times D^2$ be a solid torus. Attach the $B_i$ to $C$ by gluing the
binding boundaries to parallel annuli in $\bd C$, each of which intersects
a meridian disk exactly once. The result is a book of $I$-bundles $M$. Let
$A_1$, $A_2$, and $A_3$ be the respective gluing annuli. Then $M=\cB\cup_\cA
\cC$ where $\cB=B_1 \cup B_2 \cup B_3$, $\cC=C$, and $\cA = A_1 \cup A_2 \cup
A_3$.

Now $\bd M \cap C$ is a union of 3 annuli. Let these annuli be $A_{12}'$,
$A_{23}'$, and $A_{31}'$, where each boundary annulus is labeled based on its
adjacent gluing annuli. We choose orientations and a cyclic order for the
gluing such that the 3 boundary components of M are precisely $bd_+B_1 \cup
A_{12}' \cup \bd_-B_2)$, $\bd_+B_2 \cup A_{23}' \cup \bd_-B_3$, and $\bd_+B_3
\cup A_{31}' \cup \bd_-B_1$. We'll refer to these as $\bd_{12}M$, $\bd_{23}M$,
and $\bd_{31}M$, respectively.  Each boundary component of $M$ consists of two
copies of $\Si_{1,1}$ glued along an annulus. Topologically, this forms a genus
two surface.

We know from Proposition~\ref{P:annuli} that any pared structure $P$ on $M$
consists entirely of annuli. Furthermore, we can apply the Pared Decomposition
Lemma to homotope $P$ to minimal position with respect to $\cA$. We then have
the following consequences of the pared lifting criterion. (For all of the
following, we assume $P$ is in minimal position).

\begin{prop}

Suppose $P$ has a component $P_0$ such that $P_0 \cin C$. Then $(M,P)$ does not
contain a (QF) surface.

\end{prop}

\begin{proof}

Suppose it does. Let $S$ be a (QF) surface, and homotope $S$ to minimal
position with the surface decomposition lemma. Applying the pared lifting
criterion, any annulus component $S \cap C$ will cover the binding annuli at
$C$ and allow $P_0$ to lift to a closed curve in the pared lifting pattern.
This would show $S$ is not (QF), by the pared lifting criterion. So $S$ must be
disjoint from $C$.  The pages $B_i$ have free fundamental group, so none of
them admits a $\pi_1$-injective map from a closed surface.  Therefore such an
$S$ cannot exist.

\end{proof}

\begin{prop}

Suppose $P$ has no components contained in $C$, but does have a component $P_0$
such that $P_0 \cin B_i$, for some $i$.  Without loss of generality we can let
$P_0 \cin \bd B_1$.  Then there exists a (QF) surface if and only if $P$
contains no components that are contained in $\bd_-B_2 \sqcup \bd_{23}M \sqcup
\bd_+B_3$.

\end{prop}

\begin{proof}

For the forward direction, take $S=\bd_{23}M$. Intuitively, it's easy to see
that the conditions on $P$ force all its components to lie on surfaces where
they can't be homotoped into $S$.

Algebraically, $\pi_1S$ is the subgroup generated by $a_2,b_2,a_3,b_3 \in
\pi_1M$. We know that
\[
\pi_1S = \langle a_2,b_2,a_3,b_3 \mid [a_2,b_2]=[a_3,b_3]\rangle.
\]
$P$ contains no components in $\bd_{23}M$, so an arbitrary component $P_k$ of
$P$ must be contained in $\bd_{12}M$ or $\bd_{31}M$, which have fundamental
groups generated by $a_1,b_1,a_2,b_2$ and $a_3,b_3,a_1,b_1$ respectively.
$\pi_1P_k$ is cyclic, so let $g$ be its generator. No matter if it's contained
in $\bd_{12}M$ or $\bd_{31}M$, we can see that in order for $\pi_1P_k$ to
overlap with $\pi_1S$, it must have generator some word in $a_2,b_2$ (if $P_k
in \bd_{12}M$), or some word in $a_3,b_3$ (if $P_k in \bd_{31}M$). In either
case, this is precisely the condition for such a word to correspond to a curve
contained in $\bd B_2$ or $\bd B_3$, respectively, contradicting our assumption
on $P$.

($\Longrightarrow$) Let $S$ be a surface satisfying \eqref{E:qf}. Cut $S$ into
components in $B_1,B_2,B_3$ (after homotoping to minimal intersections with the
annuli).  Applying Lemma~\ref{L:sc}, we can see that $S \cap B_1 = \emptyset$,
as otherwise since it's a finite-sheeted cover it would have to contain
a multiple of $P_0$.  So $S \cin B_2 \cup B_3 \cup C$, which is homeomorphic
to $S_2 \x I$.  Deformation retracting this to a surface $S_2$ and applying
a covering argument like in Lemma~\ref{L:sc}, we can see that $S$ is a cover of
$S_2$.  Since it's a finite-sheeted cover, it will have to have $\pi_1$
intersecting any component $P_k$ that violates the conditions stated above.
This completes the proof.

\end{proof}

The proof above obviously generalizes to more complex books of $I$-bundles.
Later we'll use it to simplify any pared structures containing annuli that fit
in a single $I$-bundle.

\begin{thm}\label{T:ex1}

Let $P$ be a pared structure containing no components $P_0$ as in the above
proposition. Then there exists a surface satisfying \eqref{E:qf}.

\end{thm}

This is the main theorem we prove for this example. A few preliminary facts are
required. Note that we can first isotope P such that it's in minimal position
on each boundary component with respect to the gluing circle.

\begin{lemma}

Under the hypotheses of the theorem, each ``$I$-bundle piece'' boundary
component $S=\bd_\pm B_i$ intersects $P$ in a thickened set of disjoint
essential arcs, each arc connecting $\bd S$ to itself. The arcs form at most
3 ``bands'', where all the arcs in each band are parallel.  Furthermore, if we
choose a representative arc from each band (yielding a set of at most
3 disjoint non-parallel arcs in $\Si_{1,1}$), there exists an automorphism of
$\Si_{1,1}$ taking these arcs to a standard set of 3 disjoint non-parallel
arcs.  See diagram for an illustration of the standard set.

\end{lemma}
\begin{proof}

We first need a preliminary definition. After cutting a surface with boundary
along arcs, we'll obtain one or more connected surfaces, each with one or more
boundary components. Each boundary component after cutting will have pieces
from the original boundary as well as pieces from the arcs that we cut along.
Given labels $\gamma_1,...,\gamma_k$ for the boundary components and
$\alpha_1,...,\alpha_l$ for the arcs (on the original surface with boundary),
we can describe each boundary component of the cut surfaces as a union of arcs,
each labeled with $\gamma_1,...,\gamma_k,\alpha_1,...,\alpha_l$. We call this
an \emph{arc pattern} for that boundary component of the new surface.

We first show that $\Si_{1,1}$ admitts at most 3 disjoint essential
non-parallel arcs, and the possible surfaces and arc patterns obtained by
cutting along these arcs are very restricted.

Label the boundary component of $\Si_{1,1}$ by $\gamma$. Consider a proper
essential arc $\alpha_1 \cin \Si_{1,1}$. Fix an orientation for $\alpha_1$.
Since $\Si_{1,1}$ is orientable, we can look at local neighborhoods of
$\alpha_1$ and see that $\alpha_1$ has a well-defined ``left side'' and ``right
side'' as we travel along it. Looking at the endpoints of $\alpha_1$ along
$\gamma$, we are forced to connect certain endpoints in the cut-up $\gamma$
with $\alpha_1$, in order to preserve the parity. This tells us the (possibly
disconnected) cut surface $S_1$ will have two boundary components. Each will
have an arc pattern consisting of two arcs, one labeled $\gamma$ and one
labeled $\alpha_1$.

Since we cut along a properly embedded arc, the Euler characteristic increases
by one. $\chi(S_1)=0$ and $S_1$ has two boundary components. By classification
of surfaces, $S_1 = \Si_{0,2}$ or $\Si_{0,1} \sqcup \Si_{1,1}$. But if $S_1$
contained a disk with the arc pattern described above, embedding that disk back
in $\Si_{1,1}$ would describe a homotopy of $\alpha_1$ into the boundary. So
$S_1 = \Si_{0,2}$.

Now suppose we had a second proper essential arc $\alpha_2 \cin \Si_{1,1}$,
disjoint from and non-parallel to $\alpha_1$. Since it's disjoint from
$\alpha_1$, $\alpha_2$ induces a proper arc in $S_1$ which connects two regions
in the arc pattern labeled $\gamma$.  $\alpha_2$ must have one endpoint on each
boundary component of $S_1$. If both are on the same side, it's either
homotopic to the boundary of $\Si_{1,1}$ or parallel to $\alpha_1$. Cutting
along $\alpha_2$ yields a new surface $S_2$. Topologically $S_2=D^2$, with arc
pattern consisting of 8 components in the cyclic order
$(\gamma,\alpha_1,\gamma,\alpha_2,\gamma,\alpha_1,\gamma,\alpha_2)$.

Finally, adding our 3rd proper essential arc $\alpha_3$, disjoint and
non-parallel to the first two arcs, a similar argument shows that $\alpha_3$
must connect opposite $\gamma$ pieces in the arc pattern. Cutting along
$\alpha_3$ yields two disks with the same arc pattern. Depending on the choice
of $\alpha_3$, the arc pattern on these disks is either
$(\gamma,\alpha_1,\gamma,\alpha_2,\gamma,\alpha_3)$ or
$(\gamma,\alpha_1,\gamma,\alpha_3,\gamma,\alpha_2)$. So up to relabeling
$\alpha_1,\alpha_2,\alpha_3$, cutting along 3 proper essential disjoint
non-parallel arcs has only one possible choice of cut surfaces and arc
patterns.

Observe that it is not possible to add any more disjoint non-parallel arcs. In
particular, any arc we draw between $\gamma$ components of the arc pattern on
either disk is homotopic to the boundary or parallel to an existing arc.
Furthermore, if we add new disjoint arcs and allow them to be parallel, we can
see that they must form ``bands'' around the existing 3 arcs in order to remain
disjoint. That is, we can homotope all the arcs parallel to a given arc into
a small neighborhood of that arc in the disk, without intersecting any of the
non-parallel arcs.

We claim there exists an automorphism of $\Si_{1,1}$ taking any set of 3 such
arcs to any other set (in particular, to the standard set, as illustrated).
Since there is only one topological result of cutting along the arcs, choose
a homeomorphism of the cut surfaces. Up to relabeling the arcs, we can choose
a homeomorphism that identifies matching arcs in the arc patterns (as shown
above, there is only one possible arc pattern up to relabeling). Glue both
sides along the arcs to obtain the desired automorphism of $\Si_{1,1}$.

We now consider $P \cap \bd_S \cin S \cong \Si_{1,1}$. As shown, $P$ is a union
of annuli. By the assumptions of the theorem, no annulus of $P$ is contained in
$S$, so each annulus intersects $S$ in a union of rectangles, where two sides
of the rectangle are embedded in the boundary $\bd S$.  Since we homotoped $P$
to have minimal intersections, all the core arcs of the rectangles (pieces of
the core curve of the annulus) must be essential. They are disjoint by
definition of $P$. The statement of the lemma follows from applying the above
argument to these core arcs.

\end{proof}

We consider connected double covers $\widetilde{S} \to S \cong \Si_{1,1}$.
These covers have 2 boundary components. Every core arc of $P \cap S$ lifts to
2 arcs in $\widetilde{S}$.  We say an arc in $S$ is \emph{cis} for a given
$\widetilde{S}$ if any (ie all) lifts of that arc have both endpoints in the
same boundary component of $\widetilde{S}$.  Otherwise, we say it's
\emph{trans} for that cover.

\begin{lemma}

Given any 3 disjoint non-parallel proper arcs in $\Si_{1,1}$, and any double
cover of $\Si_{1,1}$, 2 of the 3 arcs are trans, and the 3rd is cis.
Furthermore, we can choose any two of the three we wish to be trans with an
appropriate cover.

\end{lemma}
\begin{proof}

As in earlier lemma, there exists an automorphism of $\Si_{1,1}$ taking these
arcs to the standard set of 3 disjoint non-parallel proper arcs.  Now it
suffices to observe, by looking at relative first homology or just by
construction, that each of the three standard connected double covers
(corresponding to nontrivial maps $\mathbb{Z}^2 \to \mathbb{Z}/2$) makes two of
the three arcs trans and the third cis.

\end{proof}

\begin{proof}[Proof of Theorem~\ref{T:ex1}]

Since $P$ has no components $P_0$, every $P \cap \bd_+B_i$ is a union of
thickened arcs (that is, there are no full annulus components in any $\bd_+B_i$
or $\bd_-B_i$). Apply the first lemma to break these into bands. The problem is
most constrained when there are 3 bands, so we'll consider that case (if there
are fewer than 3, just draw some more arcs on that component arbitrarily, and
the proof still works).

We will build our surface $S$ by taking a double cover of the core $\Si_{1,1}$
surface for each $I$-bundle page $B_1,B_2,B_3$. Call these covers
$S_1,S_2,S_3$.  Each of these has two boundary components. We'll then attach
the boundary components such that each of $S_1,S_2,S_3$ has exactly one
boundary component connected to each of the others. See diagram.

Choose an arbitrary connected double cover for $S_1$. We can view this as
a cover of $\bd_+B_1$ or $\bd_-B_1$, deformation retracting either way. By the
lemma, two of the
3 bands on $\bd_+B_1$ are trans, and the other is cis. The same holds for
  $\bd_-B_1$.

We cannot choose $S_2$ arbitrarily, as a cis arc for $S_1$ in $\bd_+B_1$ may
connect (in $S$) to a cis arc for $S_2$ in $\bd_-B_2$. These together would
form a closed curve that lifts to $S$, which once $S$ is immersed in $M$ will
yield a violation of condition \eqref{E:qf}. Instead, observe that there is at
most one band of the three in $\bd_-B_2$ containing arcs that, when glued
across the core circle, match up to arcs in the cis band of $\bd_+B_1$ to form
closed curves containing only arcs in those two bands. This is because once one
band has that behavior, by endpoint parity the vertices can't also match up for
a different band, if the arcs they have to match with on the other side are
parallel. Looking at the boundary circle, non-parallel arcs have endpoints in
cyclic order, but parallel arcs don't. See diagram. It suffices to check this
for our standard set of non-parallel arcs, by the same automorphism argument.

Since there is only one such band on $\bd_-B_2$, choose $S_2$ such that this
band is not cis.  Finally, for $S_3$ we have two connecting constraints. We
have a cis band on $\bd_-B_1$ from our choice of $S_1$, and a cis band on
$\bd_+B_2$ from our choice of $S_2$.  Applying the same argument, there is at
most one band on $\bd_-B_3$ and one band on $\bd_+B_3$ that will connect to
form closed curves. But we can choose $S_3$ such that both of these bands are
trans. (This requires a slight modification to the lemma - since these bands
are on different boundary surfaces, they may not be disjoint non-parallel. If
they aren't disjoint, we can tweak them by local modifications so they are, and
then ``untweak'' them in the cover. If they are parallel, just ignore one set
of bands)

Glue $S_1,S_2,S_3$ as described to obtain $S$. We claim that $S$ satisfies
condition \eqref{E:qf}. $S$ is closed and immersed by construction (as in
Lemma~\ref{L:sc}). It is $\pi_1$-injective inside each $I$-bundle by
Lemma~\ref{L:sc}, and inside the core $C$ because it's just a union of
incompressible gluing annuli there. It suffices to show that $\pi_1S \cap
\pi_1P_k = 1$ for each component $P_k$ of the pared locus. Looking at the core
curve of the annulus $P_k$, this implies that some multiple of that core curve
is homotopic into $S$.

But this is impossibly by the above construction. Every such curve contains at
least two bands, on two different components $\bd_\pm B_i$. By the construction
of $S$, of any two bands which connect up when the boundary components are
glued across circles, at least one must be trans. This means that when we try
to homotope the core curve multiple into $S$, in order to follow along $S$
locally (in each page, where $S$ is locally a cover of the core surface, it
must be a lift from that core surface) it would have to traverse between all
three pieces of the cover, as that's how we connected up the $Si$. Every trans
arc lifts to an arc that connects two different boundary components of an $Si$,
which are ``pointed in different directions.'' But this is obviously
impossible, as our $P_k$ is restricted to be in a single boundary component, so
up to homotopy it must be generated by those two $I$-bundle pages only.

\end{proof}


%%%%%%%%%%%%%%%%%%%%%%%%%%%%%%%%%%%%%%%%%%%%%%%%%
\section{The case of no quasi-Fuchsian surface}
%%%%%%%%%%%%%%%%%%%%%%%%%%%%%%%%%%%%%%%%%%%%%%%%%

We prove a generalization of the above example. Basically, we want to first
identify the cases that obviously can't have a surface that is (QF). Then we'll
prove by construction that every other book of $I$-bundles does in fact contain
such a surface. We first introduce an additional definition.

\begin{defn}

Let $(M,P)$ be a pared book of $I$-bundles, possibly with compressible
boundary. We say $(M,P)$ is \emph{reduced} if it satisfies the following
criteria.

\begin{enumerate}

\item A meridian disk of each spine intersects the union of all binding annuli
at least three times.

\item $M$ is fully bound --- that is, there are no free binding boundaries of
pages. Note that this would be a consequence of Proposition~\ref{P:1ptcap}, but
we are allowing $M$ to have compressible boundary.

\item Each component of $P$ (once placed in minimal position with respect to
the gluing annuli) intersects at least one page $B$. Furthermore, if it
intersects exactly one page, it traverses at least one spine $C$ attached to
that page. In this case, either $C$ is of valence 1, or it's of valence greater
than 1 and $B$ is attached to $C$ along more than one boundary component.  If
valence 1, we require that $B$ is attached to $C$ along a gluing of degree at
least 3.

Note that any spines of valence 1 and degree 1 would also be eliminated by
Proposition~\ref{P:1ptcap}.  Also note that since it lies on a single boundary
component of $M$, a parabolic cannot enter and exit the same binding annulus at
a spine unless that spine has valence 1. Therefore, we can divide the cases
with a parabolic that intersects a spine but only hits a single page into
valence 1 spines, and multiple attached binding annuli from same page.  The
above criterion restricts the possible valence 1 spine cases to those which
have degree at least 3.

\end{enumerate}

\end{defn}

This definition is motivated by the following important idea. If $(M,P)$ is not
reduced, we can attempt to reduce it inductively. At each step, we remove pages
or spines from our pared books of $I$-bundles to produce new pared books of
$I$-bundles.  Because we're deleting pieces, we may end up with multiple
connected components. Each component is either a pared book of $I$-bundles,
possibly with compressible boundary, or an isolated page with no remaining
boundary components --- that is, a thickened closed surface. This is why we
must consider $M$ with compressible boundary in the above definition. However,
if $M$ is reduced, it will in fact have incompressible boundary by
Proposition~\ref{P:boibincomp}.

\begin{defn}

We \emph{reduce} a book of $I$-bundles as follows. We can repeat this
inductively on connected components of the result of a reduction.

\begin{enumerate}

\item[(A)] If $(M,P)$ has a spine $C$ which violates (1):

\begin{enumerate}

\item[(A1)] If $C$ has 0 binding annuli, delete $C$.

\item[(A2)] If $C$ has 1 binding annulus, note that the annulus may intersect
a meridian disk once or twice. If it intersects once, delete $C$. Note that the
attached page will soon be deleted by criterion (2).

\item[(A3)] If it has 1 binding annulus which intersects twice, the book of
$I$-bundles is homeomorphic to one obtained from the following.  Delete $C$ and
glue the attached page to a twisted $I$-bundle over a Mobius strip. This
corresponds to attaching a Mobius strip to the page's base surface.

\item[(A4)] If $C$ has 2 binding annuli, note that they must be parallel and
each intersect a meridian disk only once. The book of $I$-bundles is
homeomorphic to one obtained from the following. Delete $C$ and glue two pages
(or possibly two locations on the same page) along the now-free binding annuli.
Notice that this also matches up parabolics.

Note that (A3) and (A4) are the only steps that can produce a thickened closed
surface (instead of a true book of $I$-bundles).

\end{enumerate}

\item[(B)] If $(M,P)$ has a page $B$ which violates (2), delete $B$.

\item[(C)] If $(M,P)$ has a component $P_0$ of $P$ which violates (3), $P_0$
intersects at most one page:

\begin{enumerate}

\item[(C1)] If $P_0$ intersects a single page $B$, delete $B$.

\item[(C2)] If $P_0$ intersects no pages, it lies entirely inside a spine $C$.
Delete $C$.

\end{enumerate}

\end{enumerate}

In all cases, modify $P$ by removing all components which have nonempty minimal
intersection with the deleted pages or spines.

\end{defn}

Notice that in all cases we have a pared embedding of our new manifold into the
original. See Figure~\ref{F:reductionex} for an example of this reduction
algorithm. Note that in the figure, all pages are trivial $I$-bundles, and all
binding annuli intersect a meridian disk of the spine exactly once. To simplify
the picture, the annuli are omitted and the spines are drawn as their base
surfaces.

\myrotfig{fig-reductionex}{F:reductionex}{Example of the reduction algorithm}

\begin{thm}(Reduction theorem)

Let $(M,P)$ be a pared book of $I$-bundles. Repeatedly reducing terminates
after finitely many reductions in some $(M_r,P_r)$, with an associated pared
embedding $\iota\colon (M_r,P_r) \to (M,P)$. Each resulting connected component
is either a reduced book of $I$-bundles or a thickened closed surface with
empty pared locus.

Furthermore, any (QF) surface in $(M,P)$ is contained in $\iota(M_r,P_r)$. In
particular, if $(M_r,P_r)=\emptyset$, $(M,P)$ does not contain a (QF) surface.

\end{thm}

\begin{proof}

Write $\cM_0 = (M,P)$ and $\cM_i$ for the (possibly disconnected) manifold we
obtain after reducing $i$ times.

Let $\operatorname{size}(\cM_i) = \Sigma_{M_{ij} \text{ component of } \cM_i}
\operatorname{pages}(M_{ij}) + \operatorname{spines}(M_{ij})$. We claim that
any reduction must decrease the size. If we consider a thickened closed surface
to be made out of one page and no spines, it's immediately clear that
$\operatorname{size}(\cM_i)>0$ if $\cM_i \neq \emptyset$.  So this will imply
that reducing repeatedly must terminate after finitely many reductions.
Furthermore, to show that (QF) surfaces are included, we must show that any
reduction deletes a set disjoint from all (QF) surfaces.

We consider each possible reduction.

Reduction (A1) removes spines without attached pages. No such spine can contain
a $\pi_1$-injective closed surface, so it's clearly disjoint from (QF)
surfaces.

Reduction (A2) removes spines attached to a single page along a binding annulus
intersecting a meridian disk only once. Again, no $\pi_1$-injective closed
surface can cross this spine, as by Proposition~\ref{P:1ptcap} the boundary
admits a compression which our surface can be made disjoint from.

Reductions (A3) and (A4) do not actually delete anything. They produce
a homeomorphic book of $I$-bundles (or possibly a thickened closed surface)
with a homeomorphically corresponding pared structure. Therefore there's
nothing to prove in this case.

Reduction (B) removes a page with a boundary component that's not glued to
a spine. By the surface decomposition lemma, any $\pi_1$-injective surface that
intersects the page at all must be a cover of the that page's core surface.
Therefore it must have boundary components covering all boundary components of
the page under consideration. This includes the free boundary component, but
then by the surface decomposition lemma again, these have nothing they can glue
to. So it is impossible to produce a closed $\pi_1$-injective surface (without
boundary) that intersects this page.

Alternatively, apply Proposition~\ref{P:1ptcap} to obtain a compression of the
boundary, which we can make any $\pi_1$-injective surface disjoint from. By the
surface decomposition lemma, this forces our surface to be disjoint from that
page.

Reduction (C1) removes a page $B$ which intersects a parabolic $P_0$. There are
two cases to consider here.

\begin{enumerate}

\item If $P_0$ is contained entirely inside $B$, that is, it doesn't intersect
any spines at all, it's easy to see why no (QF) surface can intersect $B$. By
the surface decomposition lemma, any surface intersecting $B$ would have to be
a cover of $B$. But then $P_0$ will lift to any such cover, violating the
parabolic lifting criterion.

\item If $P_0$ intersects a spine, in order to satisfy the (C1) reduction
criterion, it must be a valence 1 spine $C$ with $B$ attached by degree 2.  $C$
cannot be valence 1 degree 1 because otherwise we'd violate (A2). Now it
suffices to consider a surface that intersects the page $B$. Such a surface
must be a cover of $B$. By the surface decomposition lemma, it must be glued to
itself by annuli along the spine $C$. But since the gluing of $B$ to $C$ is
degree 2, there can only be one topological choice of gluing through $C$, and
this is the same choice that the parabolic $P_0$ makes. This is easy to see if
we look at a double cover near where $B$ is glued to $C$. Since our parabolic
only traverses spines of this form, our surface will necessarily remain
parallel to the parabolic through each spine, and the resulting surface will
violate the parabolic lifting criterion for the parabolic $P_0$. So no (QF)
surface can intersect a page which satisfies the (C1) reduction criterion.

\end{enumerate}

Reduction (C2) removes a spine $C$ which intersects a parabolic $P_0$. Again,
by the surface decomposition lemma, any surface which traverses that spine will
have to contain an annulus connecting two page gluings --- that is, contain
a multiple of that parabolic. This violates the pared lifting criterion.

Finally, observe that in all cases, the parabolic components we remove from
$(M,P)$ as we decompose --- i.e., those that intersect the deleted pieces ---
cannot possibly affect whether a surface outside the deleted pieces is (QF), as
any surface that was otherwise (QF) but contained a multiple of any deleted
parabolic would have to intersect a deleted piece. So as we inductively delete
pieces of our book of $I$-bundles, we preserve the property that all (QF)
surfaces and parabolics we need to determine which surfaces are (QF) remain
outside the deleted parts of the book of $I$-bundles. Since our induction is
guaranteed to terminate by the size measure above, and we know that every
termination is a reduced book of $I$-bundles, a thickened closed surface with
no parabolics, or empty, this completes the proof.

\end{proof}

As motivation, also observe that our specific example in the earlier section is
one of the simplest possible topological structures for a reduced book of
$I$-bundles.

Note that the surface case is very easy.

\begin{prop}

Let $(M,P) = (\Si \times I, \emptyset)$ be a thickened closed surface with
empty pared locus. Then every cover of $\Sigma$ induces a (QF) surface
in $(M,P)$.

\end{prop}

\begin{proof}

$P = \emptyset$, so this follows directly from the covering lemma. Every closed
$\pi_1$-injective surface is (QF).

\end{proof}

%%%%%%%%%%%%%%%%%%%%%%%%%%%%%%%%%%%%%%%%%%%%%%%%%
\section{Statement of main theorem}
%%%%%%%%%%%%%%%%%%%%%%%%%%%%%%%%%%%%%%%%%%%%%%%%%

We are now ready to state the precise theorem.

\begin{thm}[Main Theorem]

Every reduced book of $I$-bundles contains a (QF) surface.

\end{thm}

Note that together with the earlier theorem, we've fully covered the
non-reduced case also.

\begin{thm}[Main Theorem, non-reduced case]

Let $M$ be a book of $I$-bundles. If the reduction theorem yields a nonempty
set of reduced book of $I$-bundles in $M$, then $M$ contains a (QF) surface.
Otherwise, it does not.

\end{thm}

%%%%%%%%%%%%%%%%%%%%%%%%%%%%%%%%%%%%%%%%%%%%%%%%%
\section{Proof of main theorem - preliminaries}
%%%%%%%%%%%%%%%%%%%%%%%%%%%%%%%%%%%%%%%%%%%%%%%%%

Now there are a number of topological simplifications we make, by passing to an
appropriate finite-sheeted cover of the book of $I$-bundles.  Once we can
construct a surface satisfying (QF) inside this cover, we will push it down and
perturb to obtain a surface satisfying (QF) downstairs (since it
$\pi_1$-injects into a subgroup, it will definitely still be
$\pi_1$-injective).

% TODO expand the above into an actual (short) proof

\begin{defn}

A \emph{good} book of $I$-bundles $M$ is a reduced book of I-bundles which
satisfies the following additional conditions.

\begin{enumerate}

\item Each binding annulus on a spine intersects a meridian disk of that spine
exactly once. That is, every spine has degree one.

\item Each page is glued to a given spine at most once. That is, for any page
$B$ and spine $C$, $B \cap C$ is at most a single component of $\cA$.

\item The two endpoints of a fiber in each page are in different boundary
components of $M$. In particular, each page is a trivial $I$-bundle over an
oriented surface.

\item Each spine intersects each boundary component of $M$ in at most a single
annulus.

\item Every arc of $P$ on a page connects two different binding annuli. That
is, there are no essential arcs that begin and end at the same binding annulus.

\end{enumerate}

\end{defn}

\begin{thm}

Let $M$ be a reduced book of $I$-bundles. Then $M$ has a finite-sheeted regular
cover which is good.

\end{thm}

\begin{proof}

To prove this, we'll use the fact that $\pi_1M$ is LERF to construct finite
sheeted covers with nice properties. Note that (1), (2), (3), and (4) are all
properties that once true, remain true when we lift to finite-sheeted covers.
So it suffices to guarantee each property one at a time by lifting to
finite-sheeted covers, because we can lift repeatedly and the old properties
will still hold.

We first take care of condition (1). For each spine $C$, let $k=d(C)$.  Let
$A_1,\dots,A_n$ be the binding annuli on $C$. These annuli must be parallel.
Notice that $\pi_1C$ is infinite cyclic, as its only torsion could come from
the identification with the binding annuli, but we've assumed that these are
$\pi_1$-injective. Now if $\alpha$ is a generator for the infinite cyclic group
$\pi_1C$, $\alpha^k$ is generates each $\pi_1A_i \cin \pi_1C$. Fix a point $x_0
\in C$, and apply Proposition~\ref{P:lerf3} to $\alpha$ and $k$ to obtain
a finite-sheeted regular cover $M' \to M$..  For any spine $C'$ covering $C$,
we claim that $C'$ has degree 1. Observe that $C'\to C$ has degree $d$ by
construction, with $k \mid d$. Therefore $\alpha^d$, which lifts to traverse
$C'$ exactly once, is a power of $\alpha^k$, the generator of each binding
annulus. Hence each binding annulus has preimage a union of components, each of
which traverses $C'$ exactly once. Hence $C'$ has degree 1.

We now produce a cover $M'$ that satisfies (1). Perform the above construction
for each spine to produce generators $\alpha_1,\dots,\alpha_n$ and degrees
$k_1,\dots,k_n$. Applying Proposition~\ref{P:lerf3'}, we can follow the above
argument locally on each spine in $M'$ to show that it has degree 1.

So now assume $M$ is reduced and satisfies (1). We claim $M$ has
a finite-sheeted cover which satisfies (2).

Let $B$ be a page intersecting a spine $C$ more than once. For each pair of
binding annuli $A_1$,$A_2$ which attach $B$ to $C$, we'll draw a closed curve
as follows.  Fix a meridian disk $D$ of $C$. Choose an arbitrary point $x_1 \in
A_1 \cap D$, $x_2 \in A_2 \cap D$. Choose an arbitrary arc $\alpha' \cin B$
connecting $x_1$ and $x_2$.  Let $\alpha$ be the closed curve obtained by
closing up $\alpha'$ with an arc along the interior of $D$. Denote this arc by
$\alpha''$.

Fix $x_1$ to be our basepoint. We claim that $\alpha \notin \pi_1B$. $\alpha$
intersects each of the binding annuli $A_1$ and $A_2$ in a single point. Any
homotopy of $\alpha$ to a curve in $\pi_1B$ would have to homotope $\alpha''$
rel boundary into $B$, but this is clearly impossible. So $\alpha \notin
\pi_1B$.  $\pi_1B$ is finitely generated. Apply Proposition~\ref{P:lerf1} to
$\alpha$ and $B$, obtaining a finite-sheeted cover.

Any lift of $\alpha$ to $\alpha'$ in this cover must connect two different
components of $p^{-1}(B)$.  Any lift of $\alpha''$ still lies on a meridian
disk of a spine covering $C$, and connects binding annuli covering $A_1$ and
$A_2$.  But $\alpha'$ lifts to arcs inside $p^{-1}(B)$. So in order for a lift
of $\alpha$ to connect two different components of $pi-1(B)$, the lifts of
$\alpha''$ must have endpoints in different components of $p^{-1}(B)$.

We now construct a cover $M'$ that satisfies (2). Repeat the above construction
for each pair of binding annuli for each page with multiple attachments to
a spine. This yields closed cuvres $\alpha_1,\dots,\alpha_n$ and corresponding
pages $B_1,\dots,B_n$. Note that there may be duplicates in this list of pages,
but the argument remains the same. Apply Proposition~\ref{P:lerf1'}. By
following the above argument locally, we see that any page upstairs with
multiple gluings to the same spine would correspond to a lift of some
$\alpha_i''$ with endpoints in the same component of $p^{-1}(Bi)$,
a contradiction.

So we have a cover which satisfies (1) and (2). We now show that $M$ satisfying
(1) and (2) has a finite-sheeted cover satisfying (3).

$M$ is orientable by definition, so each page must be orientable as well. So
the pages which are not trivial $I$-bundles over oriented pages must be twisted
$I$-bundles over nonoriented pages (in order for the resulting page to be
orientable). In particular, the endpoints of the $I$ fibers of a twisted
$I$-bundle will connect globally to form a single side of the page. So if we
look locally at an $I$ fiber, the two endpoints are guaranteed to be in the
same boundary component of $M$ (since they're in the same boundary component
even if we just look at that page). This explains why ensuring distinct
boundary components for each fiber guarantees trivial $I$-bundles.

This argument is very similar to (2). Let $B$ be a page such that fibers have
both endpoints on the same boundary component $\bd_i M \cin \bd M$.  Choose
a binding annulus $A$ in $\bd B$, and an arbitrary transversal $\alpha'' \cin
A$.  Let $\alpha' \cin \bd_a M$ be an arc in the boundary connecting the
endpoints of $\alpha''$. Such an $\alpha'$ must exist as the two endpoints are
part of the same boundary component. $\alpha'$ and $\alpha''$ combine to form
a closed curve $\alpha$.  Fix one endpoint of $\alpha'$ to be our basepoint.
$\bd M$ is a compact surface, so $\pi_1 \bd_i M$ is finitely generated.

We claim that $\alpha \notin \pi_1 \bd_i M$. As above, any such homotopy would
correspond to a homotopy rel boundary of $\alpha''$ into $\bd_i M$. Since
$\alpha''$ is a transversal of the binding annulus $A$, such a homotopy would
correspond to a boundary compression of $A$, which is impossible as $A$ is
boundary incompressible. Apply Proposition~\ref{P:lerf1} to $\alpha$ and $\bd_i
M$, obtaining a finite-sheeted cover. By the same argument as in (2),
$\alpha''$ must lift to connect two different components of p$^{-1}(\bd_i M).$
Since $\bd M' = p^{-1}(\bd M)$, this shows that the pages covering $B$ have
fibers that connect two different boundary components. We can therefore
construct $M'$ satisfying (3), by taking all such pages with both sides on the
same boundary component and applying Proposition~\ref{P:lerf1'}.

We can satisfy (4) by a similar argument. Let $C$ be a spine that intersects
the same boundary component $\bd_a M$ twice, and $\alpha'$ to be an arc of
a meridian disk of $C$ which connects two components of $C \cap \bd_a M$.
Connect the endpoints of $\alpha'$ with an arc $\alpha'' \cin \bd_a M$. By the
same argument as above with boundary compressions, $\alpha=\alpha' \cup
\alpha'' \notin \pi_1 \bd_a M$, so following the same argument proves (4).

So we have a cover which satisfies (1), (2), (3), and (4). To satisfy (5), let
$\alpha'$ be the core of a component of $P \cap B$ which has both endpoints in
the same binding annulus $A$. Let $\alpha''$ be a segment in $\bd A$ which
connects the two endpoints of $\alpha'$. Let $\alpha = \alpha' \cup \alpha''$
be the resulting closed curve. Since $\alpha'$ is essential, $\alpha \notin
\pi_1A$.  Apply Proposition~\ref{P:lerf1} to $\alpha$ and $A$, and follow the
argument used in (2) and (3).  This proves that $M$ has a cover satisfying
(1)-(5), completing the proof.

\end{proof}

%%%%%%%%%%%%%%%%%%%%%%%%%%%%%%%%%%%%%%%%%%%%%%%%%
\section{Proof of main theorem}
%%%%%%%%%%%%%%%%%%%%%%%%%%%%%%%%%%%%%%%%%%%%%%%%%

We've now reduced to the case of a good book of $I$-bundles. However, the
remaining work is still quite involved.

This is the main proof, and is quite involved. We proceed somewhat similarly to
the first example. We take covers over the pages and glue them together
cleverly to construct our surface. We check that the resulting surface
satisfies the pared lifting criterion. As in the first example, this suffices
to prove that our surface satisfies (QF).  We first need to take the correct
cover over each page. However, the details are more complex. We first will need
the following important notion.

\begin{defn}

Let $M$ be a book of $I$-bundles. A \emph{partially decomposed surface} $Q \cin
M$ is obtained from a closed pi1-injective surface S in minimal position
(following the surface decomposition lemma) by deleting any number of
components of Q cap C, for each spine C of M. The result is a possibly
disconnected surface with boundary, where each component is pi1-injective and
in minimal position. If (M,P) is a pared book of I-bundles, the \emph{pared
lifting pattern} on Q is defined componentwise in the surface decomposition,
just like for closed surfaces. We say that Q satisfies the \emph{pared lifting
criterion} if its pared lifting pattern contains no closed curves.

\end{defn}

Note that at each spine C of M, there must be an even number of boundary
components of Q which lie in C, by parity considerations with the deleted
annuli. Also note that, assuming M is reduced, it is trivial to build
a partially decomposed surface that satisfies the pared lifting criterion. To
do so, simply take a closed pi1-injective surface and delete all the
intersections with spines.  No individual page contains a pared closed curve,
so this trivial partially decomposed surface (which is just a disjoint union of
page covers) cannot either.

\begin{defn}

Let Q be a partially decomposed surface in a book of I-bundles M. The
\emph{defect} of Q is the total number of boundary components, summing over all
connected components. The \emph{local defect} at a spine C cin M is the number
of boundary components which intersect C. The \emph{degree} of Q is the
covering degree on each page. (Note that since some spine components were
deleted, the covering degree on a spine may be smaller.)

\end{defn}

Our strategy is as follows. For each boundary component of M, we will build
a partially decomposed surface. This surface will be obtained by removing spine
components from a cover of that boundary component. We will then align the
number of boundary components of the different partially decomposed surfaces
where they meet at spines, and glue them all together with new annuli. This
will produce a closed surface. As suggested by the above terminology, we want
our partially decomposed surface to have as small a defect as possible.

We will also need our partially decomposed surface to satisfy the following
technical conditions:

\begin{defn}

Let Q be a partially decomposed surface in a book of I-bundles M. We say that
Q is \emph{usable} if it satisfies the following:

begin enumerate

item[dag] The pared lifting pattern on Q contains no closed curves.

item[ddag] Every arc in the pared lifting pattern on Q begins and ends at two
different boundary components of Q

\end{defn}

\begin{defn}

We say that Q \emph{has sparse defect} if, in the surface decomposition of Q,
each component of Q cap B contains at most one component of bd Q. That is, the
``defective'' (non-glued) boundary components are ``spread out'' among the
components of Q in each page.

\end{defn}

\begin{lemma}\label{L:jigsaw}

Let M be a good book of I-bundles. Fix a boundary component F of M. Then, for
any sufficiently large d, there exists a partially decomposed surface Q in
M with the following properties:

begin enumerate

item Q is obtained by removing spine components from a finite-sheeted cover of
F of degree d.

item Q is usable.

item Q has sparse defect.

item Q has defect bounded above by a constant C depending only on (M,P) (ie,
not depending on d)

end enumerate

\end{lemma}

\begin{proof}

Suppose that F intersects the spines C1,dots,Cr and pages B1,dots,Bs of M.
Since M is good, F intersects each of these in a single component. Denote the
page intersections by Fi = F cap Bi and the spine intersectons by Fj' = F cap
Cj.

Let Q0 be the surface obtained by taking d many disjoint copies of each of the
Fi. Q0 is obviously a partially decomposed surface obtained by removing all
spine components from the trivial d-fold cover of F (that is, d many disjoint
copies of F), which is a closed pi1-injective surface in M. The pared lifting
pattern on each component of Q0 is a homeomorphic copy of (the core of) P cap
Fi. Since F is reduced, this pattern has no closed curves. Since F is good,
every arc in the pattern connects two different boundary curves of its
component. Hence Q0 is useful.

However, Q0 obviously fails conditions (3) and (4) of the lemma. We need to add
some annuli to ensure these conditions hold.

Fix a large integer $m>0$. Now, reserve a subset $\cR$ of the set of boundary
components of $Q_0$.  This subset should have the following properties:

(i) Each boundary component of each $F_i$ is covered by exactly $m$ elements of
$\cR$.

(ii) Every connected component of $F_i$, that is, lifted copy of each page,
contains at most one element of $\cR$.

This is possible as long as $d \geq C_0m$, where $C_0$ is the maximum number of
boundary components of any $F_i$. $C_0$ is fixed and depends only on $(M,P)$.
If $d\geq C_0m$,  each local cover $Q_0 \cap B_i$ has at least $C_0m$ many
disjoint copies above its respective boundary piece $F_i$.  So we have enough
copies to make our choices above.

We begin adding annuli that are copies of the annuli F cap Cj = Fj'. Given any
such annulus Fj', we know there are exactly two pages Bi1 and Bi2 such that the
two boundary components of Fj' are attached to Fi1 and Fi2 in F. There are in
fact two specific boundary components gamma1 of Fi1 and gamm2 of Fi2 that are
attached in this way. Each time we add an annulus copy of Fj' to Q0, we must
attach it to two curves widetilde gamma1 and widetilde gamma2 which are lifted
copies of gamma1 and gamma2 in components of Q0 cap Bi1 and Q0 cap Bi2,
respectively.  This is necessary to ensure that Q0 remains a partially
decomposed surface that can be extended to a cover of F.

We add annuli one at a time. Each time we add an annulus, we make sure to add
in such a way that the resulting surface is still usable. We also require that
we do \emph{not} attach any annuli to the boundary components in cR. Aside from
these requirements and the condition above (that our gluing extends to a cover
of F), we repeatedly choose an arbitrary annulus. This process will terminate
at some some surface Q1, after there are no more possible gluings to make.

\begin{claim}

We claim that $\#(\bd Q_1 \setminus \cR)$ is bounded by $C$, a constant
depending only on $(M,P)$. In particular, $C$ is independent of the choice of
$d$ and $m$ made earlier.

\end{claim}

\begin{proof}[Proof of Claim]

Consider an arbitrary boundary component $\widetilde{\gamma} \cin \bd Q_1
- \cR$.  It covers some boundary component $\gamma \cin F_i$. Look at the
number of arcs of $P \cap F_i$ which are incident to $\gamma$. This number
depends only on $(M,P)$. Let $C_1$ be the maximum such number of incident arcs
of $P$ for any boundary component $\gamma$ of any $F_i \cin F$. Let $C_2$ be
the total number of boundary components of the $F_i$. Let $\gamma' \cin F_{i'}$
be the boundary component matching $\gamma$ - that is, the boundary component
that is attached to $\gamma$ in $F$ by some spine annulus. Let $F_j'= F \cap
C_j$ be the annulus in $F$ that connects them.  Suppose we attempt to add an
annulus to $Q_1$ that covers $F_j'$, with one of its boundary components
attached to $\gamma$.

The cover extension condition allows us to attach the other boundary component
to any component of $\bd(Q_1 \cap F_{i'}) - \cR$ above $\gamma'$.  Because we
always glue matching boundary components (and we started with the same number
of each), $\bd Q_1 - \cR$ has the same number of remaining boundary components
above $\gamma'$ as it does above $\gamma$. So there must be at least one to
glue to.  Since we stopped gluing at $Q_1$, this implies that any gluing we
could make would force the resulting surface to not be useful.

Let $\widetilde{\gamma}' \cin \bd(Q_1 \cap F_{i'}) - \cR$ be an arbitrary
available lift of $\gamma'$.  Suppose gluing to $\widetilde{\gamma}'$ produces
a closed curve $\alpha$ in the pared lifting pattern. Obviously $\alpha$
intersects the newly added annulus $A$, otherwise $Q_1$ would already contain
a closed curve. $P$ is in minimal position and $M$ is reduced, so within $A$
the pared lifting pattern is just a union of transverse arcs. Removing $A$
therefore divides $\alpha$ into a union of arcs with endpoints on either
$\gamma$ or $\gamma'$.  $Q_1$ is useful, so none of these arcs can have both
endpoints on the same boundary component. So there exists an arc in $Q_1$
connecting $\gamma$ to $\gamma'$.  (This is the reason for the second condition
in our definition of a useful surface).  The pared lifting pattern in $Q_1$
near $\gamma$ is a homeomorphic copy of (the core of) $P \cap F_i$, so the
number of arcs incident to $\gamma$ is bounded by $C_1$.  Tracing these arcs
through $Q_1$, they can hit at most $C_1$ other boundary components.  These are
the only boundary components we can attach our annulus to to produce a closed
curve.

Similarly, suppose gluing to $\widetilde{\gamma}'$ produces an arc $\alpha$
with endpoints on the same boundary component. Let $\widetilde{\delta}$ be this
boundary component. Again, $\alpha$ intersects the added annulus $A$, otherwise
$Q_1$ would already not be useful. Removing $A$ divides $\alpha$ into a union
of arcs. Except for the two endpoints of $\alpha$ (which now lie on two
different subarcs - call these subarcs $\alpha_0$ and $\alpha_1$), all other
endpoints of these arcs must either lie on $\gamma$ or $\gamma'$.  But no arc
can have both endpoints on the same boundary component, or $Q_1$ would already
fail to be useful.  So any subarcs except $\alpha_0$ and $\alpha_1$ must
connect $\gamma$ to $\gamma'$.  By parity we can see that $\alpha_0$ and
$\alpha_1$ both have one endpoint on $\widetilde{\delta}$, but their other
endpoints must be different.  That is, either $\alpha_0$ ends on
$\widetilde{\gamma}$ and $\alpha_1$ on $\widetilde{\gamma}'$, or vice versa.
This is the set of circumstances that leads to a returing arc.

Now, as above, the arcs incident to $\widetilde{\gamma}$ hit at most $C_1$
other boundary components. These are our possible $\widetilde{\delta}$. Each of
these has at most $C_1$ incident arcs itself, one of which returns to
$\widetilde{\gamma}$, leaving $C_1-1$ that we need to care about. We have
a total of at most $i_a*(C_1-1)$ many "distance two" available boundary
components. If we attach our annulus to one of these boundary components, we
will produce a non-useful surface.

By construction of $Q_1$, there are no more legal gluings. So every remaining
boundary component must be disallowed for one of the above reasons. So $Q_1$
can have at most $C_1+(C_1*(C_1-1)) = C_1^2$ many leftover boundary components
(that is, components of $\bd Q_1 - \cR$) above $\gamma'$.  Since our choice of
$\widetilde{\gamma}$ was arbitrary, this implies that there are at most $C
= C_1^2*C_2$ many leftover boundary components, that is, components of $\bd
Q_1-\cR$. This proves the claim.

\end{proof}

\noindent \emph{Proof of Lemma~\ref{L:jigsaw} continued.} We now attach more
annuli to build a surface $Q$ which satisfies (3) and (4). Intuitively, the
boundary components that make up the reserve are sparse. As long as our reserve
is sufficiently large we can make use of it to attach enough annuli that only
reserve boundary components remain. All the remaining non-reserve boundary
components will be attached to reserve boundary components by annuli.

Construct $Q$ from $Q_1$ as follows.  For each boundary component $\gamma$
downstairs, as discussed above, there are at most $C_1^2$ many elements of $\bd
Q_1 - \cR$ above it.  For each of these, there are at most $C_1^2$ we could
glue to above $\gamma'$ that produce a non-useful surface. We now allow
ourselves to glue to the reserve boundary components. Assume that $m\geq
2C_1^2$.  Then attach each leftover (non-reserve) lift of $\gamma$ to an
element of $\cR$ above $\gamma'$ with a copy of the annulus $F_j'$, one at
a time.  At any point there will be at least $i^2+1$ elements of $\cR$
remaining above $\gamma'$, so we'll always be able to choose one such that our
surface is still useful.  Repeat this process for each $\gamma$ to construct
$Q$.  Since we glued all the leftovers, it immediately follows that $\bd Q \cin
\cR$, and therefore $Q$ satisfies (3).

Let's analyze the possibilities for the defect of $Q$, that is, the number of
boundary components.  Above each $\gamma \cin \bd F_i$, $\cR$ has $m$ elements.
So $\bd Q$ has at most $m$ components above $\gamma$, or $C_2m$ many boundary
components in total. The lower bound is determined by the maximum number of
leftovers, since the only way we'll remove reserve components from $\bd Q$ is
by gluing them to non-reserve leftover components. There are at most $C_1^2$
leftover components above $\gamma$, and similarly above $\gamma'$, so there
will be at least $m - C_1^2$ components remaining in $\bd Q$ above $\gamma$.
We find that

\[ C_2(m-C_1^2) \leq defect(Q) \leq C_2m \]

Any choice of $m\geq 2C_1^2$ works, as long as $d$ is sufficiently large
relative to $m$.  This proves (4).

\end{proof}

\begin{proof}[Proof of Main Theorem]

We want to use Lemma~\ref{L:jigsaw} to construct a partially decomposed surface
$Q_a$ for each boundary component $F_a$ of $M$. Now we want to glue these
together, but they may have very different numbers of boundary components! So
we'll need to normalize them so they all have the same number of boundary
components near each spine, so we can do a local construction with annuli in
each spine.

Let $M_1 = max a C_1(a)$, that is, the maximum number of components of $P \cap
\cap B_i$ incident to a single boundary component of $\bd M \cap B_i$ for any
spine $B_i$. By the inequality in Lemma~\ref{L:jigsaw}, for each $\gamma \cin
\bd F_i$, $\#\{\bd F_i \text{ above } \gamma\} \geq m - M_1^2$. We want to make
this an equality for each $Q_a$ by adding some more annuli to $Q_a$ in the same
way that we added annuli in Lemma~\ref{L:jigsaw}.

Again, we use the size of the reserve to our advantage. Assuming $m\geq M_1^2
+ C_1(a)^2$ for each $F_a$, by the same argument as the lemma, there will
always be choices remaining from the reserve to make the gluing. Combining
these all we need is $m\geq 2M_1^2$.

By abuse of notation, we'll call these normalized surfaces $Q_a$ also (the
non-normalized ones will not come up again). Each has exactly $M = m - M_1^2$
many boundary components above any $\gamma \cin \bd F_i$.

Build a single closed surface $S$ by attaching annuli to the $Q_a$ as follows.

Let $C$ be an arbitrary spine of $M$. There are exactly $d(C)$ many binding
annuli in $\bd C$, and $C$ is attached to $d(C)$ distinct pages, since $M$ is
good. Each page has two sides, so there are $2d(C)$ many ``incident neighbor
types'' in the $Q_a$, corresponding to the $2d(C)$ boundary components of
binding annuli in $\bd C$. These are attached to the $2d(C)$ side boundary
components of pages adjacent to $C$.  Because $M$ is good, we know that all
these pages are distinct and all the corresponding binding annuli have degree
1.

Introduce notation as follows. A neighborhood $\cN(C)$ consists of $d(C)$
neighborhoods inside pages, attached to $C$ by annuli. The meridian of $C$
gives a cyclic order to the attached pages.  Fix an orientation for the
meridian and an (arbitrary) starting point along it, and label the pages $1,
\dots, d(C)$ under this ordering.  The orientation of the meridian also gives
an ordering of the boundary components of each binding  annulus.  Using this,
label the $d(C)$ gluing annuli by $A_1,\dots,A_{d(C)}$. Also label the $2q$
binding annulus boundary components by $\gamma_1^-,\gamma_1^+,\dots,
\gamma_q^-,\gamma_q^+$, where $\gamma_i^-$ and $\gamma_i^+$ are the two sides
of the binding annulus for the $i$th page.  Extend these labels to the
corresponding boundary components of the page boundaries attached to the
binding annuli.  Remember that all these labels are downstairs - that is,
they're labels of pieces of $\bd M$.

By construction of the $Q_a$, there are $2d(C)M$ many components of
$\bd(\bigsqcup_a Q_a) \cap C$. Attach them as follows. Let $A_{1,3}$ be an
embedded valence 1 annulus in $C$ with one boundary component in $A_1$ and one
in $A_3$. Define $A_{2,4},A_{3,5},\dots,A_{(d(C)-1),1},A_{d(C),2}$ similarly.
Now construct the partially decomposed surface $S$ by $M$ many copies of each
$A_{i,i+2}$, and attaching their boundary components to lifts of $\gamma_i^-$
and $\gamma_{i+2}^+$. By construction of the $Q_a$ there will be exactly $M$
many available lifts of each to attach to. Therefore, when we repeat this for
each spine $C$, the result is a closed surface $S$. Note that since $M$ is
reduced, we know that each spine has at least 3 annuli. This means we'll never
attempt to attach an annulus with both boundary components in the same binding
annulus.

We claim that $S$ satisfies the pared lifting criterion.



% d = lcm degree of subgp sep cover (lcm of degree on each page) After taking
% copies of each subgp sep cover to ensure all are same degree: n = (large)
% # of copies of each cover above m = # of copies of each bc to reserve
% i(M,P,S_a) = max # incident arcs at a bc on S_a boundary component TODO call
% this ia j(M,P,S_a) = total # bcs on page boundaries making up S_a
% ("junctions")

Consider an arbitrary boundary component $F_a$ of the book of $I$-bundles $M$.
Cutting $M$ along the gluing annuli decomposes $\bd_aM$ into a union of pieces
each of which is one side of an $I$-bundle page. Call these pieces $S_{a,k}$.
Then the page is $S_{a,k}\times I$, with the piece embedded as $S_{a,k}\times
0$ or $S_{a,k}\times 1$. Without loss of generality let it be $S_{a,k}\times
0$. We can look at the pattern of arcs $P_{a,k} = P \cap (S_{a,k}\times 0)$ on
that side of the page, which is part of $\bd_aM$. We can also look at the
pattern of arcs on the opposite side of the page $\overline{P_{a,k}} = P \cap
(S_{a,k}\times 1)$, and temporarily view it as living inside $S_{a,k}$
(flattening $S_{a,k}\times I \to S_{a,k}$).  For each piece, apply the lemma
above to ($S_{a,k},P_{a,k} \cup barP_{a,k}$) to obtain a cover $S'_{a,k}$. If
we view this as a cover of $S_{a,k}x0$ and $S_{a,k}x1$, it has the property
that any arc in $P_{a,k}$ or $barP_{a,k}$ lifts to an arc connecting two
different boundary components of $S'_{a,k}$.

Let $d_{a,k}$ be the degree of this cover. Let $d = lcm a,k d_{a,k}$.

Fix some large integer $n>0$. Now for each $k$, let $\pi_{a,k}
\colon \widetilde{S_{a,k}} \to S'_{a,k} \to S_{a,k}$ be a disjoint union of $nd/d_{a,k}$
copies of $S'_{a,k}$. The purpose of the $d/d_{a,k}$ normalization is to make sure
each $\pi_{a,k}$ is a cover of the same degree, $n*d$.  We'll call the connected
components of this cover "pieces". We can also combine these into a single
cover $\pi \colon \bigsqcup \widetilde{S_{a,k}} \to \bigsqcup S_{a,k}$

Fix a large integer $m>0$. Now, reserve a subset $\cR$ of the boundary
components of $\bigsqcup \widetilde{S_{a,k}}$.  This subset should have the following
properties:

(1) Each boundary component of each $S_{a,k}$ is covered by exactly $m$ elements
of $\cR$.

(2) Every connected component of $\bigsqcup \widetilde{S_{a,k}}$ contains at most one element
of $\cR$.

This is possible as long as $n \geq mb$, where $b$ is the maximum number of
boundary components of any $S_{a,k}$. If $n\geq mb$, we know $d/d_{a,k} \geq
1$, so each disjoint union $\widetilde{S_{a,k}}$ has at least $mb$ many
disjoint copies of each $S'_{a,k}$.  So we have enough copies to make our choices
above that $S_{a,k}$ disjoint.

Each $S_{a,k}$ is homeomorphic to the core surface of its page, so by the covering
lemma any closed surface formed by gluing the $\widetilde{S_{a,k}}$ along appropriate
boundaries will induce an immersed $\pi_1$-injective surface in $M$. This is
our strategy.

Write $\widetilde{P_{a,k}}=\pi_{a,k}^{-1}(P_{a,k})$. As constructed, $P_{a,k}$ consists of proper arcs
connecting two different boundary components of $\widetilde{S_{a,k}}$.

We first glue the pieces "along" $\bd_aM$ in the following sense. Given any two
pieces $S_{a,k_1}$, $S_{a,k_2}$ in the decomposition of $\bd_aM$, we know that
$S_{a,k_1}$ has some boundary components $\gamma_1,\gamma_2,\dots \cin
dS_{a,k_1}$, and $\gamma_1',\gamma_2',\dots \cin dS_{a,k_2}$ such that before we
cut $\bd_aM$ apart, $\gamma_1$ was glued to $\gamma_1'$, $\gamma_2$ to
$\gamma_2'$, etc.  That is, these components form the locus we glue $S_{a,k_1}$
and $S_{a,k_2}$ together along when we glue the $S_{a,k}$ to form $\bd_aM$.  For
each such pair, look at the boundary components of $\widetilde{S_{a,k}}1$ and
$\widetilde{S_{a,k}}2$.  We allow ourselves to glue components of
$\pi^{-1}(\gamma_1)$ to components of $\pi^{-1}(\gamma_1')$, components of
$\pi^{-1}(\gamma_2)$ to components of $\pi^{-1}(\gamma_2')$, etc. We forbid
ourselves from gluing components of $\pi^{-1}(\gamma_1)$ to any other lifted
boundary components of $S_{a,k_2}$ or of any other pieces. Only on the very last
step of our construction will we break this rule.

Note that since our gluing is locally a cover we can only glue each pair of
boundary components upstairs in one possible way - we could choose a different
gluing, but this would just correspond to a different choice of cover by the
covering lemma.

We introduce some notation. Bbegin by taking $\widetilde{S}^{(0)} = \bigsqcup \widetilde{S_{a,k}}$ to
be a disjoint union of these covers - that is, we haven't done any gluing yet.
Think of these as pieces we'll use to build our cover. At each step in the
gluing, we have a gluing map $f(i+1) \colon \widetilde{S}^{(i)} \to
\widetilde{S}^{(i+1)}$ by attaching two boundary comonents of
$\widetilde{S}^{(i)}$. Write $\widetilde{P}^{(i)}$ for the arcs and curves on
$\widetilde{S}^{(i)}$ induced by gluing the $\widetilde{P_{a,k}}$.  That is,
$\widetilde{P}^{(i)} = f(i) \circ ...  \circ f(1) (\bigsqcup
\widetilde{P_{a,k}})$.

We'll want to glue boundary components other than $\cR$ first - these are the
"reserve" that we'll only use at the end. This guarantees that our final glued
surface will have leftover boundary components that are sufficiently far apart.
By abuse of notation, we'll also use $\cR$ to refer to the induced set of
boundary components in each $\widetilde{S}^{(i)}$.

In addition to gluing along $\bd_aM$ and avoiding gluing members of $\cR$, we want
to preserve the following two invariants.

% FIXME decide what to call it in the paper...  still need to figure out how
% I'm going to do the references too

\begin{enumerate}

\item[(\dag)] $\widetilde{P}^{(i)} \cin \widetilde{S}^{(i)}$ contain no closed
curves.  That is, it is a union of proper arcs. \label{I:dag}

\item[(\dag')] No arc of $\widetilde{P}^{(i)}$ begins and ends at the same
boundary component of $\widetilde{S}^{(i)}$. \label{I:dag'}

\end{enumerate}

By the use of the lemma and the assumption of the theorem (no closed curves in
individual pages), $\widetilde{S}^{(0)}$ satisfies (\dag) and (\dag') We
repeatedly make an arbitrary gluing that lies along $\bd_aM$ in the above
sense, doesn't glue any elements of $\cR$, and preserves (\dag) and (\dag').  This
process will terminate at some $\widetilde{S}^{(N)}$, after there are no more
gluings left to make. $\widetilde{S}^{(N)}$ is a surface with boundary, because
there are "leftover" boundary components that we couldn't glue without
violating one of the above conditions.

% FIXME restructure into lemmas?
%Claim.

We claim that $\#(d\widetilde{S}^{(N)} - \cR)$ is bounded by $C(M,P,S_a)$.  In
particular, this bound is independent of the choice of $n$ and $m$ made
earlier.

%Proof of Claim.

Consider an arbitrary boundary component $\widetilde{\gamma} \cin
d\widetilde{S}^{(N)} - \cR$.  It lives above some boundary component $\gamma
\cin dS_{a,k}$. Look at the number of arcs of $P_{a,k}$ which are incident to
$\gamma$. This number depends only on $M$ and $P$ (not on $n$). Let $i_a
= i(M,P,S_a)$ be the maximum such number of incident arcs of $P$ for any
boundary component $\gamma$ of any $S_{a,k} in S_a$. Let $j_a = j(M,P,S_a)$ be
the total number of boundary components of the $S_{a,k}$.

Let $\gamma' \cin dS_{a,k}'$ be the boundary component matching $\gamma$. We
consider gluing $\gamma$ along $M$ to any component of $\bd\widetilde{S}^{(N)}
- \cR$ above $\gamma'$.  Because we always glue matching boundary components
(and we started with the same number of each), $\bd\widetilde{S} - \cR$ has the
same number of leftover boundary components above $\gamma'$ as it does above
$\gamma$. So there must be at least one to glue to.  The only reason we'd be
unable to glue is if gluing to any leftover lift of $\gamma'$ violates (\dag) or
(\dag').

Let $\widetilde{\gamma}' \cin \bd\widetilde{S}^{(N)} - \cR$ be an arbitrary
leftover lift of $\gamma'$.  Suppose gluing to $\widetilde{\gamma}'$ violates
(\dag). Let $\alpha$ be a closed curve produced by the gluing.  Obviously
$\alpha$ intersects the $\widetilde{\gamma} = \widetilde{\gamma}'$ gluing
circle, otherwise $\widetilde{S}^{(N)}$ would already violate (\dag).  Cutting
along this circle divides $\alpha$ into a union of arcs with endpoints on
either $\gamma$ or $\gamma'$.  $\widetilde{S}$ satisfies (\dag'), so no arc can
have both endpoints on the same boundary component. So there exists an arc
connecting $\gamma$ to $\gamma'$. This is the only way that an additional
gluing would violate (\dag). But since $\widetilde{S}^{(N)}$ is locally a cover
on each piece, the number of arcs incident to $\gamma$ is bounded by $i_a$.
Tracing these arcs through $\widetilde{S}^{(N)}$, they can hit at most $i_a$
other boundary components.  These are the only boundary components we can glue
to to violate (\dag).

Similarly, suppose gluing to $\widetilde{\gamma}'$ violates (\dag'). Let $\alpha$ be an
arc with endpoints on the same boundary component produced by the gluing. Let
$\widetilde{\delta}$ be this boundary component. Again, $\alpha$ intersects the
$\widetilde{\gamma} = \widetilde{\gamma}'$ gluing circle, otherwise $\widetilde{S}^{(N)}$ would already
violate (\dag').  Cutting along this circle divides $\alpha$ into a union of
arcs.  Except for the two original endpoints (which now lie on two different
subarcs - call these subarcs $\alpha_0$ and $\alpha_1$), all other endpoints of
these arcs must either lie on $\gamma$ or $\gamma'$.  But no arc can have both
endpoints on the same boundary component, or $\widetilde{S}^{(N)}$ would arleady violate
(\dag'). So any subarcs except $\alpha_0$ and $\alpha_1$ must connect $\gamma$ to
$\gamma'$.  By parity we can see that $\alpha_0$ and $\alpha_1$ both have one
endpoint on $\widetilde{\delta}$, but their other endpoints must be different. That is,
either $\alpha_0$ ends on $\widetilde{\gamma}$ and $\alpha_1$ on $\widetilde{\gamma}'$, or vice
versa.

Now the arcs incident to $\widetilde{\gamma}$ hit at most $ia$ other boundary
components. These are our possible $\widetilde{\delta}$. Each of these has at
most $i_a$ incident arcs itself, one of which returns to $\widetilde{\gamma}$,
leaving $i_a-1$ that we need to care about. We have a total of at most
$i_a*(i_a-1)$ many "distance two" leftover boundary components. These are the
only boundary components we can glue to to violate (\dag').

But by construction of $\widetilde{S}^{(N)}$, there are no more legal gluings. So
$\widetilde{S}$ can have at most $i_a+(i_a*(i_a-1)) = i_a^2$ many leftover boundary
components (that is, components of $\bd\widetilde{S}^{(N)}-\cR$) above $\gamma'$.  But since
$\widetilde{S}^{(0)}$ has the same number of boundary components above $\gamma$ and
$\gamma'$, and we're only allowed to glue them to each other, this implies that
there can be at most $i_a$ boundary components above $\gamma$ as well.  Since
our choice of $\widetilde{\gamma}$ was arbitrary, this implies that there are
at most $C(M,P,S_a) = i_a^2*j(M,P,S_a)$ many leftover boundary components, that
is, components of $\bd\widetilde{S}^{(N)}-\cR$.

%Proof of Theorem contd.

Now that we've constructed $\widetilde{S}^{(N)}$, we want to perform some additional
gluings to construct $\widetilde{S}$. In addition to (\dag) and (\dag'), $\widetilde{S}$ should
satisfy

\begin{enumerate}

\item[(\dag'')] Any two boundary components of $\widetilde{S}$ lie on different
pieces. That is, they correspond to boundary components of different connected
components of $\widetilde{S}^{(0)}$ under the gluing map $\widetilde{S}^{(0)}
\to \widetilde{S}$.

\end{enumerate}

Intuitively, this is easy using the reserve. The reserve on its own satisfies
(\dag''), so we only have to deal with the leftovers discussed above.  The trick
is the reserve is much larger than the leftovers, so if we allow ourselves to
glue the leftovers to the reserve we can take care of all the leftovers without
violating (\dag) or (\dag'). Our remaining boundary components will be a subset
of the reserve.

To be precise, construct $\widetilde{S}$ from $\widetilde{S}^{(N)}$ as follows.
For each boundary component $\gamma$ downstairs, as discussed above, there are
at most $i^2$ many elements of $\bd\widetilde{S}^{(N)}-\cR$ above it. For each of
these, there are at most $i^2$ we could glue to above $\gamma'$ that violate
(\dag) or (\dag').  Assume that $m\geq 2i^2$.  Then glue each leftover lift of
$\gamma$ to an element of $\cR$ above gama', one at a time. At any point there
will be at least $i^2+1$ elements of $\cR$ remaining above $\gamma'$, so we'll
always be able to choose one that doesn't violate (\dag) or (\dag'). Repeat
this process for each $\gamma$ to construct $\widetilde{S}$.  Since we glued
all the leftovers, it immediately follows that $\bd\widetilde{S} \cin \cR$, and
therefore $\widetilde{S}$ satisfies (\dag'').

Let's analyze the possibilities for $\#\bd\widetilde{S}$, the number of
boundary components. Above each $\gamma \cin dS_{a,k}$,
$\cR$ has $m$ elements. So $\bd\widetilde{S}$ has at most $m$ components above
$\gamma$, or $m*j$ many boundary components in total. The lower bound is
determined by the maximum number of leftovers, since the only way we'll remove
reserve components from $\bd\widetilde{S}$ is by gluing them to leftover
components. There are at most $i^2$ leftover components above $\gamma$, and
similarly above $\gamma'$, so there will be at least $m - i^2$ components
remaining in $\bd\widetilde{S}$ above $\gamma$. So

\[ m-i_a^2 \leq \#(\bd\widetilde{S} \text{ above }\gamma) \leq m \]

and

\[ (m-i_a^2)j_a \leq \#\bd\widetilde{S} \leq mj_a \]

Finally, suppose we've done the entire above construction for each boundary
component of our book of $I$-bundles $M$ to produce a partially-glued surface
with some boundary components left over. For each boundary component $\bd_aM$,
we'll call the surface $\widetilde{S_a}$. Note that we must be sure to make the same
choice of $n$ and $m$ for all these surfaces. Now we want to glue these
together, but they may have very different numbers of boundary components! So
we'll need to normalize them so they all have the same number of boundary
components above each $\gamma \cin dS_{a,k}$, so we can do a local construction
above a neighborhood of each spine in $M$.

Let $i_M = max a i_a$. By the inequality above, for each $\gamma \cin S_{a,k}$,
$\#(d\widetilde{S_a} \text{ above } \gamma) \geq m - i_M^2$. We want to make
this an equality for each $\widetilde{S_a}$ by performing some more gluings.

Again, we use the size of the reserve to our advantage. Assuming $m\geq i_M^2
+ i_a^2$ for each $\widetilde{S_a}$, by the same argument as above there will
always be choices remaining from the reserve to make the gluing. Combining
these all we need is $m\geq 2i_M^2$.

By abuse of notation, we'll call these normalized surfaces $\widetilde{S_a}$ also (the
non-normalized ones will not come up again). Each has exactly $C = m - i_M^2$
many boundary components above any $\gamma \cin dS_{a,k}$.

Build a single surface $\widetilde{S}$ by gluing the $\widetilde{S_a}$ as follows. Begin with
$\bigsqcup_a \widetilde{S_a}$.

For each spine $M_c$ of $M$, look at all the incident pages. Say there are $q$
of them.  Each page has two sides, so there are $2q$ many pieces $S_{a,k}$
downstairs near $M_c$, each with one boundary component glued along the spine
$M_c$.  Because $M$ is good, we know that all these $S_{a,k}$ are distinct and
they are all glued along simple parallel closed curves on $\bd M_c$ (do we know
they're longitudes?  See above \textbf{ TODO}).

Introduce notation as follows. $\cN(M_c)$ consists of $q$ neighborhoods inside
pages, attached to $M_c$ by annuli. The meridian of $M_c$ gives a cyclic order to
the attached pages.  Fix an orientation for the meridian and an (arbitrary)
starting point along it, and label the pages $1, \dots, q$ under this ordering.
The orientation of the meridian also gives an ordering of the boundary
components of each attaching annulus.  Using this, label the $2q$ boundary
components by $\gamma_1^-,\gamma_1^+,\dots, \gamma_q^-,\gamma_q^+$, where
$\gamma_i^-$ and $\gamma_i^+$ are the two sides of the attaching annulus for
the ith page.

Now consider $\bd M \cap \cN(M_c)$. Notice that the $2q$ boundary components of these
attaching annuli that we just labeled are precisely the $2q$ boundary curves of
the boundary pieces $S_{a,k}$ that are near $M_c$. We don't know how many global
boundary components $a$ are involved, but locally we do know which pieces are
attached inside $\bd M \cap \cN(M_c)$. $\bd M \cap M_c$ consists of q parallel annuli
along $\bd M_c$ that connect adjacent pages together. That is, the first annulus
has boundary $\gamma_1^+ \cup \gamma_2^-$, the second $\gamma_2^+ \cup
\gamma^3_-$, and so on.  It follows that these are the "pairs downstairs" in
the above construction of each $\widetilde{S_a}$. That is, these are the curves
we called $\gamma$ and $\gamma'$ earlier.  So within $\bigsqcup \widetilde{S_a}$,
the existing gluings only glue lifts of $\gamma_1^+$ to lifts of $\gamma_2^-$,
lifts of $\gamma_2^+$ to lifts of $\gamma_3^-$, etc etc.

Finally, the gluing. Because we normalized, each $\bigsqcup_a \widetilde{S_a}$ has
exactly $C$ boundary components above each $\gamma_i^\pm$.  Glue these $2qC$
boundary components as follows. Glue lifts of $\gamma_1^+$ to lifts of
$\gamma_3^-$, lifts of $\gamma_2^+$ to lifts of $\gamma_4^-$, and so on. In
general glue lifts of $\gamma_i^+$ to lifts of $\gamma_{i+2}^-$. Within each
subset of lifts, choose the gluing arbitrarily. We know the numbers on each
side are equal, so they'll match up.

Every boundary component of $\bigsqcup_a \widetilde{S_a}$ lives above a boundary component
of some $S_{a,k}$ which means it is attached to some core $M_c$. So once we've done
the above gluing step for all cores, the resulting surface $\widetilde{S}$ is closed.

%Claim.

Finally, we claim $\widetilde{S}$ satisfies (QF).

%Proof of Claim.

% TODO
\textbf{TODO simplify this argument using good property (3) as written up
earlier and discussed with Ian!!}

By construction, $\widetilde{S}$ consists of covers of each page of $M$ glued along
parallel longitudinal annuli at the spines of $M$. By the covering lemma, it
induces an immersed $\pi_1$-injective surface. It remains to show that
$\widetilde{S}$ cannot contain any lifts of parabolic curves.

Let $\alpha$ be a parabolic curve in some boundary component $S_a$ of $M$. Cut
$\alpha$ into arcs $\alpha_i$ along the page boundaries, so each $\alpha_i \cin
S_{a,k_i}$.  We can think about this as follows. For each $\alpha_i$, there are two
kinds of lifts to $\widetilde{S}$. We can view it as an arc in $P_{a,k_i} \cin
S_{a,k_i}$, and lift it to a cover of $S_{a,k_i}$, that is, a component of
$\widetilde{S_{a,k_i}}$. Since $\alpha \cin S_a$, this will necessarily be part
of $\widetilde{S_a} \cin \widetilde{S}$. Furthermore, by construction of
$\widetilde{S_a}$, lifting the $\alpha_i$ to $\widetilde{S_a}$ pieces and
gluing cannot yield closed curves.  It only yields a union of arcs, by (\dag).

However, the second way we can lift $\alpha_i \cin S_{a,k_i}$ is to a cover of the
opposite page boundary $\overline{S_{a,k_i}}$. Let $S_b$ be the boundary component
containing $\overline{S_{a,k_i}}$, and write $S_{b,l_i} = \overline{S_{a,k_i}}$, we can see that $\alpha_i$
in $\overline{P_{b,l_i}}$, the set of "opposite parabolic arcs" that we defined earlier.
So $\alpha_i$ lifts to $S'b,li$ as a union of arcs, each of which connects two
different boundary components of $S'b,li$. This produces a set of arcs in
$\widetilde{S_b}$. Note that it may be possible to have $S_a = S_b$, depending on
how our book of $I$-bundles is constructed.  Regardless, we want to think of
these as a different kind of lift, because we're using the opposite page side
to lift, rather than the side the arc "naturally lives on."

We now claim that these arcs cannot be used to form closed curves in
$\widetilde{S_b}$.  Intuitively, this is because the opposite side of the
boundary does not remain parallel to $\alpha$ for long enough, so we'll soon
reach a piece of $\widetilde{S_b}$ where our lift can't continue.  To be
precise, let $\widetilde{\alpha_i}$ be a lift of $\alpha_i$ to
$\widetilde{S_b}$.  Suppose $\alpha_i$ has an endpoint at a spine $M_c$. If we
look at how the boundary components of a book of $I$-bundles behave near
a spine, we that locally the two boundary sides of a page are attached to sides
of two different pages.  In the terminology we used earlier for gluing pages at
a spine, $\gamma_i^-$ and $\gamma_i^+$ are attached to the attaching annulus
boundary curves of two different pages: $\gamma_i^-$ attaches to
$\gamma_{i+1}^+$, but $\gamma_i^+$ attaches to $\gamma_{i-1}^-$.  Since there
are at least three pages at $M_c$ by (3), the $(i-1)$st and $(i+1)$st pages are
distinct.

Suppose that the endpoint of $\alpha_i$ we're considering lies on $\gamma_i^-$. That
is, $\gamma_i^-$ is a boundary component of $S_{a,k_i}$, in our other notation. And
$\gamma_i^+$ is a boundary component of the opposite page $\overline{S_{a,k_i}} = S_{b,l_i}$.
Since $\gamma_i^-$ attaches to $\gamma_{i+1}^+$, any lifted arc we attach
$\widetilde{\alpha_i}$ to in that direction must lift from that $(i+1)$st page at $M_c$.
But in $\widetilde{S_b}$, we attach the surface pieces according to the gluings needed
for the $S_b$ boundary component.  Locally, $\widetilde{\alpha_i} \cin \widetilde{S_b},li$, where
lifts of $\gamma_i^+$ only attach to lifts of $\gamma_{i-1}^-$. But $\gamma_{i-1}^-$ is
a boundary component of a different page, so $\widetilde{\alpha_i}$ connects to a next
page with no lift of the next segment of $\alpha_i$. So there is no possible way
to continue $\widetilde{\alpha_i}$.  The same argument applies at the opposite endpoint
of $\widetilde{\alpha_i}$. This argument shows that lifts of arcs $\alpha_i$ as "opposite
parabolic arcs" to $S_b$ have no "opposite arcs" on either side that they can
glue to.

Finally, suppose our parabolic curve $\alpha \cin S_a$, after cutting into
$\alpha_i$, lifts to pieces $\widetilde{\alpha_i}$ which form a closed curve $\widetilde{\alpha}
\cin \widetilde{S}$.  Cut $\widetilde{\alpha}$ along only the gluings between different
$\widetilde{S_b}$ done in the final gluing step. Each piece $\widetilde{\alpha_j}$ now consists
of multiple $\widetilde{\alpha_i}$, which are either all "natural lifts" or all
"opposite lifts" (in the above sense). This is simply because each $\widetilde{S_b}$
is, except for its extra boundary components, a cover of the corresponding
$S_b$, and there's no way $\alpha$ to locally jump from being on the opposite
side of a page from $S_b$ to suddenly being on the same side. However, at the
gluings in the final step, we attach covers of different boundary components
together, so we may have attached multiple pieces $\widetilde{\alpha_j}$ of the same
type, or different types.

It is impossible to attach two natural segments $\widetilde{\alpha_j}$ in the final
step.  Suppose we had such an attachment, and look locally at the spine where
they're attached. Our attachment must yield a lift of $\alpha$, so we can look
at the local neighborhood in $\alpha$ covered by a neighborhood of our
attachment point.  Near the spine, this neighborhood must live in a single
local boundary component. But as we discussed in detail above, each local
boundary component simply consists of small neighborhoods on two page
boundaries, joined together by an annulus. Locally, a natural segment can only
be obtained by lifting to a cover of one of these two page boundaries, not any
of the other page boundaries near this spine. So one side of our attachment
must lie on one page boundary, and the other side on the other. But by
definition of how we do our final attaching step, we never attach in such a way
that we follow along the local boundary components! This is the point of having
valence at least 3 at each spine, and attaching $\gamma_i^+$ to
$\gamma_{i+2}^-$, is to avoid this problem. Since we never do attachments of
this form, we can't have an attachment with natural lifts on both sides.

So each attachment must have an opposite lift on one or both sides. But recall
that by (\dag''), each piece of each $\widetilde{S_b}$ has at most one free boundary
component. So all the $\widetilde{\alpha_j}$ must traverse at least two pieces, as we
ensured by construction of the $S'_{a,k}$ that they couldn't begin and end at the
same boundary component. This is where we finally use all those
carefully-established earlier criteria.

Because finally, as shown above, opposite lifts of individual $\alpha_i$ cannot
connect to opposite lifts on either side. So it's impossible to construct
a $\widetilde{\alpha_j}$ made out of opposite lifts, since it has to traverse
more than one piece.  This contradicts our assumption that the
$\widetilde{\alpha_j}$ glue to form a closed curve.

So we cannot join up the lifts of $\alpha_i$ to form a lift of $\alpha$.
Applying the parabolic lifting criterion shows that $\widetilde{S}$ is (QF).
This completes the proof of the theorem.

\end{proof}

\begin{thm}

Let $M$ be a good book of $I$-bundles.  Suppose $M$ does contain a component of
the parabolic locus $P$ inside a single page's boundary.  Construct $M'$ by
deleting all such pages from $M$, and $P'$ by removing all components of the
pared locus that intersect those pages. Then $(M,P)$ contains a surface
satisfying (QF) if and only if $(M',P')$ is nonempty and contains
a surface satisfying (QF).

\end{thm}

\begin{proof}

First observe that if $(M,P)$ contains a surface satisfying (QF), it cannot
traverse any of the pages we deleted to obtain $M'$, by the same argument as in
our first example.  Since our surface exists, $M'$ must be nonempty.
Furthermore, since $(M',P') \cin (M,P)$, it satisfies (QF) for $M',P'$ as well.
Conversely, if we have such a surface, we can obviously view it as contained in
$(M,P)$, where it must satisfy (QF) because it doesn't intersect the deleted
pages at all.

Note that $(M',P')$ may be nonelementary, or it still may not be good.
However, we can repeat the steps needed to guarantee that it's good, and obtain
a manifold where we can either apply this theorem again, or use the first
theorem. It is not immediately clear that this process terminates, because
taking the finite-sheeted cover we use for (2) and (3) makes the manifold more
complicated, possibly increasing the number of parabolics. We'll have to do
this carefully. Possibly we should do this step BEFORE the other
simplification, but then we'll have to make sure the parabolics pass through
that simplification nicely.

\end{proof}

\textbf{ The following sections ended up being unnecessary. I'll keep it here just
in case we need something. }

{\tiny

%%%%%%%%%%%%%%%%%%%%%%%%%%%%%%%%%%%%%%%%%%%%%%%%%
\section{Surface background details}
%%%%%%%%%%%%%%%%%%%%%%%%%%%%%%%%%%%%%%%%%%%%%%%%%

These are some background facts that we'll need to go through the examples in
detail.

Arcs in surfaces with boundary - from Fathi-Laudenback-Poenaru, Exposes 2 and
4 (see p. 21-24, 43-52).
This is an exposition of coordinates for classes of arcs in a surface
with boundary, aka Dehn-Thurston coordinates.

Let $A(N)$ be the set of isotopy classe sof simple closed proper arcs $I\in N$
such that they represent nontrivial elements of pi1 rel boundary. We isotope
them with the ends of the arcs free to move within the boundary components they
are contained in. Similarly $A'(N)$ with simple closed multi-arcs, up to
isotopy.

\begin{thm}[FLP 2.11]

Let $P^2$ be the standard pair of pants. $A(P^2)$ consists of exactly six
elements, classified by the boundary components of their endpoints.

\end{thm}

\begin{thm}[FLP 2.12]

$A'(P^2)$ is isomorphic to $A'={(a1,a2,a3) in Z | sum ai is even }$ via the map
$i:A'(P^2)->A', i(\tau) = (i(\tau, d_1),i(\tau,d_2),i(\tau,d_3))$ where
$d_1,d_2,d_3$ are the boundary components of $P^2$, and $i(\cdot,\cdot)$ is
geometric intersection number.

\end{thm}

That is, simple closed multicurves exist and are uniquely determined, for each
choice of boundary intersection numbers of the correct parity.

Finally, given a closed surface, let $N$, let $S(N)$ be the set of isotopy
classes of simple closed curves on $N$, and $S'(N)$ simple closed multicurves.

\begin{thm}[FLP 4.8]

Let $N$ be a closed surface of negative Euler characteristic.  Fix a pants
decomposition of $N$ into $2g-2$ pairs of pants along $3g-3$ disjoint simple
closed curves $K_1,...,K_{3g-3}$. Then

$S'(N)$ is isomorphic to $B_0 = {(m_i,s_i,t_i) \in \mathbb{Z}, i = 1,...,3g-3
\mid \text{ all coords }\geq 0, \text{ and for each pair of pants the
corresponding }m_i \text{ sum is even}}$ where the $m_i$ measure intersection
numbers with the curves $K1,...,K3g-3$, and the $s_i,t_i$ measure twisting
around at each intersection curve as a rational number.  See FLP for details of
the calculation of the twisting coordinates - we won't be needing it here (at
least, not yet!).

\end{thm}

We actually require a slight generalization of this result, but the proof is
identical.

\begin{thm}

Let $SA'(N)$ be the set of disjoint multi-curves or arcs on $N$, a surface
possibly with boundary. That is, each element of $SA'(N)$ corresponds to
a union of simple closed curves and simple arcs on $N$, all disjoint, such that
the arcs are properly embedded, up to isotopy where the arcs are permitted to
slide on each boundary component. Then $SA'(N)$ is isomorphic to the obvious
choice, where we again decompose into pairs of pants, have $mi$ for each
boundary curve or decomposition curve, but $s_i,t_i$ only for each
decomposition curve.

\end{thm}

Finally, let $A''(N)$ (or maybe some better notation, etc) be the set of simple
proper multi-arcs satisfying the condition that no two components of the
multi-arc are parallel, ie, isotopic. Alternatively we can think of $A''(N)$ as
consisting of simplices in the arc complex of $N$, or as a quotient of $A'(N)$
where we identify multicurves along splitting / joining of parallel arcs. In
general, $A''(N)$ is more difficult to parametrize, as purely from the
coordinates it is difficult to determine which multicurves or multiarcs will
contain parallel components.

We define $S''(N)$, $SA''(N)$ similarly.

\begin{example}

Consider the pair of pants $P^2$. We know $A(P^2)$ has 6 elements, so
$A''(P^2)$ has at most $2^6$ elements, corresponding to which arcs are present
in our multi-arc. It suffices to check that all 6 arcs can be embedded in $P^2$
without intersecting.

\end{example}

\begin{example}

Consider the punctured torus $\Si_{1,1}$. We first compute $SA'(\Si_{1,1})$.
By the generalized Dehn-Thurston coordinates theorem above, (work this out on
paper first).

\end{example}

%%%%%%%%%%%%%%%%%%%%%%%%%%%%%%%%%%%%%%%%%%%%%%%%%
\section{Old reduction}
%%%%%%%%%%%%%%%%%%%%%%%%%%%%%%%%%%%%%%%%%%%%%%%%%

We first restrict ourselves to considering only Kleinian groups which are
geometrically finite. In the geometrically infinite case, look at the simply
degenerate (geometrically infinite) ends of the group. By the Canary covering
theorem, tameness, and some other results, it's known that any finitely
generated subgroup of a Kleinian group must be either geometrically finite or
a virtual surface fiber - that is, it corresponds to a fiber surface in
a finite-sheeted cover which is fibered over the circle. See (AFW p117) for
this fact and various spots in the book for arguments.

% No virtual surface fiber subgroup can possibly be quasifuchsian,

Only geometrically finite subgroups can be quasifuchsian. This is basically by
definition (check this). The problematic case is when it's possible to have
a geometrically finite quasifuchsian surface subgroup of a geometrically
infinite Kleinian group. Looking at the simply degenerate ends, and comparing
to the ends of the quasifuchsian surface subgroup itself (viewed as a Kleinian
group), the relative ends must finite-to-one cover relative ends of the
original Kleinian group, by the Canary covering theorem. But now there's some
problem with different types of (relative) ends covering different types of
ends. I still don't understand this.

Also, note that quasifuchsian here is intended in the sense of Kahn-Markovic as
discussed in AFW p81. As discussed there (will get references), quasifuchsian
surface subgroups must be geometrically finite - which rules out the virtual
fiber case by the above dichotomy - and directly correspond to geometrically
finite subgroups that avoid the cusps.

One final consideration is to ensure that geometrically finite surface
subgroups cannot exhibit certain pathological behavior inside geometrically
infinite Kleinian groups. In the geometrically finite case this is easy,
because it'll necessarily also be geometrically finite, but in the
geometrically infinite case we need to either guarantee that the surface we
find by the below, purely topological construction is geometrically finite, or
show that actually no such surface exists and our construction is irrelevant.
Check with Ian (and references) about which of these we're actually doing!

Of course, so long as we are only studying books of $I$-bundles topologically,
this is irrelevant, as we can restrict ourselves to considering the
geometrically finite realizations of these hyperbolic 3-manifolds. However, it
is nice to be able to state the result more generally. Also if we want to
generalize to a more complete consideration of (finitely generated) Kleinian
groups at some point, not just books of $I$-bundles, it will be nice to say that
our work applies to more than just geometrically finite cases.

Anyway, I believe the geometrically infinite case is much simpler once
I understand what existing work it's based on. I'll need to talk to Ian about
this. In what follows, just assume our Kleinian group is geometrically finite.

% OLD REFERENCE - DOESN'T WORK BY ITSELF!
%%(Baker-Cooper Theorem 1.7 - "work of Bonahon and Thurston").
%\begin{thm}
%
%Let $M$ be a complete hyperbolic 3-manifold with finite volume, and $S$
%a closed oriented $\pi_1$-injective immersed surface of negative Euler
%characteristic. Then either $S$ is a virtual fiber, or $S$ is geometrically
%finite. In the second case, it's either quasifuchsian or some element of
%$\pi_1S$ is parabolic.
%
%\end{thm}
%

We assume this reduction still works in our cases (which are not finite
volume). I hope some reference can fix this!

So in our case, since the parabolic elements precisely correspond to the pared
structure, our problem becomes a purely topological one: can we find a closed
immersed $\pi_1$-injective surface of negative Euler characteristic, that
doesn't contain any of the pared locus, up to homotopy?

Since we're working in the case of books of $I$-bundles, we can reduce the
pared locus by observing that there are no torus boundary components of M, so
all the components of the pared locus must be annuli (as they're
$\pi_1$-injective). For each annulus, since we're only concerned with its
homotopy class, it suffices to consider the core curve.

%%%%%%%%%%%%%%%%%%%%%%%%%%%%%%%%%%%%%%%%%%%%%%%%%
\section{LERF and lifting properties}
%%%%%%%%%%%%%%%%%%%%%%%%%%%%%%%%%%%%%%%%%%%%%%%%%
\textbf{We had trouble with some technical details here, but we ended up not
needing these facts anyways.}

This is relevant to us because manifolds with LERF fundamental groups have nice
topological properties. In particular, we have the following fact.

\begin{thm}

Let $M$ be a compact $n$-manifold such that $\pi_1M$ is LERF. Let $M'$ to $M$
be a (possibly infinite-sheeted) cover such that $\pi_1M'$ is finitely
generated.  Suppose $C \cin M'$ is a compact subset which avoids the boundary
of $M$. Then there exists a intermediate cover $M' \to M'' \to M$, which is
finite-sheeted over $M$, such that the covering map $M' \to M''$ is an
embedding on $C$.

\end{thm}

\begin{proof}

See Long-Reid. Note that their proof is only stated for closed manifolds $M$,
but it carries over to all compact manifolds also if we restrict $C$ to avoid
the boundary of $M$.

% I think? Might want to check w Ian.

\end{proof}

This leads us to the following well-known fact. I haven't found a written proof
in the literature, so I've provided one. Thanks to Ian Agol for pointing this
out.

\begin{thm}

Let $M$ be a compact n-manifold such that $\pi_1M$ is LERF. Let $C$ be
a compact (n-1)-complex, and $\phi \colon C \looparrowright M$
a $\pi_1$-injective immersion.  Then there exists a finite sheeted cover $M'
\to M$ such that $\phi$ lifts to a map $\phi' \colon C \hookrightarrow M'$
which is homotopic to an embedding.

\end{thm}

Note that we can't guarantee that $\phi'$ is itself an embedding, because for
instance it's easy to map a small straight segment of $C$ so that its image
contains a small homotopicall trivial loop. This loop will be present in every
lift of $C$ to a cover of $M$.

\begin{proof}

Note that when we use this, both $C$ and $M$ will be aspherical, so an
alternative proof for that case is to invoke Whitehead's theorem to show
$\phi_H$ is a homotopy equivalence, followed by some obstruction theory facts
(??  - similar to the ones invoked in Long-Reid) to show it can be homotoped to
an embedding.

\textbf{ TODO is this a real proof? not sure...}

%Since C is a compact complex, pi1C is finitely generated. Let H=phi*(pi1C), and
%let pH colon MH to M be the corresponding cover of M. By our choice of H, phi
%lifts to a map phiH colon C to MH. We claim that phiH is homotopic to an
%embedding. Since C is compact and phi is an immersion, it suffices to show we
%can nicely homotope phiH to be injective.
%
%Since pH*(pi1MH) = H = phi*(pi1C), phiH must induce an isomorphism on pi1 in
%order to have pH circ phiH = phi. Since C is an immersed 1-complex, without
%loss of generality the only failures to be embedded we need to consider are
%double points.  Let x1,x2 in C such that phiH(x1)=phiH(x2)=y.  Let alpha cin
%C be a path from x1 to x2.  phiH(alpha) is a closed curve in MH.  If
%phiH(alpha) is homotopically trivial, we'll apply the loop theorem and homotope
%locally to remove the intersection at y.
%
%To be precise, we have two cases to consider. First, if phiH(alpha) is a simple
%closed curve, that is except for its two endpoints alpha embeds in MH. Then we
%can apply the loop theorem directly to obtain a disc D cin MH with boundary
%phiH(alpha). By homotoping across D we can ensure that phiH(C) cap int(D)
%= empyset.

% FIXME Ok HOLD ON this isn't right. Dimensions don't add up. If dim C = 1 and
% dim M = 3 it's trivial! Because just homotope into the extra dimension
% (general position) to fix -> in fact, I think this can be done generically.
%
% So really what I'm concerned with is (1) dim C = 1, dim M = 2. And also (2)
% dim C = 2, dim M = 3.
%
% I hope I don't have to consider these two cases separately...

\end{proof}

We use the above condition repeatedly to show that we can construct
a finite-sheeted cover of our book of $I$-bundles that satisfies certain nice
properties, by starting with an infinite-sheeted cover and then pushing down.

%%%%%%%%%%%%%%%%%%%%%%%%%%%%%%%%%%%%%%%%%%%%%%%%%
\section{Some generalizations}
%%%%%%%%%%%%%%%%%%%%%%%%%%%%%%%%%%%%%%%%%%%%%%%%%

We use the following general notation for books of $I$-bundles. Let
$M_{c1},...,M_{cm}$ be the spines, all solid tori. Let $M_1,...,M_n$ be the
pages, all thickened surfaces. Each $M_i$ has boundary components
$A_{i1},...,A_{ik_i}$ all annuli. We glue each annulus to a boundary annulus on
some spine. For each spine, all these gluing annuli must be disjoint parallel
incompressible. If $M$ is nonelementary, We can assume without loss of
generality that all pages have negative Euler characteristic, and all spines
have at least 3 pages glued to them (otherwise, just consolidate into fewer
pages / spines).

Given such a book of $I$-bundles $M$, let $G=(V,E)$ be the graph associated to
the embedded surface $A = \bigcup_{i,j} A_{ij} \cin M$ which is the union of
all the page gluing annuli. Combinatorially, $G$ has a vertex for each spine or
page, and an edge for each gluing of a page to a spine along an annulus. We
call $G$ the gluing graph of the book of $I$-bundles.

We say a book of $I$-bundles is \emph{tree-shaped} if its gluing graph is
a tree.

As above, if the pared locus $P$ contains any components that fit inside
a single $I$-bundle, then we can ``cross off that page.'' Just like in our
earlier lemma, any $\pi_1$-injective surface that passed through that page
would have would have to contain the component of the pared locus. By
repeatedly crossing off pages and consolidating into fewer pages / spines, we
can reduce any pared book of $I$-bundles to one with no such components.

\begin{conj}

Let $M$ be a tree-shaped book of $I$-bundles. Suppose $P$ has no components
$P_0\cin \bd_\pm M_i$ for any $i$. Then there exists a surface satisfying
\eqref{E:qf}.

\end{conj}

We're quite certain that this is true. It should just be a hairy inductive
adaptation of the above argument. In particular, it is totally unclear how one
might construct a counterexample. However, we haven't completed a full proof
yet.

\begin{thm}

Let $M$ be a tree-shaped book of $I$-bundles. Suppose $P$ has no components
$P_0 \cin \bd_\pm M_i$ for any $i$. Suppose further that every
component of $P$ crosses at least one gluing circle at a spine where at least
4 pages are attached. Then there exists a surface satisfying \eqref{E:qf}.

In particular, note that if all spines are at least valence 4, the technical
crossing condition must be satisfied.

\end{thm}
\begin{proof}

Our construction of $S$ is very simple. In fact, in this case, $S$ embeds in
our book of $I$-bundles. Construct $S$ by starting with a single page's core
surface, and traversing the tree-shaped gluing as follows. For each boundary
component of $S$, look at the valence of the associated spine. If it's valence
3, choose an arbitrary one of the other two pages. If it's valence at least 4,
choose a page that's non-boundary-adjacent to $S$, in that neither boundary
component of the incoming page attaches to a boundary component of the newly
chosen page when we glue the book of $I$-bundles. This is always possible for
valence at least 4, because at a spine, each incoming boundary component
connects to a single outgoing boundary component, and there are only 2 incoming
boundary components but at least 3 new pages to choose from. Attach the core
surface of the chosen page to $S$ with an annulus across the core. Since $M$ is
tree-shaped, we can build $S$ inductively without it running into itself.

We claim $S$ satisfies \eqref{E:qf}. $S$ is immediately properly immersed (in
fact, embedded) and $\pi_1$-injective, as it's a union of page cores and spine
annuli. It suffices to check the pared locus. Suppose some $P_k$ had $\pi_1S
\cap \pi_1P_k neq 1$, ie some multiple of its core curve was homotopic into
$S$. By the crossing condition, $P_k$ must traverse a gluing circle on a spine
of valence at least 4.  But at such a spine, $P_k$, which is contained in
a boundary component of $M$, must connect two boundary-adjacent pages. But we
chose $S$ so it would connect at such a spine to a non-boundary-adjacent page.
Since $M$ is tree-shaped, applying van Kampen shows that these pages correspond
to different pieces of the amalgamated free product for $\pi_1M$, so it's
impossble for this curve to be in $\pi_1S$. This completes the proof.

\end{proof}

Finally, we have one more technical intermediate result. We hoped this would
extend to more cases, but we haven't been able to do anything with it yet.

Let $M$ be a book of $I$-bundles. Associated to every surface we construct in
an analogous way to what we've been doing (taking covers of the page core
surfaces and gluing them together), there's a set of fairly complicated graphs
we can construct as follows. For each $(M_i,A_{ij},\widetilde{S_i})$, where
$M_i$ is a page, $A_{ij}$ gluing annulus of that page, and $\widetilde{S_i}$
finite-sheeted cover of the page core surface, we can draw a \emph{local
pared-arc graph} $G=G(M_i,A_{ij},\widetilde{S_i})$.

The vertices are boundary components of $\widetilde{S_i}$ which sit above
$A_{ij}$. Two vertices are connected by an edge if there is an arc in the pared
locus which lifts to connect those two boundary components. For instance,
a double cover of the punctured torus page, with 3 disjoint non-parallel arcs
on each boundary, has a graph, over the single boundary annulus downstairs,
with 2 vertices and 3 edges (one between the two vertices, and one loop at each
vertex). This graph is a generalization of the cis-trans terminology we used in
that example.

We say a cover $\widetilde{S_i}$ is \emph{pared-bipartite} if each of its local
pared-arc graphs is bipartite. Note that none of the covers we constructed in
our first example are pared-bipartite. This is a very specialized condition.
Note also that if the pared locus does not intersect a page, all covers of that
page are pared-bipartite.

\begin{thm}

Let $M$ be a tree-shaped book of $I$-bundles. Suppose that $P$ has no
components in $\bd_\pm M_i$ for any $i$. Suppose that every page of $M$
has a pared-bipartite cover of its core surface. Then $M$ contains a surface
satisfying \eqref{E:qf}.

\end{thm}
\begin{proof}

First observe that every cover of a pared-bipartite cover is itself
pared-bipartite. Simply lift the partition of the vertices to obtain a new
partition of each local pared-arc graph. The arcs cannot break this
partition because otherwise they wouldn't cover arcs in the downstairs cover.

Choose a sufficiently large cover of a pared-bipartite cover of each page such
that at each spine, the number of incoming cover boundary components above each
incident page equal for that spine. We can do this taking a ``least common
multiple cover,'' unless pages are glued to the same spine multiple times. In
that case, choose a finite-sheeted cover of the book of $I$-bundles itself that
doesn't have this issue, lift the pared locus, and construct a surface there.
Afterwards, we can push it down and it will still satisfy \eqref{E:qf}.

% careful! finite subgp separability covers of tree-shaped things aren't
% necessarily tree-shaped! might need another condition / argument here.

Now, we construct $S$ by gluing these large covers. For each $\widetilde{S_i}$
and downstairs gluing annulus $A_{ij}$ there will be two adjacent annuli in the
associated spine. For each $\widetilde{S_i}$, attach half the boundary
components above $A_{ij}$ to the boundary components coming from each of the
``neighboring'' $\widetilde{S_i}$, via annuli. Use the pared-bipartite
structure to split the set of boundary components above $A_{ij}$ in half. Since
the number of incoming boundary components match at each spine, as constructed
above, the boundary components will match up. We have a ``cover boundary
component bipartition'' above each $A_{ij}$. Note that once we have the
bipartition, the details of how we glue the pieces inside the bipartition are
irrelevant.

We claim the resulting $S$ satisfies \eqref{E:qf}. We already know it's proper
immersed $\pi_1$-injective by the same arguments. To show it doesn't overlap
with the pared locus, observe that by the definition of local pared-arc graphs,
every core curve in the pared structure must correspond to a sequence of arcs
in covers, where each arc either connects boundary components above different
annuli $A_{ij}$, $A_{ij'}$, or it connects 2 boundary components above the same
annulus $A_{ij}$. But it cannot connect two boundary components above the same
annulus, as we glued those according to the pared bipartition in such a way
that the pared arc component, which is locally restricted to a single boundary
component of the $I$-bundles, can't follow. This is a generalization of the
first argument in our simplest case.

Note that if the arc connects boundary components above different annuli, these
annuli must attach to different spines (by our earlier simplifying assumption).
But now because $M$ is tree-shaped, no arc that's been redirected toward
different spines can possibly form a closed loop. This completes the proof.

\end{proof}

We tried to do this proof in the non-tree-shaped case, but it doesn't work!
There are counterexamples of pared-bipartite covers where the obvious fix
relies on messy combinatorics inside the boundary component matching.


%%%%%%%%%%%%%%%%%%%%%%%%%%%%%%%%%%%%%%%%%%%%%%%%%
\section{Next steps}
%%%%%%%%%%%%%%%%%%%%%%%%%%%%%%%%%%%%%%%%%%%%%%%%%

$I$-bundles. In fact, combinatorially, it seems that all books of $I$-bundles
(satisfying an appropriate condition on components of $P$) should admit these
surfaces. But we haven't proved anything yet. The general case is much messier
(combinatorially) than the tree-shaped case, because in addition to cis arcs
we'll also need to consider trans arcs that ``wrap around'' a loop in the
gluing graph. So even ensuring that every closed curve in the pared structure
is broken into arcs, at least one of which is trans, is insufficient.

We chose to study books of $I$-bundles first. Ian thinks the other infinite
volume cases ought to be easier. What about acylindrical manifolds?  Books of
$I$-bundles are somehow the opposite end of the spectrum of acylindrical
- they're glued together along a bunch of cylinders, and the pieces are all
very simple ($I$-bundles).  If we can address the acylindrical and book of
$I$-bundle cases we ought to be able to put these together to solve the entire
problem. Ian thinks maybe the acylindrical case can be addressed by a variant
of the Baker-Cooper argument.

}%tiny

% TODO rewrite
\textbf{ TODO Rewrite next steps after seeing how far we get.}


\bibliographystyle{plain}
\bibliography{refs}

\end{document}
