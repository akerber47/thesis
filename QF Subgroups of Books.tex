\documentclass[12pt]{amsart}
\usepackage{amsmath,amscd,amssymb,amsthm,amsfonts}

\setlength{\topmargin}{0.5cm}
\setlength{\oddsidemargin}{-0.2cm}
\setlength{\evensidemargin}{-0.2cm}
\textheight = 22cm
\textwidth = 16.2cm

\newtheorem{theorem}{Theorem}[section]
\newtheorem{thm}[theorem]{Theorem}
\newtheorem{lemma}[theorem]{Lemma}
\newtheorem{prop}[theorem]{Proposition}
\newtheorem{cor}[theorem]{Corollary}

\theoremstyle{definition}
\newtheorem{Def}[theorem]{Definition}
\newtheorem{conj}[theorem]{Conjecture}

\theoremstyle{remark}
\newtheorem{example}[theorem]{Example}
\newtheorem*{rmk}{Remark}
\newtheorem*{claim}{Claim}
\newtheorem*{ntn}{Notation}

\newcommand{\x}{\times}
\newcommand{\bd}{\partial}
\newcommand{\Om}{\Omega}
\newcommand{\Si}{\Sigma}
\newcommand{\cin}{\subseteq}

\begin{document}

\title{Quasifuchsian surface subgroups of books of $I$-bundles}

\author{Alvin Kerber}
%\address{University of California, Berkeley\\
%970 Evans Hall \\
%Berkeley, CA, 94720}
    \email{alvin@math.berkeley.edu}
%\thanks{Agol supported by something or other}
%\date{%
%\today}

\begin{abstract}

Given a Kleinian group $\Gamma$, one can ask whether the group contains any
quasifuchsian surface subgroups. Equivalently, given a pared 3-manifold
$(M,P)$, one can ask whether there exists a closed immersed $\pi_1$-injective
surface in $M$ that avoids the peripheral subgroups associated to $P$.  This is
known to be true for closed hyperbolic 3-manifolds, and more generally for
finite volume hyperbolic 3-manifolds. We derive explicit results for pared
3-manifolds $(M,P)$ where $M$ is a book of $I$-bundles.

\end{abstract}

\maketitle
\section{Introduction}

Let $M$ be a complete hyperbolic 3-manifold. $M$ can be realized as the
quotient manifold of a Kleinian group $\Gamma$. One can ask whether $\pi_1M$
contains any quasifuchsian surface subgroups. To tackle this problem
topologically, we represent a Kleinian group with a pared 3-manifold $(M,P)$.
The pared structure $P\cin\bd M$ specifies the boundary locus corresponding to
cusps of the parabolic group elements.

Finding a quasifuchsian surface subgroup of the Kleinian group boils down to
asking whether we can find a closed immersed $\pi_1$-injective surface which
avoids this pared locus up to homotopy. That is, we're trying to choose
a surface subgroup of $\pi_1M$ which does not contain any of the peripherial
subgroups associated to components of $P$. The general question is: for which
choices of $(M,P)$ can we find such a surface subgroup?

There is some existing work on this problem. Cooper-Long-Reid \cite{CLR}
addressed the underlying topological problem, showing that a compact connected
irreducible 3-manifold with non-empty incompressible boundary  must contain an
essential closed surface, unless it's covered by a product $F\times I$.
However, this surface may contain accidental parabolics, that is, overlap with
the pared locus $P$. If this occurs it will not be quasifuchsian.

% XXX something is wrong with Masters-Zhang citation / bib entry - I can only
% find their older paper (hyp knot complements) on mathscinet, not the full new
% one with link complements

In the closed case, there is the work of Kahn-Markovic \cite{KM}. They proved
that there always exists an immersed closed $\pi_1$-injective surface in any
closed hyperbolic 3-manifold.  Such a surface is necessarily quasifuchsian
because the surface is closed, so the pared structure is empty. Finally,
Masters-Zhang \cite{MZ} showed that every finite volume hyperbolic 3-manifold
with cusps contains a quasifuchsian surface subgroup. An alternate proof was
provided by Baker-Cooper \cite{BC}.

The remaining case is infinite volume hyperbolic 3-manifolds. So far we've only
addressed a particular example of this case --- books of $I$-bundles. We
conjecture that all books of $I$-bundles admit quasifuchsian surface subgroups,
except for a few obvious negative cases where the pared locus is too large. We
prove this in a few special cases, so far.

% I'll figure out if I need the Fathi-Laudenbach-Poenaru stuff or not. Right now
% it looks like I don't, at least for the cases I've done so far. But it might
% come in handy later for more complicated pages? Not sure.

%\section{Surface background details}
%
%These are some background facts that we'll need to go through the examples in
%detail.
%
%Arcs in surfaces with boundary - from Fathi-Laudenback-Poenaru, Exposes 2 and
%4 (see p. 21-24, 43-52).
%This is an exposition of coordinates for classes of arcs in a surface
%with boundary, aka Dehn-Thurston coordinates.
%
%Let A(N) be the set of isotopy classe sof simple closed proper arcs I\in N such
%that they represent nontrivial elements of pi1 rel boundary. We isotope them
%with the ends of the arcs free to move within the boundary components they are
%contained in. Similarly A'(N) with simple closed multi-arcs, up to isotopy.
%
%Theorem (FLP 2.11). Let P2 be the standard pair of pants. A(P2) consists of
%exactly six elements, classified by the boundary components of their endpoints.
%
%Theorem (FLP 2.12). A'(P2) is isomorphic to
%A'={(a1,a2,a3) in Z | sum ai is even }
%via the map
%i:A'(P2)->A'
%i(tau) = (i(tau, d1),i(tau,d2),i(tau,d3)) where d1,d2,d3 are the boundary
%components of P2, and i(,) is geometric intersection number.
%
%That is, simple closed multicurves exist and are uniquely determined, for each
%choice of boundary intersection numbers of the correct parity.
%
%Finally, given a closed surface, let N, let S(N) be the set of isotopy classes
%of simple closed curves on N, and S'(N) simple closed multicurves.
%
%Theorem (FLP 4.8). Let N be a closed surface of negative Euler characteristic.
%Fix a pants decomposition of N into 2g-2 pairs of pants along 3g-3 disjoint
%simple closed curves K1,...,K3g-3. Then
%
%S'(N) is isomorphic to B0 = {(mi,si,ti) in Z, i = 1,...,3g-3 | all coords >= 0,
%and for each pair of pants the corresponding mi sum is even}
%where the mi measure intersection numbers with the curves K1,...,K3g-3, and the
%si,ti measure twisting around at each intersection curve as a rational number.
%See FLP for details of the calculation of the twisting coordinates - we won't
%be needing it here (at least, not yet!).
%
%We actually require a slight generalization of this result, but the proof is
%identical.
%
%Theorem. Let SA'(N) be the set of disjoint multi-curves or arcs on N, a surface
%possibly with boundary. That is, each element of SA'(N) corresponds to a union
%of simple closed curves and simple arcs on N, all disjoint, such that the arcs
%are properly embedded, up to isotopy where the arcs are permitted to slide on
%each boundary component. Then SA'(N) is isomorphic to the obvious choice, where
%we again decompose into pairs of pants, have mi for each boundary curve or
%decomposition curve, but si,ti only for each decomposition curve.
%
%Finally, let A''(N) (or maybe some better notation, etc) be the set of simple
%proper multi-arcs satisfying the condition that no two components of the
%multi-arc are parallel, ie, isotopic. Alternatively we can think of A''(N) as
%consisting of simplices in the arc complex of N, or as a quotient of A'(N)
%where we identify multicurves along splitting / joining of parallel arcs. In
%general, A''(N) is more difficult to parametrize, as purely from the
%coordinates it is difficult to determine which multicurves or multiarcs will
%contain parallel components.
%
%We define S''(N), SA''(N) similarly.
%
%Example. Consider the pair of pants P2. We know A(P2) has 6 elements, so
%A''(P2) has at most 2**6 elements, corresponding to which arcs are present in
%our multi-arc. It suffices to check that all 6 arcs can be
%embedded in P2 without intersecting.
%
%Example. Consider the punctured torus \Si_{1,}. We first compute SA'(\Si_{1,1}). By the
%generalized Dehn-Thurston coordinates theorem above, (work this out on paper
%first)

\section{Background}

Material in this section is from \cite{Mo} and \cite{CMc}. Look there for
further details.

%Definition and important facts about pared manifolds. After Morgan, The Smith
%Conjecture, V ("Uniformization Theorem for Three-Dimensional Manifolds"),
%p 58-60. Or Canary-McCullough, Homotopy Equivalences of 3-Manifolds and
%Deformation Theory of Kleinian Groups, Ch 5 p. 87-92. Also Ch 7 p.105-107.
%Note that Canary-McCullough is much more recent.
%
%(Morgan p58) (Canary-McCullough p87)
\begin{Def}

A \emph{pared manifold} $(M,P)$ is a compact orientable irreducible 3-manifold
$M$, together with $P\cin\bd M$, such that the following conditions hold:

\begin{enumerate}
\item Every component of $P$ is a torus or annulus, incompressible in $M$.

\item Every noncyclic abelian subgroup of $\pi_1M$ is peripheral with respect
to $P$ -- ie, conjugate to the fundamental group of a component of $P$.

\item $(M,P)$ is ``annulus-incompressible'': every $\pi_1$-injective map $(A^2,
\bd A^2) \to (M,P)$ is homotopic (as a map of pairs) to a map into $P$.

\end{enumerate}

We call $P$ the \emph{pared locus} or \emph{parabolic locus} of the pared
manifold $(M,P)$.

\end{Def}

Note that there are a few pared manifolds that are special cases, like with
elementary Kleinian groups. In fact, these are precisely the pared manifolds
that correspond to elementary Kleinian groups when we construct pared
3-manifolds from Kleinian groups below.

%(Canary-McCullough p88)
\begin{Def}

A pared manifold $(M,P)$ is \emph{elementary} if it is homeomorphic (as a pair)
to one of the following: $(T^2\x I,T^2\x 0)$, $(A^2\x I,A^2\x 0)$, or $(A^2\x
I,\emptyset)$.
%(or (Morgan only) (S3,empty)).

\end{Def}

%(Morgan p59, Canary-McCullough p88)
\begin{prop}

Let $(M,P)$ nonelementary. The following facts hold:

%In fact, Canary-McCullough makes these 3 statements. Morgan only has the 4th.
\begin{enumerate}
\item M is not a $T^2\x I$, $K^2$ $I$-bundle, or $A^2\x I$ (solid torus).
\item M does not contain an embedded $K^2$
\item For every component $Q$ of $P$, $\pi_1Q$ is a maximal abelian subgroup of
$\pi_1M$.
\item Every component of $\bd M-P$ has negative Euler characteristic.
% Yes this is correct. But the only extra info we get is that there's no
% parallel annuli. Canary-McCullough includes the statement:
% "Every toroidal component of dM is contained in P"
% but this is redundant with item 4 above (from Morgan)
\end{enumerate}

\end{prop}
\begin{proof}
See \cite{CMc} or \cite{Mo}.
\end{proof}

Pared manifolds arise naturally in the study of Kleinian groups. Given
a geometrically finite torsion-free Kleinian group $G$, we construct $(M,P)$ by
truncating small neighborhoods of the cusps of the quotient manifold $M(G)
= \left(\mathbb{H}^3\cup \Om(G)\right)/G$.  Then we let $M$ be the resulting
compact manifold with boundary, and $P$ the boundary locus along which we
truncated.  That is, we can rebuild the quotient manifold from $(M,P)$ by
gluing cusp neighborhoods onto each component of $P$. It follows from basic
Kleinian group theory that the resulting $(M,P)$ will satisfy conditions 1-3
above, making it a pared manifold.  The parabolic elements of $G$ precisely
correspond to (conjugacy classes of) the peripheral subgroups $\pi_1P$ in
$\pi_1M$.

Conversely, Thurston's famous uniformization theorem states that for every
pared manifold $(M,P)$ with $M$ Haken, there exists a finite volume (in
particular, geometrically finite) hyperbolic 3-manifold structure on the
interior satisfying the appropriate conditions for a pared 3-manifold! To be
precise:

%(see Morgan for details of this construction)

%(Morgan p60, but his statement is confusing / less modern definitions. He does
%state the more general theorem for Haken pared manifolds though.)
%(Canary-McCullough p105-106)

\begin{thm}

If $(M,P)$ is an oriented pared 3-manifold with nonempty boundary, then
there exists a geometrically finite uniformization of $(M,P)$, that is, a map
$\rho: \pi_1M \to \mbox{\rm{PSL}}_2\mathbb{C}$ such that $\rho(g)$ is parabolic
if $g \in \pi_1(P)$, and an orientation-preserving homeomorphism $M-P \to
\left(\mathbb{H}^3 \cup \Om(\rho(\pi_1M))\right)/\rho(\pi_1M)$.

\end{thm}
\begin{proof}

See \cite{CMc}.
Also see discussion on (CMc p106, bottom of page) for why this definition of
geometrically finite is equivalent to the standard one.

\end{proof}

% Example. Maybe? XXX TODO

\section{Our reduction}

In our result, we study the problem of finding quasifuchsian surface subgroups
of Kleinian groups from the topological perspective. We convert our given
Kleinian group to a pared 3-manifold, and look for surface subgroups of this
Kleinian group that avoid the peripheral pared structure. Finding
a quasifuchsian surface subgroup of a Kleinian group is reduced by the
following theorem:

%(Baker-Cooper Theorem 1.7 - "work of Bonahon and Thurston").
\begin{thm}

Let $M$ be a complete hyperbolic 3-manifold with finite volume, and $S$
a closed oriented $\pi_1$-injective immersed surface of negative Euler
characteristic. Then either $S$ is a virtual fiber, or $S$ is geometrically
finite. In the second case, it's either quasifuchsian or some element of
$\pi_1S$ is parabolic.

\end{thm}
% XXX I need a reference or explanation of why qf sfc subgp <=> closed immersed
% pi1-injective sfc IN MY CASE TOO

We assume this reduction still works in our cases (which are not finite
volume). I hope some reference can fix this! TODO

So in our case, since the parabolic elements precisely correspond to the pared
structure, our problem becomes a purely topological one: can we find a closed
immersed $\pi_1$-injective surface of negative Euler characteristic, that
doesn't contain any of the pared locus, up to homotopy?

Since we're working in the case of books of $I$-bundles, we can reduce the
pared locus by observing that there are no torus boundary components of M, so
all the components of the pared locus must be annuli (as they're
$\pi_1$-injective). For each annulus, since we're only concerned with its
homotopy class, it suffices to consider the core curve.

\section{Our first example}

We construct a book of $I$-bundles $M$ as follows. In the following we take
$\Si_{1,1}$ to be the compact surface of genus 1 with a single boundary
component, a circle. Let $M_1$,$M_2$,$M_3$ be 3 punctured torus $I$-bundles,
$M_1=M_2=M_3=\Si_{1,1}\x I$. For each $M_i$, write \[ \bd M_i = \bd \Si_{1,1}\x
I \cup \Si_{1,1}\x0 \cup \Si_{1,1} \x 1 \] and label these boundary pieces as
\[ \bd M_i = A_i \cup \bd_- M_i \cup \bd_+M_i \].

Let $M_c = S^1xD^2$ be a solid torus. Attach the $M_i$ to $M_c$ by gluing the
$A_i$ to parallel annuli in $\bd M_c$, each with longitudinal core curve. The
result is a compact 3-manifold with boundary. This is our $M$. We choose
orientations and a cyclic order for the gluing such that the 3 boundary
components of M are precisely
\[
\bd M = (\bd_+M_1 \cup_{S^1} \bd_-M_2) \sqcup (\bd_+M_2 \cup_{S^1} \bd_-M_3)
\sqcup (\bd_+M_3 \cup_{S^1} \bd_-M_1)
\]
and label these
\[
\bd M=\bd_{12}M \sqcup \bd_{23}M \sqcup \bd_{31}M
\].
Each boundary
component of $M$ consists of two punctured tori glued along their boundary
circles, topologically a genus two surface.

We consider possible pared structures $P$ on $M$. Our goal is to find
a surface $S \cin M$ such that
\begin{equation}\label{E:qf}
S \text{ is closed immersed $\pi_1$-injective, and  $\pi_1P_k \cap \pi_1S
= 1$
for each component $P_k$ of $P$} \tag{\textasteriskcentered}
\end{equation}

% XXX straighten this out
%Note that since we
%haven't fixed a basepoint, these subgroups are really only defined up to
%conjugacy - what we're saying is they fail to intersect for an arbitrary choice
%of conjugacy class for each subgroup. (ask Ian about this? seems fuzzy)

\begin{lemma}\label{L:sc}

Let $S$ be a surface satisfying \eqref{E:qf}. Homotope $S$ to have minimal
intersection with each $A_i$, that is, so there are no ``bumps''.  Then for
each $M_i$, each component of $S \cap M_i$ is a finite-sheeted covering of the
core surface.  That is, given such a component $S' \cin M_i$, the map $S' \to
M_i = \Si_{1,1}\x I \to \Si_{1,1}\x{1/2}$ is homotopic to a finite-sheeted
covering map. Conversely, given any finite-sheeted covering $\widetilde{S} \to
\Si_{1,1}$, there exists a corresponding proper immersed $\pi_1$-injective
surface $S \cin \Si_{1,1}\x I$.

\end{lemma}
\begin{proof}

$S$ is compact, so every such component $S'$ is a compact surface with
boundary, properly immersed in $M_i$. Since $S$ is $\pi_1$-injective in $M$,
$S'$ must be $\pi_1$-injective in $M_i$. Otherwise, we'd have a nontrivial
element of $\pi_1S'$ which is trivial in $\pi_1M_i$, hence in $\pi_1M$, hence
in $\pi_1S$, contradicting the minimal position homotopy above. So the map
$\phi : S'\to\Si_{1,1}\x{1,2}$ is also $\pi_1$-injective, since $M_i$
deformation retracts to its core. Let $H = \phi_*(\pi_1S')$, and let
$\Si_{1,1}^H$ be the cover of $\Si_{1,1}$ associated to $H<\pi_1\Si_{1,1}$.
$\phi$ lifts to $\widetilde{\phi}\colon S'\to \Si_{1,1}^H$. This is a proper
map of compact surfaces which is an isomorphism on $\pi_1$.  By the
classification of surfaces, it must be homotopic (as a proper map) to
a homeomorphism. So $\phi$ is homotopic to a covering map. It must be
finite-sheeted as $S'$ is compact (by classification of surfaces again).

Conversely, given a finite-sheeted cover $\widetilde{S}\to \Si_{1,1}$, compose
with the embedding $\Si_{1,1} = \Si_{1,1}\x{1/2} \cin \Si_{1,1}\x I$. This is
proper and $\pi_1$-injective.  Perturb locally to obtain an immersion.

\end{proof}

\begin{lemma}

Let $P$ be a pared structure on $M$. $P$ cannot contain any tori. That is,
$P$ consists entirely of annuli.

\end{lemma}
\begin{proof}

We can compute $\pi_1M$ directly from the van Kampen theorem. Let $a_i,b_i$ be
the generators of $\pi_1(M_i)$, and $c$ be the generator of $\pi_1(M_c)$. Our
gluing yields
\[
\pi_1(M) = \langle a_1,b_1,a_2,b_2,a_3,b_3,c \mid
[a_1,b_1]=[a_2,b_2]=[a_3,b_3]=c \rangle
\]
It immediately follows that $\pi_1M$ contains no rank 2 abelian subgroups, as
no two elements commute. Since every component of $P$ is $\pi_1$-injective,
they must all be annuli.

Alternatively, simply observe that all the boundary components of $M$ have
negative Euler characteristic, so they don't admit $\pi_1$-injective maps from
a torus.

\end{proof}

\begin{prop}

Suppose $P$ contains a component $P_0$ that, up to homotopy, is contained
entirely within one ``$I$-bundle half'' of a boundary component. That is, $P_0
\cin \bd M_i$, for some i. Without loss of generality we can let $P_0 \cin \bd
M_1$.  Then there exists a surface satisfying \eqref{E:qf} if and only if $P$
contains no components that are (up to homotopy) contained in $\bd M_2 \cup \bd
M_3 = \bd_-M_2 \sqcup \bd_{23}M \sqcup \bd_+M_3$.

\end{prop}
\begin{proof}

($\Longleftarrow$) Take $S=\bd_{23}M$. Intuitively, it's easy to see that the
conditions on $P$ force all its components to lie on surfaces where they can't
be homotoped into $S$.

Algebraically, $\pi_1S$ is the subgroup generated by $a_2,b_2,a_3,b_3 \in
\pi_1M$. We know that
\[
\pi_1S = \langle a_2,b_2,a_3,b_3 \mid [a_2,b_2]=[a_3,b_3]\rangle.
\]
$P$ contains no components in $\bd_{23}M$, so an arbitrary component $P_k$ of
$P$ must be contained in $\bd_{12}M$ or $\bd_{31}M$, which have fundamental
groups generated by $a_1,b_1,a_2,b_2$ and $a_3,b_3,a_1,b_1$ respectively.
$\pi_1P_k$ is cyclic, so let $g$ be its generator. No matter if it's contained
in $\bd_{12}M$ or $\bd_{31}M$, we can see that in order for $\pi_1P_k$ to
overlap with $\pi_1S$, it must have generator some word in $a_2,b_2$ (if $P_k
in \bd_{12}M$), or some word in $a_3,b_3$ (if $P_k in \bd_{31}M$). In either
case, this is precisely the condition for such a word to correspond to a curve
contained in $\bd M_2$ or $\bd M_3$, respectively, contradicting our assumption
on $P$.

($\Longrightarrow$) Let $S$ be a surface satisfying \eqref{E:qf}. Cut $S$ into
components in $M_1,M_2,M_3$ (after homotoping to minimal intersections with the
annuli).  Applying Lemma~\ref{L:sc}, we can see that $S \cap M_1 = \emptyset$,
as otherwise since it's a finite-sheeted cover it would have to contain
a multiple of $P_0$.  So $S \cin M_2 \cup M_3 \cup M_c$, which is homeomorphic
to $S_2 \x I$.  Deformation retracting this to a surface $S_2$ and applying
a covering argument like in Lemma~\ref{L:sc}, we can see that $S$ is a cover of
$S_2$.  Since it's a finite-sheeted cover, it will have to have $\pi_1$
intersecting any component $P_k$ that violates the conditions stated above.
This completes the proof.

\end{proof}

The proof above obviously generalizes to more complex books of $I$-bundles.
Later we'll use it to simplify any pared structures containing annuli that fit
in a single $I$-bundle.

\begin{thm}\label{T:ex1}

Let $P$ be a pared structure containing no components $P_0$ as in the above
proposition. Then there exists a surface satisfying \eqref{E:qf}.

\end{thm}

This is the main theorem we prove for this example. A few preliminary facts are
required. Note that we can first isotope P such that it's in minimal position
on each boundary component with respect to the gluing circle.

\begin{lemma}

Under the hypotheses of the theorem, each ``$I$-bundle piece'' boundary
component $S=\bd_\pm M_i$ intersects $P$ in a thickened set of disjoint
essential arcs, each arc connecting $\bd S$ to itself. The arcs form at most
3 ``bands'', where all the arcs in each band are parallel.  Furthermore, if we
choose a representative arc from each band (yielding a set of at most
3 disjoint non-parallel arcs in $\Si_{1,1}$), there exists an automorphism of
$\Si_{1,1}$ taking these arcs to a standard set of 3 disjoint non-parallel
arcs.  See diagram for an illustration of the standard set.

\end{lemma}
\begin{proof}

We first need a preliminary definition. After cutting a surface with boundary
along arcs, we'll obtain one or more connected surfaces, each with one or more
boundary components. Each boundary component after cutting will have pieces
from the original boundary as well as pieces from the arcs that we cut along.
Given labels $\gamma_1,...,\gamma_k$ for the boundary components and
$\alpha_1,...,\alpha_l$ for the arcs (on the original surface with boundary),
we can describe each boundary component of the cut surfaces as a union of arcs,
each labeled with $\gamma_1,...,\gamma_k,\alpha_1,...,\alpha_l$. We call this
an \emph{arc pattern} for that boundary component of the new surface.

We first show that $\Si_{1,1}$ admitts at most 3 disjoint essential
non-parallel arcs, and the possible surfaces and arc patterns obtained by
cutting along these arcs are very restricted.

Label the boundary component of $\Si_{1,1}$ by $\gamma$. Consider a proper
essential arc $\alpha_1 \cin \Si_{1,1}$. Fix an orientation for $\alpha_1$.
Since $\Si_{1,1}$ is orientable, we can look at local neighborhoods of
$\alpha_1$ and see that $\alpha_1$ has a well-defined ``left side'' and ``right
side'' as we travel along it. Looking at the endpoints of $\alpha_1$ along
$\gamma$, we are forced to connect certain endpoints in the cut-up $\gamma$
with $\alpha_1$, in order to preserve the parity. This tells us the (possibly
disconnected) cut surface $S_1$ will have two boundary components. Each will
have an arc pattern consisting of two arcs, one labeled $\gamma$ and one
labeled $\alpha_1$.

Since we cut along a properly embedded arc, the Euler characteristic increases
by one. $\chi(S_1)=0$ and $S_1$ has two boundary components. By classification
of surfaces, $S_1 = \Si_{0,2}$ or $\Si_{0,1} \sqcup \Si_{1,1}$. But if $S_1$
contained a disk with the arc pattern described above, embedding that disk back
in $\Si_{1,1}$ would describe a homotopy of $\alpha_1$ into the boundary. So
$S_1 = \Si_{0,2}$.

Now suppose we had a second proper essential arc $\alpha_2 \cin \Si_{1,1}$,
disjoint from and non-parallel to $\alpha_1$. Since it's disjoint from
$\alpha_1$, $\alpha_2$ induces a proper arc in $S_1$ which connects two regions
in the arc pattern labeled $\gamma$.  $\alpha_2$ must have one endpoint on each
boundary component of $S_1$. If both are on the same side, it's either
homotopic to the boundary of $\Si_{1,1}$ or parallel to $\alpha_1$. Cutting
along $\alpha_2$ yields a new surface $S_2$. Topologically $S_2=D^2$, with arc
pattern consisting of 8 components in the cyclic order
$(\gamma,\alpha_1,\gamma,\alpha_2,\gamma,\alpha_1,\gamma,\alpha_2)$.

Finally, adding our 3rd proper essential arc $\alpha_3$, disjoint and
non-parallel to the first two arcs, a similar argument shows that $\alpha_3$
must connect opposite $\gamma$ pieces in the arc pattern. Cutting along
$\alpha_3$ yields two disks with the same arc pattern. Depending on the choice
of $\alpha_3$, the arc pattern on these disks is either
$(\gamma,\alpha_1,\gamma,\alpha_2,\gamma,\alpha_3)$ or
$(\gamma,\alpha_1,\gamma,\alpha_3,\gamma,\alpha_2)$. So up to relabeling
$\alpha_1,\alpha_2,\alpha_3$, cutting along 3 proper essential disjoint
non-parallel arcs has only one possible choice of cut surfaces and arc
patterns.

Observe that it is not possible to add any more disjoint non-parallel arcs. In
particular, any arc we draw between $\gamma$ components of the arc pattern on
either disk is homotopic to the boundary or parallel to an existing arc.
Furthermore, if we add new disjoint arcs and allow them to be parallel, we can
see that they must form ``bands'' around the existing 3 arcs in order to remain
disjoint. That is, we can homotope all the arcs parallel to a given arc into
a small neighborhood of that arc in the disk, without intersecting any of the
non-parallel arcs.

We claim there exists an automorphism of $\Si_{1,1}$ taking any set of 3 such
arcs to any other set (in particular, to the standard set, as illustrated).
Since there is only one topological result of cutting along the arcs, choose
a homeomorphism of the cut surfaces. Up to relabeling the arcs, we can choose
a homeomorphism that identifies matching arcs in the arc patterns (as shown
above, there is only one possible arc pattern up to relabeling). Glue both
sides along the arcs to obtain the desired automorphism of $\Si_{1,1}$.

We now consider $P \cap \bd_S \cin S \cong \Si_{1,1}$. As shown, $P$ is a union
of annuli. By the assumptions of the theorem, no annulus of $P$ is contained in
$S$, so each annulus intersects $S$ in a union of rectangles, where two sides
of the rectangle are embedded in the boundary $\bd S$.  Since we homotoped $P$
to have minimal intersections, all the core arcs of the rectangles (pieces of
the core curve of the annulus) must be essential. They are disjoint by
definition of $P$. The statement of the lemma follows from applying the above
argument to these core arcs.

\end{proof}

We consider connected double covers $\widetilde{S} \to S \cong \Si_{1,1}$.
These covers have 2 boundary components. Every core arc of $P \cap S$ lifts to
2 arcs in $\widetilde{S}$.  We say an arc in $S$ is \emph{cis} for a given
$\widetilde{S}$ if any (ie all) lifts of that arc have both endpoints in the
same boundary component of $\widetilde{S}$.  Otherwise, we say it's
\emph{trans} for that cover.

\begin{lemma}

Given any 3 disjoint non-parallel proper arcs in $\Si_{1,1}$, and any double
cover of $\Si_{1,1}$, 2 of the 3 arcs are trans, and the 3rd is cis.
Furthermore, we can choose any two of the three we wish to be trans with an
appropriate cover.

\end{lemma}
\begin{proof}

As in earlier lemma, there exists an automorphism of $\Si_{1,1}$ taking these
arcs to the standard set of 3 disjoint non-parallel proper arcs.  Now it
suffices to observe, by looking at relative first homology or just by
construction, that each of the three standard connected double covers
(corresponding to nontrivial maps $\mathbb{Z}^2 \to \mathbb{Z}/2$) makes two of
the three arcs trans and the third cis.

\end{proof}

\begin{proof}[Proof of Theorem~\ref{T:ex1}]

Since $P$ has no components $P_0$, every $P \cap \bd_+M_i$ is a union of
thickened arcs (that is, there are no full annulus components in any $\bd_+M_i$
or $\bd_-M_i$). Apply the first lemma to break these into bands. The problem is
most constrained when there are 3 bands, so we'll consider that case (if there
are fewer than 3, just draw some more arcs on that component arbitrarily, and
the proof still works).

We will build our surface $S$ by taking a double cover of the core $\Si_{1,1}$
surface for each $I$-bundle page $M_1,M_2,M_3$. Call these covers
$S_1,S_2,S_3$.  Each of these has two boundary components. We'll then attach
the boundary components such that each of $S_1,S_2,S_3$ has exactly one
boundary component connected to each of the others. See diagram.

Choose an arbitrary connected double cover for $S_1$. We can view this as
a cover of $\bd_+M_1$ or $\bd_-M_1$, deformation retracting either way. By the
lemma, two of the
3 bands on $\bd_+M_1$ are trans, and the other is cis. The same holds for
  $\bd_-M_1$.

We cannot choose $S_2$ arbitrarily, as a cis arc for $S_1$ in $\bd_+M_1$ may
connect (in $S$) to a cis arc for $S_2$ in $\bd_-M_2$. These together would
form a closed curve that lifts to $S$, which once $S$ is immersed in $M$ will
yield a violation of condition \eqref{E:qf}. Instead, observe that there is at
most one band of the three in $\bd_-M_2$ containing arcs that, when glued
across the core circle, match up to arcs in the cis band of $\bd_+M_1$ to form
closed curves containing only arcs in those two bands. This is because once one
band has that behavior, by endpoint parity the vertices can't also match up for
a different band, if the arcs they have to match with on the other side are
parallel. Looking at the boundary circle, non-parallel arcs have endpoints in
cyclic order, but parallel arcs don't. See diagram. It suffices to check this
for our standard set of non-parallel arcs, by the same automorphism argument.

Since there is only one such band on $\bd_-M_2$, choose $S_2$ such that this
band is not cis.  Finally, for $S_3$ we have two connecting constraints. We
have a cis band on $\bd_-M_1$ from our choice of $S_1$, and a cis band on
$\bd_+M_2$ from our choice of $S_2$.  Applying the same argument, there is at
most one band on $\bd_-M_3$ and one band on $\bd_+M_3$ that will connect to
form closed curves. But we can choose $S_3$ such that both of these bands are
trans. (This requires a slight modification to the lemma - since these bands
are on different boundary surfaces, they may not be disjoint non-parallel. If
they aren't disjoint, we can tweak them by local modifications so they are, and
then ``untweak'' them in the cover. If they are parallel, just ignore one set
of bands)

Glue $S_1,S_2,S_3$ as described to obtain $S$. We claim that $S$ satisfies
condition \eqref{E:qf}. $S$ is closed and immersed by construction (as in
Lemma~\ref{L:sc}). It is $\pi_1$-injective inside each $I$-bundle by
Lemma~\ref{L:sc}, and inside the core $M_c$ because it's just a union of
incompressible gluing annuli there. It suffices to show that $\pi_1S \cap
\pi_1P_k = 1$ for each component $P_k$ of the pared locus. Looking at the core
curve of the annulus $P_k$, this implies that some multiple of that core curve
is homotopic into $S$.

But this is impossibly by the above construction. Every such curve contains at
least two bands, on two different components $\bd_\pm M_i$. By the construction
of $S$, of any two bands which connect up when the boundary components are
glued across circles, at least one must be trans. This means that when we try
to homotope the core curve multiple into $S$, in order to follow along $S$
locally (in each page, where $S$ is locally a cover of the core surface, it
must be a lift from that core surface) it would have to traverse between all
three pieces of the cover, as that's how we connected up the $Si$. Every trans
arc lifts to an arc that connects two different boundary components of an $Si$,
which are ``pointed in different directions.'' But this is obviously
impossible, as our $P_k$ is restricted to be in a single boundary component, so
up to homotopy it must be generated by those two $I$-bundle pages only.

\end{proof}

\section{General case - preliminaries}

We prove a generalization of the above example. Basically, we want to first
identify the cases that obviously can't have a surface satisfying (*). Then
we'll prove by construction that every other book of I-bundles does in fact
contain such a surface.

We use the following general notation for books of $I$-bundles. Let
$M_{c1},...,M_{cm}$ be the spines, all solid tori. Let $M_1,...,M_n$ be the
pages, all thickened surfaces. Each $M_i$ has boundary components
$A_{i1},...,A_{ik_i}$ all annuli. We glue each annulus by a homeomorphism phiij
to a boundary annulus Aij' some dMck. Note that Aij' might not be a simple
longitude of, that is, a page can wrap multiple times around a spine. For each
spine, we require all these Aij' to be disjoint parallel incompressible.  If
$M$ is a nonelementary Kleinian manifold with incompressible boundary, we can
assume without loss of generality that all pages have negative Euler
characteristic, and all spines have at least 3 pages glued to them.

To see this, observe that each page which is a copy of S0,1 means that that
page and its attaching spine form a 3-manifold with a finite-sheeted cover by
a ball (we can arrange things so that in the cover, the S0,1 attaching map only
traverses the longitude once, and we get a thickened disk). This means that
we'll have finite order summands in our Kleinian group, which is impossible as
it's torsion free. Or some argument with JSJ decomposition having elliptic
piece? Either way, this can't happen with the manifolds we care about.

If we have a spine with only a single page attached, we can pass to
a finite-sheeted cover and obtain a boundary compression, so this case can't
occur (?). Begin with a meridian disk of the spine.  This intersects the page's
attaching curve in a single point, since we passed to an appropriate cover. Now
... something. % TODO figure this out.

Finally, for any spine with 2 pages attached, we can simplify the book of
I-bundles structure by attaching the 2 pages along their boundary annuli to
make a single larger page. This is homeomorphic to the original manifold.
Similarly, for any S0,2 page, we can glue the two spine boundaries together
along that page's gluing curves to make a single spine. The other attached
pages of each spine will be carried along by the gluing to form a single set of
attached pages which is the union of the two original sets, minus the S0,2
page.

We also want to clarify the possible gluing annuli on the spines. Given a solid
torus boundary curve (l,m) that travels m times around a meridian and l times
around a longitude, we can ask whether it is permissible to glue a page along
an annulus parallel to this curve. Obviously if we glue our pages along
meridian curves (0,1), we'll have a boundary compression along a parallel
meridian disk.

% I believe all other possibilities are ok, as long as we don't violate the
% one-point intersection constraint Ian mentioned.

Lemma ("One-point intersection lemma").

Consider a solid torus, or more generally a handlebody, with a collection of
I-bundles glued to it along annuli. The resulting manifold has a boundary
compression iff there exists a meridian disk that intersects the boundary
gluing curves exactly once.

Proof.

Not immediately obvious. Use some cut and paste topology? % TODO

Anyway, now we have a general book of I-bundles. Observe first that the pared
locus P must be a union of annuli, just as in the specific example above.  Now
there are a number of topological simplifications we make, by passing to an
appropriate finite-sheeted cover of the book of I-bundles.  Once we can
construct a surface satisfying (*) inside this cover, we can push it down and
perturb slightly to obtain a surface satisfying (*) downstairs (since it
pi1-injects into a subgroup, it will definitely still be pi1-injective).

We pass to a book of I-bundles satisfying the following conditions:

(0) M has incompressible boundary, every page core surface has negative Euler
characteristic, and every spine has at least 3 gluing annuli.

(1) Each page is glued to a given spine at most once.

(2) Each gluing annulus on a spine is parallel to a simple longitude.

Lemma.

Let M be a nonelementary Kleinian manifold which is a book of I-bundles with
incompressible boundary. Then M has a finite-sheeted cover satisfying (1) and
(2).

Proof.

We use subgroup separability. Look at Long-Reid, fill in details here.

\section{Main theorem}

Suppose we have a book of I-bundles satisfying (1) and (2).

Theorem.

Let M be a book of I-bundles Kleinian manifold with incompressible boundary
satisfying (0), (1) and (2). Suppose further that no components of the
parabolic locus P are contained in a single page's boundary. Then M contains
a surface satisfying (*).

Proof.

This is the main proof, and is quite involved. We proceed somewhat similarly to
the first example. We take covers over the pages and glue them together
cleverly to construct our surface. We check that the parabolic locus arcs on
each page lift to arcs on these covers that do not connect up to form a closed
curve. As in the first example, this suffices to prove that our surface
satisfies (*). We first need to take the correct cover over each page.

Lemma.

Let S be a surface with boundary, and P a union of disjoint proper arcs on S.
Then there exists a finite-sheeted cover S' of S such that the preimage pi-1(P)
consists of arcs each of which connects 2 different boundary components of S'.

Proof.

Subgroup separability again. The eyeglasses trick. Look up and fill in.


Fix some large integer N>0.

Theorem.

Let M be a book of I-bundles Kleinian manifold with incompressible boundary
satisfying (0), (1) and (2). Suppose M does contain a component of the
parabolic locus P inside a single page's boundary. Construct M' by deleting all
such pages from M, and P' by removing all components of the pared locus that
intersect those pages. Then (M,P) contains a surface satisfying (*) if and only
if (M',P') is nonempty and contains a surface satisfying (*).

Proof.

First observe that if (M,P) contains a surface satisfying (*), it cannot
traverse any of the pages we deleted to obtain M', by the same argument as in
our first example.  Since our surface exists, M' must be nonempty.
Furthermore, since (M',P') cin (M,P), it satisfies (*) for M',P' as well.
Conversely, if we have such a surface, we can obviously view it as contained in
(M,P), where it must satisfy (*) because it doesn't intersect the deleted pages
at all.

Note that (M',P') may be nonelementary, or it still may not satisfy (0), (1),
and (2).  However, we can repeat the steps needed to guarantee (0), (1), and
(2), and obtain a manifold where we can either apply this theorem again, or use
the first theorem. It is not immediately clear that this process terminates,
because taking the finite-sheeted cover we use for (1) and (2) makes the
manifold more complicated, possibly increasing the number of parabolics. We'll
have to do this carefully. Possibly we should do this step BEFORE the other
simplification, but then we'll have to make sure the parabolics pass through
that simplification nicely.

\section{Some generalizations - SCRUBBED}

The following section is obviated by the general case above. I'll keep it here
just in case we need something from it.

We use the following general notation for books of $I$-bundles. Let
$M_{c1},...,M_{cm}$ be the spines, all solid tori. Let $M_1,...,M_n$ be the
pages, all thickened surfaces. Each $M_i$ has boundary components
$A_{i1},...,A_{ik_i}$ all annuli. We glue each annulus to a boundary annulus on
some spine. For each spine, all these gluing annuli must be disjoint parallel
incompressible. If $M$ is nonelementary, We can assume without loss of
generality that all pages have negative Euler characteristic, and all spines
have at least 3 pages glued to them (otherwise, just consolidate into fewer
pages / spines).

Given such a book of $I$-bundles $M$, let $G=(V,E)$ be the graph associated to
the embedded surface $A = \bigcup_{i,j} A_{ij} \cin M$ which is the union of
all the page gluing annuli. Combinatorially, $G$ has a vertex for each spine or
page, and an edge for each gluing of a page to a spine along an annulus. We
call $G$ the gluing graph of the book of $I$-bundles.

We say a book of $I$-bundles is \emph{tree-shaped} if its gluing graph is
a tree.

As above, if the pared locus $P$ contains any components that fit inside
a single $I$-bundle, then we can ``cross off that page.'' Just like in our
earlier lemma, any $\pi_1$-injective surface that passed through that page
would have would have to contain the component of the pared locus. By
repeatedly crossing off pages and consolidating into fewer pages / spines, we
can reduce any pared book of $I$-bundles to one with no such components.

\begin{conj}

Let $M$ be a tree-shaped book of $I$-bundles. Suppose $P$ has no components
$P_0\cin \bd_\pm M_i$ for any $i$. Then there exists a surface satisfying
\eqref{E:qf}.

\end{conj}

We're quite certain that this is true. It should just be a hairy inductive
adaptation of the above argument. In particular, it is totally unclear how one
might construct a counterexample. However, we haven't completed a full proof
yet.

\begin{thm}

Let $M$ be a tree-shaped book of $I$-bundles. Suppose $P$ has no components
$P_0 \cin \bd_\pm M_i$ for any $i$. Suppose further that every
component of $P$ crosses at least one gluing circle at a spine where at least
4 pages are attached. Then there exists a surface satisfying \eqref{E:qf}.

In particular, note that if all spines are at least valence 4, the technical
crossing condition must be satisfied.

\end{thm}
\begin{proof}

Our construction of $S$ is very simple. In fact, in this case, $S$ embeds in
our book of $I$-bundles. Construct $S$ by starting with a single page's core
surface, and traversing the tree-shaped gluing as follows. For each boundary
component of $S$, look at the valence of the associated spine. If it's valence
3, choose an arbitrary one of the other two pages. If it's valence at least 4,
choose a page that's non-boundary-adjacent to $S$, in that neither boundary
component of the incoming page attaches to a boundary component of the newly
chosen page when we glue the book of $I$-bundles. This is always possible for
valence at least 4, because at a spine, each incoming boundary component
connects to a single outgoing boundary component, and there are only 2 incoming
boundary components but at least 3 new pages to choose from. Attach the core
surface of the chosen page to $S$ with an annulus across the core. Since $M$ is
tree-shaped, we can build $S$ inductively without it running into itself.

We claim $S$ satisfies \eqref{E:qf}. $S$ is immediately properly immersed (in
fact, embedded) and $\pi_1$-injective, as it's a union of page cores and spine
annuli. It suffices to check the pared locus. Suppose some $P_k$ had $\pi_1S
\cap \pi_1P_k neq 1$, ie some multiple of its core curve was homotopic into
$S$. By the crossing condition, $P_k$ must traverse a gluing circle on a spine
of valence at least 4.  But at such a spine, $P_k$, which is contained in
a boundary component of $M$, must connect two boundary-adjacent pages. But we
chose $S$ so it would connect at such a spine to a non-boundary-adjacent page.
Since $M$ is tree-shaped, applying van Kampen shows that these pages correspond
to different pieces of the amalgamated free product for $\pi_1M$, so it's
impossble for this curve to be in $\pi_1S$. This completes the proof.

\end{proof}

Finally, we have one more technical intermediate result. We hoped this would
extend to more cases, but we haven't been able to do anything with it yet.

Let $M$ be a book of $I$-bundles. Associated to every surface we construct in
an analogous way to what we've been doing (taking covers of the page core
surfaces and gluing them together), there's a set of fairly complicated graphs
we can construct as follows. For each $(M_i,A_{ij},\widetilde{S_i})$, where
$M_i$ is a page, $A_{ij}$ gluing annulus of that page, and $\widetilde{S_i}$
finite-sheeted cover of the page core surface, we can draw a \emph{local
pared-arc graph} $G=G(M_i,A_{ij},\widetilde{S_i})$.

The vertices are boundary components of $\widetilde{S_i}$ which sit above
$A_{ij}$. Two vertices are connected by an edge if there is an arc in the pared
locus which lifts to connect those two boundary components. For instance,
a double cover of the punctured torus page, with 3 disjoint non-parallel arcs
on each boundary, has a graph, over the single boundary annulus downstairs,
with 2 vertices and 3 edges (one between the two vertices, and one loop at each
vertex). This graph is a generalization of the cis-trans terminology we used in
that example.

We say a cover $\widetilde{S_i}$ is \emph{pared-bipartite} if each of its local
pared-arc graphs is bipartite. Note that none of the covers we constructed in
our first example are pared-bipartite. This is a very specialized condition.
Note also that if the pared locus does not intersect a page, all covers of that
page are pared-bipartite.

\begin{thm}

Let $M$ be a tree-shaped book of $I$-bundles. Suppose that $P$ has no
components in $\bd_\pm M_i$ for any $i$. Suppose that every page of $M$
has a pared-bipartite cover of its core surface. Then $M$ contains a surface
satisfying \eqref{E:qf}.

\end{thm}
\begin{proof}

First observe that every cover of a pared-bipartite cover is itself
pared-bipartite. Simply lift the partition of the vertices to obtain a new
partition of each local pared-arc graph. The arcs cannot break this
partition because otherwise they wouldn't cover arcs in the downstairs cover.

Choose a sufficiently large cover of a pared-bipartite cover of each page such
that at each spine, the number of incoming cover boundary components above each
incident page equal for that spine. We can do this taking a ``least common
multiple cover,'' unless pages are glued to the same spine multiple times. In
that case, choose a finite-sheeted cover of the book of $I$-bundles itself that
doesn't have this issue, lift the pared locus, and construct a surface there.
Afterwards, we can push it down and it will still satisfy \eqref{E:qf}.

% XXX careful! finite subgp separability covers of tree-shaped things aren't
% necessarily tree-shaped! might need another condition / argument here.

Now, we construct $S$ by gluing these large covers. For each $\widetilde{S_i}$
and downstairs gluing annulus $A_{ij}$ there will be two adjacent annuli in the
associated spine. For each $\widetilde{S_i}$, attach half the boundary
components above $A_{ij}$ to the boundary components coming from each of the
``neighboring'' $\widetilde{S_i}$, via annuli. Use the pared-bipartite
structure to split the set of boundary components above $A_{ij}$ in half. Since
the number of incoming boundary components match at each spine, as constructed
above, the boundary components will match up. We have a ``cover boundary
component bipartition'' above each $A_{ij}$. Note that once we have the
bipartition, the details of how we glue the pieces inside the bipartition are
irrelevant.

We claim the resulting $S$ satisfies \eqref{E:qf}. We already know it's proper
immersed $\pi_1$-injective by the same arguments. To show it doesn't overlap
with the pared locus, observe that by the definition of local pared-arc graphs,
every core curve in the pared structure must correspond to a sequence of arcs
in covers, where each arc either connects boundary components above different
annuli $A_{ij}$, $A_{ij'}$, or it connects 2 boundary components above the same
annulus $A_{ij}$. But it cannot connect two boundary components above the same
annulus, as we glued those according to the pared bipartition in such a way
that the pared arc component, which is locally restricted to a single boundary
component of the $I$-bundles, can't follow. This is a generalization of the
first argument in our simplest case.

Note that if the arc connects boundary components above different annuli, these
annuli must attach to different spines (by our earlier simplifying assumption).
But now because $M$ is tree-shaped, no arc that's been redirected toward
different spines can possibly form a closed loop. This completes the proof.

\end{proof}

We tried to do this proof in the non-tree-shaped case, but it doesn't work!
There are counterexamples of pared-bipartite covers where the obvious fix
relies on messy combinatorics inside the boundary component matching.

\section{Next steps - SCRUBBED}

% XXX talk a bit about my ideas for cis/trans and gen bipartition
It should be possible to prove the case of general tree-shaped books of
$I$-bundles. In fact, combinatorially, it seems that all books of $I$-bundles
(satisfying an appropriate condition on components of $P$) should admit these
surfaces. But we haven't proved anything yet. The general case is much messier
(combinatorially) than the tree-shaped case, because in addition to cis arcs
we'll also need to consider trans arcs that ``wrap around'' a loop in the
gluing graph. So even ensuring that every closed curve in the pared structure
is broken into arcs, at least one of which is trans, is insufficient.

We chose to study books of $I$-bundles first. Ian thinks the other infinite
volume cases ought to be easier. What about acylindrical manifolds?  Books of
$I$-bundles are somehow the opposite end of the spectrum of acylindrical
- they're glued together along a bunch of cylinders, and the pieces are all
very simple ($I$-bundles).  If we can address the acylindrical and book of
$I$-bundle cases we ought to be able to put these together to solve the entire
problem. Ian thinks maybe the acylindrical case can be addressed by a variant
of the Baker-Cooper argument.

\bibliographystyle{plain}
\bibliography{refs}

\end{document}
