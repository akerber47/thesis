\documentclass[12pt]{amsart}
\usepackage{amsmath,amscd,amssymb,amsthm,amsfonts}

\setlength{\topmargin}{0.5cm}
\setlength{\oddsidemargin}{-0.2cm}
\setlength{\evensidemargin}{-0.2cm}
\textheight = 22cm
\textwidth = 16.2cm

% For proof reading
%\renewcommand{\baselinestretch}{2.5}

\newtheorem{theorem}{Theorem}[section]
\newtheorem{thm}[theorem]{Theorem}
\newtheorem{lemma}[theorem]{Lemma}
\newtheorem{prop}[theorem]{Proposition}
\newtheorem{cor}[theorem]{Corollary}

\theoremstyle{definition}
\newtheorem{defn}[theorem]{Definition}
\newtheorem{conj}[theorem]{Conjecture}

\theoremstyle{remark}
\newtheorem{example}[theorem]{Example}
\newtheorem*{rmk}{Remark}
\newtheorem*{claim}{Claim}
\newtheorem*{ntn}{Notation}

\newcommand{\x}{\times}
\newcommand{\bd}{\partial}
\newcommand{\Om}{\Omega}
\newcommand{\Si}{\Sigma}
\newcommand{\cin}{\subseteq}

\newcommand{\cC}{\mathcal{C}}
\newcommand{\cN}{\mathcal{N}}

\begin{document}

\title{Quasifuchsian surface subgroups of books of $I$-bundles}

\author{Alvin Kerber}
%\address{University of California, Berkeley\\
%970 Evans Hall \\
%Berkeley, CA, 94720}
    \email{alvin@math.berkeley.edu}
%\thanks{Agol supported by something or other}
%\date{%
%\today}

\begin{abstract}

Given a Kleinian group $\Gamma$, one can ask whether the group contains any
quasifuchsian surface subgroups. Equivalently, given a pared 3-manifold
$(M,P)$, one can ask whether there exists a closed immersed $\pi_1$-injective
surface in $M$ that avoids the peripheral subgroups associated to $P$.  This is
known to be true for closed hyperbolic 3-manifolds, and more generally for
finite volume hyperbolic 3-manifolds. We derive explicit results for pared
3-manifolds $(M,P)$ where $M$ is a book of $I$-bundles.

\end{abstract}

\maketitle
%%%%%%%%%%%%%%%%%%%%%%%%%%%%%%%%%%%%%%%%%%%%%%%%%
\section{Introduction}
%%%%%%%%%%%%%%%%%%%%%%%%%%%%%%%%%%%%%%%%%%%%%%%%%

Let $M$ be a complete hyperbolic 3-manifold. $M$ can be realized as the
quotient manifold of a Kleinian group $\Gamma$. One can ask whether $\pi_1M$
contains any quasifuchsian surface subgroups. To tackle this problem
topologically, we represent a Kleinian group with a pared 3-manifold $(M,P)$.
The pared structure $P\cin\bd M$ specifies the boundary locus corresponding to
cusps of the parabolic group elements.

Finding a quasifuchsian surface subgroup of the Kleinian group boils down to
asking whether we can find a closed immersed $\pi_1$-injective surface which
avoids this pared locus up to homotopy. That is, we're trying to choose
a surface subgroup of $\pi_1M$ which does not contain any of the peripherial
subgroups associated to components of $P$. The general question is: for which
choices of $(M,P)$ can we find such a surface subgroup?

There is some existing work on this problem. Cooper-Long-Reid \cite{CLR}
addressed the underlying topological problem, showing that a compact connected
irreducible 3-manifold with non-empty incompressible boundary  must contain an
essential closed surface, unless it's covered by a product $F\times I$.
However, this surface may contain accidental parabolics, that is, overlap with
the pared locus $P$. If this occurs it will not be quasifuchsian.

% FIXME something is wrong with Masters-Zhang citation / bib entry - I can only
% find their older paper (hyp knot complements) on mathscinet, not the full new
% one with link complements

In the closed case, there is the work of Kahn-Markovic \cite{KM}. They proved
that there always exists an immersed closed $\pi_1$-injective surface in any
closed hyperbolic 3-manifold.  Such a surface is necessarily quasifuchsian
because the surface is closed, so the pared structure is empty. Finally,
Masters-Zhang \cite{MZ} showed that every finite volume hyperbolic 3-manifold
with cusps contains a quasifuchsian surface subgroup. An alternate proof was
provided by Baker-Cooper \cite{BC}.

The remaining case is infinite volume hyperbolic 3-manifolds. So far we've only
addressed a particular example of this case --- books of $I$-bundles. We
conjecture that all books of $I$-bundles admit quasifuchsian surface subgroups,
except for a few obvious negative cases where the pared locus is too large. We
prove this in a few special cases, so far.

%%%%%%%%%%%%%%%%%%%%%%%%%%%%%%%%%%%%%%%%%%%%%%%%%
\section{Background}
%%%%%%%%%%%%%%%%%%%%%%%%%%%%%%%%%%%%%%%%%%%%%%%%%

Material in this section is from \cite{Mo} and \cite{CMc}. Look there for
further details.

%Definition and important facts about pared manifolds. After Morgan, The Smith
%Conjecture, V ("Uniformization Theorem for Three-Dimensional Manifolds"),
%p 58-60. Or Canary-McCullough, Homotopy Equivalences of 3-Manifolds and
%Deformation Theory of Kleinian Groups, Ch 5 p. 87-92. Also Ch 7 p.105-107.
%Note that Canary-McCullough is much more recent.
%
%(Morgan p58) (Canary-McCullough p87)
\begin{defn}

A \emph{pared manifold} $(M,P)$ is a compact orientable irreducible 3-manifold
$M$, together with $P\cin\bd M$, such that the following conditions hold:

\begin{enumerate}
\item Every component of $P$ is a torus or annulus, incompressible in $M$.

\item Every noncyclic abelian subgroup of $\pi_1M$ is peripheral with respect
to $P$ -- ie, conjugate to the fundamental group of a component of $P$.

\item $(M,P)$ is ``annulus-incompressible'': every $\pi_1$-injective map $(A^2,
\bd A^2) \to (M,P)$ is homotopic (as a map of pairs) to a map into $P$.

\end{enumerate}

We call $P$ the \emph{pared locus} or \emph{parabolic locus} of the pared
manifold $(M,P)$.

\end{defn}

Note that there are a few pared manifolds that are special cases, like with
elementary Kleinian groups. In fact, these are precisely the pared manifolds
that correspond to elementary Kleinian groups when we construct pared
3-manifolds from Kleinian groups below.

%(Canary-McCullough p88)
\begin{defn}

A pared manifold $(M,P)$ is \emph{elementary} if it is homeomorphic (as a pair)
to one of the following: $(T^2\x I,T^2\x 0)$, $(A^2\x I,A^2\x 0)$, or $(A^2\x
I,\emptyset)$.
%(or (Morgan only) (S3,empty)).

\end{defn}

%(Morgan p59, Canary-McCullough p88)
\begin{prop}

Let $(M,P)$ nonelementary. The following facts hold:

%In fact, Canary-McCullough makes these 3 statements. Morgan only has the 4th.
\begin{enumerate}
\item M is not a $T^2\x I$, $K^2$ $I$-bundle, or $A^2\x I$ (solid torus).
\item M does not contain an embedded $K^2$
\item For every component $Q$ of $P$, $\pi_1Q$ is a maximal abelian subgroup of
$\pi_1M$.
\item Every component of $\bd M-P$ has negative Euler characteristic.
% Yes this is correct. But the only extra info we get is that there's no
% parallel annuli. Canary-McCullough includes the statement:
% "Every toroidal component of dM is contained in P"
% but this is redundant with item 4 above (from Morgan)
\end{enumerate}

\end{prop}
\begin{proof}
See \cite{CMc} or \cite{Mo}.
\end{proof}

Pared manifolds arise naturally in the study of Kleinian groups. Given
a geometrically finite torsion-free Kleinian group $G$, we construct $(M,P)$ by
truncating small neighborhoods of the cusps of the quotient manifold $M(G)
= \left(\mathbb{H}^3\cup \Om(G)\right)/G$.  Then we let $M$ be the resulting
compact manifold with boundary, and $P$ the boundary locus along which we
truncated.  That is, we can rebuild the quotient manifold from $(M,P)$ by
gluing cusp neighborhoods onto each component of $P$. It follows from basic
Kleinian group theory that the resulting $(M,P)$ will satisfy conditions 1-3
above, making it a pared manifold.  The parabolic elements of $G$ precisely
correspond to (conjugacy classes of) the peripheral subgroups $\pi_1P$ in
$\pi_1M$.

Conversely, Thurston's famous uniformization theorem states that for every
pared manifold $(M,P)$ with $M$ Haken, there exists a finite volume (in
particular, geometrically finite) hyperbolic 3-manifold structure on the
interior satisfying the appropriate conditions for a pared 3-manifold! To be
precise:

%(see Morgan for details of this construction)

%(Morgan p60, but his statement is confusing / less modern definitions. He does
%state the more general theorem for Haken pared manifolds though.)
%(Canary-McCullough p105-106)

\begin{thm}

If $(M,P)$ is an oriented pared 3-manifold with nonempty boundary, then
there exists a geometrically finite uniformization of $(M,P)$, that is, a map
$\rho: \pi_1M \to \mbox{\rm{PSL}}_2\mathbb{C}$ such that $\rho(g)$ is parabolic
if $g \in \pi_1(P)$, and an orientation-preserving homeomorphism $M-P \to
\left(\mathbb{H}^3 \cup \Om(\rho(\pi_1M))\right)/\rho(\pi_1M)$.

\end{thm}
\begin{proof}

See \cite{CMc}.
Also see discussion on (CMc p106, bottom of page) for why this definition of
geometrically finite is equivalent to the standard one.

\end{proof}

% Example. Maybe? TODO

\textbf{ TODO Include an example.}

%%%%%%%%%%%%%%%%%%%%%%%%%%%%%%%%%%%%%%%%%%%%%%%%%
\section{Our reduction}
%%%%%%%%%%%%%%%%%%%%%%%%%%%%%%%%%%%%%%%%%%%%%%%%%

% TODO
\textbf{ TODO This entire section needs a lot of work. Rewrite and expand.}

{\tiny

In our result, we study the problem of finding quasifuchsian surface subgroups
of finitely generated Kleinian groups from the topological perspective. We
convert our given Kleinian group to a pared 3-manifold, and look for surface
subgroups of this Kleinian group that avoid the peripheral pared structure. We
can reduce the problem of finding a quasifuchsian surface subgroup of
a Kleinian group as follows.

We first restrict ourselves to considering only Kleinian groups which are
geometrically finite. In the geometrically infinite case, look at the simply
degenerate (geometrically infinite) ends of the group. By the Canary covering
theorem, tameness, and some other results, it's known that any finitely
generated subgroup of a Kleinian group must be either geometrically finite or
a virtual surface fiber - that is, it corresponds to a fiber surface in
a finite-sheeted cover which is fibered over the circle. See (AFW p117) for
this fact and various spots in the book for arguments.

% No virtual surface fiber subgroup can possibly be quasifuchsian,

Only geometrically finite subgroups can be quasifuchsian. This is basically by
definition (check this). The problematic case is when it's possible to have
a geometrically finite quasifuchsian surface subgroup of a geometrically
infinite Kleinian group. Looking at the simply degenerate ends, and comparing
to the ends of the quasifuchsian surface subgroup itself (viewed as a Kleinian
group), the relative ends must finite-to-one cover relative ends of the
original Kleinian group, by the Canary covering theorem. But now there's some
problem with different types of (relative) ends covering different types of
ends. I still don't understand this.

Also, note that quasifuchsian here is intended in the sense of Kahn-Markovic as
discussed in AFW p81. As discussed there (will get references), quasifuchsian
surface subgroups must be geometrically finite - which rules out the virtual
fiber case by the above dichotomy - and directly correspond to geometrically
finite subgroups that avoid the cusps.

One final consideration is to ensure that geometrically finite surface
subgroups cannot exhibit certain pathological behavior inside geometrically
infinite Kleinian groups. In the geometrically finite case this is easy,
because it'll necessarily also be geometrically finite, but in the
geometrically infinite case we need to either guarantee that the surface we
find by the below, purely topological construction is geometrically finite, or
show that actually no such surface exists and our construction is irrelevant.
Check with Ian (and references) about which of these we're actually doing!

Of course, so long as we are only studying books of $I$-bundles topologically,
this is irrelevant, as we can restrict ourselves to considering the
geometrically finite realizations of these hyperbolic 3-manifolds. However, it
is nice to be able to state the result more generally. Also if we want to
generalize to a more complete consideration of (finitely generated) Kleinian
groups at some point, not just books of $I$-bundles, it will be nice to say that
our work applies to more than just geometrically finite cases.

Anyway, I believe the geometrically infinite case is much simpler once
I understand what existing work it's based on. I'll need to talk to Ian about
this. In what follows, just assume our Kleinian group is geometrically finite.

% OLD REFERENCE - DOESN'T WORK BY ITSELF!
%%(Baker-Cooper Theorem 1.7 - "work of Bonahon and Thurston").
%\begin{thm}
%
%Let $M$ be a complete hyperbolic 3-manifold with finite volume, and $S$
%a closed oriented $\pi_1$-injective immersed surface of negative Euler
%characteristic. Then either $S$ is a virtual fiber, or $S$ is geometrically
%finite. In the second case, it's either quasifuchsian or some element of
%$\pi_1S$ is parabolic.
%
%\end{thm}
%

We assume this reduction still works in our cases (which are not finite
volume). I hope some reference can fix this!

So in our case, since the parabolic elements precisely correspond to the pared
structure, our problem becomes a purely topological one: can we find a closed
immersed $\pi_1$-injective surface of negative Euler characteristic, that
doesn't contain any of the pared locus, up to homotopy?

Since we're working in the case of books of $I$-bundles, we can reduce the
pared locus by observing that there are no torus boundary components of M, so
all the components of the pared locus must be annuli (as they're
$\pi_1$-injective). For each annulus, since we're only concerned with its
homotopy class, it suffices to consider the core curve.

}

%%%%%%%%%%%%%%%%%%%%%%%%%%%%%%%%%%%%%%%%%%%%%%%%%
\section{Our first example}
%%%%%%%%%%%%%%%%%%%%%%%%%%%%%%%%%%%%%%%%%%%%%%%%%

\textbf{ TODO shorten this section. Line it up with general result.}

We construct a book of $I$-bundles $M$ as follows. In the following we take
$\Si_{1,1}$ to be the compact surface of genus 1 with a single boundary
component, a circle. Let $M_1$,$M_2$,$M_3$ be 3 punctured torus $I$-bundles,
$M_1=M_2=M_3=\Si_{1,1}\x I$. For each $M_i$, write \[ \bd M_i = \bd \Si_{1,1}\x
I \cup \Si_{1,1}\x0 \cup \Si_{1,1} \x 1 \] and label these boundary pieces as
\[ \bd M_i = A_i \cup \bd_- M_i \cup \bd_+M_i \].

Let $M_c = S^1xD^2$ be a solid torus. Attach the $M_i$ to $M_c$ by gluing the
$A_i$ to parallel annuli in $\bd M_c$, each with longitudinal core curve. The
result is a compact 3-manifold with boundary. This is our $M$. We choose
orientations and a cyclic order for the gluing such that the 3 boundary
components of M are precisely
\[
\bd M = (\bd_+M_1 \cup_{S^1} \bd_-M_2) \sqcup (\bd_+M_2 \cup_{S^1} \bd_-M_3)
\sqcup (\bd_+M_3 \cup_{S^1} \bd_-M_1)
\]
and label these
\[
\bd M=\bd_{12}M \sqcup \bd_{23}M \sqcup \bd_{31}M
\].
Each boundary
component of $M$ consists of two punctured tori glued along their boundary
circles, topologically a genus two surface.

We consider possible pared structures $P$ on $M$. Our goal is to find
a surface $S \cin M$ such that
\begin{equation}\label{E:qf}
S \text{ is closed immersed $\pi_1$-injective, and  $\pi_1P_k \cap \pi_1S
= 1$
for each component $P_k$ of $P$} \tag{\textasteriskcentered}
\end{equation}
%%% Equation (*) (star)

% FIXME straighten this out
%Note that since we
%haven't fixed a basepoint, these subgroups are really only defined up to
%conjugacy - what we're saying is they fail to intersect for an arbitrary choice
%of conjugacy class for each subgroup. (ask Ian about this? seems fuzzy)
%
% FIXME FIXME fix perturb nonsense by reference to obstruction theory (as
%discussed in Long-Reid, p11

\begin{lemma}\label{L:sc}

Let $S$ be a surface satisfying \eqref{E:qf}. Homotope $S$ to have minimal
intersection with each $A_i$, that is, so there are no ``bumps''.  Then for
each $M_i$, each component of $S \cap M_i$ is a finite-sheeted covering of the
core surface.  That is, given such a component $S' \cin M_i$, the map $S' \to
M_i = \Si_{1,1}\x I \to \Si_{1,1}\x{1/2}$ is homotopic to a finite-sheeted
covering map. Conversely, given any finite-sheeted covering $\widetilde{S} \to
\Si_{1,1}$, there exists a corresponding proper immersed $\pi_1$-injective
surface $S \cin \Si_{1,1}\x I$.

\end{lemma}
\begin{proof}

$S$ is compact, so every such component $S'$ is a compact surface with
boundary, properly immersed in $M_i$. Since $S$ is $\pi_1$-injective in $M$,
$S'$ must be $\pi_1$-injective in $M_i$. Otherwise, we'd have a nontrivial
element of $\pi_1S'$ which is trivial in $\pi_1M_i$, hence in $\pi_1M$, hence
in $\pi_1S$, contradicting the minimal position homotopy above. So the map
$\phi : S'\to\Si_{1,1}\x{1,2}$ is also $\pi_1$-injective, since $M_i$
deformation retracts to its core. Let $H = \phi_*(\pi_1S')$, and let
$\Si_{1,1}^H$ be the cover of $\Si_{1,1}$ associated to $H<\pi_1\Si_{1,1}$.
$\phi$ lifts to $\widetilde{\phi}\colon S'\to \Si_{1,1}^H$. This is a proper
map of compact surfaces which is an isomorphism on $\pi_1$.  By the
classification of surfaces, it must be homotopic (as a proper map) to
a homeomorphism. So $\phi$ is homotopic to a covering map. It must be
finite-sheeted as $S'$ is compact (by classification of surfaces again).

Conversely, given a finite-sheeted cover $\widetilde{S}\to \Si_{1,1}$, compose
with the embedding $\Si_{1,1} = \Si_{1,1}\x{1/2} \cin \Si_{1,1}\x I$. This is
proper and $\pi_1$-injective.  Perturb locally to obtain an immersion.

\end{proof}

\begin{lemma}

Let $P$ be a pared structure on $M$. $P$ cannot contain any tori. That is,
$P$ consists entirely of annuli.

\end{lemma}
\begin{proof}

We can compute $\pi_1M$ directly from the van Kampen theorem. Let $a_i,b_i$ be
the generators of $\pi_1(M_i)$, and $c$ be the generator of $\pi_1(M_c)$. Our
gluing yields
\[
\pi_1(M) = \langle a_1,b_1,a_2,b_2,a_3,b_3,c \mid
[a_1,b_1]=[a_2,b_2]=[a_3,b_3]=c \rangle
\]
It immediately follows that $\pi_1M$ contains no rank 2 abelian subgroups, as
no two elements commute. Since every component of $P$ is $\pi_1$-injective,
they must all be annuli.

Alternatively, simply observe that all the boundary components of $M$ have
negative Euler characteristic, so they don't admit $\pi_1$-injective maps from
a torus.

\end{proof}

\begin{prop}

Suppose $P$ contains a component $P_0$ that, up to homotopy, is contained
entirely within one ``$I$-bundle half'' of a boundary component. That is, $P_0
\cin \bd M_i$, for some i. Without loss of generality we can let $P_0 \cin \bd
M_1$.  Then there exists a surface satisfying \eqref{E:qf} if and only if $P$
contains no components that are (up to homotopy) contained in $\bd M_2 \cup \bd
M_3 = \bd_-M_2 \sqcup \bd_{23}M \sqcup \bd_+M_3$.

\end{prop}
\begin{proof}

($\Longleftarrow$) Take $S=\bd_{23}M$. Intuitively, it's easy to see that the
conditions on $P$ force all its components to lie on surfaces where they can't
be homotoped into $S$.

Algebraically, $\pi_1S$ is the subgroup generated by $a_2,b_2,a_3,b_3 \in
\pi_1M$. We know that
\[
\pi_1S = \langle a_2,b_2,a_3,b_3 \mid [a_2,b_2]=[a_3,b_3]\rangle.
\]
$P$ contains no components in $\bd_{23}M$, so an arbitrary component $P_k$ of
$P$ must be contained in $\bd_{12}M$ or $\bd_{31}M$, which have fundamental
groups generated by $a_1,b_1,a_2,b_2$ and $a_3,b_3,a_1,b_1$ respectively.
$\pi_1P_k$ is cyclic, so let $g$ be its generator. No matter if it's contained
in $\bd_{12}M$ or $\bd_{31}M$, we can see that in order for $\pi_1P_k$ to
overlap with $\pi_1S$, it must have generator some word in $a_2,b_2$ (if $P_k
in \bd_{12}M$), or some word in $a_3,b_3$ (if $P_k in \bd_{31}M$). In either
case, this is precisely the condition for such a word to correspond to a curve
contained in $\bd M_2$ or $\bd M_3$, respectively, contradicting our assumption
on $P$.

($\Longrightarrow$) Let $S$ be a surface satisfying \eqref{E:qf}. Cut $S$ into
components in $M_1,M_2,M_3$ (after homotoping to minimal intersections with the
annuli).  Applying Lemma~\ref{L:sc}, we can see that $S \cap M_1 = \emptyset$,
as otherwise since it's a finite-sheeted cover it would have to contain
a multiple of $P_0$.  So $S \cin M_2 \cup M_3 \cup M_c$, which is homeomorphic
to $S_2 \x I$.  Deformation retracting this to a surface $S_2$ and applying
a covering argument like in Lemma~\ref{L:sc}, we can see that $S$ is a cover of
$S_2$.  Since it's a finite-sheeted cover, it will have to have $\pi_1$
intersecting any component $P_k$ that violates the conditions stated above.
This completes the proof.

\end{proof}

The proof above obviously generalizes to more complex books of $I$-bundles.
Later we'll use it to simplify any pared structures containing annuli that fit
in a single $I$-bundle.

\begin{thm}\label{T:ex1}

Let $P$ be a pared structure containing no components $P_0$ as in the above
proposition. Then there exists a surface satisfying \eqref{E:qf}.

\end{thm}

This is the main theorem we prove for this example. A few preliminary facts are
required. Note that we can first isotope P such that it's in minimal position
on each boundary component with respect to the gluing circle.

\begin{lemma}

Under the hypotheses of the theorem, each ``$I$-bundle piece'' boundary
component $S=\bd_\pm M_i$ intersects $P$ in a thickened set of disjoint
essential arcs, each arc connecting $\bd S$ to itself. The arcs form at most
3 ``bands'', where all the arcs in each band are parallel.  Furthermore, if we
choose a representative arc from each band (yielding a set of at most
3 disjoint non-parallel arcs in $\Si_{1,1}$), there exists an automorphism of
$\Si_{1,1}$ taking these arcs to a standard set of 3 disjoint non-parallel
arcs.  See diagram for an illustration of the standard set.

\end{lemma}
\begin{proof}

We first need a preliminary definition. After cutting a surface with boundary
along arcs, we'll obtain one or more connected surfaces, each with one or more
boundary components. Each boundary component after cutting will have pieces
from the original boundary as well as pieces from the arcs that we cut along.
Given labels $\gamma_1,...,\gamma_k$ for the boundary components and
$\alpha_1,...,\alpha_l$ for the arcs (on the original surface with boundary),
we can describe each boundary component of the cut surfaces as a union of arcs,
each labeled with $\gamma_1,...,\gamma_k,\alpha_1,...,\alpha_l$. We call this
an \emph{arc pattern} for that boundary component of the new surface.

We first show that $\Si_{1,1}$ admitts at most 3 disjoint essential
non-parallel arcs, and the possible surfaces and arc patterns obtained by
cutting along these arcs are very restricted.

Label the boundary component of $\Si_{1,1}$ by $\gamma$. Consider a proper
essential arc $\alpha_1 \cin \Si_{1,1}$. Fix an orientation for $\alpha_1$.
Since $\Si_{1,1}$ is orientable, we can look at local neighborhoods of
$\alpha_1$ and see that $\alpha_1$ has a well-defined ``left side'' and ``right
side'' as we travel along it. Looking at the endpoints of $\alpha_1$ along
$\gamma$, we are forced to connect certain endpoints in the cut-up $\gamma$
with $\alpha_1$, in order to preserve the parity. This tells us the (possibly
disconnected) cut surface $S_1$ will have two boundary components. Each will
have an arc pattern consisting of two arcs, one labeled $\gamma$ and one
labeled $\alpha_1$.

Since we cut along a properly embedded arc, the Euler characteristic increases
by one. $\chi(S_1)=0$ and $S_1$ has two boundary components. By classification
of surfaces, $S_1 = \Si_{0,2}$ or $\Si_{0,1} \sqcup \Si_{1,1}$. But if $S_1$
contained a disk with the arc pattern described above, embedding that disk back
in $\Si_{1,1}$ would describe a homotopy of $\alpha_1$ into the boundary. So
$S_1 = \Si_{0,2}$.

Now suppose we had a second proper essential arc $\alpha_2 \cin \Si_{1,1}$,
disjoint from and non-parallel to $\alpha_1$. Since it's disjoint from
$\alpha_1$, $\alpha_2$ induces a proper arc in $S_1$ which connects two regions
in the arc pattern labeled $\gamma$.  $\alpha_2$ must have one endpoint on each
boundary component of $S_1$. If both are on the same side, it's either
homotopic to the boundary of $\Si_{1,1}$ or parallel to $\alpha_1$. Cutting
along $\alpha_2$ yields a new surface $S_2$. Topologically $S_2=D^2$, with arc
pattern consisting of 8 components in the cyclic order
$(\gamma,\alpha_1,\gamma,\alpha_2,\gamma,\alpha_1,\gamma,\alpha_2)$.

Finally, adding our 3rd proper essential arc $\alpha_3$, disjoint and
non-parallel to the first two arcs, a similar argument shows that $\alpha_3$
must connect opposite $\gamma$ pieces in the arc pattern. Cutting along
$\alpha_3$ yields two disks with the same arc pattern. Depending on the choice
of $\alpha_3$, the arc pattern on these disks is either
$(\gamma,\alpha_1,\gamma,\alpha_2,\gamma,\alpha_3)$ or
$(\gamma,\alpha_1,\gamma,\alpha_3,\gamma,\alpha_2)$. So up to relabeling
$\alpha_1,\alpha_2,\alpha_3$, cutting along 3 proper essential disjoint
non-parallel arcs has only one possible choice of cut surfaces and arc
patterns.

Observe that it is not possible to add any more disjoint non-parallel arcs. In
particular, any arc we draw between $\gamma$ components of the arc pattern on
either disk is homotopic to the boundary or parallel to an existing arc.
Furthermore, if we add new disjoint arcs and allow them to be parallel, we can
see that they must form ``bands'' around the existing 3 arcs in order to remain
disjoint. That is, we can homotope all the arcs parallel to a given arc into
a small neighborhood of that arc in the disk, without intersecting any of the
non-parallel arcs.

We claim there exists an automorphism of $\Si_{1,1}$ taking any set of 3 such
arcs to any other set (in particular, to the standard set, as illustrated).
Since there is only one topological result of cutting along the arcs, choose
a homeomorphism of the cut surfaces. Up to relabeling the arcs, we can choose
a homeomorphism that identifies matching arcs in the arc patterns (as shown
above, there is only one possible arc pattern up to relabeling). Glue both
sides along the arcs to obtain the desired automorphism of $\Si_{1,1}$.

We now consider $P \cap \bd_S \cin S \cong \Si_{1,1}$. As shown, $P$ is a union
of annuli. By the assumptions of the theorem, no annulus of $P$ is contained in
$S$, so each annulus intersects $S$ in a union of rectangles, where two sides
of the rectangle are embedded in the boundary $\bd S$.  Since we homotoped $P$
to have minimal intersections, all the core arcs of the rectangles (pieces of
the core curve of the annulus) must be essential. They are disjoint by
definition of $P$. The statement of the lemma follows from applying the above
argument to these core arcs.

\end{proof}

We consider connected double covers $\widetilde{S} \to S \cong \Si_{1,1}$.
These covers have 2 boundary components. Every core arc of $P \cap S$ lifts to
2 arcs in $\widetilde{S}$.  We say an arc in $S$ is \emph{cis} for a given
$\widetilde{S}$ if any (ie all) lifts of that arc have both endpoints in the
same boundary component of $\widetilde{S}$.  Otherwise, we say it's
\emph{trans} for that cover.

\begin{lemma}

Given any 3 disjoint non-parallel proper arcs in $\Si_{1,1}$, and any double
cover of $\Si_{1,1}$, 2 of the 3 arcs are trans, and the 3rd is cis.
Furthermore, we can choose any two of the three we wish to be trans with an
appropriate cover.

\end{lemma}
\begin{proof}

As in earlier lemma, there exists an automorphism of $\Si_{1,1}$ taking these
arcs to the standard set of 3 disjoint non-parallel proper arcs.  Now it
suffices to observe, by looking at relative first homology or just by
construction, that each of the three standard connected double covers
(corresponding to nontrivial maps $\mathbb{Z}^2 \to \mathbb{Z}/2$) makes two of
the three arcs trans and the third cis.

\end{proof}

\begin{proof}[Proof of Theorem~\ref{T:ex1}]

Since $P$ has no components $P_0$, every $P \cap \bd_+M_i$ is a union of
thickened arcs (that is, there are no full annulus components in any $\bd_+M_i$
or $\bd_-M_i$). Apply the first lemma to break these into bands. The problem is
most constrained when there are 3 bands, so we'll consider that case (if there
are fewer than 3, just draw some more arcs on that component arbitrarily, and
the proof still works).

We will build our surface $S$ by taking a double cover of the core $\Si_{1,1}$
surface for each $I$-bundle page $M_1,M_2,M_3$. Call these covers
$S_1,S_2,S_3$.  Each of these has two boundary components. We'll then attach
the boundary components such that each of $S_1,S_2,S_3$ has exactly one
boundary component connected to each of the others. See diagram.

Choose an arbitrary connected double cover for $S_1$. We can view this as
a cover of $\bd_+M_1$ or $\bd_-M_1$, deformation retracting either way. By the
lemma, two of the
3 bands on $\bd_+M_1$ are trans, and the other is cis. The same holds for
  $\bd_-M_1$.

We cannot choose $S_2$ arbitrarily, as a cis arc for $S_1$ in $\bd_+M_1$ may
connect (in $S$) to a cis arc for $S_2$ in $\bd_-M_2$. These together would
form a closed curve that lifts to $S$, which once $S$ is immersed in $M$ will
yield a violation of condition \eqref{E:qf}. Instead, observe that there is at
most one band of the three in $\bd_-M_2$ containing arcs that, when glued
across the core circle, match up to arcs in the cis band of $\bd_+M_1$ to form
closed curves containing only arcs in those two bands. This is because once one
band has that behavior, by endpoint parity the vertices can't also match up for
a different band, if the arcs they have to match with on the other side are
parallel. Looking at the boundary circle, non-parallel arcs have endpoints in
cyclic order, but parallel arcs don't. See diagram. It suffices to check this
for our standard set of non-parallel arcs, by the same automorphism argument.

Since there is only one such band on $\bd_-M_2$, choose $S_2$ such that this
band is not cis.  Finally, for $S_3$ we have two connecting constraints. We
have a cis band on $\bd_-M_1$ from our choice of $S_1$, and a cis band on
$\bd_+M_2$ from our choice of $S_2$.  Applying the same argument, there is at
most one band on $\bd_-M_3$ and one band on $\bd_+M_3$ that will connect to
form closed curves. But we can choose $S_3$ such that both of these bands are
trans. (This requires a slight modification to the lemma - since these bands
are on different boundary surfaces, they may not be disjoint non-parallel. If
they aren't disjoint, we can tweak them by local modifications so they are, and
then ``untweak'' them in the cover. If they are parallel, just ignore one set
of bands)

Glue $S_1,S_2,S_3$ as described to obtain $S$. We claim that $S$ satisfies
condition \eqref{E:qf}. $S$ is closed and immersed by construction (as in
Lemma~\ref{L:sc}). It is $\pi_1$-injective inside each $I$-bundle by
Lemma~\ref{L:sc}, and inside the core $M_c$ because it's just a union of
incompressible gluing annuli there. It suffices to show that $\pi_1S \cap
\pi_1P_k = 1$ for each component $P_k$ of the pared locus. Looking at the core
curve of the annulus $P_k$, this implies that some multiple of that core curve
is homotopic into $S$.

But this is impossibly by the above construction. Every such curve contains at
least two bands, on two different components $\bd_\pm M_i$. By the construction
of $S$, of any two bands which connect up when the boundary components are
glued across circles, at least one must be trans. This means that when we try
to homotope the core curve multiple into $S$, in order to follow along $S$
locally (in each page, where $S$ is locally a cover of the core surface, it
must be a lift from that core surface) it would have to traverse between all
three pieces of the cover, as that's how we connected up the $Si$. Every trans
arc lifts to an arc that connects two different boundary components of an $Si$,
which are ``pointed in different directions.'' But this is obviously
impossible, as our $P_k$ is restricted to be in a single boundary component, so
up to homotopy it must be generated by those two $I$-bundle pages only.

\end{proof}

%%%%%%%%%%%%%%%%%%%%%%%%%%%%%%%%%%%%%%%%%%%%%%%%%
\section{Some additional topological background}
%%%%%%%%%%%%%%%%%%%%%%%%%%%%%%%%%%%%%%%%%%%%%%%%%

The following is based on \cite{AFW} and \cite{LR}. It is definitely not
original.

\begin{defn}

Let G be a group. G is residually finite, or RF, if for any g in G, there
exists a finite index subgroup H<G such that g notin H.

Let $G$ be a group, and $H$ a subgroup of $G$. $H$ is \emph{separable} in $G$
if for any $g \in G$, $g \notin H$, there exists a finite index subgroup $H'<G$
such that $H'>H$, and $g \notin H'$.

$G$ is \emph{subgroup separable}, or \emph{LERF}, if all its finitely generated
subgroups are separable.

\end{defn}

It is well-known that these are equivalent to the following facts: (see
\cite{LR}). We'll use these facts later in our proof.

- G is RF if and only if for any g in G there exists a map phi colon G to F,
where F is a finite group, such that phi(g) neq id.

- G is LERF if and only if for any g in G, H < G, g notin H, there exists a map
phi colon G to F such that phi(g) notin phi(H).

Note that in this entire paper, \emph{surface groups} will refer to fundamental
groups of closed surfaces (compact without boundary).
% TODO move this sentence somewhere. Probably want to clarify what I mean by
% hyperbolic 3-manifold, etc also...

\begin{thm}[Hall] Free groups are LERF. \end{thm}

\begin{thm}[Scott] Surface groups are LERF. \end{thm}

\begin{thm}[Agol--Kahn--Markovic--Wise] Let $N$ by a hyperbolic 3-manifold,
possibly infinite volume.  Then $\pi_1N$ is LERF. \end{thm}

\begin{proof}

See citations in \cite{LR}, \cite{AFW}.

\end{proof}

Note that the last theorem is actually an extremely deep fact. Wise's proof
actually shows that hyperbolic 3-manifold with boundary groups satisfy
a technical condition, namely that they are virtually compact special. (Other
cases of hyperbolic 3-manifolds, including the particularly difficult closed
case, were addressed by the overlapping work of Agol, Kahn-Markovic, Sageev,
and Bergeron-Wise).  Additional work by Haglund and Haglund-Wise demonstrates
that a virtual compact special group is virtually a quasiconvex subgroup of
a right-angled Artin group, and geometrically finite subgroups are separable.
It follows by the tameness theorem that hyperbolic 3-manifold groups are LERF.


%%%%%%%%%%%%%%%%%%%%%%%%%%%%%%%%%%%%%%%%%%%%%%%%%
\section{General case - preliminaries}
%%%%%%%%%%%%%%%%%%%%%%%%%%%%%%%%%%%%%%%%%%%%%%%%%

We prove a generalization of the above example. Basically, we want to first
identify the cases that obviously can't have a surface satisfying \eqref{E:qf}.
Then we'll prove by construction that every other book of $I$-bundles does in
fact contain such a surface.

We use the following general description for books of $I$-bundles. Let
$M_{c1},...,M_{cr}$ be the spines, all solid tori. Let $M_{p1},...,M_{ps}$ be
the pages, all thickened surfaces with boundary. Each page has boundary
components $A_{ij}$, all annuli. We glue each annulus by a homeomorphism
$\phi_{ij}$ to a boundary annulus $A_{ij'}\cin \bd M_{ck}$. Note that $Aij'$
might not be a simple longitude of, that is, a page can wrap multiple times
around a spine.  However, for each spine, all the $A_{ij'}$ must be disjoint
and parallel, or the result of gluing will not be a manifold.

We first make the following simple observation.

prop

Let M be a book of I-bundles, and M' to M be a finite-sheeted covering space.
Then M' is also a book of I-bundles.

proof

This is straightforward. Lift the spines and pages of M to obtain spines and
pages of M'. Lift the gluing annuli of M to gluing annuli of M'.

We want to consider books of $I$-bundles with some basic topological
properties, to make sure they are Kleinian manifolds.

\begin{defn}

A book of $I$-bundles $M$ is \emph{admissible} if it satisfies the following
conditions.

\begin{enumerate}

\item The underlying manifold is a compact connected orientable hyperbolic
3-manifold.

\item Each page is an I-bundle over a surface of nonpositive Euler
characteristic. That is, no page has base surface a disk or Mobius strip, (or
Klein bottle?).

% FIXME figure out what Ian said  about Klein bottles

\end{enumerate}

\end{defn}

Henceforth we will only consider admissible books of $I$-bundles. Let's briefly
explain why.

Obviously we want M to be compact connected orientable hyperbolic, as this is
the general case we're studying (Kleinian manifolds). If M were nonorientable
we could reduce to the orientable double cover.

Each page's base surface must have at least one boundary component to glue to
in order for M to be connected. We don't want to consider base surfaces which
are disks.  Observe that each page which is a copy of $S0,1$ means that that
page and its attached spine form a 3-manifold with a finite-sheeted cover by
a ball (we can arrange things so that in the cover, the $S0,1$ attaching map
only traverses the longitude once, and we get a thickened disk).  This means
that we'll have finite order summands in our group, which correspond to
elliptic pieces, for instance lens spaces, in the JSJ decomposition. These
cases are not hyperbolic.

prop

Let M be an admissible book of I-bundles. Then M must have incompressible
boundary.

% TODO
\textbf{TODO figure out/throw in Ian's explanation of incompressible boundary.}

Now consider a pared structure P on a book of I-bundles. We'll say (M,P) is
admissible if the book of I-bundles M is admissible. We introduce an additional
definition.

\begin{defn}

Let (M,P) be a pared admissible book of I-bundles. We say (M,P) is
\emph{reduced} if it satisfies the following.

\begin{enumerate}

\item Each page is an I-bundle over a surface of \emph{negative} Euler
characteristic. That is, no page has base surface an annulus or Klein bottle.

\item Each spine contains at least three gluing annuli.

\item All page boundary components are glued (there are no "free boundary
components").

\item Each page intersects P in a disjoint union of rectangles. That is, no
component of P is contained in a single page.

\end{enumerate}

\end{defn}

Note that the first requirement forces all boundary components of M to have
negative Euler characteristic. Because components of P are pi1-injective,
P therefore must be a union of annuli in any reduced book of I-bundles.

This definition is motivated by the following fact.

\begin{thm}

Let (M,P) be a pared admissible book of I-bundles. Then exactly one of the
following holds:

\begin{enumerate}

\item There exists a set of reduced pared books of I-bundles (M',P') and
disjoint embeddings (M',P') hookrightarrow (M,P) such that any (QF) surface in
(M,P) is contained in one of the (M',P').

\item (M,P) does not contain a (QF) surface.

\end{enumerate}

\end{thm}

proof.

For any page with base surface an annulus, we can glue the two spine boundaries
together along that page's gluing curves to make a single spine. The other
attached pages of each spine will be carried along by the gluing to form
a single set of attached pages which is the union of the two original sets
(without the page we just eliminated).

For any spine with exactly two pages attached, we can simplify the book of
$I$-bundles structure by attaching the two pages along their boundary annuli to
make a single larger page. This is homeomorphic to the original manifold.

Spines which have exactly one page attached. Cross them off.

Cross off pages with closed parabolics on them. Keep reducing, might break into
multiple pieces. There's a (QF) surface if and only if our reduction terminates
in a nonempty surface.

As motivation, also observe that our specific example in the earlier section is
the simplest possible topological structure for a reduced book of I-bundles.

%%%%%%%%%%%%%%%%%%%%%%%%%%%%%%%%%%%%%%%%%%%%%%%%%
\section{Statement of main theorem}
%%%%%%%%%%%%%%%%%%%%%%%%%%%%%%%%%%%%%%%%%%%%%%%%%

We are now ready to state the precise theorem.

thm

Every reduced book of I-bundles contains a (QF) surface.

Note that together with the earlier theorem, we've fully covered the
non-reduced case also.

thm

Let M be non-reduced book of I-bundles. If the reduction of the earlier theorem
yields at least one reduced book of I-bundles, then M contains a QF surface.
Otherwise, if the reduction terminates in an empty set, it does not.



%%%%%%%%%%%%%%%%%%%%%%%%%%%%%%%%%%%%%%%%%%%%%%%%%
\section{Proof of main theorem - preliminaries}
%%%%%%%%%%%%%%%%%%%%%%%%%%%%%%%%%%%%%%%%%%%%%%%%%

Now there are a number of topological simplifications we make, by passing to an
appropriate finite-sheeted cover of the book of $I$-bundles.  Once we can
construct a surface satisfying \eqref{E:qf} inside this cover, we can push it
down and perturb slightly to obtain a surface satisfying \eqref{E:qf}
downstairs (since it $\pi_1$-injects into a subgroup, it will definitely still
be $\pi_1$-injective).

\begin{defn}

A \emph{good} book of $I$-bundles $M$ is a reduced book of I-bundles which
satisfies the following additional conditions.

\begin{enumerate}

\item Each gluing annulus on a spine traverses longitudinally exactly once. (It
may traverse one or more meridians, but this irrelevant for our proof. Note
that this corresponds to a different, but topologically arbitrary (upto
diffeomorphism) choice of longitude for our spine.)

\item Each page is glued to a given spine at most once.

\item The two endpoints of each fiber in each page are in different boundary
components of M.  In particular, each page is a trivial I-bundle over an
oriented surface.

\end{enumerate}

\end{defn}

\begin{lemma}

Let M be a reduced book of I-bundles. Then M has a finite-sheeted regular cover
which is good.

\end{lemma}

\begin{proof}

To prove this, we'll repeatedly use the fact that pi1M is LERF to construct
finite sheeted covers with nice properties.

We first take care of condition (1). Let C be a spine, and let A1,dots,An be
the gluing annuli on C. These annuli must be parallel. Let [alpha] be
a generator for pi1C (a solid torus). Choose k such that alpha^k is homotopic
in C to each Ai.

Since M is admissible, C cin M is pi1-injective (alpha^k, ie a page gluing
annulus, can't be homotopically trivial in an attached page). So alpha
generates a cyclic subgroup of pi1M.

pi1M is LERF. Let phiC colon pi1M to GC be a map to a finite group such that
phiC(alpha),phiC(alpha^2),dots,phiC(alpha^{k-1}) notin phiC(<alpha^k>). We can
do this by repeatedly applying LERF to get a map for phiC(alpha), a map for
phiC(alpha^2), etc, and taking the product of all these maps. Let phi colon
pi1M to G = prod GC be the product of all these maps of the appropriate orders
for each C.  Again, we elide changes of basepoint.  Let M' be the associated
regular cover of M.

We claim that M' satisfies (1). The spines of M' are precisely connected
components of preimages of spines of M. Let C' be such a spine upstairs, and
C=pi(C') the associated spine downstairs. Let alpha be a generator for pi1C.
Recallthat in our construction above, we chose a value k associated to each
alpha. Let d be the degree of the restricted cover C' to C. Note that d may not
be the degree of the cover M' to M, as pi-1(C) may not be connected. We claim
that k|d.

Suppose not. Then d = m1k - m2, for some m1,m2 in Z, 0 < m2 < k.  Rearranging,
m2 = m1k - d. Choose x0 in alpha, and let x0' be a lift of x0 inside C'. Since
C' to C is a d-fold cover of a solid torus (with cyclic pi1), alpha^d lifts to
C' as a closed curve. That is, phi(alpha^d) is trivial. But this implies
phi(alpha^m2) = phi(alpha^m1k)phi(alpha^-d) = phi(alpha^m1k), contradicting our
LERF assumption that phiC(alpha^m2) notin phiC(<alpha^k>).

So k|d. Write d = m1k. Then each gluing annulus will, when raised to the m1th
power, lift to an embedded annulus which traverses the longitude C' exactly
once.  By regularity of the cover this implies that the union of all lifts of
these annuli is a union of embedded annuli traversing the longitude of C'
exactly once. Since this argument holds for any C' upstairs and C downstairs,
M' satisfies (1).

Note that any further finite-sheeted cover of M' will also satisfy (1), by
looking locally at the lifting of gluing annuli above each spine.

So now assume M is reduced and satisfies (1). We claim M has a finite-sheeted
cover which satisfies (2).

Let B be a page intersecting a spine C more than once. For each pair of gluing
annuli A1,A2 which attach B to C, we'll draw a closed curve as follows. Fix
a meridian disk D of C. Choose an arbitrary point x1 in A1 cap D, x2 in A2 cap
D. Choose an arbitrary arc beta cin B connecting x1 and x2. Let gamma be the
closed curve obtained by closing up beta with an arc along the interior of D.
Orient gamma so that it first traverses this arc through the interior of D, and
then traverses beta.  Fix x1 to be our basepoint. Using LERF, choose a map
phi_C,B,A1,A2 such that phi_C,B,A1,A2(gamma) notin phi_C,B,A1,A2(pi1B). Repeat
this construction for each A1,A2 and each B,C with multiple gluings. Let phi be
the product of all these, and M' the associated regular cover, just like above.
Same basepoint trickery as earlier.

We claim that M' satisfies (2). Suppose not. Let B' be a page upstairs glued to
a spine C' more than once. Then B=pi(B') is also a page, C=pi(C') is a spine.
Since pi is a cover. either the multiple gluings of B' to C' project down to
the same gluing of B to C, or there are multiple gluings of B to C. The first
case is impossible, as all the gluings to a spine in a book of I-bundles must
be parallel, but pi|C' colon C' to C is a covering map of solid tori and
downstairs each gluing only wraps once around the longitude by (1). So there
must be multiple gluings of B to C. Let A1, A2 be annuli, and A1',A2' the
corresponding gluing annuli upstairs. Fixing our basepoint x1 in A1 as above,
x1 has a lift x1' in A1'. Lift gamma to an arc gamma' cin A1' starting at x1'.
Now gamma' first traverses a lifted meridian disk across the spine, meaning it
crosses over to A2'. The remainder of gamma' is a lift of beta beginning at
a point in A2', but beta is contained in the page B, so gamma' must remain
inside the page B' upstairs. So gamma' is an arc with both endpoints in B'.

But B' is a path connected component of pi-1(B). So covering translations of M'
over M which take x1' to other lifts of x1 in B' are precisely the group
elements phi(pi1B), that is, lifts of loops inside B. But gamma lifts to an arc
with both endpoints in B', so phi(gamma) is also such a covering translation.
Hence phi(gamma) in phi(pi1B), a contradiction.

So we have a cover which satisfies (1) and (2). Since property (2) is also
preserved under taking finite-sheeted covers, as discussed above with A1',A2'
lying above A1,A2, it now suffices to show M satisfying (1) and (2) has
a finite-sheeted cover satisfying (3).

M is orientable by definition, so each page must be orientable as well. So the
pages which are not trivial I-bundles over oriented pages must be twisted
I-bundles over nonoriented pages (in order for the resulting page to be
orientable). In particular, the endpoints of the I fibers of a twisted I-bundle
will connect globally to form a single side of the page. So if we look locally
at an I fiber, the two endpoints are guaranteed to be in the same boundary
component of M (since they're in the same boundary component even if we just
look at that page). This explains why ensuring distinct boundary components for
each fiber guarantees trivial I-bundles.

Now we proceed by a similar argument as above. Let B be a page such that fibers
have both endpoints on the same boundary component bd_a M cin bd M. Choose an
arbitrary fiber [0,1] isom I cin B. Let beta cin bd_a M be an arc in the
boundary connecting the endpoints of I. This forms a curve gamma. Fix one
endpoint of I to be our basepoint. Using LERF, choose a map phi_{bd_aM,B} colon
pi1M to G, where G is a finite group, such that phi_{bd_aM,B}(gamma) notin
phi_{bd_aM,B}(pi1bd_aM). Note that M has incompressible boundary, so pi1bd_aM
embeds as a subgroup of pi1M. Also note that pi1bd_aM is finitely generated, as
it's a compact surface. As above, repeat this construction for each page where
fibers have both endpoints on the same boundary component. Take the product phi
colon pi1M to G, and let M' be the associated regular cover.

We claim that M' satisfies (3).

\end{proof}

{\tiny

% TODO
\textbf{TODO figure out what to do with this mini section here. Does any of it
work? Is it necessary.}

We also want to clarify the possible gluing annuli on the spines. Given a solid
torus boundary curve $(l,m)$ that travels $m$ times around a meridian and $l$
times around a longitude, we can ask whether it is permissible to glue a page
along an annulus parallel to this curve. Obviously if we glue our pages along
meridian curves $(0,1)$, we'll have a boundary compression along a parallel
meridian disk.

% I believe all other possibilities are ok, as long as we don't violate the
% one-point intersection constraint Ian mentioned.

\begin{lemma}["One-point intersection lemma"]

Consider a solid torus, or more generally a handlebody, with a collection of
$I$-bundles glued to it along annuli. The resulting manifold has a boundary
compression iff there exists a meridian disk that intersects the boundary
gluing curves exactly once.

\end{lemma}

\begin{proof}

%Not immediately obvious. Use some cut and paste topology? TODO
\textbf{TODO}.

\end{proof}

}
%%%%%%%%%%%%%%%%%%%%%%%%%%%%%%%%%%%%%%%%%%%%%%%%%
\section{Proof of main theorem}
%%%%%%%%%%%%%%%%%%%%%%%%%%%%%%%%%%%%%%%%%%%%%%%%%

We've now reduced to the case of a good book of I-bundles. However, the
remaining work is still quite involved.

This is the main proof, and is quite involved. We proceed somewhat similarly to
the first example. We take covers over the pages and glue them together
cleverly to construct our surface. We check that the parabolic locus arcs on
each page lift to arcs on these covers that do not connect up to form a closed
curve. As in the first example, this suffices to prove that our surface
satisfies \eqref{E:qf}. We first need to take the correct cover over each page.

This is a somewhat elaborate construction. First, we need as our building
blocks covers where it is not so easy to join up the parabolic arcs to make
closed curves. In particular, we use the following lemma.

\begin{lemma}

Let $S$ be a compact surface with boundary, and $P$ a finite set of simple
proper essential arcs on $S$ (not necessarily disjoint).  Then there exists
a finite-sheeted cover $S'$ of $S$ such that the preimage $\pi^{-1}(P)$
consists of arcs each of which connects 2 different boundary components of
$S'$.

\end{lemma}

\begin{proof}

Let alpha cin P be an arc, and suppose that both endpoints of alpha lie on the
same boundary component gamma cin bd S. Fix a basepoint x0 in gamma.  Homotope
alpha by moving both endpoints to x0 along gamma. (The choice of homotopy is
arbitrary). This gives a closed curve beta through x0. Since alpha is simple
and essential, [beta] notin <[gamma]> cin pi1S. If beta were a multiple of
gamma, we could isotope it to a neighborhood of gamma, and then undoing the
earlier homotopy alpha would have to be non-essential (isotoping the endpoints
around to "unwind the spiral".

pi1S is free, so it's LERF. So there exists a map phialpha to a finite group
Galpha, such that phialpha([beta]) notin phialpha(<[gamma]>). Repeat this
construction for each alpha cin P. Let G = prod alpha Galpha, and phi : pi1S to
G the product map. Since G is finite, phi induces a finite-sheeted regular
cover pi : S' to S. We claim that pi-1(P) consists of arcs connecting two
different boundary components of S'.

Note that we're eliding the fact that these maps phialpha are defined on pi1S
with different basepoints. However, if we choose an arbitrary path connecting
each additional basepoint to a fixed basepoint we choose, the induced
isomorphisms on pi1S carry over the nice cover lifting properties between
different basepoints. Our curves beta and gamma still correspond to elements of
pi1 which lift in the appropriate way, once we do a reverse change of basepoint
upstairs. So this is not a problem.

Obviously if alpha cin S is an arc which already connects two distinct boundary
components downstairs, the arc covering it must also connect two boundary
components (the possible boundary component preimages are disjoint). So we need
only consider alpha with both endpoints on the same boundary component gamma.
Fix x0 as above, and let tildex0 be an arbitrary lifted basepoint in S'.  Look
at a lift of beta to a path tildebeta starting at tildex0. tildebeta must end
at some lift of x0.

Since S' is a regular cover, the covering transformation group Gamma is
precisely image(phi). Since phi([beta]) is nontrivial, it corresponds to
a covering transformation. This transformation must map tildex0 to the other
endpoint of tildebeta, by basic covering space theory. But the boundary
component of S' containing tildex0 is a finite-sheeted cover of gamma, and all
the lifts of tildex0 contained in this boundary component are connected to
tildex0 by a lift of a finite power of gamma. That is, tildex0 is mapped to
these other points by transformations corresponding to powers of phi([gamma]).
But since phi([beta]) notin phi([gamma]), the phi([beta]) transformation can't
be any of these. So it must send tildex0 to a point in a different boundary
component. tildebeta obviously connects the same boundary components as a lift
of alpha to tildex0, because we just homotope along the boundaries. So the lift
of alpha to tildex0 connects two different boundaries. S' is a regular cover,
so this applies to any lift. This completes the proof.

\end{proof}

\begin{proof}[Proof of Main Theorem]

% d = lcm degree of subgp sep cover (lcm of degree on each page)
% After taking copies of each subgp sep cover to ensure all are same degree:
% n = (large) # of copies of each cover above
% m = # of copies of each bc to reserve
% i(M,P,S_a) = max # incident arcs at a bc on S_a boundary component
% TODO call this ia
% j(M,P,S_a) = total # bcs on page boundaries making up S_a ("junctions")

% TODO
\textbf{ TODO This notation is completely different from my other notation! Be
clear and consistent.}

Consider an arbitrary boundary component $S_a$ of the book of $I$-bundles $M$.
Cutting $M$ along the gluing annuli decomposes $\bd_aM$ into a union of pieces
each of which is one side of an $I$-bundle page. Call these pieces $S_{a,k}$.
Then the page is $S_{a,k}\times I$, with the piece embedded as $S_{a,k}\times
0$ or $S_{a,k}\times 1$. Without loss of generality let it be $S_{a,k}\times
0$. We can look at the pattern of arcs $P_{a,k} = P \cap (S_{a,k}\times 0)$ on
that side of the page, which is part of $\bd_aM$. We can also look at the pattern
of arcs on the opposite side of the page $\overline{P_{a,k}} = P \cap
(S_{a,k}\times 1)$, and temporarily view it as living inside $S_{a,k}$
(flattening $S_{a,k}\times I \to S_{a,k}$).  For each piece, apply the lemma
above to ($S_{a,k},P_{a,k} \cup barP_{a,k}$) to obtain a cover $S'_{a,k}$. If we
view this as a cover of $S_{a,k}x0$ and $S_{a,k}x1$, it has the property that
any arc in $P_{a,k}$ or $barP_{a,k}$ lifts to an arc connecting two different
boundary components of $S'_{a,k}$.

Let $d_{a,k}$ be the degree of this cover. Let $d = lcm a,k d_{a,k}$.

Fix some large integer $n>0$. Now for each $k$, let $\pi_{a,k}
\colon \widetilde{S_{a,k}} \to S'_{a,k} \to S_{a,k}$ be a disjoint union of $nd/d_{a,k}$
copies of $S'_{a,k}$. The purpose of the $d/d_{a,k}$ normalization is to make sure
each $\pi_{a,k}$ is a cover of the same degree, $n*d$.  We'll call the connected
components of this cover "pieces". We can also combine these into a single
cover $\pi \colon \bigsqcup \widetilde{S_{a,k}} \to \bigsqcup S_{a,k}$

Fix a large integer $m>0$. Now, reserve a subset $\cC$ of the boundary
components of $\bigsqcup \widetilde{S_{a,k}}$.  This subset should have the following
properties:

(1) Each boundary component of each $S_{a,k}$ is covered by exactly $m$ elements
of $\cC$.

(2) Every connected component of $\bigsqcup \widetilde{S_{a,k}}$ contains at most one element
of $\cC$.

This is possible as long as $n \geq mb$, where $b$ is the maximum number of
boundary components of any $S_{a,k}$. If $n\geq mb$, we know $d/d_{a,k} \geq
1$, so each disjoint union $\widetilde{S_{a,k}}$ has at least $mb$ many
disjoint copies of each $S'_{a,k}$.  So we have enough copies to make our choices
above that $S_{a,k}$ disjoint.

Each $S_{a,k}$ is homeomorphic to the core surface of its page, so by the covering
lemma any closed surface formed by gluing the $\widetilde{S_{a,k}}$ along appropriate
boundaries will induce an immersed $\pi_1$-injective surface in $M$. This is
our strategy.

Write $\widetilde{P_{a,k}}=\pi_{a,k}^{-1}(P_{a,k})$. As constructed, $P_{a,k}$ consists of proper arcs
connecting two different boundary components of $\widetilde{S_{a,k}}$.

We first glue the pieces "along" $\bd_aM$ in the following sense. Given any two
pieces $S_{a,k_1}$, $S_{a,k_2}$ in the decomposition of $\bd_aM$, we know that
$S_{a,k_1}$ has some boundary components $\gamma_1,\gamma_2,\dots \cin
dS_{a,k_1}$, and $\gamma_1',\gamma_2',\dots \cin dS_{a,k_2}$ such that before we
cut $\bd_aM$ apart, $\gamma_1$ was glued to $\gamma_1'$, $\gamma_2$ to
$\gamma_2'$, etc.  That is, these components form the locus we glue $S_{a,k_1}$
and $S_{a,k_2}$ together along when we glue the $S_{a,k}$ to form $\bd_aM$.  For
each such pair, look at the boundary components of $\widetilde{S_{a,k}}1$ and
$\widetilde{S_{a,k}}2$.  We allow ourselves to glue components of
$\pi^{-1}(\gamma_1)$ to components of $\pi^{-1}(\gamma_1')$, components of
$\pi^{-1}(\gamma_2)$ to components of $\pi^{-1}(\gamma_2')$, etc. We forbid
ourselves from gluing components of $\pi^{-1}(\gamma_1)$ to any other lifted
boundary components of $S_{a,k_2}$ or of any other pieces. Only on the very last
step of our construction will we break this rule.

Note that since our gluing is locally a cover we can only glue each pair of
boundary components upstairs in one possible way - we could choose a different
gluing, but this would just correspond to a different choice of cover by the
covering lemma.

We introduce some notation. Bbegin by taking $\widetilde{S}^{(0)} = \bigsqcup \widetilde{S_{a,k}}$ to
be a disjoint union of these covers - that is, we haven't done any gluing yet.
Think of these as pieces we'll use to build our cover. At each step in the
gluing, we have a gluing map $f(i+1) \colon \widetilde{S}^{(i)} \to
\widetilde{S}^{(i+1)}$ by attaching two boundary comonents of
$\widetilde{S}^{(i)}$. Write $\widetilde{P}^{(i)}$ for the arcs and curves on
$\widetilde{S}^{(i)}$ induced by gluing the $\widetilde{P_{a,k}}$.  That is,
$\widetilde{P}^{(i)} = f(i) \circ ...  \circ f(1) (\bigsqcup
\widetilde{P_{a,k}})$.

We'll want to glue boundary components other than $\cC$ first - these are the
"reserve" that we'll only use at the end. This guarantees that our final glued
surface will have leftover boundary components that are sufficiently far apart.
By abuse of notation, we'll also use $\cC$ to refer to the induced set of
boundary components in each $\widetilde{S}^{(i)}$.

In addition to gluing along $\bd_aM$ and avoiding gluing members of $\cC$, we want
to preserve the following two invariants.

% FIXME decide what to call it in the paper...  still need to figure out how
% I'm going to do the references too

\begin{enumerate}

\item[(\dag)] $\widetilde{P}^{(i)} \cin \widetilde{S}^{(i)}$ contain no closed
curves.  That is, it is a union of proper arcs. \label{I:dag}

\item[(\dag')] No arc of $\widetilde{P}^{(i)}$ begins and ends at the same
boundary component of $\widetilde{S}^{(i)}$. \label{I:dag'}

\end{enumerate}

By the use of the lemma and the assumption of the theorem (no closed curves in
individual pages), $\widetilde{S}^{(0)}$ satisfies (\dag) and (\dag') We
repeatedly make an arbitrary gluing that lies along $\bd_aM$ in the above
sense, doesn't glue any elements of $\cC$, and preserves (\dag) and (\dag').  This
process will terminate at some $\widetilde{S}^{(N)}$, after there are no more
gluings left to make. $\widetilde{S}^{(N)}$ is a surface with boundary, because
there are "leftover" boundary components that we couldn't glue without
violating one of the above conditions.

% FIXME restructure into lemmas?
%Claim.

We claim that $\#(d\widetilde{S}^{(N)} - \cC)$ is bounded by $C(M,P,S_a)$.  In
particular, this bound is independent of the choice of $n$ and $m$ made
earlier.

%Proof of Claim.

Consider an arbitrary boundary component $\widetilde{\gamma} \cin
d\widetilde{S}^{(N)} - \cC$.  It lives above some boundary component $\gamma
\cin dS_{a,k}$. Look at the number of arcs of $P_{a,k}$ which are incident to
$\gamma$. This number depends only on $M$ and $P$ (not on $n$). Let $i_a
= i(M,P,S_a)$ be the maximum such number of incident arcs of $P$ for any
boundary component $\gamma$ of any $S_{a,k} in S_a$. Let $j_a = j(M,P,S_a)$ be
the total number of boundary components of the $S_{a,k}$.

Let $\gamma' \cin dS_{a,k}'$ be the boundary component matching $\gamma$. We
consider gluing $\gamma$ along $M$ to any component of $\bd\widetilde{S}^{(N)}
- \cC$ above $\gamma'$.  Because we always glue matching boundary components
(and we started with the same number of each), $\bd\widetilde{S} - \cC$ has the
same number of leftover boundary components above $\gamma'$ as it does above
$\gamma$. So there must be at least one to glue to.  The only reason we'd be
unable to glue is if gluing to any leftover lift of $\gamma'$ violates (\dag) or
(\dag').

Let $\widetilde{\gamma}' \cin \bd\widetilde{S}^{(N)} - \cC$ be an arbitrary
leftover lift of $\gamma'$.  Suppose gluing to $\widetilde{\gamma}'$ violates
(\dag). Let $\alpha$ be a closed curve produced by the gluing.  Obviously
$\alpha$ intersects the $\widetilde{\gamma} = \widetilde{\gamma}'$ gluing
circle, otherwise $\widetilde{S}^{(N)}$ would already violate (\dag).  Cutting
along this circle divides $\alpha$ into a union of arcs with endpoints on
either $\gamma$ or $\gamma'$.  $\widetilde{S}$ satisfies (\dag'), so no arc can
have both endpoints on the same boundary component. So there exists an arc
connecting $\gamma$ to $\gamma'$. This is the only way that an additional
gluing would violate (\dag). But since $\widetilde{S}^{(N)}$ is locally a cover
on each piece, the number of arcs incident to $\gamma$ is bounded by $i_a$.
Tracing these arcs through $\widetilde{S}^{(N)}$, they can hit at most $i_a$
other boundary components.  These are the only boundary components we can glue
to to violate (\dag).

Similarly, suppose gluing to $\widetilde{\gamma}'$ violates (\dag'). Let $\alpha$ be an
arc with endpoints on the same boundary component produced by the gluing. Let
$\widetilde{\delta}$ be this boundary component. Again, $\alpha$ intersects the
$\widetilde{\gamma} = \widetilde{\gamma}'$ gluing circle, otherwise $\widetilde{S}^{(N)}$ would already
violate (\dag').  Cutting along this circle divides $\alpha$ into a union of
arcs.  Except for the two original endpoints (which now lie on two different
subarcs - call these subarcs $\alpha_0$ and $\alpha_1$), all other endpoints of
these arcs must either lie on $\gamma$ or $\gamma'$.  But no arc can have both
endpoints on the same boundary component, or $\widetilde{S}^{(N)}$ would arleady violate
(\dag'). So any subarcs except $\alpha_0$ and $\alpha_1$ must connect $\gamma$ to
$\gamma'$.  By parity we can see that $\alpha_0$ and $\alpha_1$ both have one
endpoint on $\widetilde{\delta}$, but their other endpoints must be different. That is,
either $\alpha_0$ ends on $\widetilde{\gamma}$ and $\alpha_1$ on $\widetilde{\gamma}'$, or vice
versa.

Now the arcs incident to $\widetilde{\gamma}$ hit at most $ia$ other boundary
components. These are our possible $\widetilde{\delta}$. Each of these has at
most $i_a$ incident arcs itself, one of which returns to $\widetilde{\gamma}$,
leaving $i_a-1$ that we need to care about. We have a total of at most
$i_a*(i_a-1)$ many "distance two" leftover boundary components. These are the
only boundary components we can glue to to violate (\dag').

But by construction of $\widetilde{S}^{(N)}$, there are no more legal gluings. So
$\widetilde{S}$ can have at most $i_a+(i_a*(i_a-1)) = i_a^2$ many leftover boundary
components (that is, components of $\bd\widetilde{S}^{(N)}-\cC$) above $\gamma'$.  But since
$\widetilde{S}^{(0)}$ has the same number of boundary components above $\gamma$ and
$\gamma'$, and we're only allowed to glue them to each other, this implies that
there can be at most $i_a$ boundary components above $\gamma$ as well.  Since
our choice of $\widetilde{\gamma}$ was arbitrary, this implies that there are
at most $C(M,P,S_a) = i_a^2*j(M,P,S_a)$ many leftover boundary components, that
is, components of $\bd\widetilde{S}^{(N)}-\cC$.

%Proof of Theorem contd.

Now that we've constructed $\widetilde{S}^{(N)}$, we want to perform some additional
gluings to construct $\widetilde{S}$. In addition to (\dag) and (\dag'), $\widetilde{S}$ should
satisfy

\begin{enumerate}

\item[(\dag'')] Any two boundary components of $\widetilde{S}$ lie on different
pieces. That is, they correspond to boundary components of different connected
components of $\widetilde{S}^{(0)}$ under the gluing map $\widetilde{S}^{(0)}
\to \widetilde{S}$.

\end{enumerate}

Intuitively, this is easy using the reserve. The reserve on its own satisfies
(\dag''), so we only have to deal with the leftovers discussed above.  The trick
is the reserve is much larger than the leftovers, so if we allow ourselves to
glue the leftovers to the reserve we can take care of all the leftovers without
violating (\dag) or (\dag'). Our remaining boundary components will be a subset
of the reserve.

To be precise, construct $\widetilde{S}$ from $\widetilde{S}^{(N)}$ as follows.
For each boundary component $\gamma$ downstairs, as discussed above, there are
at most $i^2$ many elements of $\bd\widetilde{S}^{(N)}-\cC$ above it. For each of
these, there are at most $i^2$ we could glue to above $\gamma'$ that violate
(\dag) or (\dag').  Assume that $m\geq 2i^2$.  Then glue each leftover lift of
$\gamma$ to an element of $\cC$ above gama', one at a time. At any point there
will be at least $i^2+1$ elements of $\cC$ remaining above $\gamma'$, so we'll
always be able to choose one that doesn't violate (\dag) or (\dag'). Repeat
this process for each $\gamma$ to construct $\widetilde{S}$.  Since we glued
all the leftovers, it immediately follows that $\bd\widetilde{S} \cin \cC$, and
therefore $\widetilde{S}$ satisfies (\dag'').

Let's analyze the possibilities for $\#\bd\widetilde{S}$, the number of
boundary components. Above each $\gamma \cin dS_{a,k}$,
$\cC$ has $m$ elements. So $\bd\widetilde{S}$ has at most $m$ components above
$\gamma$, or $m*j$ many boundary components in total. The lower bound is
determined by the maximum number of leftovers, since the only way we'll remove
reserve components from $\bd\widetilde{S}$ is by gluing them to leftover
components. There are at most $i^2$ leftover components above $\gamma$, and
similarly above $\gamma'$, so there will be at least $m - i^2$ components
remaining in $\bd\widetilde{S}$ above $\gamma$. So

\[ m-i_a^2 \leq \#(\bd\widetilde{S} \text{ above }\gamma) \leq m \]

and

\[ (m-i_a^2)j_a \leq \#\bd\widetilde{S} \leq mj_a \]

Finally, suppose we've done the entire above construction for each boundary
component of our book of $I$-bundles $M$ to produce a partially-glued surface
with some boundary components left over. For each boundary component $\bd_aM$,
we'll call the surface $\widetilde{S_a}$. Note that we must be sure to make the same
choice of $n$ and $m$ for all these surfaces. Now we want to glue these
together, but they may have very different numbers of boundary components! So
we'll need to normalize them so they all have the same number of boundary
components above each $\gamma \cin dS_{a,k}$, so we can do a local construction
above a neighborhood of each spine in $M$.

Let $i_M = max a i_a$. By the inequality above, for each $\gamma \cin S_{a,k}$,
$\#(d\widetilde{S_a} \text{ above } \gamma) \geq m - i_M^2$. We want to make
this an equality for each $\widetilde{S_a}$ by performing some more gluings.

Again, we use the size of the reserve to our advantage. Assuming $m\geq i_M^2
+ i_a^2$ for each $\widetilde{S_a}$, by the same argument as above there will
always be choices remaining from the reserve to make the gluing. Combining
these all we need is $m\geq 2i_M^2$.

By abuse of notation, we'll call these normalized surfaces $\widetilde{S_a}$ also (the
non-normalized ones will not come up again). Each has exactly $C = m - i_M^2$
many boundary components above any $\gamma \cin dS_{a,k}$.

Build a single surface $\widetilde{S}$ by gluing the $\widetilde{S_a}$ as follows. Begin with
$\bigsqcup_a \widetilde{S_a}$.

For each spine $M_c$ of $M$, look at all the incident pages. Say there are $q$
of them.  Each page has two sides, so there are $2q$ many pieces $S_{a,k}$
downstairs near $M_c$, each with one boundary component glued along the spine
$M_c$.  Because $M$ is good, we know that all these $S_{a,k}$ are distinct and
they are all glued along simple parallel closed curves on $\bd M_c$ (do we know
they're longitudes?  See above \textbf{ TODO}).

Introduce notation as follows. $\cN(M_c)$ consists of $q$ neighborhoods inside
pages, attached to $M_c$ by annuli. The meridian of $M_c$ gives a cyclic order to
the attached pages.  Fix an orientation for the meridian and an (arbitrary)
starting point along it, and label the pages $1, \dots, q$ under this ordering.
The orientation of the meridian also gives an ordering of the boundary
components of each attaching annulus.  Using this, label the $2q$ boundary
components by $\gamma_1^-,\gamma_1^+,\dots, \gamma_q^-,\gamma_q^+$, where
$\gamma_i^-$ and $\gamma_i^+$ are the two sides of the attaching annulus for
the ith page.

Now consider $\bd M \cap \cN(M_c)$. Notice that the $2q$ boundary components of these
attaching annuli that we just labeled are precisely the $2q$ boundary curves of
the boundary pieces $S_{a,k}$ that are near $M_c$. We don't know how many global
boundary components $a$ are involved, but locally we do know which pieces are
attached inside $\bd M \cap \cN(M_c)$. $\bd M \cap M_c$ consists of q parallel annuli
along $\bd M_c$ that connect adjacent pages together. That is, the first annulus
has boundary $\gamma_1^+ \cup \gamma_2^-$, the second $\gamma_2^+ \cup
\gamma^3_-$, and so on.  It follows that these are the "pairs downstairs" in
the above construction of each $\widetilde{S_a}$. That is, these are the curves
we called $\gamma$ and $\gamma'$ earlier.  So within $\bigsqcup \widetilde{S_a}$,
the existing gluings only glue lifts of $\gamma_1^+$ to lifts of $\gamma_2^-$,
lifts of $\gamma_2^+$ to lifts of $\gamma_3^-$, etc etc.

Finally, the gluing. Because we normalized, each $\bigsqcup_a \widetilde{S_a}$ has
exactly $C$ boundary components above each $\gamma_i^\pm$.  Glue these $2qC$
boundary components as follows. Glue lifts of $\gamma_1^+$ to lifts of
$\gamma_3^-$, lifts of $\gamma_2^+$ to lifts of $\gamma_4^-$, and so on. In
general glue lifts of $\gamma_i^+$ to lifts of $\gamma_{i+2}^-$. Within each
subset of lifts, choose the gluing arbitrarily. We know the numbers on each
side are equal, so they'll match up.

Every boundary component of $\bigsqcup_a \widetilde{S_a}$ lives above a boundary component
of some $S_{a,k}$ which means it is attached to some core $M_c$. So once we've done
the above gluing step for all cores, the resulting surface $\widetilde{S}$ is closed.

%Claim.

Finally, we claim $\widetilde{S}$ satisfies \eqref{E:qf}.

%Proof of Claim.

By construction, $\widetilde{S}$ consists of covers of each page of $M$ glued along
parallel longitudinal annuli at the spines of $M$. By the covering lemma, it
induces an immersed $\pi_1$-injective surface. It remains to show that
$\widetilde{S}$ cannot contain any lifts of parabolic curves.

Let $\alpha$ be a parabolic curve in some boundary component $S_a$ of $M$. Cut
$\alpha$ into arcs $\alpha_i$ along the page boundaries, so each $\alpha_i \cin
S_{a,k_i}$.  We can think about this as follows. For each $\alpha_i$, there are two
kinds of lifts to $\widetilde{S}$. We can view it as an arc in $P_{a,k_i} \cin
S_{a,k_i}$, and lift it to a cover of $S_{a,k_i}$, that is, a component of
$\widetilde{S_{a,k_i}}$. Since $\alpha \cin S_a$, this will necessarily be part
of $\widetilde{S_a} \cin \widetilde{S}$. Furthermore, by construction of
$\widetilde{S_a}$, lifting the $\alpha_i$ to $\widetilde{S_a}$ pieces and
gluing cannot yield closed curves.  It only yields a union of arcs, by (\dag).

However, the second way we can lift $\alpha_i \cin S_{a,k_i}$ is to a cover of the
opposite page boundary $\overline{S_{a,k_i}}$. Let $S_b$ be the boundary component
containing $\overline{S_{a,k_i}}$, and write $S_{b,l_i} = \overline{S_{a,k_i}}$, we can see that $\alpha_i$
in $\overline{P_{b,l_i}}$, the set of "opposite parabolic arcs" that we defined earlier.
So $\alpha_i$ lifts to $S'b,li$ as a union of arcs, each of which connects two
different boundary components of $S'b,li$. This produces a set of arcs in
$\widetilde{S_b}$. Note that it may be possible to have $S_a = S_b$, depending on
how our book of $I$-bundles is constructed.  Regardless, we want to think of
these as a different kind of lift, because we're using the opposite page side
to lift, rather than the side the arc "naturally lives on."

We now claim that these arcs cannot be used to form closed curves in $\widetilde{S_b}$.
Intuitively, this is because the opposite side of the boundary does not remain
parallel to $\alpha$ for long enough, so we'll soon reach a piece of $\widetilde{S_b}$
where our lift can't continue.  To be precise, let $\widetilde{\alpha_i}$ be a lift of
$\alpha_i$ to $\widetilde{S_b}$.  Suppose $\alpha_i$ has an endpoint at a spine $M_c$. If we
look at how the boundary components of a book of $I$-bundles behave near
a spine, we that locally the two boundary sides of a page are attached to sides
of two different pages.  In the terminology we used earlier for gluing pages at
a spine, $\gamma_i^-$ and $\gamma_i^+$ are attached to the attaching annulus boundary
curves of two different pages: $\gamma_i^-$ attaches to $\gamma_{i+1}^+$, but
$\gamma_i^+$ attaches to $\gamma_{i-1}^-$.  Since there are at least three pages at
$M_c$ by (3), the $(i-1)$st and $(i+1)$st pages are distinct.

Suppose that the endpoint of $\alpha_i$ we're considering lies on $\gamma_i^-$. That
is, $\gamma_i^-$ is a boundary component of $S_{a,k_i}$, in our other notation. And
$\gamma_i^+$ is a boundary component of the opposite page $\overline{S_{a,k_i}} = S_{b,l_i}$.
Since $\gamma_i^-$ attaches to $\gamma_{i+1}^+$, any lifted arc we attach
$\widetilde{\alpha_i}$ to in that direction must lift from that $(i+1)$st page at $M_c$.
But in $\widetilde{S_b}$, we attach the surface pieces according to the gluings needed
for the $S_b$ boundary component.  Locally, $\widetilde{\alpha_i} \cin \widetilde{S_b},li$, where
lifts of $\gamma_i^+$ only attach to lifts of $\gamma_{i-1}^-$. But $\gamma_{i-1}^-$ is
a boundary component of a different page, so $\widetilde{\alpha_i}$ connects to a next
page with no lift of the next segment of $\alpha_i$. So there is no possible way
to continue $\widetilde{\alpha_i}$.  The same argument applies at the opposite endpoint
of $\widetilde{\alpha_i}$. This argument shows that lifts of arcs $\alpha_i$ as "opposite
parabolic arcs" to $S_b$ have no "opposite arcs" on either side that they can
glue to.

Finally, suppose our parabolic curve $\alpha \cin S_a$, after cutting into
$\alpha_i$, lifts to pieces $\widetilde{\alpha_i}$ which form a closed curve $\widetilde{\alpha}
\cin \widetilde{S}$.  Cut $\widetilde{\alpha}$ along only the gluings between different
$\widetilde{S_b}$ done in the final gluing step. Each piece $\widetilde{\alpha_j}$ now consists
of multiple $\widetilde{\alpha_i}$, which are either all "natural lifts" or all
"opposite lifts" (in the above sense). This is simply because each $\widetilde{S_b}$
is, except for its extra boundary components, a cover of the corresponding
$S_b$, and there's no way $\alpha$ to locally jump from being on the opposite
side of a page from $S_b$ to suddenly being on the same side. However, at the
gluings in the final step, we attach covers of different boundary components
together, so we may have attached multiple pieces $\widetilde{\alpha_j}$ of the same
type, or different types.

It is impossible to attach two natural segments $\widetilde{\alpha_j}$ in the final
step.  Suppose we had such an attachment, and look locally at the spine where
they're attached. Our attachment must yield a lift of $\alpha$, so we can look
at the local neighborhood in $\alpha$ covered by a neighborhood of our
attachment point.  Near the spine, this neighborhood must live in a single
local boundary component. But as we discussed in detail above, each local
boundary component simply consists of small neighborhoods on two page
boundaries, joined together by an annulus. Locally, a natural segment can only
be obtained by lifting to a cover of one of these two page boundaries, not any
of the other page boundaries near this spine. So one side of our attachment
must lie on one page boundary, and the other side on the other. But by
definition of how we do our final attaching step, we never attach in such a way
that we follow along the local boundary components! This is the point of having
valence at least 3 at each spine, and attaching $\gamma_i^+$ to
$\gamma_{i+2}^-$, is to avoid this problem. Since we never do attachments of
this form, we can't have an attachment with natural lifts on both sides.

So each attachment must have an opposite lift on one or both sides. But recall
that by (\dag''), each piece of each $\widetilde{S_b}$ has at most one free boundary
component. So all the $\widetilde{\alpha_j}$ must traverse at least two pieces, as we
ensured by construction of the $S'_{a,k}$ that they couldn't begin and end at the
same boundary component. This is where we finally use all those
carefully-established earlier criteria.

Because finally, as shown above, opposite lifts of individual $\alpha_i$ cannot
connect to opposite lifts on either side. So it's impossible to construct
a $\widetilde{\alpha_j}$ made out of opposite lifts, since it has to traverse
more than one piece.  This contradicts our assumption that the
$\widetilde{\alpha_j}$ glue to form a closed curve.

So we cannot join up the lifts of $\alpha_i$ to form a lift of $\alpha$. This
completes the proof of the theorem.

\end{proof}

\begin{thm}

Let $M$ be a good book of $I$-bundles.  Suppose $M$ does contain a component of
the parabolic locus $P$ inside a single page's boundary.  Construct $M'$ by
deleting all such pages from $M$, and $P'$ by removing all components of the
pared locus that intersect those pages. Then $(M,P)$ contains a surface
satisfying \eqref{E:qf} if and only if $(M',P')$ is nonempty and contains
a surface satisfying \eqref{E:qf}.

\end{thm}

\begin{proof}

First observe that if $(M,P)$ contains a surface satisfying \eqref{E:qf}, it cannot
traverse any of the pages we deleted to obtain $M'$, by the same argument as in
our first example.  Since our surface exists, $M'$ must be nonempty.
Furthermore, since $(M',P') \cin (M,P)$, it satisfies \eqref{E:qf} for $M',P'$ as well.
Conversely, if we have such a surface, we can obviously view it as contained in
$(M,P)$, where it must satisfy \eqref{E:qf} because it doesn't intersect the deleted
pages at all.

Note that $(M',P')$ may be nonelementary, or it still may not be good.
However, we can repeat the steps needed to guarantee that it's good, and obtain
a manifold where we can either apply this theorem again, or use the first
theorem. It is not immediately clear that this process terminates, because
taking the finite-sheeted cover we use for (2) and (3) makes the manifold more
complicated, possibly increasing the number of parabolics. We'll have to do
this carefully. Possibly we should do this step BEFORE the other
simplification, but then we'll have to make sure the parabolics pass through
that simplification nicely.

\end{proof}

\textbf{ The following sections ended up being unnecessary. I'll keep it here just
in case we need something. }

{\tiny

%%%%%%%%%%%%%%%%%%%%%%%%%%%%%%%%%%%%%%%%%%%%%%%%%
\section{Surface background details}
%%%%%%%%%%%%%%%%%%%%%%%%%%%%%%%%%%%%%%%%%%%%%%%%%

These are some background facts that we'll need to go through the examples in
detail.

Arcs in surfaces with boundary - from Fathi-Laudenback-Poenaru, Exposes 2 and
4 (see p. 21-24, 43-52).
This is an exposition of coordinates for classes of arcs in a surface
with boundary, aka Dehn-Thurston coordinates.

Let $A(N)$ be the set of isotopy classe sof simple closed proper arcs $I\in N$
such that they represent nontrivial elements of pi1 rel boundary. We isotope
them with the ends of the arcs free to move within the boundary components they
are contained in. Similarly $A'(N)$ with simple closed multi-arcs, up to
isotopy.

\begin{thm}[FLP 2.11]

Let $P^2$ be the standard pair of pants. $A(P^2)$ consists of exactly six
elements, classified by the boundary components of their endpoints.

\end{thm}

\begin{thm}[FLP 2.12]

$A'(P^2)$ is isomorphic to $A'={(a1,a2,a3) in Z | sum ai is even }$ via the map
$i:A'(P^2)->A', i(\tau) = (i(\tau, d_1),i(\tau,d_2),i(\tau,d_3))$ where
$d_1,d_2,d_3$ are the boundary components of $P^2$, and $i(\cdot,\cdot)$ is
geometric intersection number.

\end{thm}

That is, simple closed multicurves exist and are uniquely determined, for each
choice of boundary intersection numbers of the correct parity.

Finally, given a closed surface, let $N$, let $S(N)$ be the set of isotopy
classes of simple closed curves on $N$, and $S'(N)$ simple closed multicurves.

\begin{thm}[FLP 4.8]

Let $N$ be a closed surface of negative Euler characteristic.  Fix a pants
decomposition of $N$ into $2g-2$ pairs of pants along $3g-3$ disjoint simple
closed curves $K_1,...,K_{3g-3}$. Then

$S'(N)$ is isomorphic to $B_0 = {(m_i,s_i,t_i) \in \mathbb{Z}, i = 1,...,3g-3
\mid \text{ all coords }\geq 0, \text{ and for each pair of pants the
corresponding }m_i \text{ sum is even}}$ where the $m_i$ measure intersection
numbers with the curves $K1,...,K3g-3$, and the $s_i,t_i$ measure twisting
around at each intersection curve as a rational number.  See FLP for details of
the calculation of the twisting coordinates - we won't be needing it here (at
least, not yet!).

\end{thm}

We actually require a slight generalization of this result, but the proof is
identical.

\begin{thm}

Let $SA'(N)$ be the set of disjoint multi-curves or arcs on $N$, a surface
possibly with boundary. That is, each element of $SA'(N)$ corresponds to
a union of simple closed curves and simple arcs on $N$, all disjoint, such that
the arcs are properly embedded, up to isotopy where the arcs are permitted to
slide on each boundary component. Then $SA'(N)$ is isomorphic to the obvious
choice, where we again decompose into pairs of pants, have $mi$ for each
boundary curve or decomposition curve, but $s_i,t_i$ only for each
decomposition curve.

\end{thm}

Finally, let $A''(N)$ (or maybe some better notation, etc) be the set of simple
proper multi-arcs satisfying the condition that no two components of the
multi-arc are parallel, ie, isotopic. Alternatively we can think of $A''(N)$ as
consisting of simplices in the arc complex of $N$, or as a quotient of $A'(N)$
where we identify multicurves along splitting / joining of parallel arcs. In
general, $A''(N)$ is more difficult to parametrize, as purely from the
coordinates it is difficult to determine which multicurves or multiarcs will
contain parallel components.

We define $S''(N)$, $SA''(N)$ similarly.

\begin{example}

Consider the pair of pants $P^2$. We know $A(P^2)$ has 6 elements, so
$A''(P^2)$ has at most $2^6$ elements, corresponding to which arcs are present
in our multi-arc. It suffices to check that all 6 arcs can be embedded in $P^2$
without intersecting.

\end{example}

\begin{example}

Consider the punctured torus $\Si_{1,1}$. We first compute $SA'(\Si_{1,1})$.
By the generalized Dehn-Thurston coordinates theorem above, (work this out on
paper first).

\end{example}

%%%%%%%%%%%%%%%%%%%%%%%%%%%%%%%%%%%%%%%%%%%%%%%%%
\section{LERF and lifting properties}
%%%%%%%%%%%%%%%%%%%%%%%%%%%%%%%%%%%%%%%%%%%%%%%%%
\textbf{We had trouble with some technical details here, but we ended up not
needing these facts anyways.}

This is relevant to us because manifolds with LERF fundamental groups have nice
topological properties. In particular, we have the following fact.

\begin{thm}

Let $M$ be a compact $n$-manifold such that $\pi_1M$ is LERF. Let $M'$ to $M$
be a (possibly infinite-sheeted) cover such that $\pi_1M'$ is finitely
generated.  Suppose $C \cin M'$ is a compact subset which avoids the boundary
of $M$. Then there exists a intermediate cover $M' \to M'' \to M$, which is
finite-sheeted over $M$, such that the covering map $M' \to M''$ is an
embedding on $C$.

\end{thm}

\begin{proof}

See Long-Reid. Note that their proof is only stated for closed manifolds $M$,
but it carries over to all compact manifolds also if we restrict $C$ to avoid
the boundary of $M$.

% I think? Might want to check w Ian.

\end{proof}

This leads us to the following well-known fact. I haven't found a written proof
in the literature, so I've provided one. Thanks to Ian Agol for pointing this
out.

\begin{thm}

Let $M$ be a compact n-manifold such that $\pi_1M$ is LERF. Let $C$ be
a compact (n-1)-complex, and $\phi \colon C \looparrowright M$
a $\pi_1$-injective immersion.  Then there exists a finite sheeted cover $M'
\to M$ such that $\phi$ lifts to a map $\phi' \colon C \hookrightarrow M'$
which is homotopic to an embedding.

\end{thm}

Note that we can't guarantee that $\phi'$ is itself an embedding, because for
instance it's easy to map a small straight segment of $C$ so that its image
contains a small homotopicall trivial loop. This loop will be present in every
lift of $C$ to a cover of $M$.

\begin{proof}

Note that when we use this, both $C$ and $M$ will be aspherical, so an
alternative proof for that case is to invoke Whitehead's theorem to show
$\phi_H$ is a homotopy equivalence, followed by some obstruction theory facts
(??  - similar to the ones invoked in Long-Reid) to show it can be homotoped to
an embedding.

\textbf{ TODO is this a real proof? not sure...}

%Since C is a compact complex, pi1C is finitely generated. Let H=phi*(pi1C), and
%let pH colon MH to M be the corresponding cover of M. By our choice of H, phi
%lifts to a map phiH colon C to MH. We claim that phiH is homotopic to an
%embedding. Since C is compact and phi is an immersion, it suffices to show we
%can nicely homotope phiH to be injective.
%
%Since pH*(pi1MH) = H = phi*(pi1C), phiH must induce an isomorphism on pi1 in
%order to have pH circ phiH = phi. Since C is an immersed 1-complex, without
%loss of generality the only failures to be embedded we need to consider are
%double points.  Let x1,x2 in C such that phiH(x1)=phiH(x2)=y.  Let alpha cin
%C be a path from x1 to x2.  phiH(alpha) is a closed curve in MH.  If
%phiH(alpha) is homotopically trivial, we'll apply the loop theorem and homotope
%locally to remove the intersection at y.
%
%To be precise, we have two cases to consider. First, if phiH(alpha) is a simple
%closed curve, that is except for its two endpoints alpha embeds in MH. Then we
%can apply the loop theorem directly to obtain a disc D cin MH with boundary
%phiH(alpha). By homotoping across D we can ensure that phiH(C) cap int(D)
%= empyset.

% FIXME Ok HOLD ON this isn't right. Dimensions don't add up. If dim C = 1 and
% dim M = 3 it's trivial! Because just homotope into the extra dimension
% (general position) to fix -> in fact, I think this can be done generically.
%
% So really what I'm concerned with is (1) dim C = 1, dim M = 2. And also (2)
% dim C = 2, dim M = 3.
%
% I hope I don't have to consider these two cases separately...

\end{proof}

We use the above condition repeatedly to show that we can construct
a finite-sheeted cover of our book of $I$-bundles that satisfies certain nice
properties, by starting with an infinite-sheeted cover and then pushing down.

%%%%%%%%%%%%%%%%%%%%%%%%%%%%%%%%%%%%%%%%%%%%%%%%%
\section{Some generalizations}
%%%%%%%%%%%%%%%%%%%%%%%%%%%%%%%%%%%%%%%%%%%%%%%%%

We use the following general notation for books of $I$-bundles. Let
$M_{c1},...,M_{cm}$ be the spines, all solid tori. Let $M_1,...,M_n$ be the
pages, all thickened surfaces. Each $M_i$ has boundary components
$A_{i1},...,A_{ik_i}$ all annuli. We glue each annulus to a boundary annulus on
some spine. For each spine, all these gluing annuli must be disjoint parallel
incompressible. If $M$ is nonelementary, We can assume without loss of
generality that all pages have negative Euler characteristic, and all spines
have at least 3 pages glued to them (otherwise, just consolidate into fewer
pages / spines).

Given such a book of $I$-bundles $M$, let $G=(V,E)$ be the graph associated to
the embedded surface $A = \bigcup_{i,j} A_{ij} \cin M$ which is the union of
all the page gluing annuli. Combinatorially, $G$ has a vertex for each spine or
page, and an edge for each gluing of a page to a spine along an annulus. We
call $G$ the gluing graph of the book of $I$-bundles.

We say a book of $I$-bundles is \emph{tree-shaped} if its gluing graph is
a tree.

As above, if the pared locus $P$ contains any components that fit inside
a single $I$-bundle, then we can ``cross off that page.'' Just like in our
earlier lemma, any $\pi_1$-injective surface that passed through that page
would have would have to contain the component of the pared locus. By
repeatedly crossing off pages and consolidating into fewer pages / spines, we
can reduce any pared book of $I$-bundles to one with no such components.

\begin{conj}

Let $M$ be a tree-shaped book of $I$-bundles. Suppose $P$ has no components
$P_0\cin \bd_\pm M_i$ for any $i$. Then there exists a surface satisfying
\eqref{E:qf}.

\end{conj}

We're quite certain that this is true. It should just be a hairy inductive
adaptation of the above argument. In particular, it is totally unclear how one
might construct a counterexample. However, we haven't completed a full proof
yet.

\begin{thm}

Let $M$ be a tree-shaped book of $I$-bundles. Suppose $P$ has no components
$P_0 \cin \bd_\pm M_i$ for any $i$. Suppose further that every
component of $P$ crosses at least one gluing circle at a spine where at least
4 pages are attached. Then there exists a surface satisfying \eqref{E:qf}.

In particular, note that if all spines are at least valence 4, the technical
crossing condition must be satisfied.

\end{thm}
\begin{proof}

Our construction of $S$ is very simple. In fact, in this case, $S$ embeds in
our book of $I$-bundles. Construct $S$ by starting with a single page's core
surface, and traversing the tree-shaped gluing as follows. For each boundary
component of $S$, look at the valence of the associated spine. If it's valence
3, choose an arbitrary one of the other two pages. If it's valence at least 4,
choose a page that's non-boundary-adjacent to $S$, in that neither boundary
component of the incoming page attaches to a boundary component of the newly
chosen page when we glue the book of $I$-bundles. This is always possible for
valence at least 4, because at a spine, each incoming boundary component
connects to a single outgoing boundary component, and there are only 2 incoming
boundary components but at least 3 new pages to choose from. Attach the core
surface of the chosen page to $S$ with an annulus across the core. Since $M$ is
tree-shaped, we can build $S$ inductively without it running into itself.

We claim $S$ satisfies \eqref{E:qf}. $S$ is immediately properly immersed (in
fact, embedded) and $\pi_1$-injective, as it's a union of page cores and spine
annuli. It suffices to check the pared locus. Suppose some $P_k$ had $\pi_1S
\cap \pi_1P_k neq 1$, ie some multiple of its core curve was homotopic into
$S$. By the crossing condition, $P_k$ must traverse a gluing circle on a spine
of valence at least 4.  But at such a spine, $P_k$, which is contained in
a boundary component of $M$, must connect two boundary-adjacent pages. But we
chose $S$ so it would connect at such a spine to a non-boundary-adjacent page.
Since $M$ is tree-shaped, applying van Kampen shows that these pages correspond
to different pieces of the amalgamated free product for $\pi_1M$, so it's
impossble for this curve to be in $\pi_1S$. This completes the proof.

\end{proof}

Finally, we have one more technical intermediate result. We hoped this would
extend to more cases, but we haven't been able to do anything with it yet.

Let $M$ be a book of $I$-bundles. Associated to every surface we construct in
an analogous way to what we've been doing (taking covers of the page core
surfaces and gluing them together), there's a set of fairly complicated graphs
we can construct as follows. For each $(M_i,A_{ij},\widetilde{S_i})$, where
$M_i$ is a page, $A_{ij}$ gluing annulus of that page, and $\widetilde{S_i}$
finite-sheeted cover of the page core surface, we can draw a \emph{local
pared-arc graph} $G=G(M_i,A_{ij},\widetilde{S_i})$.

The vertices are boundary components of $\widetilde{S_i}$ which sit above
$A_{ij}$. Two vertices are connected by an edge if there is an arc in the pared
locus which lifts to connect those two boundary components. For instance,
a double cover of the punctured torus page, with 3 disjoint non-parallel arcs
on each boundary, has a graph, over the single boundary annulus downstairs,
with 2 vertices and 3 edges (one between the two vertices, and one loop at each
vertex). This graph is a generalization of the cis-trans terminology we used in
that example.

We say a cover $\widetilde{S_i}$ is \emph{pared-bipartite} if each of its local
pared-arc graphs is bipartite. Note that none of the covers we constructed in
our first example are pared-bipartite. This is a very specialized condition.
Note also that if the pared locus does not intersect a page, all covers of that
page are pared-bipartite.

\begin{thm}

Let $M$ be a tree-shaped book of $I$-bundles. Suppose that $P$ has no
components in $\bd_\pm M_i$ for any $i$. Suppose that every page of $M$
has a pared-bipartite cover of its core surface. Then $M$ contains a surface
satisfying \eqref{E:qf}.

\end{thm}
\begin{proof}

First observe that every cover of a pared-bipartite cover is itself
pared-bipartite. Simply lift the partition of the vertices to obtain a new
partition of each local pared-arc graph. The arcs cannot break this
partition because otherwise they wouldn't cover arcs in the downstairs cover.

Choose a sufficiently large cover of a pared-bipartite cover of each page such
that at each spine, the number of incoming cover boundary components above each
incident page equal for that spine. We can do this taking a ``least common
multiple cover,'' unless pages are glued to the same spine multiple times. In
that case, choose a finite-sheeted cover of the book of $I$-bundles itself that
doesn't have this issue, lift the pared locus, and construct a surface there.
Afterwards, we can push it down and it will still satisfy \eqref{E:qf}.

% careful! finite subgp separability covers of tree-shaped things aren't
% necessarily tree-shaped! might need another condition / argument here.

Now, we construct $S$ by gluing these large covers. For each $\widetilde{S_i}$
and downstairs gluing annulus $A_{ij}$ there will be two adjacent annuli in the
associated spine. For each $\widetilde{S_i}$, attach half the boundary
components above $A_{ij}$ to the boundary components coming from each of the
``neighboring'' $\widetilde{S_i}$, via annuli. Use the pared-bipartite
structure to split the set of boundary components above $A_{ij}$ in half. Since
the number of incoming boundary components match at each spine, as constructed
above, the boundary components will match up. We have a ``cover boundary
component bipartition'' above each $A_{ij}$. Note that once we have the
bipartition, the details of how we glue the pieces inside the bipartition are
irrelevant.

We claim the resulting $S$ satisfies \eqref{E:qf}. We already know it's proper
immersed $\pi_1$-injective by the same arguments. To show it doesn't overlap
with the pared locus, observe that by the definition of local pared-arc graphs,
every core curve in the pared structure must correspond to a sequence of arcs
in covers, where each arc either connects boundary components above different
annuli $A_{ij}$, $A_{ij'}$, or it connects 2 boundary components above the same
annulus $A_{ij}$. But it cannot connect two boundary components above the same
annulus, as we glued those according to the pared bipartition in such a way
that the pared arc component, which is locally restricted to a single boundary
component of the $I$-bundles, can't follow. This is a generalization of the
first argument in our simplest case.

Note that if the arc connects boundary components above different annuli, these
annuli must attach to different spines (by our earlier simplifying assumption).
But now because $M$ is tree-shaped, no arc that's been redirected toward
different spines can possibly form a closed loop. This completes the proof.

\end{proof}

We tried to do this proof in the non-tree-shaped case, but it doesn't work!
There are counterexamples of pared-bipartite covers where the obvious fix
relies on messy combinatorics inside the boundary component matching.


%%%%%%%%%%%%%%%%%%%%%%%%%%%%%%%%%%%%%%%%%%%%%%%%%
\section{Next steps}
%%%%%%%%%%%%%%%%%%%%%%%%%%%%%%%%%%%%%%%%%%%%%%%%%

$I$-bundles. In fact, combinatorially, it seems that all books of $I$-bundles
(satisfying an appropriate condition on components of $P$) should admit these
surfaces. But we haven't proved anything yet. The general case is much messier
(combinatorially) than the tree-shaped case, because in addition to cis arcs
we'll also need to consider trans arcs that ``wrap around'' a loop in the
gluing graph. So even ensuring that every closed curve in the pared structure
is broken into arcs, at least one of which is trans, is insufficient.

We chose to study books of $I$-bundles first. Ian thinks the other infinite
volume cases ought to be easier. What about acylindrical manifolds?  Books of
$I$-bundles are somehow the opposite end of the spectrum of acylindrical
- they're glued together along a bunch of cylinders, and the pieces are all
very simple ($I$-bundles).  If we can address the acylindrical and book of
$I$-bundle cases we ought to be able to put these together to solve the entire
problem. Ian thinks maybe the acylindrical case can be addressed by a variant
of the Baker-Cooper argument.

}%tiny

% TODO rewrite
\textbf{ TODO Rewrite next steps after seeing how far we get.}

\bibliographystyle{plain}
\bibliography{refs}

\end{document}
