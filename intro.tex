%%%%%%%%%%%%%%%%%%%%%%%%%%%%%%%%%%%%%%%%%%%%%%%%%
\section{Introduction}
%%%%%%%%%%%%%%%%%%%%%%%%%%%%%%%%%%%%%%%%%%%%%%%%%

Let $N$ be a complete hyperbolic 3-manifold. $N$ can be realized as the
quotient manifold of a Kleinian group $\Gamma=\pi_1N$. This paper is devoted to
the following general question:

\begin{prob}

Does $\pi_1N$ contains any quasi-Fuchsian surface subgroups?

\end{prob}

In this paper, we answer this question in the special case where $N$ is a book
of $I$-bundles. We tackle this problem topologically, representing $N$ with
a pared 3-manifold $(M,P)$.  The pared structure $P\cin\bd M$ specifies the
locus corresponding to cusps of $N$. Finding a quasi-Fuchsian surface subgroup
of $\pi_1N$ amounts to asking whether we can find a closed immersed
$\pi_1$-injective surface in $M$ which avoids $P$ up to homotopy.  That is,
we're trying to choose a surface subgroup of $\pi_1M$ which trivially intersect
any of the peripherial subgroups associated to components of $P$. We prove full
results on which $(M,P)$ admit a surface subgroup in the book of $I$-bundles
case.

To motivate this choice of problem and special case to solve, we briefly
discuss the history of this problem and previous work on it. We begin with
a straightforward, purely topological question:

\begin{prob}

Which 3-manifold groups contain surface subgroups? Equivalently, which
3-manifolds contain closed $\pi_1$-injective surfaces?

\end{prob}

This problem arises quite directly when considering 3-manifold groups. The
first explicit answer was proposed by Waldhausen (cite: Kirby problem list) in
the Surface Subgroup Conjecture: every closed irreducible 3-manifold with
infinite $\pi_1$ contains a surface subgroup. As originally stated, the
Conjecture only applies to closed 3-manifolds. It was mostly solved by
geometrization, as the question is straightforward for Seifert fibered spaces.
After some preliminary efforts in the hyperbolic case by Cooper-Long
\cite{CooperLong} and Li \cite{Li}, it was finally
resolved in general by Kahn and Markovic in 2012 \cite{KM}.  In fact, the
surface subgroups they find are quasi-Fuchsian, and they also obtained
a density result stating that there are many such surfaces \cite{KM2}.

We can naturally ask this question for manifolds with boundary as well.
Cooper, Long, and Reid showed that a compact connected irreducible 3-manifold
with non-empty incompressible boundary  must contain an essential closed
surface, unless it's covered by a product $F\times I$ \cite{CLR}. In
particular, this shows that every non-closed hyperbolic 3-manifold group
contains a surface subgroup.

However, since the result of Kahn and Markovic provided us with surface
subgroups which are quasi-Fuchsian, we might ask whether we can do this in the
non-closed case as well. Note that the Cooper-Long-Reid result is insufficient,
as the surfaces obtained may contain accidental parabolics, that is, overlap in
$\pi_1$ with the pared locus $P$. If this occurs they will not be
quasi-Fuchsian. This does commonly occur in practice --- see \cite{CLRbundles}
and \cite{MenascoReid} for examples. Of course, in the closed case there cannot
be any parabolics to worry about, as the boundary is empty.

% FIXME something is wrong with Masters-Zhang citation / bib entry - I can only
% find their older paper (hyp knot complements) on mathscinet, not the full new
% one with link complements

More recently, there have been efforts to find quasi-Fuchsian surface subgroups
specifically. Masters and Zhang \cite{MZ} showed that every hyperbolic knot
complement contains a quasi-Fuchsian surface subgroup, and later extended this
result to link complements \cite{MZ2}. This fully addresses the case of finite
volume cusped hyperbolic 3-manifolds. An alternate proof was provided by
Baker-Cooper \cite{BC}.

The remaining case is infinite volume hyperbolic 3-manifolds. We So far we've only
addressed a particular example of this case --- books of $I$-bundles. We
conjecture that all books of $I$-bundles admit quasifuchsian surface subgroups,
except for a few obvious negative cases where the pared locus is too large. We
prove this in a few special cases, so far.

