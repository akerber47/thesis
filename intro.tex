%%%%%%%%%%%%%%%%%%%%%%%%%%%%%%%%%%%%%%%%%%%%%%%%%
\section{Introduction}
%%%%%%%%%%%%%%%%%%%%%%%%%%%%%%%%%%%%%%%%%%%%%%%%%

Let $N$ be a complete hyperbolic 3-manifold. $N$ can be realized as the
quotient manifold of a Kleinian group $\Gamma=\pi_1N$. This paper is devoted to
the following general question:

\begin{prob}

Does $\pi_1N$ contains any quasi-Fuchsian surface subgroups?

\end{prob}

In this paper, we answer this question in the special case where $N$ is a book
of $I$-bundles. We tackle this problem topologically, representing $N$ with
a pared 3-manifold $(M,P)$.  The pared structure $P\cin\bd M$ specifies the
locus corresponding to cusps of $N$. Finding a quasi-Fuchsian surface subgroup
of $\pi_1N$ amounts to asking whether we can find a closed immersed
$\pi_1$-injective surface in $M$ which avoids $P$ up to homotopy.  That is,
we're trying to choose a surface subgroup of $\pi_1M$ which trivially intersect
any of the peripherial subgroups associated to components of $P$. We prove full
results on which $(M,P)$ admit a surface subgroup in the book of $I$-bundles
case.

To motivate this choice of problem and special case to solve, we briefly
discuss the history of this problem and previous work on it. We begin with
a straightforward, purely topological question:

\begin{prob}

Which 3-manifold groups contain surface subgroups? Equivalently, which
3-manifolds contain closed $\pi_1$-injective surfaces of negative Euler
characteristic?

\end{prob}

This problem arises quite directly when considering 3-manifold groups. The
first explicit answer was proposed by Waldhausen \cite{Kirby} in
the Surface Subgroup Conjecture: every closed irreducible 3-manifold with
infinite $\pi_1$ contains a surface subgroup. As originally stated, the
Conjecture only applies to closed 3-manifolds. By geometrization, we can
consider two cases: Seifert fibered spaces or, more generally, graph manifolds,
and manifolds with at least one hyperbolic piece. For graph manifolds, the
answer is negative (TODO cite Neumann).  After some preliminary efforts in the
hyperbolic case by Cooper--Long \cite{CooperLong} and Li \cite{Li}, it was
finally resolved in general by Kahn and Markovic in
2012 \cite{KM}.  In fact, the surface subgroups they find are quasi-Fuchsian,
and they also obtained a density result stating that there are many such
surfaces \cite{KM2}.

We can naturally ask this question for manifolds with boundary as well.
Cooper, Long, and Reid showed \cite{CLR} that a compact connected irreducible
3-manifold with non-empty incompressible boundary  must contain an essential
closed surface, unless it's covered by a product $F\times I$. In particular,
this shows that every non-closed hyperbolic 3-manifold group contains a surface
subgroup.

However, since the result of Kahn and Markovic provided us with surface
subgroups which are quasi-Fuchsian, we might ask whether we can do this in the
non-closed case as well. Note that the Cooper-Long-Reid result is insufficient,
as the surfaces obtained may contain accidental parabolics, that is, overlap in
$\pi_1$ with the pared locus $P$. If this occurs they will not be
quasi-Fuchsian. This does commonly occur in practice --- see other work of

Cooper--Long--Reid \cite{CLRbundles} and Menasco--Reid \cite{MenascoReid} for
examples.  Of course, in the closed case there cannot be any parabolics to
worry about, as the boundary is empty.

More recently, there have been efforts to find quasi-Fuchsian surface subgroups
specifically. Masters and Zhang \cite{MZ} showed that every hyperbolic knot
complement contains a quasi-Fuchsian surface subgroup, and later extended this
result to link complements \cite{MZ2}. This fully addresses the case of finite
volume cusped hyperbolic 3-manifolds. An alternate proof was provided by Baker
and Cooper \cite{BC}. The densite result of Kahn--Markovic was also extended by
work of Cooper and Futer \cite{CooperFuter}.

The remaining case is infinite volume hyperbolic 3-manifolds. This paper
presents a preliminary approach to the problem. We outline a general argument
for all infinite volume hyperbolic 3-manifolds. We explicitly solve a special
case, books of $I$-bundles, which we believe will generalize. We prove that all
books of $I$-bundles admit quasi-Fuchsian surface subgroups, except for a few
obvious negative cases where the pared locus is too large. We now present
a brief outline of the topics we'll cover.

In Chapter 2, we provide some brief background on relevant topics. We define
quasi-Fuchsian subgroups and describe various criteria we can use to establish
that a given surface subgroup is quasi-Fuchsian. We define pared 3-manifolds
and explain how to translate these criteria to topological criteria for
surfaces in pared 3-manifolds. We briefly discuss the case of geometrically
infinite hyperbolic 3-manifolds. We restrict to the geometrically finite case,
with reference to hyperbolization.

In Chapter 3, we outline a general procedure to either find a closed
quasi-Fuchsian surface in a pared 3-manifold, or conclude that there is no such
surface.  We decompose the pared 3-manifold along disks and annuli in an
alternating hierarchy, and prove that any quasi-Fuchsian surface ond pared
structure behave nicely under this decomposition. We prove that the hierarchy
is finite and terminates in balls and acylindrical 3-manifolds. We briefly
describe a strategy for the case of an acylindrical. We explain why the case of
an arbitrary pared 3-manifold with no acylindrical pieces is a generalization
of the case of $I$-bundles. We briefly describe how we might generalize the
proof.

In Chapter 4, we prove preliminary facts we will need for the main theorem on
books of $I$-bundles. The quasi-Fuchsian surface construction relies on
subgroup separability arguments to simplify the 3-manifold under consideration.
So we give a brief overview of subgroup separability, and prove key topological
lemmas using it that we'll need later. We also prove many topological
properties of books of $I$-bundles, and quasi-Fuchsian surfaces and pared loci
inside them.  These results refine the general decomposition properties proven
in Chapter 3.  We use these properties to prove a criterion that establishes
whether or not a given closed surface in a book of $I$-bundles is
quasi-Fuchsian.  Finally, we use this criterion to explicitly identify closed
quasi-Fuchsian surfaces in an example book of $I$-bundles. We describe which
pared structures on this example allow closed quasi-Fuchsian surfaces to be
present. This example will motivate our main theorem.

In Chapter 5, we state and prove the main theorem on books of $I$-bundles. We
first describe a reduction process that removes parts of the book of
$I$-bundles that cannot intersect a closed quasi-Fuchsian surface. We state and
prove the negative case of the main theorem --- that is, which pared books of
$I$-bundles do not admit quasi-Fuchsian surfaces.

We then state and prove the positive case, that all other pared books of
$I$-bundles do admit quasi-Fuchsian surfaces. This proof relies on an elaborate
construction. First, we use the separability properties from Chapter 4 to
simplify the topology of the book of $I$-bundles $M$ and its pared locus $P$.
We then construct a surface with boundary corresponding to each boundary
component of $M$, with certain nice properties with respect to $P$. We then
attach these surfaces with boundary together to form a closed surface. Our
construction allows us to use the criterion from Chapter 4 to show that the
resulting surface is quasi-Fuchsian. This will complete the main proof.

Note that all later chapters depend on Chapter 2. However, Chapters 4 and 5 may
be read independently of Chapter 3. Chapter 4 is largely elementary, and
readers familiar with subgroup separability and cut-and-paste arguments may
wish to read only the definitions and the statement of the pared lifting
criterion before moving to Chapter 5.
