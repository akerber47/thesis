%%%%%%%%%%%%%%%%%%%%%%%%%%%%%%%%%%%%%%%%%%%%%%%%%
\section{Separability properties of groups and spaces}
%%%%%%%%%%%%%%%%%%%%%%%%%%%%%%%%%%%%%%%%%%%%%%%%%

We discuss some algebraic properties that will allow us to perform nice
topological constructions. These are standard definitions. See
\cite{Agolsurvey}, \cite{LR}, or \cite{AFW} for an overview and survey of
recent work.

\begin{defn}

Let $G$ be a group. $G$ is \emph{residually finite}, or \emph{RF}, if for any
$g \in G$, there exists a finite index subgroup $H<G$ such that $g \notin H$.

Let $G$ be a group, and $H$ a subgroup of $G$. $H$ is \emph{separable} in $G$
if for any $g \in G$, $g \notin H$, there exists a finite index subgroup $H'<G$
such that $H' \geq H$, and $g \notin H'$.

$G$ is \emph{subgroup separable}, or \emph{LERF}, if all of its finitely
generated subgroups are separable.

\end{defn}

The following equivalences are well-known.  We'll use these facts later in our
proofs of topological properties.

\begin{prop}\label{P:lerfmap}

$G$ is RF if and only if for any $g \in G$ there exists a map $\phi \colon
G \to F$, where $F$ is a finite group, such that $\phi(g) \neq id$.

$G$ is LERF if and only if for any $g \in G$, $H < G$, $g \notin H$, there
exists a map $\phi \colon G \to F$ such that $\phi(g) \notin \phi(H)$.

\end{prop}

\begin{proof}

See \cite{LR}.

\end{proof}

\begin{cor}\label{C:lerfmap'}

$G$ is LERF if and only if for any $g_1,\dots,g_n \in G$, $H<G$, $g_1,\dots,g_n
\notin H$ there exists a map $\phi \colon G \to F$ to a finite group such that
$\phi(g_i) \notin \phi(H)$ for all $i$.

\end{cor}

\begin{proof}

Let $G$ be LERF. For each $g_i$, $g_i \notin H$, so there exists a map $\phi_i
\colon G \to F_i$, a finite group, such that $\phi_i(g_i) \notin \phi_i(H)$.
Now let $F = F_1 \times \dots \times F_n$, and $\phi = \phi_1 \times \dots
\times \phi_n$.  For each $i$, $\phi(g_i)_{(i)} \notin \phi(H)_{(i)}$, so
$\phi(g_i) \notin \phi(H)$.  This proves the forward direction.  The converse
is trivial.

\end{proof}

A priori it is not obvious that any well-known groups are LERF. It is
a classical theorem of Hall \cite{Hall} that free groups are LERF. Peter Scott
\cite{Scott} showed that surface groups are LERF as well. However, we will need
stronger results in this paper. Deep work of Wise \cite{Wise}, building on work
of Haglund--Wise \cite{HaglundWise} and Hsu--Wise \cite{HsuWise} shows that
every non-closed hyperbolic $3$-manifold has LERF fundamental group.  We note
that the closed case has also been settled by Agol \cite{Agol}, incorporating
work of Bergeron--Wise \cite{BergeronWise} and Kahn--Markovic \cite{KM}. Since
we're studying books of $I$-bundles, we will use the former result in this
paper.

Note that Wise's theorem is actually an extremely deep fact. Wise's proof
actually shows that hyperbolic $3$-manifold with boundary groups satisfy
a technical condition, namely that they are virtually compact special.
Additional work by Haglund--Wise \cite{HaglundWise1} demonstrates that
a virtual compact special group is virtually a quasi-convex subgroup of
a right-angled Artin group. Therefore later work by Haglund \cite{Haglund}
applies in this case, showing that these groups are in fact LERF. See Agol's
survey \cite{Agolsurvey} for a more detailed overview.

We now describe some elementary topological consequences of LERF. These exact
statements are new, but most of the following propositions are simple
translations of the conclusions of LERF into topological statements about loops
and covers. In all of the following we assume $X$ is connected, path connected,
and semi-locally simply connected. That is, we make the assumptions needed to
apply elementary covering space theory. \emph{We also assume that $\pi_1X$ is
LERF.}


\begin{prop}\label{P:lerf1}

Fix a basepoint $x_0 \in X$.  Suppose that $Y \cin X$ is a homotopically
nontrivial connected path-connected semi-locally simply connected subspace
containing $x_0$, $\pi_1Y$ is finitely generated, and $\alpha$ is a loop at
$x_0$ which is not homotopic into $Y$ (relative to the basepoint $x_0$). Then
there is a finite-sheeted regular cover $p \colon (X',x_0') \to (X,x_0)$ with
the property that the lift $\alpha'$ of $\alpha$ to $x_0'$ connects two
different connected components of $p^{-1}(Y)$.

\end{prop}

\begin{proof}

We apply Proposition~\ref{P:lerfmap}. Let $g = \alpha$ and $H = \pi_1Y \leq
\pi_1X$.  By assumption, $H$ is finitely generated and $g \notin H$. Therefore
there exists a map $\phi \colon G \to F$ such that $\phi(g) \notin \phi(H)$,
where $F$ is a finite group.  Without loss of generality we can assume this map
is surjective (take its image). Let $H' = ker \phi$. Now $H'$ is a finite index
normal subgroup, so it induces a finite sheeted regular cover $p \colon X' \to
X$.  Let $x_0'$ be an arbitrary lift of $x_0$ to $X'$. We claim that
$(X',x_0')$ has the desired property.  $X'$ is a regular cover, so it has
covering transformation group $F$.  $F$ acts on the fiber $p^{-1}(x_0)$. We
know that any loop $\gamma \cin X$ based at $x_0$ lifts to a loop $\gamma'$
which starts at $x_0'$ and ends at $\phi(\gamma') \cdot x_0'$.  Since
$\phi(\alpha) \notin \phi(\pi_1Y)$, $\alpha$ cannot end at any point that is
connected to $x_0'$ by a lift of a loop in $\pi_1Y$.  Therefore its right
endpoint must lie in a different connected component of $p^{-1}(Y)$.

\end{proof}

\begin{prop}\label{P:lerf2}

Fix a basepoint $x_0 \in X$.  Suppose that $\alpha$ and $\beta$ are
homotopically nontrivial loops at $x_0$ such that $\alpha$ is not homotopic
(relative to the basepoint $x_0$) to a multiple of $\beta$.  Then there is
a finite-sheeted regular cover $p \colon (X',x_0') \to (X,x_0)$ with the
property that, when we consider the lift $\alpha'$ of $\alpha$ to $x_0'$ has
its right endpoint at a point not reachable by lifting multiples of $\beta$ to
$x_0'$.

\end{prop}

\begin{proof}

Let $g = \alpha$ and $H = \langle\beta\rangle$. Since $\alpha$ is not homotopic
to a multiple of $\beta$, $\alpha \notin H$. Apply Proposition~\ref{P:lerfmap}
to find a map $\phi \colon G \to F$ such that $\phi(g) \notin \phi(H)$, where
$F$ is a finite group.  Let $H' = ker \phi$, and $X'$ be the corresponding
finite-sheeted regular cover of $X$. By the same argument as above, a lift
$\alpha'$ of $\alpha$ to a lifted basepoint $x_0'$ cannot end at any point that
is connected to $x_0'$ by a lift of a multiple of $\beta$. This completes the
proof.

\end{proof}

\begin{prop}\label{P:lerf3}

Fix a basepoint $x_0 \in X$.  Suppose that we have a homotopically nontrivial
loop $\alpha$ at $x_0$ of infinite order in $\pi_1X$.  Then given an integer
$k>0$, there is a finite-sheeted regular cover $p \colon (X',x_0') \to (X,x_0)$
with the following property.  Consider the subset $C$ of $p^{-1}(\alpha)$
obtained by lifting multiples of $\alpha$ to $x_0'$. Let $d$ be the degree of
the restricted covering map $p|C \colon C \to \alpha$. Then $k \mid d$.

\end{prop}

\begin{proof}

Let $H = \langle\alpha^k\rangle$. Since $\alpha$ has infinite order,
$\alpha,\alpha^2,\dots,\alpha^{k-1} \notin H$. Apply Corollary~\ref{C:lerfmap'}
to find a map $\phi \colon G \to F$ such that
$\phi(\alpha),\dots,\phi(\alpha^{k-1}) \notin \phi(H)$. Let $H' = ker \phi$,
and $p \colon X' \to X$ be the associated finite-sheeted regular cover.  We
claim that $X'$ has the desired property. To see this, observe that $d$ is the
smallest integer such that $\alpha^d$ lifts to a closed curve in $X'$. $C$ is
a finite-sheeted cover of the loop $\alpha$, so it must be cyclic. Because $X'$
is regular, this means $d$ is the smallest integer such that $\phi(\alpha)^d$
is trivial, that is, the deck transformation induced by $\alpha$ has order $d$.

We cannot have $d<k$, as $\phi(\alpha^d)=\phi(\alpha)^d=1 \in \phi(H)$
contradicts the LERF assumption on $\phi$. Write $d = m_1k + m_2$, for some
$m_1,m_2 \in Z, 0 \leq m_2 < k$.
$\phi(\alpha^d)=1=\phi(\alpha^{m_1k})\phi(\alpha^{m_2})$.  That is,
$\phi(\alpha^{m_2})=\phi(\alpha^k)^{-m_1}$. The right-hand side is in
$\phi(H)$.  By the LERF assumption on $\phi$, this forces $m_2=0$. So $d \mid
k$.

\end{proof}

We also have the following propositions, which generalize the above
propositions to multiple simultaneous basepoints and loops/subspaces at those
basepoints. Note that the following three propositions are equivalent to saying
that the three properties above are preserved under taking covers.

\begin{prop}\label{P:lerf1'}

Consider a finite collection of basepoints $x_1,\dots,x_n$ with corresponding
loops $\alpha_i$ and subspaces $Y_i$ at each basepoint $x_i$. Assume all the
$Y_i$ are homotopically nontrivial connected path-connected semi-locally simply
connected subspaces with finitely generated fundamental group. Then there is
a finite-sheeted regular cover $X' \to X$ such that each pair $\alpha_i$,$Y_i$
satisfies the conclusion of Proposition~\ref{P:lerf1}.  Note that $X'$ is
regular, so the choice of lifted basepoints is arbitrary.

\end{prop}

\begin{prop}\label{P:lerf2'}

Consider a finite collection of basepoints $x_1,\dots,x_n$ with corresponding
loops $\alpha_i$,$\beta_i$ at each basepoint $x_i$. Then there is
a finite-sheeted regular cover $X' \to X$ such that each pair
$\alpha_i$,$\beta_i$ satisfies the conclusion of Proposition~\ref{P:lerf2}.

\end{prop}

\begin{prop}\label{P:lerf3'}

Consider a finite collection of basepoints $x_1,\dots,x_n$ with corresponding
loops $\alpha_i$ at each basepoint $x_i$. Then given integers
$k_1,\dots,k_n>0$, there is a finite-sheeted regular cover $p \colon X' \to X$
such that each pair $\alpha_i$,$k_i$ satisfies the conclusion of
Proposition~\ref{P:lerf3}.

\end{prop}

\begin{proof}

All these lemmas are proved the same way. Fix a basepoint $x_0$ once and for
all.  Since $X$ is connected, choose a path from $x_0$ to $x_i$ for each $x_i$,
and use this to fix an isomorphism $\sigma_i \colon \pi_1(X,x_0) \cong
\pi_1(X,x_i)$.  For each $x_i$, we apply the appropriate proposition above
(\ref{P:lerf1}, \ref{P:lerf2}, or \ref{P:lerf3}) to construct a map $\phi_i
\colon \pi_1(X,x_i) \to F_i$ with the appropriate property. Now let $F = F_1
\times \dots \times F_n$ and $\phi = (\phi_1 \circ \sigma_1) \times \dots
\times (\phi_n \circ \sigma_n)$.  By the same argument as
Corollary~\ref{C:lerfmap'}, this cover will have the appropriate property for
all $n$ conditions.

\end{proof}
