%%%%%%%%%%%%%%%%%%%%%%%%%%%%%%%%%%%%%%%%%%%%%%%%%
\section{Topology of books of I-bundles}
%%%%%%%%%%%%%%%%%%%%%%%%%%%%%%%%%%%%%%%%%%%%%%%%%

In this section we establish basic definitions and facts about books of
I-bundles and quasi-Fuchsian surfaces contained inside them.

Recall that we are considering possible pared structures $P$ on $M$ a convex
hyperbolic 3-manifold. Our goal is to find a surface $S \cin M$ such that

\begin{equation}\label{E:qf}
S \text{ is closed immersed $\pi_1$-injective, and $\pi_1P_k \cap \pi_1S = 1$
for each component $P_k$ of $P$} \tag{QF}
\end{equation}

Note that since we haven't fixed a basepoint, these subgroups are really only
defined up to conjugacy (that is, they're sets of free homotopy classes) - what
we're saying is they fail to intersect for an arbitrary choice of conjugacy
class for each subgroup. This corresponds to multiples of the parabolic curves
being freely homotopic into the immersed image of the given surface

\begin{defn}

An \emph{$I$-bundle} is a fiber bundle with fiber $I=[0,1]$, and base space
a compact surface with boundary. We'll refer to this as its \emph{base
surface}.  In this paper we adopt the convention that all $I$-bundles are
oriented.  The boundary of an $I$-bundle $I to B to Si$ decomposes into the
\emph{side boundary} $\pi^{-1}(\bd Si)$ and the \emph{binding boundary} $cup
\bd \pi^{-1}(x)$. Note that the binding boundary is a collection of disjoint
annuli. The reason for these terms is the following definition.

A \emph{book of $I$-bundles} is a 3-manifold obtained from the following
construction. Let sB be a collection of I-bundles, and sC a collection of solid
tori. Attach the I-bundles to the solid tori by attaching some (or all) of the
binding boundary components to disjoint annuli in the boundaries of the solid
tori.  If we let sA be the union of glued annuli in the resulting manifold, our
manifold can be written as $M=sB cup_sA sC$. Note that by construction each
component of sA is properly embedded, 2-sided, and incompressible in M.

\end{defn}

% cite/ref Morgan or Thurston or Culler-Shalen (or Ian's work?) for this def

Intuitively, if we take a physical book with thick pages and imagine bending it
in a circle so its top and bottom are identified, the spine becomes a solid
torus.  If we also attach each page along its top and bottom, the pages become
thickened annuli, attached to the spine at one end. This manifold can be built
from the above construction. Take $sC$ to be a single solid torus, and $sB$ to
be a union of thickened annuli, one for each page of the book. Glue one gluing
annulus of each component of $sB$, and glue them all to parallel longitudinal
annuli in $\bd sC$.

Therefore, in keeping with the "book" terminology, we will generally refer to
the components of sB as \emph{pages}, the components of sC as \emph{spines},
and the components of sA as \emph{bindings} or \emph{binding annuli}.
Furthermore, we'll say that M is \emph{fully bound} if \emph{all} the binding
boundary components are glued. For reasons we'll see later, the books of
I-bundles that we want to consider will satisfy this criterion.

Algebraically, a book of I-bundles corresponds to a particularly simple graph
of groups. The graph is bipartitie, and one side of the partition (the spines)
are just copies of $Z$.

In this paper we impose the following standard conditions on a book of
$I$-bundles, to avoid elementary or degenerate cases.

\begin{enumerate}

\item The underlying manifold is a compact connected orientable irreducible
hyperbolic 3-manifold with incompressible boundary.

\item Each page is an I-bundle over a surface with boundary of negative Euler
characteristic.  That is, no base surface is a sphere, projective plane, disk,
Mobius strip, or annulus.

\end{enumerate}

Let's briefly explain why. Obviously we want M to be compact connected
orientable hyperbolic, as this is the general case we're studying (Kleinian
manifolds). If M were nonorientable we could reduce to the orientable double
cover. Hyperbolic 3-manifolds must be irreducible.

If M had compressible boundary, let D be a compressing disk for the boundary.
Any closed pi1-injective surface S cin M that intersects D must do so in
a union of closed curves, since S is properly embedded. Every such closed curve
$\alpha$ is homotopically trivial in M, because it's a closed curve in the disk
D.  But S is pi1-injective, so $\alpha$ must be homotopically trivial in S as
well. Since S is a surface, contracting the loop $\alpha$ yields a disk $D'\cin
S$. Note that D' may be immersed, but we can use it to produce a smaller disk
D'' which is embedded and still bounded by $\alpha$. Combining D and D' yields
a sphere S2 cin M. By irreducibility of M, this bounds a ball. Using a standard
innermost disk argument, we can homotope S across these balls (starting from
the innermost and working our way out) until S and D are disjoint.  This proves
that any quasi-Fuchsian surfaces are essentially disjoint from compression
disks. Therefore, since we're interested in quasi-Fuchsian surfaces, we might
as well compress as much as possible before looking for surfaces.

Each page's base surface must have at least one boundary component to glue to
in order for M to be connected. We don't want to consider base surfaces which
are disks.  Observe that each page which is a copy of $S0,1$ means that that
page and its attached spine form a 3-manifold with a finite-sheeted cover by
a ball (we can arrange things so that in the cover, the $S0,1$ attaching map
only traverses the longitude once, and we get a thickened disk).  This means
that we'll have finite order summands in our group, which correspond to
elliptic pieces, for instance lens spaces, in the JSJ decomposition. These
cases are not hyperbolic.

%TODO
\textbf{TODO explain Mobius strips / annuli with "big bad loops" idea which
breaks our book of I-bundles reduction theorem.}

We first make the following simple observation.

prop

Let $M$ be a book of $I$-bundles, and $M'$ to $M$ be a finite-sheeted covering
space.  Then $M'$ is also a book of $I$-bundles.

proof

This is straightforward. Lift the spines and pages of $M$ to obtain spines and
pages of $M'$. Lift the gluing annuli of $M$ to gluing annuli of $M'$.

end proof

The following result follows immediately from the classification of I-bundles
over surfaces.

prop

Let M be a book of I-bundles. Recall we require that M is oriented. To be
orientable, every I-bundle in M is either a trivial bundle over an oriented
surface, or a Z/2-twisted bundle over a nonorientable surface. In the latter
case, the homotopy classes with nontrivial bundle twisting are precisely those
with orientation-reversing monodromy (word choice?).

Furthermore, every twisted I-bundle has a double cover which is a trivial
I-bundle over a base surface which is an oriented double cover of the original
nonorientable base surface.

proof

See Hempel I-bundle facts. {\bf TODO}

To analyze the boundary components of a book of $I$-bundles, we'll need
a couple more simple definitions.

defn

Let M be a book of I-bundles, and C a spine in M.  The \emph{valence} v(C) is
the number of binding annuli that intersect C or, equivalently, that are
contained in $\bd C$.

Let A be a binding annulus in M. Note that A lies in the boundary of exactly
one spine $C_A$. The \emph{degree} of A is the geometric intersection number
i(A,D), where D is a meridian disk of $C_A$. That is, it's the minimal number
of components of A cap D as we properly isotope $D cin C_A$ and $A cin bd C_A$.

In fact, note that all binding annuli in a given spine C must have the same
degree. This is because their core curves are curves on a torus, and any curves
with different slopes must intersect. Since they have the same slope, they
intersect a meridian curve (slope $\infty$) the same number of times.
Therefore, we refer to this as the \emph{degree} d(C) of the spine C.

enddefn

prop

Let M be a book of I-bundles. Recall that we require that M has incompressible
boundary. Then M cannot have any spines C such that $v(C)=0$, $d(C)=0$, or
$v(C)=d(C)=1$.

proof

Any spine C with $v(C)=0$ forces C to be a solid torus component. This does not
have incompressible boundary (it's also not hyperbolic).

Any spine C with $d(C)=0$ has a meridian disk that properly embeds in the
resulting book of I-bundles, since it's disjoint from all the binding annuli.
This is a boundary compressing disk.

% The following case is my old nemesis, the "one-point intersection lemma"

Let C be a spine with $v(C)=d(C)=1$. Let A be the sole binding annulus, B the
attached page, and Si its base surface. Let beta be the projection of A down to
Si. beta is a boundary component of Si. Let alpha be an essential arc in Si
with both (distinct) endpoints in beta. Note that such an alpha must exist as
Si has negative Euler characteristic. The preimage of alpha under the
projection is a rectangle R cin B with two edges gamma1,gamma2 in the the side
boundary of B, and two edges delta1,delta2 that are parallel proper essential
arcs in A.  Since A intersects each meridian disk of C exactly once, and does
so in a single essential arc, we can construct two meridian disks D1, D2 such
that D1 cap R = delta1, and D2 cap R = delta2. Let D = D1 cup delta1 R cup
delta2 D2.

We claim that D is a compression disk for bd M. It is properly embedded: D1 and
D2 are properly embedded in C, R is properly embedded in B, and the union lines
up the boundary components along A. It suffices to show that bd D does not
bound a disk in bd M. The options for such a disk D' are very limited. C cap bd
M is an annulus, and bd D divides it into 2 disk regions. D' cap C must be one
of these two regions, which implies it intersects A in two parallel boundary
arcs, each with one endpoint in delta1 and one in delta2. Call these arcs
epsilon1 and epsilon2. But now we can see that the only way to bound a disk in
bd M is if gamma1 cup epsilon1 or gamma2 cup epsilon2 bounded a disk. But if
this disk were inside B, either of these would project down to Si to contradict
the assumption that alpha was essential. Note that the disk cannot intersect
any spines or pages outside of B and C, as its boundary is entirely contained
within B cup C so we can use standard innermost disk/irreducibility arguments
to push it out of anywhere else. This proves that D is a compression disk.

endproof

We make a few observations about the boundary of M. Each page contributes two
(if it's a trivial bundle) or one (if it's a twisted bundle) side boundary to
the boundary of M. It also contributes any leftover binding annuli that are not
glued. Each spine contributes a number of disjoint annuli equal to its valence,
that is, the number of attached pages. These are the annuli in its boundary
which lie in between the binding annuli. As these glue up to form the boundary
components of M, each spine annulus connects two side boundaries of "adjacently
glued" pages. This intuitive picture will be very important later as we
construct quasi-Fuchsian surfaces.

We now discuss the possible closed pi1-injective surfaces inside a book of
I-bundles. These are the surfaces we'll need to consider in order to find
a closed quasi-Fuchsian surface - that is, a quasi-Fuchsian surface subgroup.

Note that since every hyperbolic 3-manifold is aspherical, by elementary
obstruction theory any injective map pi1S to pi1M is induced by a pi1-injective
map S to M.
% TODO cite Long-Reid, p11

Furthermore, by minimal surface theory, we can guarantee that any pi1-injective
surface in a hyperbolic 3-manifold is homotopic to an immersed surface.
A surface which is not immersed will contradict minimality.
% TODO cite Neumann p1

In what follows we'll speak only of pi1-injective surfaces, or possibly surface
subgroups. But it is important to note that in this situation those are the
same thing, and can be chosen to be immersed as well.

lemma (The Covering Lemma, as I actually use it)

Let phi: S to S' be a proper pi1-injective map of compact connected oriented
surfaces with boundary, and suppose that S is not a 2-sphere.  Then phi is
homotopic to a finite-sheeted covering map.  Note that the converse holds more
generally - that is, any finite-sheeted covering is a proper pi1-injective map.

proof

For the forward direction, we can apply minimal surface theory to perturb phi
via homotopy rel boundary to an immersion. It's straightforward to ensure that
phi is a local homeomorphism on boundary curves. Now think of int(S') as being
embedded as a totally geodesic core subsurface of a Fuchsian group quotient of
hyperbolic 3-space.  Then as we pull phi(int(S)) cin int(S')  to minimality its
image will necessarily remain in int(S'), as it's totally geodesic.  Since it
is connected and not a sphere, it can't shrink down to nothing. Since it's
properly embedded, the boundary components of S correspond to local maps of
cusps into cusp tubes of the thickened int(S') and must remain fixed on the
same cusp as we pull phi(int(S)) to a minimal surface. Hence the boundary
components of S do not move under this homotopy. The result is an immersion
produced via a homotopy rel boundary.

Then since phi is a proper immersion between manifolds of the same dimension,
it's a local homeomorphism on boundary and interior neighborhoods, and
therefore a covering map. It must be finite-sheeted because preimages of points
are discrete sets because it's a covering, and S is compact, so it follows that
point fibers are finite sets.

The converse follows from elementary covering space theory. As in the above
note, we can use minimal surface theory to make that proper pi1-injective map
into an immersion.

lemma (surface decomposition lemma)

Let M be a book of I-bundles, and S immerse M be a pi1-injective map, where
S is a (connected) closed surface.  Then, we can place S in minimal position
with respect to M such that:

(1) for each page B, S cap B is a finite-sheeted cover of the page core surface
(that is, if B = Si x I, S is a finite-sheeted cover of Si.

(2) for each spine C, S cap C is a union of essential annuli.

(3) the page covers and spine annuli are attached along curves parallel to the
gluing annulus core curves of M. All boundary components of the page covers and
spine annuli are necessarily attached in this way - there are no free boundary
components.

proof

% TODO

lemma (parabolic lifting criterion)

Let M be a book of I-bundles, and S immerse M be a pi1-injective map, where
S is a connected closed surface.  After placing S in minimal position as above,
S is a (QF) surface if and only if the following criterion holds. Let B be
a page in M, and C a spine in M. P cap B is a union of disjoint arcs and curves
in B. We have canonical covering maps S cap B to Si and S cap C to A2 by the
surface decomposition lemma. Let tildePS be the union of the preimages of P cap
B. % TODO finish statement here

proof

% TODO

\begin{lemma}\label{L:sc}

Let $S$ be a surface satisfying \eqref{E:qf}. Homotope $S$ to have minimal
intersection with each $A_i$, that is, so there are no ``bumps''.  Then for
each $M_i$, each component of $S \cap M_i$ is a finite-sheeted covering of the
core surface.  That is, given such a component $S' \cin M_i$, the map $S' \to
M_i = \Si_{1,1}\x I \to \Si_{1,1}\x{1/2}$ is homotopic to a finite-sheeted
covering map. Conversely, given any finite-sheeted covering $\widetilde{S} \to
\Si_{1,1}$, there exists a corresponding proper immersed $\pi_1$-injective
surface $S \cin \Si_{1,1}\x I$.

\end{lemma}
\begin{proof}

$S$ is compact, so every such component $S'$ is a compact surface with
boundary, properly immersed in $M_i$. Since $S$ is $\pi_1$-injective in $M$,
$S'$ must be $\pi_1$-injective in $M_i$. Otherwise, we'd have a nontrivial
element of $\pi_1S'$ which is trivial in $\pi_1M_i$, hence in $\pi_1M$, hence
in $\pi_1S$, contradicting the minimal position homotopy above. So the map
$\phi : S'\to\Si_{1,1}\x{1,2}$ is also $\pi_1$-injective, since $M_i$
deformation retracts to its core. Let $H = \phi_*(\pi_1S')$, and let
$\Si_{1,1}^H$ be the cover of $\Si_{1,1}$ associated to $H<\pi_1\Si_{1,1}$.
$\phi$ lifts to $\widetilde{\phi}\colon S'\to \Si_{1,1}^H$. This is a proper
map of compact surfaces which is an isomorphism on $\pi_1$.  By the
classification of surfaces, it must be homotopic (as a proper map) to
a homeomorphism. So $\phi$ is homotopic to a covering map. It must be
finite-sheeted as $S'$ is compact (by classification of surfaces again).

Conversely, given a finite-sheeted cover $\widetilde{S}\to \Si_{1,1}$, compose
with the embedding $\Si_{1,1} = \Si_{1,1}\x{1/2} \cin \Si_{1,1}\x I$. This is
proper and $\pi_1$-injective.  Perturb locally to obtain an immersion.

\end{proof}

\begin{lemma}

Let $M$ be a book of $I$-bundles, and $P$ a pared structure on $M$. $P$ cannot
contain any tori.  That is, $P$ consists entirely of annuli.

\end{lemma}
\begin{proof}

Since each page of $M$ has negative Euler characteristic, 

\end{proof}

