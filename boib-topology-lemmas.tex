%%%%%%%%%%%%%%%%%%%%%%%%%%%%%%%%%%%%%%%%%%%%%%%%%
\section{Topology of books of I-bundles}
%%%%%%%%%%%%%%%%%%%%%%%%%%%%%%%%%%%%%%%%%%%%%%%%%

In this section we establish basic facts about books of I-bundles and (QF)
surfaces contained inside them.

Recall that we are considering possible pared structures $P$ on $M$. Our goal
is to find a surface $S \cin M$ such that

\begin{equation}\label{E:qf}

S \text{ is closed immersed $\pi_1$-injective, and $\pi_1P_k \cap \pi_1S = 1$
for each component $P_k$ of $P$} \tag{QF}

\end{equation}

Note that since we haven't fixed a basepoint, these subgroups are really only
defined up to conjugacy (that is, they're sets of free homotopy classes) - what
we're saying is they fail to intersect for an arbitrary choice of conjugacy
class for each subgroup.

We first make a few observations.

Note that since every hyperbolic 3-manifold is aspherical, by elementary
obstruction theory any injective map pi1S to pi1M is induced by a pi1-injective
map S to M.
% cite Long-Reid, p11

Furthermore, by minimal surface theory, we can guarantee that any pi1-injective
surface in a hyperbolic 3-manifold is homotopic to an immersed surface.
A surface which is not immersed will contradict minimality.
% cite Neumann p1

In what follows we'll speak only of pi1-injective surfaces, or possibly surface
subgroups. But it is important to note that in this situation those are the
same thing, and can be chosen to be immersed as well.

lemma (The Covering Lemma, as I actually use it)

Let phi: S to S' be a proper pi1-injective map of compact connected oriented
surfaces with boundary, and suppose that S is not a 2-sphere.  Then phi is
homotopic to a finite-sheeted covering map.  Note that the converse holds more
generally - that is, any finite-sheeted covering is a proper pi1-injective map.

proof

For the forward direction, we can apply minimal surface theory to perturb phi
via homotopy rel boundary to an immersion. It's straightforward to ensure that
phi is a local homeomorphism on boundary curves. Now think of int(S') as being
embedded as a totally geodesic core subsurface of a Fuchsian group quotient of
hyperbolic 3-space.  Then as we pull phi(int(S)) cin int(S')  to minimality its
image will necessarily remain in int(S'), as it's totally geodesic.  Since it
is connected and not a sphere, it can't shrink down to nothing. Since it's
properly embedded, the boundary components of S correspond to local maps of
cusps into cusp tubes of the thickened int(S') and must remain fixed on the
same cusp as we pull phi(int(S)) to a minimal surface. Hence the boundary
components of S do not move under this homotopy. The result is an immersion
produced via a homotopy rel boundary.

Then since phi is a proper immersion between manifolds of the same dimension,
it's a local homeomorphism on boundary and interior neighborhoods, and
therefore a covering map. It must be finite-sheeted because preimages of points
are discrete sets because it's a covering, and S is compact, so it follows that
point fibers are finite sets.

The converse follows from elementary covering space theory. As in the above
note, we can use minimal surface theory to make that proper pi1-injective map
into an immersion.

lemma (surface decomposition lemma)

Let M be a book of I-bundles, and S immerse M be a pi1-injective map, where
S is a connected closed surface.  Then, we can place S in minimal position with
respect to M such that:

(1) for each page B, S cap B is a finite-sheeted cover of the page core surface
(that is, if B = Si x I, S is a finite-sheeted cover of Si.

(2) for each spine C, S cap C is a union of essential annuli.

(3) the page covers and spine annuli are attached along curves parallel to the
gluing annulus core curves of M. All boundary components of the page covers and
spine annuli are necessarily attached in this way - there are no free boundary
components.

proof

% TODO

lemma (parabolic lifting criterion)

Let M be a book of I-bundles, and S immerse M be a pi1-injective map, where
S is a connected closed surface.  After placing S in minimal position as above,
S is a (QF) surface if and only if the following criterion holds. Let B be
a page in M, and C a spine in M. P cap B is a union of disjoint arcs and curves
in B. We have canonical covering maps S cap B to Si and S cap C to A2 by the
surface decomposition lemma. Let tildePS be the union of the preimages of P cap
B. % TODO finish statement here

proof

% TODO

\begin{lemma}\label{L:sc}

Let $S$ be a surface satisfying \eqref{E:qf}. Homotope $S$ to have minimal
intersection with each $A_i$, that is, so there are no ``bumps''.  Then for
each $M_i$, each component of $S \cap M_i$ is a finite-sheeted covering of the
core surface.  That is, given such a component $S' \cin M_i$, the map $S' \to
M_i = \Si_{1,1}\x I \to \Si_{1,1}\x{1/2}$ is homotopic to a finite-sheeted
covering map. Conversely, given any finite-sheeted covering $\widetilde{S} \to
\Si_{1,1}$, there exists a corresponding proper immersed $\pi_1$-injective
surface $S \cin \Si_{1,1}\x I$.

\end{lemma}
\begin{proof}

$S$ is compact, so every such component $S'$ is a compact surface with
boundary, properly immersed in $M_i$. Since $S$ is $\pi_1$-injective in $M$,
$S'$ must be $\pi_1$-injective in $M_i$. Otherwise, we'd have a nontrivial
element of $\pi_1S'$ which is trivial in $\pi_1M_i$, hence in $\pi_1M$, hence
in $\pi_1S$, contradicting the minimal position homotopy above. So the map
$\phi : S'\to\Si_{1,1}\x{1,2}$ is also $\pi_1$-injective, since $M_i$
deformation retracts to its core. Let $H = \phi_*(\pi_1S')$, and let
$\Si_{1,1}^H$ be the cover of $\Si_{1,1}$ associated to $H<\pi_1\Si_{1,1}$.
$\phi$ lifts to $\widetilde{\phi}\colon S'\to \Si_{1,1}^H$. This is a proper
map of compact surfaces which is an isomorphism on $\pi_1$.  By the
classification of surfaces, it must be homotopic (as a proper map) to
a homeomorphism. So $\phi$ is homotopic to a covering map. It must be
finite-sheeted as $S'$ is compact (by classification of surfaces again).

Conversely, given a finite-sheeted cover $\widetilde{S}\to \Si_{1,1}$, compose
with the embedding $\Si_{1,1} = \Si_{1,1}\x{1/2} \cin \Si_{1,1}\x I$. This is
proper and $\pi_1$-injective.  Perturb locally to obtain an immersion.

\end{proof}

\begin{lemma}

Let $P$ be a pared structure on $M$. $P$ cannot contain any tori. That is,
$P$ consists entirely of annuli.

\end{lemma}
\begin{proof}

TODO

\end{proof}

