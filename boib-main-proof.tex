%%%%%%%%%%%%%%%%%%%%%%%%%%%%%%%%%%%%%%%%%%%%%%%%%
\section{Statement of main theorem}
%%%%%%%%%%%%%%%%%%%%%%%%%%%%%%%%%%%%%%%%%%%%%%%%%

We are now ready to state the precise theorem.

\begin{thm}[Main Theorem]

Every reduced book of $I$-bundles contains a (QF) surface.

\end{thm}

Note that together with the earlier theorem, we've fully covered the
non-reduced case also.

\begin{thm}[Main Theorem, non-reduced case]

Let $M$ be a book of $I$-bundles. If the reduction theorem yields a nonempty
set of reduced book of $I$-bundles in $M$, then $M$ contains a (QF) surface.
Otherwise, it does not.

\end{thm}

%%%%%%%%%%%%%%%%%%%%%%%%%%%%%%%%%%%%%%%%%%%%%%%%%
\section{Proof of main theorem - preliminaries}
%%%%%%%%%%%%%%%%%%%%%%%%%%%%%%%%%%%%%%%%%%%%%%%%%

Now there are a number of topological simplifications we make, by passing to an
appropriate finite-sheeted cover of the book of $I$-bundles.  Once we can
construct a surface satisfying (QF) inside this cover, we will push it down and
perturb to obtain a surface satisfying (QF) downstairs (since it
$\pi_1$-injects into a subgroup, it will definitely still be
$\pi_1$-injective).

% TODO expand the above into an actual (short) proof

\begin{defn}

A \emph{good} book of $I$-bundles $M$ is a reduced book of I-bundles which
satisfies the following additional conditions.

\begin{enumerate}

\item Each binding annulus on a spine intersects a meridian disk of that spine
exactly once. That is, every spine has degree one.

\item Each page is glued to a given spine at most once. That is, for any page
$B$ and spine $C$, $B \cap C$ is at most a single component of $\cA$.

\item The two endpoints of a fiber in each page are in different boundary
components of $M$. In particular, each page is a trivial $I$-bundle over an
oriented surface.

\item Each spine intersects each boundary component of $M$ in at most a single
annulus.

\item Every arc of $P$ on a page connects two different binding annuli. That
is, there are no essential arcs that begin and end at the same binding annulus.

\end{enumerate}

\end{defn}

\begin{thm}

Let $M$ be a reduced book of $I$-bundles. Then $M$ has a finite-sheeted regular
cover which is good.

\end{thm}

\begin{proof}

To prove this, we'll use the fact that $\pi_1M$ is LERF to construct finite
sheeted covers with nice properties. Note that (1), (2), (3), and (4) are all
properties that once true, remain true when we lift to finite-sheeted covers.
So it suffices to guarantee each property one at a time by lifting to
finite-sheeted covers, because we can lift repeatedly and the old properties
will still hold.

We first take care of condition (1). For each spine $C$, let $k=d(C)$.  Let
$A_1,\dots,A_n$ be the binding annuli on $C$. These annuli must be parallel.
Notice that $\pi_1C$ is infinite cyclic, as its only torsion could come from
the identification with the binding annuli, but we've assumed that these are
$\pi_1$-injective. Now if $\alpha$ is a generator for the infinite cyclic group
$\pi_1C$, $\alpha^k$ is generates each $\pi_1A_i \cin \pi_1C$. Fix a point $x_0
\in C$, and apply Proposition~\ref{P:lerf3} to $\alpha$ and $k$ to obtain
a finite-sheeted regular cover $M' \to M$..  For any spine $C'$ covering $C$,
we claim that $C'$ has degree 1. Observe that $C'\to C$ has degree $d$ by
construction, with $k \mid d$. Therefore $\alpha^d$, which lifts to traverse
$C'$ exactly once, is a power of $\alpha^k$, the generator of each binding
annulus. Hence each binding annulus has preimage a union of components, each of
which traverses $C'$ exactly once. Hence $C'$ has degree 1.

We now produce a cover $M'$ that satisfies (1). Perform the above construction
for each spine to produce generators $\alpha_1,\dots,\alpha_n$ and degrees
$k_1,\dots,k_n$. Applying Proposition~\ref{P:lerf3'}, we can follow the above
argument locally on each spine in $M'$ to show that it has degree 1.

So now assume $M$ is reduced and satisfies (1). We claim $M$ has
a finite-sheeted cover which satisfies (2).

Let $B$ be a page intersecting a spine $C$ more than once. For each pair of
binding annuli $A_1$,$A_2$ which attach $B$ to $C$, we'll draw a closed curve
as follows.  Fix a meridian disk $D$ of $C$. Choose an arbitrary point $x_1 \in
A_1 \cap D$, $x_2 \in A_2 \cap D$. Choose an arbitrary arc $\alpha' \cin B$
connecting $x_1$ and $x_2$.  Let $\alpha$ be the closed curve obtained by
closing up $\alpha'$ with an arc along the interior of $D$. Denote this arc by
$\alpha''$.

Fix $x_1$ to be our basepoint. We claim that $\alpha \notin \pi_1B$. $\alpha$
intersects each of the binding annuli $A_1$ and $A_2$ in a single point. Any
homotopy of $\alpha$ to a curve in $\pi_1B$ would have to homotope $\alpha''$
rel boundary into $B$, but this is clearly impossible. So $\alpha \notin
\pi_1B$.  $\pi_1B$ is finitely generated. Apply Proposition~\ref{P:lerf1} to
$\alpha$ and $B$, obtaining a finite-sheeted cover.

Any lift of $\alpha$ to $\alpha'$ in this cover must connect two different
components of $p^{-1}(B)$.  Any lift of $\alpha''$ still lies on a meridian
disk of a spine covering $C$, and connects binding annuli covering $A_1$ and
$A_2$.  But $\alpha'$ lifts to arcs inside $p^{-1}(B)$. So in order for a lift
of $\alpha$ to connect two different components of $pi-1(B)$, the lifts of
$\alpha''$ must have endpoints in different components of $p^{-1}(B)$.

We now construct a cover $M'$ that satisfies (2). Repeat the above construction
for each pair of binding annuli for each page with multiple attachments to
a spine. This yields closed cuvres $\alpha_1,\dots,\alpha_n$ and corresponding
pages $B_1,\dots,B_n$. Note that there may be duplicates in this list of pages,
but the argument remains the same. Apply Proposition~\ref{P:lerf1'}. By
following the above argument locally, we see that any page upstairs with
multiple gluings to the same spine would correspond to a lift of some
$\alpha_i''$ with endpoints in the same component of $p^{-1}(Bi)$,
a contradiction.

So we have a cover which satisfies (1) and (2). We now show that $M$ satisfying
(1) and (2) has a finite-sheeted cover satisfying (3).

$M$ is orientable by definition, so each page must be orientable as well. So
the pages which are not trivial $I$-bundles over oriented pages must be twisted
$I$-bundles over nonoriented pages (in order for the resulting page to be
orientable). In particular, the endpoints of the $I$ fibers of a twisted
$I$-bundle will connect globally to form a single side of the page. So if we
look locally at an $I$ fiber, the two endpoints are guaranteed to be in the
same boundary component of $M$ (since they're in the same boundary component
even if we just look at that page). This explains why ensuring distinct
boundary components for each fiber guarantees trivial $I$-bundles.

This argument is very similar to (2). Let $B$ be a page such that fibers have
both endpoints on the same boundary component $\bd_i M \cin \bd M$.  Choose
a binding annulus $A$ in $\bd B$, and an arbitrary transversal $\alpha'' \cin
A$.  Let $\alpha' \cin \bd_a M$ be an arc in the boundary connecting the
endpoints of $\alpha''$. Such an $\alpha'$ must exist as the two endpoints are
part of the same boundary component. $\alpha'$ and $\alpha''$ combine to form
a closed curve $\alpha$.  Fix one endpoint of $\alpha'$ to be our basepoint.
$\bd M$ is a compact surface, so $\pi_1 \bd_i M$ is finitely generated.

We claim that $\alpha \notin \pi_1 \bd_i M$. As above, any such homotopy would
correspond to a homotopy rel boundary of $\alpha''$ into $\bd_i M$. Since
$\alpha''$ is a transversal of the binding annulus $A$, such a homotopy would
correspond to a boundary compression of $A$, which is impossible as $A$ is
boundary incompressible. Apply Proposition~\ref{P:lerf1} to $\alpha$ and $\bd_i
M$, obtaining a finite-sheeted cover. By the same argument as in (2),
$\alpha''$ must lift to connect two different components of p$^{-1}(\bd_i M).$
Since $\bd M' = p^{-1}(\bd M)$, this shows that the pages covering $B$ have
fibers that connect two different boundary components. We can therefore
construct $M'$ satisfying (3), by taking all such pages with both sides on the
same boundary component and applying Proposition~\ref{P:lerf1'}.

We can satisfy (4) by a similar argument. Let $C$ be a spine that intersects
the same boundary component $\bd_a M$ twice, and $\alpha'$ to be an arc of
a meridian disk of $C$ which connects two components of $C \cap \bd_a M$.
Connect the endpoints of $\alpha'$ with an arc $\alpha'' \cin \bd_a M$. By the
same argument as above with boundary compressions, $\alpha=\alpha' \cup
\alpha'' \notin \pi_1 \bd_a M$, so following the same argument proves (4).

So we have a cover which satisfies (1), (2), (3), and (4). To satisfy (5), let
$\alpha'$ be the core of a component of $P \cap B$ which has both endpoints in
the same binding annulus $A$. Let $\alpha''$ be a segment in $\bd A$ which
connects the two endpoints of $\alpha'$. Let $\alpha = \alpha' \cup \alpha''$
be the resulting closed curve. Since $\alpha'$ is essential, $\alpha \notin
\pi_1A$.  Apply Proposition~\ref{P:lerf1} to $\alpha$ and $A$, and follow the
argument used in (2) and (3).  This proves that $M$ has a cover satisfying
(1)-(5), completing the proof.

\end{proof}

%%%%%%%%%%%%%%%%%%%%%%%%%%%%%%%%%%%%%%%%%%%%%%%%%
\section{Proof of main theorem}
%%%%%%%%%%%%%%%%%%%%%%%%%%%%%%%%%%%%%%%%%%%%%%%%%

We've now reduced to the case of a good book of $I$-bundles. However, the
remaining work is still quite involved.

This is the main proof, and is quite involved. We proceed somewhat similarly to
the first example. We take covers over the pages and glue them together
cleverly to construct our surface. We check that the resulting surface
satisfies the pared lifting criterion. As in the first example, this suffices
to prove that our surface satisfies (QF).  We first need to take the correct
cover over each page. However, the details are more complex. We first will need
the following important notion.

\begin{defn}

Let $M$ be a book of $I$-bundles. A \emph{partially decomposed surface} $Q \cin
M$ is obtained from a closed pi1-injective surface S in minimal position
(following the surface decomposition lemma) by deleting any number of
components of Q cap C, for each spine C of M. The result is a possibly
disconnected surface with boundary, where each component is pi1-injective and
in minimal position. If (M,P) is a pared book of I-bundles, the \emph{pared
lifting pattern} on Q is defined componentwise in the surface decomposition,
just like for closed surfaces. We say that Q satisfies the \emph{pared lifting
criterion} if its pared lifting pattern contains no closed curves.

\end{defn}

Note that at each spine C of M, there must be an even number of boundary
components of Q which lie in C, by parity considerations with the deleted
annuli. Also note that, assuming M is reduced, it is trivial to build
a partially decomposed surface that satisfies the pared lifting criterion. To
do so, simply take a closed pi1-injective surface and delete all the
intersections with spines.  No individual page contains a pared closed curve,
so this trivial partially decomposed surface (which is just a disjoint union of
page covers) cannot either.

\begin{defn}

Let Q be a partially decomposed surface in a book of I-bundles M. The
\emph{defect} of Q is the total number of boundary components, summing over all
connected components. The \emph{local defect} at a spine C cin M is the number
of boundary components which intersect C. The \emph{degree} of Q is the
covering degree on each page. (Note that since some spine components were
deleted, the covering degree on a spine may be smaller.)

\end{defn}

Our strategy is as follows. For each boundary component of M, we will build
a partially decomposed surface. This surface will be obtained by removing spine
components from a cover of that boundary component. We will then align the
number of boundary components of the different partially decomposed surfaces
where they meet at spines, and glue them all together with new annuli. This
will produce a closed surface. As suggested by the above terminology, we want
our partially decomposed surface to have as small a defect as possible.

We will also need our partially decomposed surface to satisfy the following
technical conditions:

\begin{defn}

Let Q be a partially decomposed surface in a book of I-bundles M. We say that
Q is \emph{usable} if it satisfies the following:

begin enumerate

item[dag] The pared lifting pattern on Q contains no closed curves.

item[ddag] Every arc in the pared lifting pattern on Q begins and ends at two
different boundary components of Q

\end{defn}

\begin{defn}

We say that Q \emph{has sparse defect} if, in the surface decomposition of Q,
each component of Q cap B contains at most one component of bd Q. That is, the
``defective'' (non-glued) boundary components are ``spread out'' among the
components of Q in each page.

\end{defn}

\begin{lemma}\label{L:jigsaw}

Let M be a good book of I-bundles. Fix a boundary component F of M. Then, for
any sufficiently large d, there exists a partially decomposed surface Q in
M with the following properties:

begin enumerate

item Q is obtained by removing spine components from a finite-sheeted cover of
F of degree d.

item Q is usable.

item Q has sparse defect.

item Q has defect bounded above by a constant C depending only on (M,P) (ie,
not depending on d)

end enumerate

\end{lemma}

\begin{proof}

Suppose that F intersects the spines C1,dots,Cr and pages B1,dots,Bs of M.
Since M is good, F intersects each of these in a single component. Denote the
page intersections by Fi = F cap Bi and the spine intersectons by Fj' = F cap
Cj.

Let Q0 be the surface obtained by taking d many disjoint copies of each of the
Fi. Q0 is obviously a partially decomposed surface obtained by removing all
spine components from the trivial d-fold cover of F (that is, d many disjoint
copies of F), which is a closed pi1-injective surface in M. The pared lifting
pattern on each component of Q0 is a homeomorphic copy of (the core of) P cap
Fi. Since F is reduced, this pattern has no closed curves. Since F is good,
every arc in the pattern connects two different boundary curves of its
component. Hence Q0 is useful.

However, Q0 obviously fails conditions (3) and (4) of the lemma. We need to add
some annuli to ensure these conditions hold.

Fix a large integer $m>0$. Now, reserve a subset $\cR$ of the set of boundary
components of $Q_0$.  This subset should have the following properties:

(i) Each boundary component of each $F_i$ is covered by exactly $m$ elements of
$\cR$.

(ii) Every connected component of $F_i$, that is, lifted copy of each page,
contains at most one element of $\cR$.

This is possible as long as $d \geq C_0m$, where $C_0$ is the maximum number of
boundary components of any $F_i$. $C_0$ is fixed and depends only on $(M,P)$.
If $d\geq C_0m$,  each local cover $Q_0 \cap B_i$ has at least $C_0m$ many
disjoint copies above its respective boundary piece $F_i$.  So we have enough
copies to make our choices above.

We begin adding annuli that are copies of the annuli F cap Cj = Fj'. Given any
such annulus Fj', we know there are exactly two pages Bi1 and Bi2 such that the
two boundary components of Fj' are attached to Fi1 and Fi2 in F. There are in
fact two specific boundary components gamma1 of Fi1 and gamm2 of Fi2 that are
attached in this way. Each time we add an annulus copy of Fj' to Q0, we must
attach it to two curves widetilde gamma1 and widetilde gamma2 which are lifted
copies of gamma1 and gamma2 in components of Q0 cap Bi1 and Q0 cap Bi2,
respectively.  This is necessary to ensure that Q0 remains a partially
decomposed surface that can be extended to a cover of F.

We add annuli one at a time. Each time we add an annulus, we make sure to add
in such a way that the resulting surface is still usable. We also require that
we do \emph{not} attach any annuli to the boundary components in cR. Aside from
these requirements and the condition above (that our gluing extends to a cover
of F), we repeatedly choose an arbitrary annulus. This process will terminate
at some some surface Q1, after there are no more possible gluings to make.

\begin{claim}

We claim that $\#(\bd Q_1 \setminus \cR)$ is bounded by $C$, a constant
depending only on $(M,P)$. In particular, $C$ is independent of the choice of
$d$ and $m$ made earlier.

\end{claim}

\begin{proof}[Proof of Claim]

Consider an arbitrary boundary component $\widetilde{\gamma} \cin \bd Q_1
- \cR$.  It covers some boundary component $\gamma \cin F_i$. Look at the
number of arcs of $P \cap F_i$ which are incident to $\gamma$. This number
depends only on $(M,P)$. Let $C_1$ be the maximum such number of incident arcs
of $P$ for any boundary component $\gamma$ of any $F_i \cin F$. Let $C_2$ be
the total number of boundary components of the $F_i$. Let $\gamma' \cin F_{i'}$
be the boundary component matching $\gamma$ - that is, the boundary component
that is attached to $\gamma$ in $F$ by some spine annulus. Let $F_j'= F \cap
C_j$ be the annulus in $F$ that connects them.  Suppose we attempt to add an
annulus to $Q_1$ that covers $F_j'$, with one of its boundary components
attached to $\gamma$.

The cover extension condition allows us to attach the other boundary component
to any component of $\bd(Q_1 \cap F_{i'}) - \cR$ above $\gamma'$.  Because we
always glue matching boundary components (and we started with the same number
of each), $\bd Q_1 - \cR$ has the same number of remaining boundary components
above $\gamma'$ as it does above $\gamma$. So there must be at least one to
glue to.  Since we stopped gluing at $Q_1$, this implies that any gluing we
could make would force the resulting surface to not be useful.

Let $\widetilde{\gamma}' \cin \bd(Q_1 \cap F_{i'}) - \cR$ be an arbitrary
available lift of $\gamma'$.  Suppose gluing to $\widetilde{\gamma}'$ produces
a closed curve $\alpha$ in the pared lifting pattern. Obviously $\alpha$
intersects the newly added annulus $A$, otherwise $Q_1$ would already contain
a closed curve. $P$ is in minimal position and $M$ is reduced, so within $A$
the pared lifting pattern is just a union of transverse arcs. Removing $A$
therefore divides $\alpha$ into a union of arcs with endpoints on either
$\gamma$ or $\gamma'$.  $Q_1$ is useful, so none of these arcs can have both
endpoints on the same boundary component. So there exists an arc in $Q_1$
connecting $\gamma$ to $\gamma'$.  (This is the reason for the second condition
in our definition of a useful surface).  The pared lifting pattern in $Q_1$
near $\gamma$ is a homeomorphic copy of (the core of) $P \cap F_i$, so the
number of arcs incident to $\gamma$ is bounded by $C_1$.  Tracing these arcs
through $Q_1$, they can hit at most $C_1$ other boundary components.  These are
the only boundary components we can attach our annulus to to produce a closed
curve.

Similarly, suppose gluing to $\widetilde{\gamma}'$ produces an arc $\alpha$
with endpoints on the same boundary component. Let $\widetilde{\delta}$ be this
boundary component. Again, $\alpha$ intersects the added annulus $A$, otherwise
$Q_1$ would already not be useful. Removing $A$ divides $\alpha$ into a union
of arcs. Except for the two endpoints of $\alpha$ (which now lie on two
different subarcs - call these subarcs $\alpha_0$ and $\alpha_1$), all other
endpoints of these arcs must either lie on $\gamma$ or $\gamma'$.  But no arc
can have both endpoints on the same boundary component, or $Q_1$ would already
fail to be useful.  So any subarcs except $\alpha_0$ and $\alpha_1$ must
connect $\gamma$ to $\gamma'$.  By parity we can see that $\alpha_0$ and
$\alpha_1$ both have one endpoint on $\widetilde{\delta}$, but their other
endpoints must be different.  That is, either $\alpha_0$ ends on
$\widetilde{\gamma}$ and $\alpha_1$ on $\widetilde{\gamma}'$, or vice versa.
This is the set of circumstances that leads to a returing arc.

Now, as above, the arcs incident to $\widetilde{\gamma}$ hit at most $C_1$
other boundary components. These are our possible $\widetilde{\delta}$. Each of
these has at most $C_1$ incident arcs itself, one of which returns to
$\widetilde{\gamma}$, leaving $C_1-1$ that we need to care about. We have
a total of at most $i_a*(C_1-1)$ many "distance two" available boundary
components. If we attach our annulus to one of these boundary components, we
will produce a non-useful surface.

By construction of $Q_1$, there are no more legal gluings. So every remaining
boundary component must be disallowed for one of the above reasons. So $Q_1$
can have at most $C_1+(C_1*(C_1-1)) = C_1^2$ many leftover boundary components
(that is, components of $\bd Q_1 - \cR$) above $\gamma'$.  Since our choice of
$\widetilde{\gamma}$ was arbitrary, this implies that there are at most $C
= C_1^2*C_2$ many leftover boundary components, that is, components of $\bd
Q_1-\cR$. This proves the claim.

\end{proof}

\noindent \emph{Proof of Lemma~\ref{L:jigsaw} continued.} We now attach more
annuli to build a surface $Q$ which satisfies (3) and (4). Intuitively, the
boundary components that make up the reserve are sparse. As long as our reserve
is sufficiently large we can make use of it to attach enough annuli that only
reserve boundary components remain. All the remaining non-reserve boundary
components will be attached to reserve boundary components by annuli.

Construct $Q$ from $Q_1$ as follows.  For each boundary component $\gamma$
downstairs, as discussed above, there are at most $C_1^2$ many elements of $\bd
Q_1 - \cR$ above it.  For each of these, there are at most $C_1^2$ we could
glue to above $\gamma'$ that produce a non-useful surface. We now allow
ourselves to glue to the reserve boundary components. Assume that $m\geq
2C_1^2$.  Then attach each leftover (non-reserve) lift of $\gamma$ to an
element of $\cR$ above $\gamma'$ with a copy of the annulus $F_j'$, one at
a time.  At any point there will be at least $i^2+1$ elements of $\cR$
remaining above $\gamma'$, so we'll always be able to choose one such that our
surface is still useful.  Repeat this process for each $\gamma$ to construct
$Q$.  Since we glued all the leftovers, it immediately follows that $\bd Q \cin
\cR$, and therefore $Q$ satisfies (3).

Let's analyze the possibilities for the defect of $Q$, that is, the number of
boundary components.  Above each $\gamma \cin \bd F_i$, $\cR$ has $m$ elements.
So $\bd Q$ has at most $m$ components above $\gamma$, or $C_2m$ many boundary
components in total. The lower bound is determined by the maximum number of
leftovers, since the only way we'll remove reserve components from $\bd Q$ is
by gluing them to non-reserve leftover components. There are at most $C_1^2$
leftover components above $\gamma$, and similarly above $\gamma'$, so there
will be at least $m - C_1^2$ components remaining in $\bd Q$ above $\gamma$.
We find that

\[ C_2(m-C_1^2) \leq defect(Q) \leq C_2m \]

Any choice of $m\geq 2C_1^2$ works, as long as $d$ is sufficiently large
relative to $m$.  This proves (4).

\end{proof}

\begin{proof}[Proof of Main Theorem]

We want to use Lemma~\ref{L:jigsaw} to construct a partially decomposed surface
$Q_a$ for each boundary component $F_a$ of $M$. Now we want to glue these
together, but they may have very different numbers of boundary components! So
we'll need to normalize them so they all have the same number of boundary
components near each spine, so we can do a local construction with annuli in
each spine.

Let $M_1 = max a C_1(a)$, that is, the maximum number of components of $P \cap
\cap B_i$ incident to a single boundary component of $\bd M \cap B_i$ for any
spine $B_i$. By the inequality in Lemma~\ref{L:jigsaw}, for each $\gamma \cin
\bd F_i$, $\#\{\bd F_i \text{ above } \gamma\} \geq m - M_1^2$. We want to make
this an equality for each $Q_a$ by adding some more annuli to $Q_a$ in the same
way that we added annuli in Lemma~\ref{L:jigsaw}.

Again, we use the size of the reserve to our advantage. Assuming $m\geq M_1^2
+ C_1(a)^2$ for each $F_a$, by the same argument as the lemma, there will
always be choices remaining from the reserve to make the gluing. Combining
these all we need is $m\geq 2M_1^2$.

By abuse of notation, we'll call these normalized surfaces $Q_a$ also (the
non-normalized ones will not come up again). Each has exactly $M = m - M_1^2$
many boundary components above any $\gamma \cin \bd F_i$.

Build a single closed surface $S$ by attaching annuli to the $Q_a$ as follows.

Let $C$ be an arbitrary spine of $M$. There are exactly $d(C)$ many binding
annuli in $\bd C$, and $C$ is attached to $d(C)$ distinct pages, since $M$ is
good. Each page has two sides, so there are $2d(C)$ many ``incident neighbor
types'' in the $Q_a$, corresponding to the $2d(C)$ boundary components of
binding annuli in $\bd C$. These are attached to the $2d(C)$ side boundary
components of pages adjacent to $C$.  Because $M$ is good, we know that all
these pages are distinct and all the corresponding binding annuli have degree
1.

Introduce notation as follows. A neighborhood $\cN(C)$ consists of $d(C)$
neighborhoods inside pages, attached to $C$ by annuli. The meridian of $C$
gives a cyclic order to the attached pages.  Fix an orientation for the
meridian and an (arbitrary) starting point along it, and label the pages $1,
\dots, d(C)$ under this ordering.  The orientation of the meridian also gives
an ordering of the boundary components of each binding  annulus.  Using this,
label the $d(C)$ gluing annuli by $A_1,\dots,A_{d(C)}$. Also label the $2q$
binding annulus boundary components by $\gamma_1^-,\gamma_1^+,\dots,
\gamma_q^-,\gamma_q^+$, where $\gamma_i^-$ and $\gamma_i^+$ are the two sides
of the binding annulus for the $i$th page.  Extend these labels to the
corresponding boundary components of the page boundaries attached to the
binding annuli.  Remember that all these labels are downstairs - that is,
they're labels of pieces of $\bd M$.

By construction of the $Q_a$, there are $2d(C)M$ many components of
$\bd(\bigsqcup_a Q_a) \cap C$. Attach them as follows. Let $A_{1,3}$ be an
embedded valence 1 annulus in $C$ with one boundary component in $A_1$ and one
in $A_3$. Define $A_{2,4},A_{3,5},\dots,A_{(d(C)-1),1},A_{d(C),2}$ similarly.
Now construct the partially decomposed surface $S$ by $M$ many copies of each
$A_{i,i+2}$, and attaching their boundary components to lifts of $\gamma_i^-$
and $\gamma_{i+2}^+$. By construction of the $Q_a$ there will be exactly $M$
many available lifts of each to attach to. Therefore, when we repeat this for
each spine $C$, the result is a closed surface $S$. Note that since $M$ is
reduced, we know that each spine has at least 3 annuli. This means we'll never
attempt to attach an annulus with both boundary components in the same binding
annulus.

We claim that $S$ satisfies the pared lifting criterion.



% d = lcm degree of subgp sep cover (lcm of degree on each page) After taking
% copies of each subgp sep cover to ensure all are same degree: n = (large)
% # of copies of each cover above m = # of copies of each bc to reserve
% i(M,P,S_a) = max # incident arcs at a bc on S_a boundary component TODO call
% this ia j(M,P,S_a) = total # bcs on page boundaries making up S_a
% ("junctions")

Consider an arbitrary boundary component $F_a$ of the book of $I$-bundles $M$.
Cutting $M$ along the gluing annuli decomposes $\bd_aM$ into a union of pieces
each of which is one side of an $I$-bundle page. Call these pieces $S_{a,k}$.
Then the page is $S_{a,k}\times I$, with the piece embedded as $S_{a,k}\times
0$ or $S_{a,k}\times 1$. Without loss of generality let it be $S_{a,k}\times
0$. We can look at the pattern of arcs $P_{a,k} = P \cap (S_{a,k}\times 0)$ on
that side of the page, which is part of $\bd_aM$. We can also look at the
pattern of arcs on the opposite side of the page $\overline{P_{a,k}} = P \cap
(S_{a,k}\times 1)$, and temporarily view it as living inside $S_{a,k}$
(flattening $S_{a,k}\times I \to S_{a,k}$).  For each piece, apply the lemma
above to ($S_{a,k},P_{a,k} \cup barP_{a,k}$) to obtain a cover $S'_{a,k}$. If
we view this as a cover of $S_{a,k}x0$ and $S_{a,k}x1$, it has the property
that any arc in $P_{a,k}$ or $barP_{a,k}$ lifts to an arc connecting two
different boundary components of $S'_{a,k}$.

Let $d_{a,k}$ be the degree of this cover. Let $d = lcm a,k d_{a,k}$.

Fix some large integer $n>0$. Now for each $k$, let $\pi_{a,k}
\colon \widetilde{S_{a,k}} \to S'_{a,k} \to S_{a,k}$ be a disjoint union of $nd/d_{a,k}$
copies of $S'_{a,k}$. The purpose of the $d/d_{a,k}$ normalization is to make sure
each $\pi_{a,k}$ is a cover of the same degree, $n*d$.  We'll call the connected
components of this cover "pieces". We can also combine these into a single
cover $\pi \colon \bigsqcup \widetilde{S_{a,k}} \to \bigsqcup S_{a,k}$

Fix a large integer $m>0$. Now, reserve a subset $\cR$ of the boundary
components of $\bigsqcup \widetilde{S_{a,k}}$.  This subset should have the following
properties:

(1) Each boundary component of each $S_{a,k}$ is covered by exactly $m$ elements
of $\cR$.

(2) Every connected component of $\bigsqcup \widetilde{S_{a,k}}$ contains at most one element
of $\cR$.

This is possible as long as $n \geq mb$, where $b$ is the maximum number of
boundary components of any $S_{a,k}$. If $n\geq mb$, we know $d/d_{a,k} \geq
1$, so each disjoint union $\widetilde{S_{a,k}}$ has at least $mb$ many
disjoint copies of each $S'_{a,k}$.  So we have enough copies to make our choices
above that $S_{a,k}$ disjoint.

Each $S_{a,k}$ is homeomorphic to the core surface of its page, so by the covering
lemma any closed surface formed by gluing the $\widetilde{S_{a,k}}$ along appropriate
boundaries will induce an immersed $\pi_1$-injective surface in $M$. This is
our strategy.

Write $\widetilde{P_{a,k}}=\pi_{a,k}^{-1}(P_{a,k})$. As constructed, $P_{a,k}$ consists of proper arcs
connecting two different boundary components of $\widetilde{S_{a,k}}$.

We first glue the pieces "along" $\bd_aM$ in the following sense. Given any two
pieces $S_{a,k_1}$, $S_{a,k_2}$ in the decomposition of $\bd_aM$, we know that
$S_{a,k_1}$ has some boundary components $\gamma_1,\gamma_2,\dots \cin
dS_{a,k_1}$, and $\gamma_1',\gamma_2',\dots \cin dS_{a,k_2}$ such that before we
cut $\bd_aM$ apart, $\gamma_1$ was glued to $\gamma_1'$, $\gamma_2$ to
$\gamma_2'$, etc.  That is, these components form the locus we glue $S_{a,k_1}$
and $S_{a,k_2}$ together along when we glue the $S_{a,k}$ to form $\bd_aM$.  For
each such pair, look at the boundary components of $\widetilde{S_{a,k}}1$ and
$\widetilde{S_{a,k}}2$.  We allow ourselves to glue components of
$\pi^{-1}(\gamma_1)$ to components of $\pi^{-1}(\gamma_1')$, components of
$\pi^{-1}(\gamma_2)$ to components of $\pi^{-1}(\gamma_2')$, etc. We forbid
ourselves from gluing components of $\pi^{-1}(\gamma_1)$ to any other lifted
boundary components of $S_{a,k_2}$ or of any other pieces. Only on the very last
step of our construction will we break this rule.

Note that since our gluing is locally a cover we can only glue each pair of
boundary components upstairs in one possible way - we could choose a different
gluing, but this would just correspond to a different choice of cover by the
covering lemma.

We introduce some notation. Bbegin by taking $\widetilde{S}^{(0)} = \bigsqcup \widetilde{S_{a,k}}$ to
be a disjoint union of these covers - that is, we haven't done any gluing yet.
Think of these as pieces we'll use to build our cover. At each step in the
gluing, we have a gluing map $f(i+1) \colon \widetilde{S}^{(i)} \to
\widetilde{S}^{(i+1)}$ by attaching two boundary comonents of
$\widetilde{S}^{(i)}$. Write $\widetilde{P}^{(i)}$ for the arcs and curves on
$\widetilde{S}^{(i)}$ induced by gluing the $\widetilde{P_{a,k}}$.  That is,
$\widetilde{P}^{(i)} = f(i) \circ ...  \circ f(1) (\bigsqcup
\widetilde{P_{a,k}})$.

We'll want to glue boundary components other than $\cR$ first - these are the
"reserve" that we'll only use at the end. This guarantees that our final glued
surface will have leftover boundary components that are sufficiently far apart.
By abuse of notation, we'll also use $\cR$ to refer to the induced set of
boundary components in each $\widetilde{S}^{(i)}$.

In addition to gluing along $\bd_aM$ and avoiding gluing members of $\cR$, we want
to preserve the following two invariants.

% FIXME decide what to call it in the paper...  still need to figure out how
% I'm going to do the references too

\begin{enumerate}

\item[(\dag)] $\widetilde{P}^{(i)} \cin \widetilde{S}^{(i)}$ contain no closed
curves.  That is, it is a union of proper arcs. \label{I:dag}

\item[(\dag')] No arc of $\widetilde{P}^{(i)}$ begins and ends at the same
boundary component of $\widetilde{S}^{(i)}$. \label{I:dag'}

\end{enumerate}

By the use of the lemma and the assumption of the theorem (no closed curves in
individual pages), $\widetilde{S}^{(0)}$ satisfies (\dag) and (\dag') We
repeatedly make an arbitrary gluing that lies along $\bd_aM$ in the above
sense, doesn't glue any elements of $\cR$, and preserves (\dag) and (\dag').  This
process will terminate at some $\widetilde{S}^{(N)}$, after there are no more
gluings left to make. $\widetilde{S}^{(N)}$ is a surface with boundary, because
there are "leftover" boundary components that we couldn't glue without
violating one of the above conditions.

% FIXME restructure into lemmas?
%Claim.

We claim that $\#(d\widetilde{S}^{(N)} - \cR)$ is bounded by $C(M,P,S_a)$.  In
particular, this bound is independent of the choice of $n$ and $m$ made
earlier.

%Proof of Claim.

Consider an arbitrary boundary component $\widetilde{\gamma} \cin
d\widetilde{S}^{(N)} - \cR$.  It lives above some boundary component $\gamma
\cin dS_{a,k}$. Look at the number of arcs of $P_{a,k}$ which are incident to
$\gamma$. This number depends only on $M$ and $P$ (not on $n$). Let $i_a
= i(M,P,S_a)$ be the maximum such number of incident arcs of $P$ for any
boundary component $\gamma$ of any $S_{a,k} in S_a$. Let $j_a = j(M,P,S_a)$ be
the total number of boundary components of the $S_{a,k}$.

Let $\gamma' \cin dS_{a,k}'$ be the boundary component matching $\gamma$. We
consider gluing $\gamma$ along $M$ to any component of $\bd\widetilde{S}^{(N)}
- \cR$ above $\gamma'$.  Because we always glue matching boundary components
(and we started with the same number of each), $\bd\widetilde{S} - \cR$ has the
same number of leftover boundary components above $\gamma'$ as it does above
$\gamma$. So there must be at least one to glue to.  The only reason we'd be
unable to glue is if gluing to any leftover lift of $\gamma'$ violates (\dag) or
(\dag').

Let $\widetilde{\gamma}' \cin \bd\widetilde{S}^{(N)} - \cR$ be an arbitrary
leftover lift of $\gamma'$.  Suppose gluing to $\widetilde{\gamma}'$ violates
(\dag). Let $\alpha$ be a closed curve produced by the gluing.  Obviously
$\alpha$ intersects the $\widetilde{\gamma} = \widetilde{\gamma}'$ gluing
circle, otherwise $\widetilde{S}^{(N)}$ would already violate (\dag).  Cutting
along this circle divides $\alpha$ into a union of arcs with endpoints on
either $\gamma$ or $\gamma'$.  $\widetilde{S}$ satisfies (\dag'), so no arc can
have both endpoints on the same boundary component. So there exists an arc
connecting $\gamma$ to $\gamma'$. This is the only way that an additional
gluing would violate (\dag). But since $\widetilde{S}^{(N)}$ is locally a cover
on each piece, the number of arcs incident to $\gamma$ is bounded by $i_a$.
Tracing these arcs through $\widetilde{S}^{(N)}$, they can hit at most $i_a$
other boundary components.  These are the only boundary components we can glue
to to violate (\dag).

Similarly, suppose gluing to $\widetilde{\gamma}'$ violates (\dag'). Let $\alpha$ be an
arc with endpoints on the same boundary component produced by the gluing. Let
$\widetilde{\delta}$ be this boundary component. Again, $\alpha$ intersects the
$\widetilde{\gamma} = \widetilde{\gamma}'$ gluing circle, otherwise $\widetilde{S}^{(N)}$ would already
violate (\dag').  Cutting along this circle divides $\alpha$ into a union of
arcs.  Except for the two original endpoints (which now lie on two different
subarcs - call these subarcs $\alpha_0$ and $\alpha_1$), all other endpoints of
these arcs must either lie on $\gamma$ or $\gamma'$.  But no arc can have both
endpoints on the same boundary component, or $\widetilde{S}^{(N)}$ would arleady violate
(\dag'). So any subarcs except $\alpha_0$ and $\alpha_1$ must connect $\gamma$ to
$\gamma'$.  By parity we can see that $\alpha_0$ and $\alpha_1$ both have one
endpoint on $\widetilde{\delta}$, but their other endpoints must be different. That is,
either $\alpha_0$ ends on $\widetilde{\gamma}$ and $\alpha_1$ on $\widetilde{\gamma}'$, or vice
versa.

Now the arcs incident to $\widetilde{\gamma}$ hit at most $ia$ other boundary
components. These are our possible $\widetilde{\delta}$. Each of these has at
most $i_a$ incident arcs itself, one of which returns to $\widetilde{\gamma}$,
leaving $i_a-1$ that we need to care about. We have a total of at most
$i_a*(i_a-1)$ many "distance two" leftover boundary components. These are the
only boundary components we can glue to to violate (\dag').

But by construction of $\widetilde{S}^{(N)}$, there are no more legal gluings. So
$\widetilde{S}$ can have at most $i_a+(i_a*(i_a-1)) = i_a^2$ many leftover boundary
components (that is, components of $\bd\widetilde{S}^{(N)}-\cR$) above $\gamma'$.  But since
$\widetilde{S}^{(0)}$ has the same number of boundary components above $\gamma$ and
$\gamma'$, and we're only allowed to glue them to each other, this implies that
there can be at most $i_a$ boundary components above $\gamma$ as well.  Since
our choice of $\widetilde{\gamma}$ was arbitrary, this implies that there are
at most $C(M,P,S_a) = i_a^2*j(M,P,S_a)$ many leftover boundary components, that
is, components of $\bd\widetilde{S}^{(N)}-\cR$.

%Proof of Theorem contd.

Now that we've constructed $\widetilde{S}^{(N)}$, we want to perform some additional
gluings to construct $\widetilde{S}$. In addition to (\dag) and (\dag'), $\widetilde{S}$ should
satisfy

\begin{enumerate}

\item[(\dag'')] Any two boundary components of $\widetilde{S}$ lie on different
pieces. That is, they correspond to boundary components of different connected
components of $\widetilde{S}^{(0)}$ under the gluing map $\widetilde{S}^{(0)}
\to \widetilde{S}$.

\end{enumerate}

Intuitively, this is easy using the reserve. The reserve on its own satisfies
(\dag''), so we only have to deal with the leftovers discussed above.  The trick
is the reserve is much larger than the leftovers, so if we allow ourselves to
glue the leftovers to the reserve we can take care of all the leftovers without
violating (\dag) or (\dag'). Our remaining boundary components will be a subset
of the reserve.

To be precise, construct $\widetilde{S}$ from $\widetilde{S}^{(N)}$ as follows.
For each boundary component $\gamma$ downstairs, as discussed above, there are
at most $i^2$ many elements of $\bd\widetilde{S}^{(N)}-\cR$ above it. For each of
these, there are at most $i^2$ we could glue to above $\gamma'$ that violate
(\dag) or (\dag').  Assume that $m\geq 2i^2$.  Then glue each leftover lift of
$\gamma$ to an element of $\cR$ above gama', one at a time. At any point there
will be at least $i^2+1$ elements of $\cR$ remaining above $\gamma'$, so we'll
always be able to choose one that doesn't violate (\dag) or (\dag'). Repeat
this process for each $\gamma$ to construct $\widetilde{S}$.  Since we glued
all the leftovers, it immediately follows that $\bd\widetilde{S} \cin \cR$, and
therefore $\widetilde{S}$ satisfies (\dag'').

Let's analyze the possibilities for $\#\bd\widetilde{S}$, the number of
boundary components. Above each $\gamma \cin dS_{a,k}$,
$\cR$ has $m$ elements. So $\bd\widetilde{S}$ has at most $m$ components above
$\gamma$, or $m*j$ many boundary components in total. The lower bound is
determined by the maximum number of leftovers, since the only way we'll remove
reserve components from $\bd\widetilde{S}$ is by gluing them to leftover
components. There are at most $i^2$ leftover components above $\gamma$, and
similarly above $\gamma'$, so there will be at least $m - i^2$ components
remaining in $\bd\widetilde{S}$ above $\gamma$. So

\[ m-i_a^2 \leq \#(\bd\widetilde{S} \text{ above }\gamma) \leq m \]

and

\[ (m-i_a^2)j_a \leq \#\bd\widetilde{S} \leq mj_a \]

Finally, suppose we've done the entire above construction for each boundary
component of our book of $I$-bundles $M$ to produce a partially-glued surface
with some boundary components left over. For each boundary component $\bd_aM$,
we'll call the surface $\widetilde{S_a}$. Note that we must be sure to make the same
choice of $n$ and $m$ for all these surfaces. Now we want to glue these
together, but they may have very different numbers of boundary components! So
we'll need to normalize them so they all have the same number of boundary
components above each $\gamma \cin dS_{a,k}$, so we can do a local construction
above a neighborhood of each spine in $M$.

Let $i_M = max a i_a$. By the inequality above, for each $\gamma \cin S_{a,k}$,
$\#(d\widetilde{S_a} \text{ above } \gamma) \geq m - i_M^2$. We want to make
this an equality for each $\widetilde{S_a}$ by performing some more gluings.

Again, we use the size of the reserve to our advantage. Assuming $m\geq i_M^2
+ i_a^2$ for each $\widetilde{S_a}$, by the same argument as above there will
always be choices remaining from the reserve to make the gluing. Combining
these all we need is $m\geq 2i_M^2$.

By abuse of notation, we'll call these normalized surfaces $\widetilde{S_a}$ also (the
non-normalized ones will not come up again). Each has exactly $C = m - i_M^2$
many boundary components above any $\gamma \cin dS_{a,k}$.

Build a single surface $\widetilde{S}$ by gluing the $\widetilde{S_a}$ as follows. Begin with
$\bigsqcup_a \widetilde{S_a}$.

For each spine $M_c$ of $M$, look at all the incident pages. Say there are $q$
of them.  Each page has two sides, so there are $2q$ many pieces $S_{a,k}$
downstairs near $M_c$, each with one boundary component glued along the spine
$M_c$.  Because $M$ is good, we know that all these $S_{a,k}$ are distinct and
they are all glued along simple parallel closed curves on $\bd M_c$ (do we know
they're longitudes?  See above \textbf{ TODO}).

Introduce notation as follows. $\cN(M_c)$ consists of $q$ neighborhoods inside
pages, attached to $M_c$ by annuli. The meridian of $M_c$ gives a cyclic order to
the attached pages.  Fix an orientation for the meridian and an (arbitrary)
starting point along it, and label the pages $1, \dots, q$ under this ordering.
The orientation of the meridian also gives an ordering of the boundary
components of each attaching annulus.  Using this, label the $2q$ boundary
components by $\gamma_1^-,\gamma_1^+,\dots, \gamma_q^-,\gamma_q^+$, where
$\gamma_i^-$ and $\gamma_i^+$ are the two sides of the attaching annulus for
the ith page.

Now consider $\bd M \cap \cN(M_c)$. Notice that the $2q$ boundary components of these
attaching annuli that we just labeled are precisely the $2q$ boundary curves of
the boundary pieces $S_{a,k}$ that are near $M_c$. We don't know how many global
boundary components $a$ are involved, but locally we do know which pieces are
attached inside $\bd M \cap \cN(M_c)$. $\bd M \cap M_c$ consists of q parallel annuli
along $\bd M_c$ that connect adjacent pages together. That is, the first annulus
has boundary $\gamma_1^+ \cup \gamma_2^-$, the second $\gamma_2^+ \cup
\gamma^3_-$, and so on.  It follows that these are the "pairs downstairs" in
the above construction of each $\widetilde{S_a}$. That is, these are the curves
we called $\gamma$ and $\gamma'$ earlier.  So within $\bigsqcup \widetilde{S_a}$,
the existing gluings only glue lifts of $\gamma_1^+$ to lifts of $\gamma_2^-$,
lifts of $\gamma_2^+$ to lifts of $\gamma_3^-$, etc etc.

Finally, the gluing. Because we normalized, each $\bigsqcup_a \widetilde{S_a}$ has
exactly $C$ boundary components above each $\gamma_i^\pm$.  Glue these $2qC$
boundary components as follows. Glue lifts of $\gamma_1^+$ to lifts of
$\gamma_3^-$, lifts of $\gamma_2^+$ to lifts of $\gamma_4^-$, and so on. In
general glue lifts of $\gamma_i^+$ to lifts of $\gamma_{i+2}^-$. Within each
subset of lifts, choose the gluing arbitrarily. We know the numbers on each
side are equal, so they'll match up.

Every boundary component of $\bigsqcup_a \widetilde{S_a}$ lives above a boundary component
of some $S_{a,k}$ which means it is attached to some core $M_c$. So once we've done
the above gluing step for all cores, the resulting surface $\widetilde{S}$ is closed.

%Claim.

Finally, we claim $\widetilde{S}$ satisfies (QF).

%Proof of Claim.

% TODO
\textbf{TODO simplify this argument using good property (3) as written up
earlier and discussed with Ian!!}

By construction, $\widetilde{S}$ consists of covers of each page of $M$ glued along
parallel longitudinal annuli at the spines of $M$. By the covering lemma, it
induces an immersed $\pi_1$-injective surface. It remains to show that
$\widetilde{S}$ cannot contain any lifts of parabolic curves.

Let $\alpha$ be a parabolic curve in some boundary component $S_a$ of $M$. Cut
$\alpha$ into arcs $\alpha_i$ along the page boundaries, so each $\alpha_i \cin
S_{a,k_i}$.  We can think about this as follows. For each $\alpha_i$, there are two
kinds of lifts to $\widetilde{S}$. We can view it as an arc in $P_{a,k_i} \cin
S_{a,k_i}$, and lift it to a cover of $S_{a,k_i}$, that is, a component of
$\widetilde{S_{a,k_i}}$. Since $\alpha \cin S_a$, this will necessarily be part
of $\widetilde{S_a} \cin \widetilde{S}$. Furthermore, by construction of
$\widetilde{S_a}$, lifting the $\alpha_i$ to $\widetilde{S_a}$ pieces and
gluing cannot yield closed curves.  It only yields a union of arcs, by (\dag).

However, the second way we can lift $\alpha_i \cin S_{a,k_i}$ is to a cover of the
opposite page boundary $\overline{S_{a,k_i}}$. Let $S_b$ be the boundary component
containing $\overline{S_{a,k_i}}$, and write $S_{b,l_i} = \overline{S_{a,k_i}}$, we can see that $\alpha_i$
in $\overline{P_{b,l_i}}$, the set of "opposite parabolic arcs" that we defined earlier.
So $\alpha_i$ lifts to $S'b,li$ as a union of arcs, each of which connects two
different boundary components of $S'b,li$. This produces a set of arcs in
$\widetilde{S_b}$. Note that it may be possible to have $S_a = S_b$, depending on
how our book of $I$-bundles is constructed.  Regardless, we want to think of
these as a different kind of lift, because we're using the opposite page side
to lift, rather than the side the arc "naturally lives on."

We now claim that these arcs cannot be used to form closed curves in
$\widetilde{S_b}$.  Intuitively, this is because the opposite side of the
boundary does not remain parallel to $\alpha$ for long enough, so we'll soon
reach a piece of $\widetilde{S_b}$ where our lift can't continue.  To be
precise, let $\widetilde{\alpha_i}$ be a lift of $\alpha_i$ to
$\widetilde{S_b}$.  Suppose $\alpha_i$ has an endpoint at a spine $M_c$. If we
look at how the boundary components of a book of $I$-bundles behave near
a spine, we that locally the two boundary sides of a page are attached to sides
of two different pages.  In the terminology we used earlier for gluing pages at
a spine, $\gamma_i^-$ and $\gamma_i^+$ are attached to the attaching annulus
boundary curves of two different pages: $\gamma_i^-$ attaches to
$\gamma_{i+1}^+$, but $\gamma_i^+$ attaches to $\gamma_{i-1}^-$.  Since there
are at least three pages at $M_c$ by (3), the $(i-1)$st and $(i+1)$st pages are
distinct.

Suppose that the endpoint of $\alpha_i$ we're considering lies on $\gamma_i^-$. That
is, $\gamma_i^-$ is a boundary component of $S_{a,k_i}$, in our other notation. And
$\gamma_i^+$ is a boundary component of the opposite page $\overline{S_{a,k_i}} = S_{b,l_i}$.
Since $\gamma_i^-$ attaches to $\gamma_{i+1}^+$, any lifted arc we attach
$\widetilde{\alpha_i}$ to in that direction must lift from that $(i+1)$st page at $M_c$.
But in $\widetilde{S_b}$, we attach the surface pieces according to the gluings needed
for the $S_b$ boundary component.  Locally, $\widetilde{\alpha_i} \cin \widetilde{S_b},li$, where
lifts of $\gamma_i^+$ only attach to lifts of $\gamma_{i-1}^-$. But $\gamma_{i-1}^-$ is
a boundary component of a different page, so $\widetilde{\alpha_i}$ connects to a next
page with no lift of the next segment of $\alpha_i$. So there is no possible way
to continue $\widetilde{\alpha_i}$.  The same argument applies at the opposite endpoint
of $\widetilde{\alpha_i}$. This argument shows that lifts of arcs $\alpha_i$ as "opposite
parabolic arcs" to $S_b$ have no "opposite arcs" on either side that they can
glue to.

Finally, suppose our parabolic curve $\alpha \cin S_a$, after cutting into
$\alpha_i$, lifts to pieces $\widetilde{\alpha_i}$ which form a closed curve $\widetilde{\alpha}
\cin \widetilde{S}$.  Cut $\widetilde{\alpha}$ along only the gluings between different
$\widetilde{S_b}$ done in the final gluing step. Each piece $\widetilde{\alpha_j}$ now consists
of multiple $\widetilde{\alpha_i}$, which are either all "natural lifts" or all
"opposite lifts" (in the above sense). This is simply because each $\widetilde{S_b}$
is, except for its extra boundary components, a cover of the corresponding
$S_b$, and there's no way $\alpha$ to locally jump from being on the opposite
side of a page from $S_b$ to suddenly being on the same side. However, at the
gluings in the final step, we attach covers of different boundary components
together, so we may have attached multiple pieces $\widetilde{\alpha_j}$ of the same
type, or different types.

It is impossible to attach two natural segments $\widetilde{\alpha_j}$ in the final
step.  Suppose we had such an attachment, and look locally at the spine where
they're attached. Our attachment must yield a lift of $\alpha$, so we can look
at the local neighborhood in $\alpha$ covered by a neighborhood of our
attachment point.  Near the spine, this neighborhood must live in a single
local boundary component. But as we discussed in detail above, each local
boundary component simply consists of small neighborhoods on two page
boundaries, joined together by an annulus. Locally, a natural segment can only
be obtained by lifting to a cover of one of these two page boundaries, not any
of the other page boundaries near this spine. So one side of our attachment
must lie on one page boundary, and the other side on the other. But by
definition of how we do our final attaching step, we never attach in such a way
that we follow along the local boundary components! This is the point of having
valence at least 3 at each spine, and attaching $\gamma_i^+$ to
$\gamma_{i+2}^-$, is to avoid this problem. Since we never do attachments of
this form, we can't have an attachment with natural lifts on both sides.

So each attachment must have an opposite lift on one or both sides. But recall
that by (\dag''), each piece of each $\widetilde{S_b}$ has at most one free boundary
component. So all the $\widetilde{\alpha_j}$ must traverse at least two pieces, as we
ensured by construction of the $S'_{a,k}$ that they couldn't begin and end at the
same boundary component. This is where we finally use all those
carefully-established earlier criteria.

Because finally, as shown above, opposite lifts of individual $\alpha_i$ cannot
connect to opposite lifts on either side. So it's impossible to construct
a $\widetilde{\alpha_j}$ made out of opposite lifts, since it has to traverse
more than one piece.  This contradicts our assumption that the
$\widetilde{\alpha_j}$ glue to form a closed curve.

So we cannot join up the lifts of $\alpha_i$ to form a lift of $\alpha$.
Applying the parabolic lifting criterion shows that $\widetilde{S}$ is (QF).
This completes the proof of the theorem.

\end{proof}

\begin{thm}

Let $M$ be a good book of $I$-bundles.  Suppose $M$ does contain a component of
the parabolic locus $P$ inside a single page's boundary.  Construct $M'$ by
deleting all such pages from $M$, and $P'$ by removing all components of the
pared locus that intersect those pages. Then $(M,P)$ contains a surface
satisfying (QF) if and only if $(M',P')$ is nonempty and contains
a surface satisfying (QF).

\end{thm}

\begin{proof}

First observe that if $(M,P)$ contains a surface satisfying (QF), it cannot
traverse any of the pages we deleted to obtain $M'$, by the same argument as in
our first example.  Since our surface exists, $M'$ must be nonempty.
Furthermore, since $(M',P') \cin (M,P)$, it satisfies (QF) for $M',P'$ as well.
Conversely, if we have such a surface, we can obviously view it as contained in
$(M,P)$, where it must satisfy (QF) because it doesn't intersect the deleted
pages at all.

Note that $(M',P')$ may be nonelementary, or it still may not be good.
However, we can repeat the steps needed to guarantee that it's good, and obtain
a manifold where we can either apply this theorem again, or use the first
theorem. It is not immediately clear that this process terminates, because
taking the finite-sheeted cover we use for (2) and (3) makes the manifold more
complicated, possibly increasing the number of parabolics. We'll have to do
this carefully. Possibly we should do this step BEFORE the other
simplification, but then we'll have to make sure the parabolics pass through
that simplification nicely.

\end{proof}
