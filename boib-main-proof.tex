%%%%%%%%%%%%%%%%%%%%%%%%%%%%%%%%%%%%%%%%%%%%%%%%%
\section{The case of a quasi-Fuchsian surface}
%%%%%%%%%%%%%%%%%%%%%%%%%%%%%%%%%%%%%%%%%%%%%%%%%

We are now ready to state the precise theorem in the positive case as well.

\begin{thm}[Main theorem, positive case]

Every reduced book of $I$-bundles contains a (QF) surface.

\end{thm}

Note that together with the earlier reduction theorem, we've fully covered the
non-reduced case also. This is the main result of this paper.

\begin{thm}[Main theorem]

Let $M$ be a book of $I$-bundles. If the reduction theorem yields a nonempty
set of reduced books of $I$-bundles and thickened closed surfaces in $M$, then
$M$ contains a (QF) surface.  Otherwise, it does not.

\end{thm}

%%%%%%%%%%%%%%%%%%%%%%%%%%%%%%%%%%%%%%%%%%%%%%%%%
\section{Simplification of main theorem}
%%%%%%%%%%%%%%%%%%%%%%%%%%%%%%%%%%%%%%%%%%%%%%%%%

We now make a number of topological simplifications by passing to an
appropriate finite-sheeted cover of the book of $I$-bundles.  Once we can
construct a surface satisfying (QF) inside this cover, we will push it down and
perturb to obtain a surface satisfying (QF) downstairs (since it
$\pi_1$-injects into a subgroup, it will definitely still be
$\pi_1$-injective).

\begin{defn}

A \emph{good} book of $I$-bundles $M$ is a reduced book of I-bundles which
satisfies the following additional conditions.

\begin{enumerate}

\item Each binding annulus on a spine intersects a meridian disk of that spine
exactly once. That is, every spine has degree 1.

\item Each page is glued to a given spine at most once. That is, for any page
$B$ and spine $C$, $B \cap C$ is at most a single component of $\cA$.

\item The two endpoints of a fiber in each page are in different boundary
components of $M$. In particular, each page is a trivial $I$-bundle over an
oriented surface.

\item Each spine intersects each boundary component of $M$ in at most a single
annulus.

\item Each arc of $P$ on a page connects two different binding annuli. That is,
there are no essential arcs that begin and end at the same binding annulus.

\end{enumerate}

\end{defn}

\begin{thm}

Let $M$ be a reduced book of $I$-bundles. Then $M$ has a finite-sheeted regular
cover which is good.

\end{thm}

\begin{proof}

To prove this, we'll use the fact that $\pi_1M$ is LERF to construct finite
sheeted covers with nice properties. Note that (1), (2), (3), and (4) are all
properties that once true, remain true when we lift to finite-sheeted covers.
So it suffices to guarantee each property one at a time by lifting to
finite-sheeted covers, because we can lift repeatedly and the old properties
will still hold. We also know that finite-sheeted covers of $M$ will be
reduced.

We first take care of condition (1). For each spine $C$, let $k=d(C)$.  Let
$A_1,\dots,A_n$ be the binding annuli on $C$. These annuli must be parallel.
Notice that $\pi_1C$ is infinite cyclic, as its only torsion could come from
the identification with the binding annuli, but we've assumed that these are
$\pi_1$-injective. Now if $\alpha$ is a generator for the infinite cyclic group
$\pi_1C$, $\alpha^k$ generates each $\pi_1A_i \cin \pi_1C$. Fix a point $x_0
\in C$, and apply Proposition~\ref{P:lerf3} to $\alpha$ and $k$ to obtain
a finite-sheeted regular cover $M' \to M$.  For any spine $C'$ covering $C$, we
claim that $C'$ has degree 1. Observe that $C'\to C$ is a degree $d$ cover with
$k$ dividing $d$.  Therefore $\alpha^d$, which lifts to traverse $C'$ exactly
once, is a power of $\alpha^k$, the generator of each binding annulus. Hence
each binding annulus has preimage a union of components, each of which
traverses $C'$ exactly once.  Hence $C'$ has degree 1.

We now produce a cover $M'$ that satisfies (1). Perform the above construction
for each spine to produce generators $\alpha_1,\dots,\alpha_n$ and degrees
$k_1,\dots,k_n$. Applying Proposition~\ref{P:lerf3'}, we can follow the above
argument locally on each spine in $M'$ to show that it has degree 1.

So now assume $M$ is reduced and satisfies (1). We claim $M$ has
a finite-sheeted cover which satisfies (2).

Let $B$ be a page intersecting a spine $C$ more than once. For each pair of
binding annuli $A_1$,$A_2$ which attach $B$ to $C$, we'll draw a closed curve
as follows.  Fix a meridian disk $D$ of $C$. Choose an arbitrary point $x_1 \in
A_1 \cap D$, $x_2 \in A_2 \cap D$. Choose an arbitrary arc $\alpha' \cin B$
connecting $x_1$ and $x_2$.  Let $\alpha$ be the closed curve obtained by
closing up $\alpha'$ with an arc along the interior of $D$. Denote this arc by
$\alpha''$.

Fix $x_1$ to be our basepoint. We claim that $\alpha \notin \pi_1B$. $\alpha$
intersects each of the binding annuli $A_1$ and $A_2$ in a single point. Any
homotopy of $\alpha$ to a curve in $\pi_1B$ would have to homotope $\alpha''$
rel boundary into $B$, but this is clearly impossible. So $\alpha \notin
\pi_1B$.  $\pi_1B$ is finitely generated. Apply Proposition~\ref{P:lerf1} to
$\alpha$ and $B$, obtaining a finite-sheeted cover $p\colon M'\to M$.

Any lift of $\alpha$ to $\alpha'$ in this cover must connect two different
components of $p^{-1}(B)$.  Any lift of $\alpha''$ still lies on a meridian
disk of a spine covering $C$, and connects binding annuli covering $A_1$ and
$A_2$.  But $\alpha'$ lifts to arcs inside $p^{-1}(B)$. So in order for a lift
of $\alpha$ to connect two different components of $p^{-1}(B)$, the lifts of
$\alpha''$ must have endpoints in different components of $p^{-1}(B)$.

We now construct a cover $M'$ that satisfies (2). Repeat the above construction
for each pair of binding annuli for each page with multiple attachments to
a spine. This yields closed curves $\alpha_1,\dots,\alpha_n$ and corresponding
pages $B_1,\dots,B_n$. Note that there may be duplicates in this list of pages,
but the argument remains the same. Apply Proposition~\ref{P:lerf1'}. By
following the above argument locally, we see that any page upstairs with
multiple gluings to the same spine would correspond to a lift of some
$\alpha_i''$ with endpoints in the same component of $p^{-1}(B_i)$,
a contradiction.

So we have a cover which satisfies (1) and (2). We now show that $M$ satisfying
(1) and (2) has a finite-sheeted cover satisfying (3).

$M$ is orientable by definition, so each page must be orientable as well. So
the pages which are not trivial $I$-bundles over oriented pages must be twisted
$I$-bundles over nonoriented pages (in order for the resulting page to be
orientable). In particular, the endpoints of the fibers of a twisted $I$-bundle
will connect globally to form a single side of the page. So if we look locally
at an fiber, the two endpoints are guaranteed to be in the same boundary
component of $M$ (since they're in the same boundary component even if we just
look at that page). This explains why ensuring distinct boundary components for
each fiber guarantees trivial $I$-bundles.

This argument is very similar to (2). Let $B$ be a page such that fibers have
both endpoints on the same boundary component $\bd_a M \cin \bd M$.  Choose
a binding annulus $A \cin \bd B$, and an arbitrary transversal $\alpha'' \cin
A$.  Let $\alpha' \cin \bd_a M$ be an arc in the boundary connecting the
endpoints of $\alpha''$. Such an $\alpha'$ must exist as the two endpoints are
part of the same boundary component. $\alpha'$ and $\alpha''$ combine to form
a closed curve $\alpha$.  Fix one endpoint of $\alpha'$ to be our basepoint.
$\bd M$ is a compact surface, so $\pi_1 \bd_a M$ is finitely generated.

We claim that $\alpha \notin \pi_1 \bd_a M$. As above, any such homotopy would
correspond to a homotopy rel boundary of $\alpha''$ into $\bd_a M$. Since
$\alpha''$ is a transversal of the binding annulus $A$, such a homotopy would
correspond to a boundary compression of $A$, which is impossible as $A$ is
boundary incompressible. Apply Proposition~\ref{P:lerf1} to $\alpha$ and $\bd_a
M$, obtaining a finite-sheeted cover. By the same argument as in (2),
$\alpha''$ must lift to connect two different components of $p^{-1}(\bd_a M)$.
Since $\bd M' = p^{-1}(\bd M)$, this shows that the pages covering $B$ have
fibers that connect two different boundary components. We can therefore
construct $M'$ satisfying (3), by taking all such pages with both sides on the
same boundary component and applying Proposition~\ref{P:lerf1'}.

We can satisfy (4) by a similar argument. Let $C$ be a spine that intersects
the same boundary component $\bd_a M$ twice, and $\alpha'$ to be an arc of
a meridian disk of $C$ which connects two components of $C \cap \bd_a M$.
Connect the endpoints of $\alpha'$ with an arc $\alpha'' \cin \bd_a M$. By the
same argument as above with boundary compressions, $\alpha=\alpha' \cup
\alpha'' \notin \pi_1 \bd_a M$, so following the same argument proves (4).

So we have a cover which satisfies (1), (2), (3), and (4). To satisfy (5), let
$\alpha'$ be the core of a component of $P \cap B$ which has both endpoints in
the same binding annulus $A$. Let $\alpha''$ be a segment in $\bd A$ which
connects the two endpoints of $\alpha'$. Let $\alpha = \alpha' \cup \alpha''$
be the resulting closed curve. Since $\alpha'$ is essential, $\alpha \notin
\pi_1A$.  Apply Proposition~\ref{P:lerf1} to $\alpha$ and $A$, and follow the
argument used in (2) and (3).  This proves that $M$ has a cover satisfying
(1)-(5), completing the proof.

\end{proof}

%%%%%%%%%%%%%%%%%%%%%%%%%%%%%%%%%%%%%%%%%%%%%%%%%
\section{Proof of main theorem}
%%%%%%%%%%%%%%%%%%%%%%%%%%%%%%%%%%%%%%%%%%%%%%%%%

We've now reduced to the case of a good book of $I$-bundles. However, the
remaining work is still quite involved. We proceed somewhat similarly to the
first example.  We take covers over the pages and glue them together cleverly
to construct our surface.  We check that the resulting surface satisfies the
pared lifting criterion. As in the first example, this suffices to prove that
our surface is (QF).  We first need to take the correct cover over each page.
However, the details are more complex. We first will need the following
important notion.

\begin{defn}

Let $M$ be a book of $I$-bundles. A \emph{jigsaw surface} $Q \cin
M$ is obtained from a closed $\pi_1$-injective surface $S$ in minimal position
(following the surface decomposition lemma, Lemma~\ref{L:sfcdecomp}) by
deleting any number of components of $Q \cap C$, for each spine $C$ of $M$. The
result is a possibly disconnected surface with boundary, where each component
is $\pi_1$-injective and in minimal position. If $(M,P)$ is a pared book of
$I$-bundles, the \emph{pared lifting pattern} on $Q$ is defined componentwise
in the surface decomposition, just like for closed surfaces. We say that $Q$
satisfies the \emph{pared lifting criterion} if its pared lifting pattern
contains no closed curves.

\end{defn}

Intuitively, a jigsaw surface looks like a partially completed jigsaw puzzle.
The ``pieces'' are components of $Q \cap B$, where $B$ is a page in $M$.  Some
boundary components of these pieces are glued together (via annuli in $Q \cap
C$), while others are exposed (as those annuli were deleted).

Note that at each spine $C$ of $M$, there must be an even number of boundary
components of $Q$ which lie in $C$, by parity considerations with the deleted
annuli. Also note that, assuming $M$ is reduced, it is trivial to build
a jigsaw surface that satisfies the pared lifting criterion. To
do so, simply take a closed $\pi_1$-injective surface and delete all the
intersections with spines.  No individual page contains a pared closed curve,
so this trivial jigsaw surface (which is just a disjoint union of
page covers) cannot either.

\begin{defn}

Let $Q$ be a jigsaw surface in a book of $I$-bundles $M$. The
\emph{defect} of $Q$ is the total number of boundary components, summing over
all connected components. The \emph{local defect} at a spine $C \cin M$ is the
number of boundary components which intersect $C$. The \emph{degree} of $Q$ is
the covering degree on each page. (Note that since some spine components were
deleted, the covering degree on a spine may be smaller.)

\end{defn}

Our strategy is as follows. For each boundary component of $M$, we will build
a jigsaw surface. This surface will be obtained by removing spine
components from a cover of that boundary component. We will then align the
number of boundary components of the different jigsaw surfaces
where they meet at spines, and glue them all together with new annuli. This
will produce a closed surface. As suggested by the above terminology, we want
our jigsaw surface to have as small a defect as possible.

We will also need our jigsaw surface to satisfy the following
technical conditions:

\begin{defn}

Let $Q$ be a jigsaw surface in a book of $I$-bundles $M$. We say that $Q$
\emph{has no protoloops} if it satisfies the following two conditions:

\begin{enumerate}

\item The pared lifting pattern on $Q$ contains no closed curves.

\item Every arc in the pared lifting pattern on $Q$ begins and ends at two
different boundary components of $Q$.

\end{enumerate}

Otherwise, we say that $Q$ \emph{has protoloops}.

\end{defn}

\begin{defn}

We say that $Q$ \emph{has sparse defect} if, in the surface decomposition of
$Q$, each component of $Q \cap B$ contains at most one component of $\bd Q$,
where $B$ is an arbitrary page of $M$.  That is, the ``defective'' (non-glued)
boundary components are ``spread out'' among the components of $Q$ in each
page.

\end{defn}

\begin{lemma}\label{L:jigsaw}

Let $M$ be a good book of $I$-bundles. Fix a boundary component $F$ of $M$.
Then, for any sufficiently large $d$, there exists a jigsaw surface $Q \cin M$
with the following properties:

\begin{enumerate}

\item $Q$ is obtained by removing spine components from (a surface homotopic
to) a finite-sheeted cover of $F$ of degree $d$.

\item $Q$ has no protoloops.

\item $Q$ has sparse defect.

\item $Q$ has defect bounded above by a constant $C$ depending only on $(M,P)$
(i.e., not depending on $d$).

\end{enumerate}

\end{lemma}

\begin{proof}

Suppose that $F$ intersects the spines $C_1,\dots,C_r$ and pages
$B_1,\dots,B_s$ of $M$.  Since $M$ is good, $F$ intersects each of these in
a single component.  Denote the page intersections by $F_i = F \cap B_i$ and
the spine intersectons by $F_j' = F \cap C_j$. See Figure~\ref{F:bdcomponent}.

\mytallrotfig{fig-bcexample}{F:bdcomponent}{Intersections of $F$ with pages,
spines, and binding annuli}

Let $Q_0$ be the surface obtained by taking $d$ many disjoint copies of each of
the $F_i$. $Q_0$ is obviously a jigsaw surface obtained by
removing all spine components from the trivial $d$-fold cover of $F$ (that is,
$d$ many disjoint copies of $F$), which is a closed $\pi_1$-injective surface
in $M$.  The pared lifting pattern on each component of $Q_0$ is a homeomorphic
copy of (the core of) $P \cap F_i$. Since $M$ is reduced, this pattern has no
closed curves.  Since $M$ is good, every arc in the pattern connects two
different boundary curves of its component. Hence $Q_0$ has no protoloops.

However, $Q_0$ obviously fails conditions (3) and (4) of the lemma. We need to
add some annuli to ensure these conditions hold.

Fix a large integer $m>0$. Now, reserve a subset $\cR \cin \bd Q_0$.  This
subset should have the following properties:

\begin{enumerate}

\item $\cR$ is a union of connected components of $\bd Q_0$.

\item Each boundary component of each $F_i$ is covered by exactly $m$ elements
of $\cR$.

\item Every connected component of $F_i$, that is, lifted copy of each page,
contains at most one component of $\cR$.

\end{enumerate}

This is possible as long as $d \geq c_0m$, where $c_0$ is the maximum number of
boundary components of any $F_i$. $c_0$ is fixed and depends only on $(M,P)$.
If $d\geq c_0m$,  each local cover $Q_0 \cap B_i$ has at least $c_0m$ many
disjoint copies above its respective boundary piece $F_i$.  So we have enough
copies to make our choices above.

We begin adding annuli that are copies of the annuli $F \cap C_j = F_j'$. Given
any such annulus $F_j'$, we know there are exactly two pages $B_{i_1}$ and
$B_{i_2}$ such that the two boundary components of $F_j'$ are attached to
$F_{i_1}$ and $F_{i_2}$ in $F$.  There are in fact two specific boundary
components $\gamma_1$ of $F_{i_1}$ and $\gamma_2$ of $F_{i_2}$ that are
attached in this way.  Each time we add an annulus copy of $F_j'$ to $Q_0$, we
must attach it to two curves $\widetilde{\gamma_1}$ and $\widetilde{\gamma_2}$
which are lifted copies of $\gamma_1$ and $\gamma_2$ in components of $Q_0 \cap
B_{i_1}$ and $Q_0 \cap B_{i_2}$, respectively.  This is necessary to ensure
that $Q_0$ remains a jigsaw surface that can be extended to a cover of $F$. See
Figure~\ref{F:buildingQ1}.

\mytallfig{fig-buildingQ1}{F:buildingQ1}{Adding annuli to $Q_0$}

We add annuli one at a time. Each time we add an annulus, we make sure to add
in such a way that the resulting surface still has no protoloops. We also
require that we do \emph{not} attach any annuli to the boundary components in
$\cR$.  Aside from these requirements and the condition above (that our gluing
extends to a cover of $F$), we repeatedly choose an arbitrary annulus. This
process will terminate at some some surface $Q_1$, after there are no more
possible gluings to make.

\begin{claim}

We claim that $\#(\bd Q_1 \setminus \cR)$ is bounded by $c$, a constant
depending only on $(M,P)$. In particular, $c$ is independent of the choice of
$d$ and $m$ made earlier.

\end{claim}

\begin{proof}[Proof of Claim]

Consider an arbitrary boundary component $\widetilde{\gamma} \cin \bd Q_1
\setminus \cR$.  It covers some boundary component $\gamma \cin F_i$. Look at
the number of arcs of $P \cap F_i$ which are incident to $\gamma$. This number
depends only on $(M,P)$. Let $c_1$ be the maximum such number of incident arcs
of $P$ for any boundary component $\gamma$ of any $F_i \cin F$. Let $c_2$ be
the total number of boundary components of the $F_i$. Let $\gamma' \cin F_{i'}$
be the boundary component matching $\gamma$ --- that is, the boundary component
that is attached to $\gamma$ in $F$ by some spine annulus. Let $F_j'= F \cap
C_j$ be the annulus in $F$ that connects them.  Suppose we attempt to add an
annulus to $Q_1$ that covers $F_j'$, with one of its boundary components
attached to $\gamma$.

The cover extension condition allows us to attach the other boundary component
to any component of $\bd(Q_1 \cap F_{i'}) \setminus \cR$ above $\gamma'$.
Because we always glue matching boundary components (and we started with the
same number of each), $\bd Q_1 \setminus \cR$ has the same number of remaining
boundary components above $\gamma'$ as it does above $\gamma$. So there must be
at least one to glue to.  Since we stopped gluing at $Q_1$, this implies that
any gluing we could make would force the resulting surface to have protoloops.

Let $\widetilde{\gamma}' \cin \bd(Q_1 \cap F_{i'}) \setminus \cR$ be an
arbitrary available lift of $\gamma'$.  Suppose gluing to $\widetilde{\gamma}'$
produces a closed curve $\alpha$ in the pared lifting pattern. Obviously
$\alpha$ intersects the newly added annulus $A$, otherwise $Q_1$ would already
contain a closed curve. $P$ is in minimal position and $M$ is reduced, so
within $A$ the pared lifting pattern is just a union of transverse arcs.
Removing $A$ therefore divides $\alpha$ into a union of arcs with endpoints on
either $\gamma$ or $\gamma'$.  $Q_1$ has no protoloops, so none of these arcs
can have both endpoints on the same boundary component. So there exists an arc
in $Q_1$ connecting $\gamma$ to $\gamma'$.  (This is the reason for the second
condition in our definition of a ``protoloop'').  The pared lifting pattern in
$Q_1$ near $\gamma$ is a homeomorphic copy of (the core of) $P \cap F_i$, so
the number of arcs incident to $\gamma$ is bounded by $c_1$.  Tracing these
arcs through $Q_1$, they can hit at most $c_1$ other boundary components.
These are the only boundary components we can attach our annulus to to produce
a closed curve.

Similarly, suppose gluing to $\widetilde{\gamma}'$ produces an arc $\alpha$
with endpoints on the same boundary component. Let $\widetilde{\delta}$ be this
boundary component. Again, $\alpha$ intersects the added annulus $A$, otherwise
$Q_1$ would already have protoloops. Removing $A$ divides $\alpha$ into a union
of arcs.  Except for the two endpoints of $\alpha$ (which now lie on two
different subarcs --- call these subarcs $\alpha_0$ and $\alpha_1$), all other
endpoints of these arcs must either lie on $\gamma$ or $\gamma'$.  But no arc
can have both endpoints on the same boundary component, or $Q_1$ would already
have protoloops.  So any subarcs except $\alpha_0$ and $\alpha_1$ must connect
$\gamma$ to $\gamma'$.  By parity we can see that $\alpha_0$ and $\alpha_1$
both have one endpoint on $\widetilde{\delta}$, but their other endpoints must
be different.  That is, either $\alpha_0$ ends on $\widetilde{\gamma}$ and
$\alpha_1$ on $\widetilde{\gamma}'$, or vice versa.  This is the set of
circumstances that leads to a returning arc.

Now, as above, the arcs incident to $\widetilde{\gamma}$ hit at most $c_1$
other boundary components. These are our possible $\widetilde{\delta}$. Each of
these has at most $c_1$ incident arcs itself, one of which returns to
$\widetilde{\gamma}$, leaving $c_1-1$ that we need to care about. We have
a total of at most $c_1(c_1-1)$ many ``distance two'' available boundary
components.  If we attach our annulus to one of these boundary components, we
will produce a surface with protoloops.

By construction of $Q_1$, there are no more legal gluings. So every remaining
boundary component must be disallowed for one of the above reasons. So $Q_1$
can have at most $c_1+(c_1(c_1-1)) = c_1^2$ many leftover boundary components
(that is, components of $\bd Q_1 \setminus \cR$) above $\gamma'$.  Since our
choice of $\widetilde{\gamma}$ was arbitrary, this implies that there are at
most $c = c_1^2c_2$ many leftover boundary components, that is, components of
$\bd Q_1\setminus\cR$. This proves the claim.

\end{proof}

\noindent \emph{Proof of Lemma~\ref{L:jigsaw} continued.} We now attach more
annuli to build a surface $Q$ which satisfies (3) and (4). Intuitively, the
boundary components that make up the reserve are sparse. As long as our reserve
is sufficiently large we can make use of it to attach enough annuli that only
reserve boundary components remain. All the remaining non-reserve boundary
components will be attached to reserve boundary components by annuli.

Construct $Q$ from $Q_1$ as follows.  For each boundary component $\gamma$
downstairs, as discussed above, there are at most $c_1^2$ many components of
$\bd Q_1 \setminus \cR$ above it.  For each of these, there are at most $c_1^2$
we could glue to above $\gamma'$ that produce a surface with protoloops. We now
allow ourselves to glue to the reserve boundary components. Assume that $m\geq
2c_1^2$.  Then attach each leftover (non-reserve) lift $\widetilde{\gamma}$ of
$\gamma$ to a component of $\cR$ above $\gamma'$ with a copy of the annulus
$F_j'$, one at a time.  At any point there will be at least $c_1^2+1$
components of $\cR$ remaining above $\gamma'$. As in the proof of the claim, at
most $c_1^2$ of these elements have arc connections with $\widetilde{\gamma}$
that will cause that gluing to yield protoloops. So we'll always be able to
choose one such that our surface still has no protoloops.  Repeat this process
for each $\gamma$ to construct $Q$.  Since we glued all the leftovers, it
immediately follows that $\bd Q \cin \cR$, and therefore $Q$ satisfies (3).

Let's analyze the possibilities for the defect of $Q$, that is, the number of
boundary components.  Above each $\gamma \cin \bd F_i$, $\cR$ has $m$ elements.
So $\bd Q$ has at most $m$ components above $\gamma$, or $c_2m$ many boundary
components in total. The lower bound is determined by the maximum number of
leftovers, since the only way we'll remove reserve components from $\bd Q$ is
by gluing them to non-reserve leftover components. There are at most $c_1^2$
leftover components above $\gamma$, and similarly above $\gamma'$, so there
will be at least $m - c_1^2$ components remaining in $\bd Q$ above $\gamma$.
We find that

\[ c_2(m-c_1^2) \leq \operatorname{defect}(Q) \leq c_2m \]

Any choice of $m\geq 2c_1^2$ works, as long as $d$ is sufficiently large
relative to $m$.  This proves (4).

\end{proof}

\begin{proof}[Proof of Main Theorem]

We use Lemma~\ref{L:jigsaw} to construct a jigsaw surface $Q_a$ for each
boundary component $F_a$ of $M$. Now we want to glue these together, but they
may have very different numbers of boundary components! So we'll need to
normalize them so they all have the same number of boundary components near
each spine, so we can do a local construction with annuli in each spine.

Let $c_a$ be the constant $c_1$ we computed above for each jigsaw surface
$Q_a$.  Let $c_{\mathrm{max}} = \operatorname{max}_a c_a$, that is, the maximum
number of components of $P \cap B_i$ incident to a single boundary component of
$\bd M \cap B_i$ for any spine $B_i$. By the inequality in
Lemma~\ref{L:jigsaw}, for each $\gamma \cin \bd F_i$, $\#\{\text{components of
} \bd F_i \text{ above } \gamma\} \geq m - c_{\mathrm{max}}^2$. We want to make
this an equality for each $Q_a$ by adding some more annuli to $Q_a$ in the same
way that we added annuli in Lemma~\ref{L:jigsaw}.

Again, we use the size of the reserve $\cR$ to our advantage. Assuming $m\geq
c_{\mathrm{max}}^2 + c_a^2$ for each $F_a$, by the same argument as the lemma,
there will always be choices remaining from the reserve $\cR$ to make the
gluing.  Combining these all we need is $m\geq 2c_{\mathrm{max}}^2$.

By abuse of notation, we'll call these normalized surfaces $Q_a$ also (the
non-normalized ones will not come up again). Each has exactly $m'
= m - c_{\mathrm{max}}^2$ many boundary components above any $\gamma \cin \bd
F_i$.

Build a single closed surface $S$ by attaching annuli to the $Q_a$ as follows.

Let $C$ be an arbitrary spine of $M$. There are exactly $d(C)$ many binding
annuli in $\bd C$, and $C$ is attached to $d(C)$ distinct pages, since $M$ is
good. Each page has two sides, so there are $2d(C)$ many ``incident neighbor
types'' in the $Q_a$, corresponding to the $2d(C)$ boundary components of
binding annuli in $\bd C$. These are attached to the $2d(C)$ side boundary
components of pages adjacent to $C$.  Because $M$ is good, we know that all
these pages are distinct and all the corresponding binding annuli have degree
1.

Introduce notation as follows. A neighborhood $\cN(C)$ consists of $d(C)$
neighborhoods inside pages, attached to $C$ by annuli. The meridian of $C$
gives a cyclic order to the attached pages.  Fix an orientation for the
meridian and an (arbitrary) starting point along it, and label the pages $1,
\dots, d(C)$ under this ordering.  The orientation of the meridian also gives
an ordering of the boundary components of each binding  annulus.  Using this,
label the $d(C)$ gluing annuli by $A_1,\dots,A_{d(C)}$. Also label the $2d(C)$
binding annulus boundary components by $\gamma_1^-,\gamma_1^+,\dots,
\gamma_{d(C)}^-,\gamma_{d(C)}^+$, where $\gamma_i^-$ and $\gamma_i^+$ are
the two boundary components of the binding annulus for the $i$th page.
Extend these labels to the corresponding boundary components of the page
boundaries attached to the binding annuli.  Remember that all these labels are
downstairs --- that is, they're labels of pieces of $\bd M$.

By construction of the $Q_a$, there are $2d(C)m'$ many components of
$\bd(\bigsqcup_a Q_a) \cap C$. Attach them as follows. Let $A_{1,3}$ be an
embedded valence 1 annulus in $C$ with one boundary component in $A_1$ and one
in $A_3$. Define $A_{2,4},A_{3,5},\dots,A_{(d(C)-1),1},A_{d(C),2}$ similarly.
Now construct the surface $S$ by taking $m'$ many copies of each $A_{i,i+2}$,
and attaching their boundary components to lifts of $\gamma_i^+$ and
$\gamma_{i+2}^-$. See Figure~\ref{F:finalgluing}. By construction of the $Q_a$
there will be exactly $m'$ many available lifts of each to attach to.
Therefore, when we repeat this for each spine $C$, the result is a closed
surface $S$.  Note that since $M$ is reduced, we know that each spine has at
least 3 annuli.  This means we'll never attempt to attach an annulus with both
boundary components in the same binding annulus.

\myfig{fig-finalgluing}{F:finalgluing}{Attaching the $Q_a$ at $C$}

We claim that $S$ satisfies the pared lifting criterion.

Fix a boundary component $F_a \cin \bd M$. Consider an arbitrary parabolic core
curve $\alpha \cin F_a$. We claim that the pared lifting pattern components of
$\alpha$ in $S$ do not connect to form a closed curve $\widetilde{\alpha}$.  If
we can prove this for every boundary component $F_a$ and parabolic curve
$\alpha$, $S$ will satisfy the pared lifting criterion.  We break down the
possibilities for $\widetilde{\alpha}$ as follows.

\begin{enumerate}

\item $\widetilde{\alpha} \cin Q_a$.

\item $\widetilde{\alpha} \cin Q_b$, for some $b \neq a$.

\item $\widetilde{\alpha}$ intersects some $Q_b$, for $b \neq a$, in at least
one arc (that is properly embedded in $Q_b$).

\end{enumerate}

It is clear that these are all the possibilities, as if $\widetilde{\alpha}$ is
not contained in a single jigsaw surface of $S$, it must intersect at least two
of them. Therefore at least one of the surfaces it intersects is not $Q_a$. We
now claim that all of these are in fact impossible.

By construction of $Q_a$, the pared lifting pattern components of $\alpha$ in
$Q_a$ form a union of arcs. Hence (1) is impossible.

We now claim that (2) is impossible as well. That is, the pared lifting pattern
components of $\alpha$ in $Q_b$ ($b \neq a$) also form a union of arcs.

Intuitively, this is because the only way to get a pattern component of
$\alpha$ in a page in $Q_b$ is to have a page $B$ such that $F_a$ and $F_b$
contain the two sides of $B$.  But then, at every spine adjacent to $B$, $F_a$
and $F_b$ will diverge, so the $\alpha$ pattern cannot continue in $Q_b$ (since
it must extend to a cover of $F_b$).

More precisely, suppose $\alpha'$ is a pattern component of $\alpha$ in a page
component of the surface decomposition of $Q_b$. Let $B$ the corresponding page
in $M$, and $C$ a spine adjacent to $B$. Since $\alpha' \cin F_a \neq F_b$, the
only way to have $\alpha'$ components in a page component of $Q_b$ is to have
$F_a \cap B$ and $F_b \cap B$ both be nonempty. That is, $F_a$ contains one
side of $B$, and $F_b$ contains the other. Let $A$ be the binding annulus
between $B$ and $C$.  Since $C$ has valence at least 3, $\bd M \cap C$ has at
least 3 components, and more importantly, no two of the components have both
boundaries on the same two binding annuli. This means that $F_a \cap C$ and
$F_b \cap C$ may have one binding annulus $A$ ``in common'' (that is, $F_a \cap
C$ and $F_b \cap C$ both have nonempty intersection with $A$).  However, they
cannot possibly have two binding annuli in common.  Since $\alpha \cin F_a$, it
induces pattern components in $F_a \cap C$, and any surface parallel to those.
However, it cannot induce pattern components in a surface parallel to $F_b \cap
C$.  Hence $\alpha'$ cannot connect to a pattern arc in any spine adjacent to
$B$.  So (2) is impossible.

Finally, we claim that (3) is impossible. We know that $Q_b$ has sparse defect.
This means that if $\alpha'$ is contained in a component of $B \cap Q_b$, at
least one endpoint of $\alpha'$ must be attached to a component of $C \cap
Q_b$, where $C$ is a spine adjacent to $B$.  That is, the endpoints of
$\alpha'$ cannot both be free to attach to the annuli we added in the final
step. But as shown above, $\alpha$ has no pattern components in $C \cap Q_b$,
so this $\alpha'$ cannot possibly extend far enough to have both endpoints in
$\bd Q_b$.  Hence $\alpha'$ cannot be properly embedded, so it cannot be part
of a closed curve in $S$. So (3) also cannot occur.

Therefore $S$ satisfies the pared lifting criterion, and hence $S$ is (QF).
This completes the proof.

\end{proof}
