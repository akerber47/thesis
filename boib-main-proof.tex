%%%%%%%%%%%%%%%%%%%%%%%%%%%%%%%%%%%%%%%%%%%%%%%%%
\section{Statement of main theorem}
%%%%%%%%%%%%%%%%%%%%%%%%%%%%%%%%%%%%%%%%%%%%%%%%%

We are now ready to state the precise theorem.

\begin{thm}[Main Theorem]

Every reduced book of $I$-bundles contains a (QF) surface.

\end{thm}

Note that together with the earlier theorem, we've fully covered the
non-reduced case also.

\begin{thm}[Main Theorem, non-reduced case]

Let M be non-reduced book of I-bundles. If the reduction of the earlier theorem
yields at least one reduced book of I-bundles, then M contains a QF surface.
Otherwise, if the reduction terminates in an empty set, it does not.

\end{thm}

%%%%%%%%%%%%%%%%%%%%%%%%%%%%%%%%%%%%%%%%%%%%%%%%%
\section{Proof of main theorem - preliminaries}
%%%%%%%%%%%%%%%%%%%%%%%%%%%%%%%%%%%%%%%%%%%%%%%%%

Now there are a number of topological simplifications we make, by passing to an
appropriate finite-sheeted cover of the book of $I$-bundles.  Once we can
construct a surface satisfying (QF) inside this cover, we can push it
down and perturb slightly to obtain a surface satisfying (QF)
downstairs (since it $\pi_1$-injects into a subgroup, it will definitely still
be $\pi_1$-injective).

\begin{defn}

A \emph{good} book of $I$-bundles $M$ is a reduced book of I-bundles which
satisfies the following additional conditions.

\begin{enumerate}

\item Each binding annulus on a spine intersects a meridian disk of that spine
exactly once. That is, every spine has degree one.

\item Each page is glued to a given spine at most once. That is, for any page
B and spine C, B cap C is at most a single component of cA.

\item The two endpoints of a fiber in each page are in different boundary
components of $M$. In particular, each page is a trivial $I$-bundle over an
oriented surface.

\item Every arc of P on a page connects two different binding annuli. That is,
there are no essential arcs that begin and end at the same binding annulus.

\end{enumerate}

\end{defn}

\begin{lemma}

Let $M$ be a reduced book of $I$-bundles. Then $M$ has a finite-sheeted regular
cover which is good.

\end{lemma}

\begin{proof}

To prove this, we'll repeatedly use the fact that $\pi_1M$ is LERF to construct
finite sheeted covers with nice properties.

We first take care of condition (1). Let $C$ be a spine, and let
$A_1,\dots,A_n$ be the gluing annuli on $C$. These annuli must be parallel. Let
$\alpha$ be a generator for $\pi_1C$ (a solid torus). Choose $k$ such that
$\alpha^k$ is homotopic in $C$ to each $A_i$.

Since $M$ is admissible, $C \cin M$ is $\pi_1$-injective ($\alpha^k$, ie a page
gluing annulus, can't be homotopically trivial in an attached page). So $\alpha$
generates a cyclic subgroup of $\pi_1M$.

$\pi_1M$ is LERF. Let $\phi_C \colon \pi_1M \to G_C$ be a map to a finite group
such that $\phi_C(\alpha),\phi_C(\alpha^2),\dots,\phi_C(\alpha^{k-1}) \notin
\phi_C(\langle\alpha^k\rangle)$.  We can do this by repeatedly applying LERF to
get a map for $\phi_C(\alpha)$, a map for $\phi_C(\alpha^2)$, etc, and taking
the product of all these maps. Let $\phi \colon \pi_1M \to G = \prod_C G_C$  be
the product of all these maps.  Again, we elide changes of basepoint.  Let $\pi
\colon M' \to M$ be the associated regular cover of $M$.

We claim that $M'$ satisfies (1). The spines of $M'$ are precisely connected
components of preimages of spines of $M$. Let $C'$ be such a spine upstairs,
and $C=\pi(C')$ the associated spine downstairs. Let $\alpha$ be a generator
for $\pi_1C$.  Recallthat in our construction above, we chose a value $k$
associated to each $\alpha$. Let $d$ be the degree of the restricted cover $C'$
to $C$. Note that $d$ may not be the degree of the cover $M' to M$, as
$\pi^{-1}(C)$ may not be connected.  We claim that $k|d$.

Suppose not. Then $d = m1k - m2$, for some $m1,m2 \in Z$, $0 < m2 < k$.
Rearranging, $m2 = m1k - d$. Choose $x_0 \in \alpha$, and let $x_0'$ be a lift of
$x_0$ inside $C'$. Since $C'$ to $C$ is a $d$-fold cover of a solid torus (with
cyclic $\pi_1$), $\alpha^d$ lifts to $C'$ as a closed curve. That is,
$\phi(\alpha^d)$ is trivial.  But this implies $\phi(\alpha^m2)
= \phi(\alpha^m1k)\phi(\alpha^-d) = \phi(\alpha^m1k)$, contradicting our LERF
assumption that $\phi_C(\alpha^m2) \notin \phi_C(\langle\alpha^k\rangle)$.

So $k|d$. Write $d = m1k$. Then each gluing annulus will, when raised to the
$m1$th power, lift to an embedded annulus which traverses the longitude $C'$
exactly once.  By regularity of the cover this implies that the union of all
lifts of these annuli is a union of embedded annuli traversing the longitude of
$C'$ exactly once. Since this argument holds for any $C'$ upstairs and $C$
downstairs, $M'$ satisfies (1).

Note that any further finite-sheeted cover of $M'$ will also satisfy (1), by
looking locally at the lifting of gluing annuli above each spine.

So now assume $M$ is reduced and satisfies (1). We claim $M$ has
a finite-sheeted cover which satisfies (2).

Let $B$ be a page intersecting a spine $C$ more than once. For each pair of
gluing annuli $A_1$,$A_2$ which attach $B$ to $C$, we'll draw a closed curve as
follows.  Fix a meridian disk $D$ of $C$. Choose an arbitrary point $x_1 \in
A_1 \cap D$, $x_2 \in A_2 \cap D$. Choose an arbitrary arc $\beta \cin B$
connecting $x_1$ and $x_2$.  Let $\gamma$ be the closed curve obtained by
closing up $\beta$ with an arc along the interior of $D$.  Orient $\gamma$ so
that it first traverses this arc through the interior of $D$, and then
traverses $\beta$.  Fix $x_1$ to be our basepoint. Using LERF, choose a map
$\phi_C,B,A_1,A_2$ such that $\phi_C,B,A_1,A_2(\gamma) \notin
\phi_C,B,A_1,A_2(\pi_1B)$. Repeat this construction for each $A_1$,$A_2$ and
each $B$,$C$ with multiple gluings. Let $\phi$ be the product of all these, and
$M'$ the associated regular cover, just like above.  Same basepoint trickery as
earlier.

We claim that $M'$ satisfies (2). Suppose not. Let $B'$ be a page upstairs
glued to a spine $C'$ more than once. Then $B=\pi(B')$ is also a page,
$C=\pi(C')$ is a spine.  Since $\pi$ is a cover. either the multiple gluings of
$B'$ to $C'$ project down to the same gluing of $B$ to $C$, or there are
multiple gluings of $B$ to $C$. The first case is impossible, as all the
gluings to a spine in a book of $I$-bundles must be parallel, but $\pi|C'
\colon C' to C$ is a covering map of solid tori and downstairs each gluing only
wraps once around the longitude by (1). So there must be multiple gluings of
$B$ to $C$.  Let $A_1$, $A_2$ be annuli, and $A_1'$,$A_2'$ the corresponding
gluing annuli upstairs.  Fixing our basepoint $x_1 \in A_1$ as above, $x_1$ has
a lift $x_1'$ in $A_1'$.  Lift $\gamma$ to an arc $\gamma' \cin A_1'$ starting
at $x_1'$.  Now $\gamma'$ first traverses a lifted meridian disk across the
spine, meaning it crosses over to $A_2'$. The remainder of $\gamma'$ is a lift
of $\beta$ beginning at a point in $A_2'$, but $\beta$ is contained in the page
$B$, so $\gamma'$ must remain inside the page $B'$ upstairs. So $\gamma'$ is an
arc with both endpoints in $B'$.

But $B'$ is a path connected component of $\pi^{-1}(B)$. So covering
translations of $M'$ over $M$ which take $x_1'$ to other lifts of $x_1 \in B'$
are precisely the group elements $\phi(\pi_1B)$, that is, lifts of loops inside
$B$. But $\gamma$ lifts to an arc with both endpoints in $B'$, so
$\phi(\gamma)$ is also such a covering translation.  Hence $\phi(\gamma) \in
\phi(\pi_1B)$, a contradiction.

So we have a cover which satisfies (1) and (2). Since property (2) is also
preserved under taking finite-sheeted covers, as discussed above with
$A_1'$,$A_2'$ lying above $A_1$,$A_2$, it now suffices to show $M$ satisfying (1)
and (2) has a finite-sheeted cover satisfying (3).

$M$ is orientable by definition, so each page must be orientable as well. So
the pages which are not trivial $I$-bundles over oriented pages must be twisted
$I$-bundles over nonoriented pages (in order for the resulting page to be
orientable). In particular, the endpoints of the $I$ fibers of a twisted
$I$-bundle will connect globally to form a single side of the page. So if we
look locally at an $I$ fiber, the two endpoints are guaranteed to be in the
same boundary component of $M$ (since they're in the same boundary component
even if we just look at that page). This explains why ensuring distinct
boundary components for each fiber guarantees trivial $I$-bundles.

Now we proceed by a similar argument as above. Let $B$ be a page such that
fibers have both endpoints on the same boundary component $\bd_a M \cin \bd M$.
Choose an arbitrary fiber $[0,1] isom I \cin B$. Let $\beta \cin \bd_a M$ be an arc
in the boundary connecting the endpoints of $I$. This forms a curve $\gamma$.
Fix one endpoint of $I$ to be our basepoint. Orient $\gamma$ so that from that
basepoint, it first traverses $I$ and then $\beta$. Using LERF, choose a map
$\phi_{\bd_aM,B} \colon \pi_1M to G$, where $G$ is a finite group, such that
$\phi_{\bd_aM,B}(\gamma) \notin \phi_{\bd_aM,B}(\pi_1\bd_aM)$.  Note that $M$ has
incompressible boundary, so $\pi_1\bd_aM$ embeds as a subgroup of $\pi_1M$. Also
note that $\pi_1\bd_aM$ is finitely generated, as it's a compact surface.  As
above, repeat this construction for each page where fibers have both endpoints
on the same boundary component. Take the product $\phi \colon \pi_1M to G$
(modifying by changes of basepoint as appropriate), and let $M'$ be the
associated regular cover.

We claim that $M'$ satisfies (3). Suppose not. Let $B'$ be a page with a fiber
violating (3). Notice that all other fibers of $B'$ must also violate (3), as
we can extend the boundary components locally between fibers.  Since boundary
components of $M'$ cover boundary components of $M$, these fibers descend to
fibers violating (3) in $B=\pi(B')$. So without loss of generality we can let
$I' \cin B'$, a violating fiber, satisfy that $\pi(I')=I \cin B$ is one of the
fibers we chose earlier in our construction.

But now just like in our proof of (2), lift the corresponding basepoint $x_0$ to
an endpoint of $I'$, call it $x_0'$, and lift $\gamma$ to $x_0'$. The resulting
arc $\gamma'$ crosses $I'$ and then remains on the same boundary component of
$M'$.  But this implies that the two endpoints of $\gamma'$ lie on the same
boundary component of $M'$. Exactly like our argument in (2), this shows
$\phi(\gamma) \in \phi(\pi_1\bd_aM)$, a contradiction. This completes the proof.

\end{proof}

%%%%%%%%%%%%%%%%%%%%%%%%%%%%%%%%%%%%%%%%%%%%%%%%%
\section{Proof of main theorem}
%%%%%%%%%%%%%%%%%%%%%%%%%%%%%%%%%%%%%%%%%%%%%%%%%

We've now reduced to the case of a good book of I-bundles. However, the
remaining work is still quite involved.

This is the main proof, and is quite involved. We proceed somewhat similarly to
the first example. We take covers over the pages and glue them together
cleverly to construct our surface. We check that the parabolic locus arcs on
each page lift to arcs on these covers that do not connect up to form a closed
curve. As in the first example, this suffices to prove that our surface
satisfies (QF). We first need to take the correct cover over each page.

This is a somewhat elaborate construction. First, we need as our building
blocks covers where it is not so easy to join up the parabolic arcs to make
closed curves. In particular, we use the following lemma.

\begin{lemma}

Let $S$ be a compact surface with boundary, and $P$ a finite set of simple
proper essential arcs on $S$ (not necessarily disjoint).  Then there exists
a finite-sheeted cover $S'$ of $S$ such that the preimage $\pi^{-1}(P)$
consists of arcs each of which connects 2 different boundary components of
$S'$.

\end{lemma}

\begin{proof}

Let $\alpha \cin P$ be an arc, and suppose that both endpoints of $\alpha$ lie
on the same boundary component $\gamma \cin bd S$. Fix a basepoint $x_0$ in
$\gamma$.  Homotope $\alpha$ by moving both endpoints to $x_0$ along $\gamma$.
(The choice of homotopy is arbitrary). This gives a closed curve $\beta$
through $x_0$. Since $\alpha$ is simple and essential, $[\beta] \notin
<[\gamma]> \cin \pi_1S$. If $\beta$ were a multiple of $\gamma$, we could
isotope it to a neighborhood of $\gamma$, and then undoing the earlier homotopy
$\alpha$ would have to be non-essential (isotoping the endpoints around to
"unwind the spiral".

$\pi_1S$ is free, so it's LERF. So there exists a map $\phi\alpha$ to a finite
group $G\alpha$, such that $\phi\alpha([\beta]) \notin \phi\alpha(<[\gamma]>)$.
Repeat this construction for each $\alpha \cin P$. Let $G = \prod \alpha
G\alpha$, and $\phi \colon\pi_1S \to G$ the product map. Since $G$ is finite,
$\phi$ induces a finite-sheeted regular cover $\pi \colon S' \to S$. We claim
that $\pi^{-1}(P)$ consists of arcs connecting two different boundary
components of $S'$.

Note that we're eliding the fact that these maps $\phi\alpha$ are defined on
$\pi_1S$ with different basepoints. However, if we choose an arbitrary path
connecting each additional basepoint to a fixed basepoint we choose, the
induced isomorphisms on $\pi_1S$ carry over the nice cover lifting properties
between different basepoints. Our curves $\beta$ and $\gamma$ still correspond
to elements of $\pi_1$ which lift in the appropriate way, once we do a reverse
change of basepoint upstairs. So this is not a problem.

Obviously if $\alpha \cin S$ is an arc which already connects two distinct
boundary components downstairs, the arc covering it must also connect two
boundary components (the possible boundary component preimages are disjoint).
So we need only consider $\alpha$ with both endpoints on the same boundary
component $\gamma$.  Fix $x_0$ as above, and let $\widetilde{x_0}$ be an
arbitrary lifted basepoint in $S'$.  Look at a lift of $\beta$ to a path
$\widetilde{\beta}$ starting at $\widetilde{x_0}$. $\widetilde{\beta}$ must end at some lift
of $x_0$.

Since $S'$ is a regular cover, the covering transformation group $\gamma$ is
precisely $image(\phi)$. Since $\phi([\beta])$ is nontrivial, it corresponds to
a covering transformation. This transformation must map $\widetilde{x_0}$ to
the other endpoint of $\widetilde{\beta}$, by basic covering space theory. But the
boundary component of $S'$ containing $\widetilde{x_0}$ is a finite-sheeted
cover of $\gamma$, and all the lifts of $\widetilde{x_0}$ contained in this
boundary component are connected to $\widetilde{x_0}$ by a lift of a finite
power of $\gamma$. That is, $\widetilde{x_0}$ is mapped to these other points
by transformations corresponding to powers of $\phi([\gamma])$.  But since
$\phi([\beta]) notin \phi([\gamma])$, the $\phi([\beta])$ transformation can't
be any of these. So it must send $\widetilde{x_0}$ to a point in a different
boundary component. $\widetilde{\beta}$ obviously connects the same boundary
components as a lift of $\alpha$ to $\widetilde{x_0}$, because we just homotope
along the boundaries. So the lift of $\alpha$ to $\widetilde{x_0}$ connects two
different boundaries.  $S'$ is a regular cover, so this applies to any lift.
This completes the proof.

\end{proof}

\begin{proof}[Proof of Main Theorem]

% d = lcm degree of subgp sep cover (lcm of degree on each page)
% After taking copies of each subgp sep cover to ensure all are same degree:
% n = (large) # of copies of each cover above
% m = # of copies of each bc to reserve
% i(M,P,S_a) = max # incident arcs at a bc on S_a boundary component
% TODO call this ia
% j(M,P,S_a) = total # bcs on page boundaries making up S_a ("junctions")

% TODO
\textbf{ TODO This notation is completely different from my other notation! Be
clear and consistent.}

Consider an arbitrary boundary component $S_a$ of the book of $I$-bundles $M$.
Cutting $M$ along the gluing annuli decomposes $\bd_aM$ into a union of pieces
each of which is one side of an $I$-bundle page. Call these pieces $S_{a,k}$.
Then the page is $S_{a,k}\times I$, with the piece embedded as $S_{a,k}\times
0$ or $S_{a,k}\times 1$. Without loss of generality let it be $S_{a,k}\times
0$. We can look at the pattern of arcs $P_{a,k} = P \cap (S_{a,k}\times 0)$ on
that side of the page, which is part of $\bd_aM$. We can also look at the pattern
of arcs on the opposite side of the page $\overline{P_{a,k}} = P \cap
(S_{a,k}\times 1)$, and temporarily view it as living inside $S_{a,k}$
(flattening $S_{a,k}\times I \to S_{a,k}$).  For each piece, apply the lemma
above to ($S_{a,k},P_{a,k} \cup barP_{a,k}$) to obtain a cover $S'_{a,k}$. If we
view this as a cover of $S_{a,k}x0$ and $S_{a,k}x1$, it has the property that
any arc in $P_{a,k}$ or $barP_{a,k}$ lifts to an arc connecting two different
boundary components of $S'_{a,k}$.

Let $d_{a,k}$ be the degree of this cover. Let $d = lcm a,k d_{a,k}$.

Fix some large integer $n>0$. Now for each $k$, let $\pi_{a,k}
\colon \widetilde{S_{a,k}} \to S'_{a,k} \to S_{a,k}$ be a disjoint union of $nd/d_{a,k}$
copies of $S'_{a,k}$. The purpose of the $d/d_{a,k}$ normalization is to make sure
each $\pi_{a,k}$ is a cover of the same degree, $n*d$.  We'll call the connected
components of this cover "pieces". We can also combine these into a single
cover $\pi \colon \bigsqcup \widetilde{S_{a,k}} \to \bigsqcup S_{a,k}$

Fix a large integer $m>0$. Now, reserve a subset $\cC$ of the boundary
components of $\bigsqcup \widetilde{S_{a,k}}$.  This subset should have the following
properties:

(1) Each boundary component of each $S_{a,k}$ is covered by exactly $m$ elements
of $\cC$.

(2) Every connected component of $\bigsqcup \widetilde{S_{a,k}}$ contains at most one element
of $\cC$.

This is possible as long as $n \geq mb$, where $b$ is the maximum number of
boundary components of any $S_{a,k}$. If $n\geq mb$, we know $d/d_{a,k} \geq
1$, so each disjoint union $\widetilde{S_{a,k}}$ has at least $mb$ many
disjoint copies of each $S'_{a,k}$.  So we have enough copies to make our choices
above that $S_{a,k}$ disjoint.

Each $S_{a,k}$ is homeomorphic to the core surface of its page, so by the covering
lemma any closed surface formed by gluing the $\widetilde{S_{a,k}}$ along appropriate
boundaries will induce an immersed $\pi_1$-injective surface in $M$. This is
our strategy.

Write $\widetilde{P_{a,k}}=\pi_{a,k}^{-1}(P_{a,k})$. As constructed, $P_{a,k}$ consists of proper arcs
connecting two different boundary components of $\widetilde{S_{a,k}}$.

We first glue the pieces "along" $\bd_aM$ in the following sense. Given any two
pieces $S_{a,k_1}$, $S_{a,k_2}$ in the decomposition of $\bd_aM$, we know that
$S_{a,k_1}$ has some boundary components $\gamma_1,\gamma_2,\dots \cin
dS_{a,k_1}$, and $\gamma_1',\gamma_2',\dots \cin dS_{a,k_2}$ such that before we
cut $\bd_aM$ apart, $\gamma_1$ was glued to $\gamma_1'$, $\gamma_2$ to
$\gamma_2'$, etc.  That is, these components form the locus we glue $S_{a,k_1}$
and $S_{a,k_2}$ together along when we glue the $S_{a,k}$ to form $\bd_aM$.  For
each such pair, look at the boundary components of $\widetilde{S_{a,k}}1$ and
$\widetilde{S_{a,k}}2$.  We allow ourselves to glue components of
$\pi^{-1}(\gamma_1)$ to components of $\pi^{-1}(\gamma_1')$, components of
$\pi^{-1}(\gamma_2)$ to components of $\pi^{-1}(\gamma_2')$, etc. We forbid
ourselves from gluing components of $\pi^{-1}(\gamma_1)$ to any other lifted
boundary components of $S_{a,k_2}$ or of any other pieces. Only on the very last
step of our construction will we break this rule.

Note that since our gluing is locally a cover we can only glue each pair of
boundary components upstairs in one possible way - we could choose a different
gluing, but this would just correspond to a different choice of cover by the
covering lemma.

We introduce some notation. Bbegin by taking $\widetilde{S}^{(0)} = \bigsqcup \widetilde{S_{a,k}}$ to
be a disjoint union of these covers - that is, we haven't done any gluing yet.
Think of these as pieces we'll use to build our cover. At each step in the
gluing, we have a gluing map $f(i+1) \colon \widetilde{S}^{(i)} \to
\widetilde{S}^{(i+1)}$ by attaching two boundary comonents of
$\widetilde{S}^{(i)}$. Write $\widetilde{P}^{(i)}$ for the arcs and curves on
$\widetilde{S}^{(i)}$ induced by gluing the $\widetilde{P_{a,k}}$.  That is,
$\widetilde{P}^{(i)} = f(i) \circ ...  \circ f(1) (\bigsqcup
\widetilde{P_{a,k}})$.

We'll want to glue boundary components other than $\cC$ first - these are the
"reserve" that we'll only use at the end. This guarantees that our final glued
surface will have leftover boundary components that are sufficiently far apart.
By abuse of notation, we'll also use $\cC$ to refer to the induced set of
boundary components in each $\widetilde{S}^{(i)}$.

In addition to gluing along $\bd_aM$ and avoiding gluing members of $\cC$, we want
to preserve the following two invariants.

% FIXME decide what to call it in the paper...  still need to figure out how
% I'm going to do the references too

\begin{enumerate}

\item[(\dag)] $\widetilde{P}^{(i)} \cin \widetilde{S}^{(i)}$ contain no closed
curves.  That is, it is a union of proper arcs. \label{I:dag}

\item[(\dag')] No arc of $\widetilde{P}^{(i)}$ begins and ends at the same
boundary component of $\widetilde{S}^{(i)}$. \label{I:dag'}

\end{enumerate}

By the use of the lemma and the assumption of the theorem (no closed curves in
individual pages), $\widetilde{S}^{(0)}$ satisfies (\dag) and (\dag') We
repeatedly make an arbitrary gluing that lies along $\bd_aM$ in the above
sense, doesn't glue any elements of $\cC$, and preserves (\dag) and (\dag').  This
process will terminate at some $\widetilde{S}^{(N)}$, after there are no more
gluings left to make. $\widetilde{S}^{(N)}$ is a surface with boundary, because
there are "leftover" boundary components that we couldn't glue without
violating one of the above conditions.

% FIXME restructure into lemmas?
%Claim.

We claim that $\#(d\widetilde{S}^{(N)} - \cC)$ is bounded by $C(M,P,S_a)$.  In
particular, this bound is independent of the choice of $n$ and $m$ made
earlier.

%Proof of Claim.

Consider an arbitrary boundary component $\widetilde{\gamma} \cin
d\widetilde{S}^{(N)} - \cC$.  It lives above some boundary component $\gamma
\cin dS_{a,k}$. Look at the number of arcs of $P_{a,k}$ which are incident to
$\gamma$. This number depends only on $M$ and $P$ (not on $n$). Let $i_a
= i(M,P,S_a)$ be the maximum such number of incident arcs of $P$ for any
boundary component $\gamma$ of any $S_{a,k} in S_a$. Let $j_a = j(M,P,S_a)$ be
the total number of boundary components of the $S_{a,k}$.

Let $\gamma' \cin dS_{a,k}'$ be the boundary component matching $\gamma$. We
consider gluing $\gamma$ along $M$ to any component of $\bd\widetilde{S}^{(N)}
- \cC$ above $\gamma'$.  Because we always glue matching boundary components
(and we started with the same number of each), $\bd\widetilde{S} - \cC$ has the
same number of leftover boundary components above $\gamma'$ as it does above
$\gamma$. So there must be at least one to glue to.  The only reason we'd be
unable to glue is if gluing to any leftover lift of $\gamma'$ violates (\dag) or
(\dag').

Let $\widetilde{\gamma}' \cin \bd\widetilde{S}^{(N)} - \cC$ be an arbitrary
leftover lift of $\gamma'$.  Suppose gluing to $\widetilde{\gamma}'$ violates
(\dag). Let $\alpha$ be a closed curve produced by the gluing.  Obviously
$\alpha$ intersects the $\widetilde{\gamma} = \widetilde{\gamma}'$ gluing
circle, otherwise $\widetilde{S}^{(N)}$ would already violate (\dag).  Cutting
along this circle divides $\alpha$ into a union of arcs with endpoints on
either $\gamma$ or $\gamma'$.  $\widetilde{S}$ satisfies (\dag'), so no arc can
have both endpoints on the same boundary component. So there exists an arc
connecting $\gamma$ to $\gamma'$. This is the only way that an additional
gluing would violate (\dag). But since $\widetilde{S}^{(N)}$ is locally a cover
on each piece, the number of arcs incident to $\gamma$ is bounded by $i_a$.
Tracing these arcs through $\widetilde{S}^{(N)}$, they can hit at most $i_a$
other boundary components.  These are the only boundary components we can glue
to to violate (\dag).

Similarly, suppose gluing to $\widetilde{\gamma}'$ violates (\dag'). Let $\alpha$ be an
arc with endpoints on the same boundary component produced by the gluing. Let
$\widetilde{\delta}$ be this boundary component. Again, $\alpha$ intersects the
$\widetilde{\gamma} = \widetilde{\gamma}'$ gluing circle, otherwise $\widetilde{S}^{(N)}$ would already
violate (\dag').  Cutting along this circle divides $\alpha$ into a union of
arcs.  Except for the two original endpoints (which now lie on two different
subarcs - call these subarcs $\alpha_0$ and $\alpha_1$), all other endpoints of
these arcs must either lie on $\gamma$ or $\gamma'$.  But no arc can have both
endpoints on the same boundary component, or $\widetilde{S}^{(N)}$ would arleady violate
(\dag'). So any subarcs except $\alpha_0$ and $\alpha_1$ must connect $\gamma$ to
$\gamma'$.  By parity we can see that $\alpha_0$ and $\alpha_1$ both have one
endpoint on $\widetilde{\delta}$, but their other endpoints must be different. That is,
either $\alpha_0$ ends on $\widetilde{\gamma}$ and $\alpha_1$ on $\widetilde{\gamma}'$, or vice
versa.

Now the arcs incident to $\widetilde{\gamma}$ hit at most $ia$ other boundary
components. These are our possible $\widetilde{\delta}$. Each of these has at
most $i_a$ incident arcs itself, one of which returns to $\widetilde{\gamma}$,
leaving $i_a-1$ that we need to care about. We have a total of at most
$i_a*(i_a-1)$ many "distance two" leftover boundary components. These are the
only boundary components we can glue to to violate (\dag').

But by construction of $\widetilde{S}^{(N)}$, there are no more legal gluings. So
$\widetilde{S}$ can have at most $i_a+(i_a*(i_a-1)) = i_a^2$ many leftover boundary
components (that is, components of $\bd\widetilde{S}^{(N)}-\cC$) above $\gamma'$.  But since
$\widetilde{S}^{(0)}$ has the same number of boundary components above $\gamma$ and
$\gamma'$, and we're only allowed to glue them to each other, this implies that
there can be at most $i_a$ boundary components above $\gamma$ as well.  Since
our choice of $\widetilde{\gamma}$ was arbitrary, this implies that there are
at most $C(M,P,S_a) = i_a^2*j(M,P,S_a)$ many leftover boundary components, that
is, components of $\bd\widetilde{S}^{(N)}-\cC$.

%Proof of Theorem contd.

Now that we've constructed $\widetilde{S}^{(N)}$, we want to perform some additional
gluings to construct $\widetilde{S}$. In addition to (\dag) and (\dag'), $\widetilde{S}$ should
satisfy

\begin{enumerate}

\item[(\dag'')] Any two boundary components of $\widetilde{S}$ lie on different
pieces. That is, they correspond to boundary components of different connected
components of $\widetilde{S}^{(0)}$ under the gluing map $\widetilde{S}^{(0)}
\to \widetilde{S}$.

\end{enumerate}

Intuitively, this is easy using the reserve. The reserve on its own satisfies
(\dag''), so we only have to deal with the leftovers discussed above.  The trick
is the reserve is much larger than the leftovers, so if we allow ourselves to
glue the leftovers to the reserve we can take care of all the leftovers without
violating (\dag) or (\dag'). Our remaining boundary components will be a subset
of the reserve.

To be precise, construct $\widetilde{S}$ from $\widetilde{S}^{(N)}$ as follows.
For each boundary component $\gamma$ downstairs, as discussed above, there are
at most $i^2$ many elements of $\bd\widetilde{S}^{(N)}-\cC$ above it. For each of
these, there are at most $i^2$ we could glue to above $\gamma'$ that violate
(\dag) or (\dag').  Assume that $m\geq 2i^2$.  Then glue each leftover lift of
$\gamma$ to an element of $\cC$ above gama', one at a time. At any point there
will be at least $i^2+1$ elements of $\cC$ remaining above $\gamma'$, so we'll
always be able to choose one that doesn't violate (\dag) or (\dag'). Repeat
this process for each $\gamma$ to construct $\widetilde{S}$.  Since we glued
all the leftovers, it immediately follows that $\bd\widetilde{S} \cin \cC$, and
therefore $\widetilde{S}$ satisfies (\dag'').

Let's analyze the possibilities for $\#\bd\widetilde{S}$, the number of
boundary components. Above each $\gamma \cin dS_{a,k}$,
$\cC$ has $m$ elements. So $\bd\widetilde{S}$ has at most $m$ components above
$\gamma$, or $m*j$ many boundary components in total. The lower bound is
determined by the maximum number of leftovers, since the only way we'll remove
reserve components from $\bd\widetilde{S}$ is by gluing them to leftover
components. There are at most $i^2$ leftover components above $\gamma$, and
similarly above $\gamma'$, so there will be at least $m - i^2$ components
remaining in $\bd\widetilde{S}$ above $\gamma$. So

\[ m-i_a^2 \leq \#(\bd\widetilde{S} \text{ above }\gamma) \leq m \]

and

\[ (m-i_a^2)j_a \leq \#\bd\widetilde{S} \leq mj_a \]

Finally, suppose we've done the entire above construction for each boundary
component of our book of $I$-bundles $M$ to produce a partially-glued surface
with some boundary components left over. For each boundary component $\bd_aM$,
we'll call the surface $\widetilde{S_a}$. Note that we must be sure to make the same
choice of $n$ and $m$ for all these surfaces. Now we want to glue these
together, but they may have very different numbers of boundary components! So
we'll need to normalize them so they all have the same number of boundary
components above each $\gamma \cin dS_{a,k}$, so we can do a local construction
above a neighborhood of each spine in $M$.

Let $i_M = max a i_a$. By the inequality above, for each $\gamma \cin S_{a,k}$,
$\#(d\widetilde{S_a} \text{ above } \gamma) \geq m - i_M^2$. We want to make
this an equality for each $\widetilde{S_a}$ by performing some more gluings.

Again, we use the size of the reserve to our advantage. Assuming $m\geq i_M^2
+ i_a^2$ for each $\widetilde{S_a}$, by the same argument as above there will
always be choices remaining from the reserve to make the gluing. Combining
these all we need is $m\geq 2i_M^2$.

By abuse of notation, we'll call these normalized surfaces $\widetilde{S_a}$ also (the
non-normalized ones will not come up again). Each has exactly $C = m - i_M^2$
many boundary components above any $\gamma \cin dS_{a,k}$.

Build a single surface $\widetilde{S}$ by gluing the $\widetilde{S_a}$ as follows. Begin with
$\bigsqcup_a \widetilde{S_a}$.

For each spine $M_c$ of $M$, look at all the incident pages. Say there are $q$
of them.  Each page has two sides, so there are $2q$ many pieces $S_{a,k}$
downstairs near $M_c$, each with one boundary component glued along the spine
$M_c$.  Because $M$ is good, we know that all these $S_{a,k}$ are distinct and
they are all glued along simple parallel closed curves on $\bd M_c$ (do we know
they're longitudes?  See above \textbf{ TODO}).

Introduce notation as follows. $\cN(M_c)$ consists of $q$ neighborhoods inside
pages, attached to $M_c$ by annuli. The meridian of $M_c$ gives a cyclic order to
the attached pages.  Fix an orientation for the meridian and an (arbitrary)
starting point along it, and label the pages $1, \dots, q$ under this ordering.
The orientation of the meridian also gives an ordering of the boundary
components of each attaching annulus.  Using this, label the $2q$ boundary
components by $\gamma_1^-,\gamma_1^+,\dots, \gamma_q^-,\gamma_q^+$, where
$\gamma_i^-$ and $\gamma_i^+$ are the two sides of the attaching annulus for
the ith page.

Now consider $\bd M \cap \cN(M_c)$. Notice that the $2q$ boundary components of these
attaching annuli that we just labeled are precisely the $2q$ boundary curves of
the boundary pieces $S_{a,k}$ that are near $M_c$. We don't know how many global
boundary components $a$ are involved, but locally we do know which pieces are
attached inside $\bd M \cap \cN(M_c)$. $\bd M \cap M_c$ consists of q parallel annuli
along $\bd M_c$ that connect adjacent pages together. That is, the first annulus
has boundary $\gamma_1^+ \cup \gamma_2^-$, the second $\gamma_2^+ \cup
\gamma^3_-$, and so on.  It follows that these are the "pairs downstairs" in
the above construction of each $\widetilde{S_a}$. That is, these are the curves
we called $\gamma$ and $\gamma'$ earlier.  So within $\bigsqcup \widetilde{S_a}$,
the existing gluings only glue lifts of $\gamma_1^+$ to lifts of $\gamma_2^-$,
lifts of $\gamma_2^+$ to lifts of $\gamma_3^-$, etc etc.

Finally, the gluing. Because we normalized, each $\bigsqcup_a \widetilde{S_a}$ has
exactly $C$ boundary components above each $\gamma_i^\pm$.  Glue these $2qC$
boundary components as follows. Glue lifts of $\gamma_1^+$ to lifts of
$\gamma_3^-$, lifts of $\gamma_2^+$ to lifts of $\gamma_4^-$, and so on. In
general glue lifts of $\gamma_i^+$ to lifts of $\gamma_{i+2}^-$. Within each
subset of lifts, choose the gluing arbitrarily. We know the numbers on each
side are equal, so they'll match up.

Every boundary component of $\bigsqcup_a \widetilde{S_a}$ lives above a boundary component
of some $S_{a,k}$ which means it is attached to some core $M_c$. So once we've done
the above gluing step for all cores, the resulting surface $\widetilde{S}$ is closed.

%Claim.

Finally, we claim $\widetilde{S}$ satisfies (QF).

%Proof of Claim.

% TODO
\textbf{TODO simplify this argument using good property (3) as written up
earlier and discussed with Ian!!}

By construction, $\widetilde{S}$ consists of covers of each page of $M$ glued along
parallel longitudinal annuli at the spines of $M$. By the covering lemma, it
induces an immersed $\pi_1$-injective surface. It remains to show that
$\widetilde{S}$ cannot contain any lifts of parabolic curves.

Let $\alpha$ be a parabolic curve in some boundary component $S_a$ of $M$. Cut
$\alpha$ into arcs $\alpha_i$ along the page boundaries, so each $\alpha_i \cin
S_{a,k_i}$.  We can think about this as follows. For each $\alpha_i$, there are two
kinds of lifts to $\widetilde{S}$. We can view it as an arc in $P_{a,k_i} \cin
S_{a,k_i}$, and lift it to a cover of $S_{a,k_i}$, that is, a component of
$\widetilde{S_{a,k_i}}$. Since $\alpha \cin S_a$, this will necessarily be part
of $\widetilde{S_a} \cin \widetilde{S}$. Furthermore, by construction of
$\widetilde{S_a}$, lifting the $\alpha_i$ to $\widetilde{S_a}$ pieces and
gluing cannot yield closed curves.  It only yields a union of arcs, by (\dag).

However, the second way we can lift $\alpha_i \cin S_{a,k_i}$ is to a cover of the
opposite page boundary $\overline{S_{a,k_i}}$. Let $S_b$ be the boundary component
containing $\overline{S_{a,k_i}}$, and write $S_{b,l_i} = \overline{S_{a,k_i}}$, we can see that $\alpha_i$
in $\overline{P_{b,l_i}}$, the set of "opposite parabolic arcs" that we defined earlier.
So $\alpha_i$ lifts to $S'b,li$ as a union of arcs, each of which connects two
different boundary components of $S'b,li$. This produces a set of arcs in
$\widetilde{S_b}$. Note that it may be possible to have $S_a = S_b$, depending on
how our book of $I$-bundles is constructed.  Regardless, we want to think of
these as a different kind of lift, because we're using the opposite page side
to lift, rather than the side the arc "naturally lives on."

We now claim that these arcs cannot be used to form closed curves in
$\widetilde{S_b}$.  Intuitively, this is because the opposite side of the
boundary does not remain parallel to $\alpha$ for long enough, so we'll soon
reach a piece of $\widetilde{S_b}$ where our lift can't continue.  To be
precise, let $\widetilde{\alpha_i}$ be a lift of $\alpha_i$ to
$\widetilde{S_b}$.  Suppose $\alpha_i$ has an endpoint at a spine $M_c$. If we
look at how the boundary components of a book of $I$-bundles behave near
a spine, we that locally the two boundary sides of a page are attached to sides
of two different pages.  In the terminology we used earlier for gluing pages at
a spine, $\gamma_i^-$ and $\gamma_i^+$ are attached to the attaching annulus
boundary curves of two different pages: $\gamma_i^-$ attaches to
$\gamma_{i+1}^+$, but $\gamma_i^+$ attaches to $\gamma_{i-1}^-$.  Since there
are at least three pages at $M_c$ by (3), the $(i-1)$st and $(i+1)$st pages are
distinct.

Suppose that the endpoint of $\alpha_i$ we're considering lies on $\gamma_i^-$. That
is, $\gamma_i^-$ is a boundary component of $S_{a,k_i}$, in our other notation. And
$\gamma_i^+$ is a boundary component of the opposite page $\overline{S_{a,k_i}} = S_{b,l_i}$.
Since $\gamma_i^-$ attaches to $\gamma_{i+1}^+$, any lifted arc we attach
$\widetilde{\alpha_i}$ to in that direction must lift from that $(i+1)$st page at $M_c$.
But in $\widetilde{S_b}$, we attach the surface pieces according to the gluings needed
for the $S_b$ boundary component.  Locally, $\widetilde{\alpha_i} \cin \widetilde{S_b},li$, where
lifts of $\gamma_i^+$ only attach to lifts of $\gamma_{i-1}^-$. But $\gamma_{i-1}^-$ is
a boundary component of a different page, so $\widetilde{\alpha_i}$ connects to a next
page with no lift of the next segment of $\alpha_i$. So there is no possible way
to continue $\widetilde{\alpha_i}$.  The same argument applies at the opposite endpoint
of $\widetilde{\alpha_i}$. This argument shows that lifts of arcs $\alpha_i$ as "opposite
parabolic arcs" to $S_b$ have no "opposite arcs" on either side that they can
glue to.

Finally, suppose our parabolic curve $\alpha \cin S_a$, after cutting into
$\alpha_i$, lifts to pieces $\widetilde{\alpha_i}$ which form a closed curve $\widetilde{\alpha}
\cin \widetilde{S}$.  Cut $\widetilde{\alpha}$ along only the gluings between different
$\widetilde{S_b}$ done in the final gluing step. Each piece $\widetilde{\alpha_j}$ now consists
of multiple $\widetilde{\alpha_i}$, which are either all "natural lifts" or all
"opposite lifts" (in the above sense). This is simply because each $\widetilde{S_b}$
is, except for its extra boundary components, a cover of the corresponding
$S_b$, and there's no way $\alpha$ to locally jump from being on the opposite
side of a page from $S_b$ to suddenly being on the same side. However, at the
gluings in the final step, we attach covers of different boundary components
together, so we may have attached multiple pieces $\widetilde{\alpha_j}$ of the same
type, or different types.

It is impossible to attach two natural segments $\widetilde{\alpha_j}$ in the final
step.  Suppose we had such an attachment, and look locally at the spine where
they're attached. Our attachment must yield a lift of $\alpha$, so we can look
at the local neighborhood in $\alpha$ covered by a neighborhood of our
attachment point.  Near the spine, this neighborhood must live in a single
local boundary component. But as we discussed in detail above, each local
boundary component simply consists of small neighborhoods on two page
boundaries, joined together by an annulus. Locally, a natural segment can only
be obtained by lifting to a cover of one of these two page boundaries, not any
of the other page boundaries near this spine. So one side of our attachment
must lie on one page boundary, and the other side on the other. But by
definition of how we do our final attaching step, we never attach in such a way
that we follow along the local boundary components! This is the point of having
valence at least 3 at each spine, and attaching $\gamma_i^+$ to
$\gamma_{i+2}^-$, is to avoid this problem. Since we never do attachments of
this form, we can't have an attachment with natural lifts on both sides.

So each attachment must have an opposite lift on one or both sides. But recall
that by (\dag''), each piece of each $\widetilde{S_b}$ has at most one free boundary
component. So all the $\widetilde{\alpha_j}$ must traverse at least two pieces, as we
ensured by construction of the $S'_{a,k}$ that they couldn't begin and end at the
same boundary component. This is where we finally use all those
carefully-established earlier criteria.

Because finally, as shown above, opposite lifts of individual $\alpha_i$ cannot
connect to opposite lifts on either side. So it's impossible to construct
a $\widetilde{\alpha_j}$ made out of opposite lifts, since it has to traverse
more than one piece.  This contradicts our assumption that the
$\widetilde{\alpha_j}$ glue to form a closed curve.

So we cannot join up the lifts of $\alpha_i$ to form a lift of $\alpha$.
Applying the parabolic lifting criterion shows that $\widetilde{S}$ is (QF).
This completes the proof of the theorem.

\end{proof}

\begin{thm}

Let $M$ be a good book of $I$-bundles.  Suppose $M$ does contain a component of
the parabolic locus $P$ inside a single page's boundary.  Construct $M'$ by
deleting all such pages from $M$, and $P'$ by removing all components of the
pared locus that intersect those pages. Then $(M,P)$ contains a surface
satisfying (QF) if and only if $(M',P')$ is nonempty and contains
a surface satisfying (QF).

\end{thm}

\begin{proof}

First observe that if $(M,P)$ contains a surface satisfying (QF), it cannot
traverse any of the pages we deleted to obtain $M'$, by the same argument as in
our first example.  Since our surface exists, $M'$ must be nonempty.
Furthermore, since $(M',P') \cin (M,P)$, it satisfies (QF) for $M',P'$ as well.
Conversely, if we have such a surface, we can obviously view it as contained in
$(M,P)$, where it must satisfy (QF) because it doesn't intersect the deleted
pages at all.

Note that $(M',P')$ may be nonelementary, or it still may not be good.
However, we can repeat the steps needed to guarantee that it's good, and obtain
a manifold where we can either apply this theorem again, or use the first
theorem. It is not immediately clear that this process terminates, because
taking the finite-sheeted cover we use for (2) and (3) makes the manifold more
complicated, possibly increasing the number of parabolics. We'll have to do
this carefully. Possibly we should do this step BEFORE the other
simplification, but then we'll have to make sure the parabolics pass through
that simplification nicely.

\end{proof}
