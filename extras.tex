\textbf{ The following sections ended up being unnecessary. I'll keep it here just
in case we need something. }

{\tiny

%%%%%%%%%%%%%%%%%%%%%%%%%%%%%%%%%%%%%%%%%%%%%%%%%
\section{Surface background details}
%%%%%%%%%%%%%%%%%%%%%%%%%%%%%%%%%%%%%%%%%%%%%%%%%

These are some background facts that we'll need to go through the examples in
detail.

Arcs in surfaces with boundary - from Fathi-Laudenback-Poenaru, Exposes 2 and
4 (see p. 21-24, 43-52).
This is an exposition of coordinates for classes of arcs in a surface
with boundary, aka Dehn-Thurston coordinates.

Let $A(N)$ be the set of isotopy classe sof simple closed proper arcs $I\in N$
such that they represent nontrivial elements of pi1 rel boundary. We isotope
them with the ends of the arcs free to move within the boundary components they
are contained in. Similarly $A'(N)$ with simple closed multi-arcs, up to
isotopy.

\begin{thm}[FLP 2.11]

Let $P^2$ be the standard pair of pants. $A(P^2)$ consists of exactly six
elements, classified by the boundary components of their endpoints.

\end{thm}

\begin{thm}[FLP 2.12]

$A'(P^2)$ is isomorphic to $A'={(a1,a2,a3) in Z | sum ai is even }$ via the map
$i:A'(P^2)->A', i(\tau) = (i(\tau, d_1),i(\tau,d_2),i(\tau,d_3))$ where
$d_1,d_2,d_3$ are the boundary components of $P^2$, and $i(\cdot,\cdot)$ is
geometric intersection number.

\end{thm}

That is, simple closed multicurves exist and are uniquely determined, for each
choice of boundary intersection numbers of the correct parity.

Finally, given a closed surface, let $N$, let $S(N)$ be the set of isotopy
classes of simple closed curves on $N$, and $S'(N)$ simple closed multicurves.

\begin{thm}[FLP 4.8]

Let $N$ be a closed surface of negative Euler characteristic.  Fix a pants
decomposition of $N$ into $2g-2$ pairs of pants along $3g-3$ disjoint simple
closed curves $K_1,...,K_{3g-3}$. Then

$S'(N)$ is isomorphic to $B_0 = {(m_i,s_i,t_i) \in \mathbb{Z}, i = 1,...,3g-3
\mid \text{ all coords }\geq 0, \text{ and for each pair of pants the
corresponding }m_i \text{ sum is even}}$ where the $m_i$ measure intersection
numbers with the curves $K1,...,K3g-3$, and the $s_i,t_i$ measure twisting
around at each intersection curve as a rational number.  See FLP for details of
the calculation of the twisting coordinates - we won't be needing it here (at
least, not yet!).

\end{thm}

We actually require a slight generalization of this result, but the proof is
identical.

\begin{thm}

Let $SA'(N)$ be the set of disjoint multi-curves or arcs on $N$, a surface
possibly with boundary. That is, each element of $SA'(N)$ corresponds to
a union of simple closed curves and simple arcs on $N$, all disjoint, such that
the arcs are properly embedded, up to isotopy where the arcs are permitted to
slide on each boundary component. Then $SA'(N)$ is isomorphic to the obvious
choice, where we again decompose into pairs of pants, have $mi$ for each
boundary curve or decomposition curve, but $s_i,t_i$ only for each
decomposition curve.

\end{thm}

Finally, let $A''(N)$ (or maybe some better notation, etc) be the set of simple
proper multi-arcs satisfying the condition that no two components of the
multi-arc are parallel, ie, isotopic. Alternatively we can think of $A''(N)$ as
consisting of simplices in the arc complex of $N$, or as a quotient of $A'(N)$
where we identify multicurves along splitting / joining of parallel arcs. In
general, $A''(N)$ is more difficult to parametrize, as purely from the
coordinates it is difficult to determine which multicurves or multiarcs will
contain parallel components.

We define $S''(N)$, $SA''(N)$ similarly.

\begin{example}

Consider the pair of pants $P^2$. We know $A(P^2)$ has 6 elements, so
$A''(P^2)$ has at most $2^6$ elements, corresponding to which arcs are present
in our multi-arc. It suffices to check that all 6 arcs can be embedded in $P^2$
without intersecting.

\end{example}

\begin{example}

Consider the punctured torus $\Si_{1,1}$. We first compute $SA'(\Si_{1,1})$.
By the generalized Dehn-Thurston coordinates theorem above, (work this out on
paper first).

\end{example}

%%%%%%%%%%%%%%%%%%%%%%%%%%%%%%%%%%%%%%%%%%%%%%%%%
\section{Old reduction}
%%%%%%%%%%%%%%%%%%%%%%%%%%%%%%%%%%%%%%%%%%%%%%%%%

We first restrict ourselves to considering only Kleinian groups which are
geometrically finite. In the geometrically infinite case, look at the simply
degenerate (geometrically infinite) ends of the group. By the Canary covering
theorem, tameness, and some other results, it's known that any finitely
generated subgroup of a Kleinian group must be either geometrically finite or
a virtual surface fiber - that is, it corresponds to a fiber surface in
a finite-sheeted cover which is fibered over the circle. See (AFW p117) for
this fact and various spots in the book for arguments.

% No virtual surface fiber subgroup can possibly be quasifuchsian,

Only geometrically finite subgroups can be quasifuchsian. This is basically by
definition (check this). The problematic case is when it's possible to have
a geometrically finite quasifuchsian surface subgroup of a geometrically
infinite Kleinian group. Looking at the simply degenerate ends, and comparing
to the ends of the quasifuchsian surface subgroup itself (viewed as a Kleinian
group), the relative ends must finite-to-one cover relative ends of the
original Kleinian group, by the Canary covering theorem. But now there's some
problem with different types of (relative) ends covering different types of
ends. I still don't understand this.

Also, note that quasifuchsian here is intended in the sense of Kahn-Markovic as
discussed in AFW p81. As discussed there (will get references), quasifuchsian
surface subgroups must be geometrically finite - which rules out the virtual
fiber case by the above dichotomy - and directly correspond to geometrically
finite subgroups that avoid the cusps.

One final consideration is to ensure that geometrically finite surface
subgroups cannot exhibit certain pathological behavior inside geometrically
infinite Kleinian groups. In the geometrically finite case this is easy,
because it'll necessarily also be geometrically finite, but in the
geometrically infinite case we need to either guarantee that the surface we
find by the below, purely topological construction is geometrically finite, or
show that actually no such surface exists and our construction is irrelevant.
Check with Ian (and references) about which of these we're actually doing!

Of course, so long as we are only studying books of $I$-bundles topologically,
this is irrelevant, as we can restrict ourselves to considering the
geometrically finite realizations of these hyperbolic 3-manifolds. However, it
is nice to be able to state the result more generally. Also if we want to
generalize to a more complete consideration of (finitely generated) Kleinian
groups at some point, not just books of $I$-bundles, it will be nice to say that
our work applies to more than just geometrically finite cases.

Anyway, I believe the geometrically infinite case is much simpler once
I understand what existing work it's based on. I'll need to talk to Ian about
this. In what follows, just assume our Kleinian group is geometrically finite.

% OLD REFERENCE - DOESN'T WORK BY ITSELF!
%%(Baker-Cooper Theorem 1.7 - "work of Bonahon and Thurston").
%\begin{thm}
%
%Let $M$ be a complete hyperbolic 3-manifold with finite volume, and $S$
%a closed oriented $\pi_1$-injective immersed surface of negative Euler
%characteristic. Then either $S$ is a virtual fiber, or $S$ is geometrically
%finite. In the second case, it's either quasifuchsian or some element of
%$\pi_1S$ is parabolic.
%
%\end{thm}
%

We assume this reduction still works in our cases (which are not finite
volume). I hope some reference can fix this!

So in our case, since the parabolic elements precisely correspond to the pared
structure, our problem becomes a purely topological one: can we find a closed
immersed $\pi_1$-injective surface of negative Euler characteristic, that
doesn't contain any of the pared locus, up to homotopy?

Since we're working in the case of books of $I$-bundles, we can reduce the
pared locus by observing that there are no torus boundary components of M, so
all the components of the pared locus must be annuli (as they're
$\pi_1$-injective). For each annulus, since we're only concerned with its
homotopy class, it suffices to consider the core curve.

%%%%%%%%%%%%%%%%%%%%%%%%%%%%%%%%%%%%%%%%%%%%%%%%%
\section{LERF and lifting properties}
%%%%%%%%%%%%%%%%%%%%%%%%%%%%%%%%%%%%%%%%%%%%%%%%%
\textbf{We had trouble with some technical details here, but we ended up not
needing these facts anyways.}

This is relevant to us because manifolds with LERF fundamental groups have nice
topological properties. In particular, we have the following fact.

\begin{thm}

Let $M$ be a compact $n$-manifold such that $\pi_1M$ is LERF. Let $M'$ to $M$
be a (possibly infinite-sheeted) cover such that $\pi_1M'$ is finitely
generated.  Suppose $C \cin M'$ is a compact subset which avoids the boundary
of $M$. Then there exists a intermediate cover $M' \to M'' \to M$, which is
finite-sheeted over $M$, such that the covering map $M' \to M''$ is an
embedding on $C$.

\end{thm}

\begin{proof}

See Long-Reid. Note that their proof is only stated for closed manifolds $M$,
but it carries over to all compact manifolds also if we restrict $C$ to avoid
the boundary of $M$.

% I think? Might want to check w Ian.

\end{proof}

This leads us to the following well-known fact. I haven't found a written proof
in the literature, so I've provided one. Thanks to Ian Agol for pointing this
out.

\begin{thm}

Let $M$ be a compact n-manifold such that $\pi_1M$ is LERF. Let $C$ be
a compact (n-1)-complex, and $\phi \colon C \looparrowright M$
a $\pi_1$-injective immersion.  Then there exists a finite sheeted cover $M'
\to M$ such that $\phi$ lifts to a map $\phi' \colon C \hookrightarrow M'$
which is homotopic to an embedding.

\end{thm}

Note that we can't guarantee that $\phi'$ is itself an embedding, because for
instance it's easy to map a small straight segment of $C$ so that its image
contains a small homotopicall trivial loop. This loop will be present in every
lift of $C$ to a cover of $M$.

\begin{proof}

Note that when we use this, both $C$ and $M$ will be aspherical, so an
alternative proof for that case is to invoke Whitehead's theorem to show
$\phi_H$ is a homotopy equivalence, followed by some obstruction theory facts
(??  - similar to the ones invoked in Long-Reid) to show it can be homotoped to
an embedding.

\textbf{ TODO is this a real proof? not sure...}

%Since C is a compact complex, pi1C is finitely generated. Let H=phi*(pi1C), and
%let pH colon MH to M be the corresponding cover of M. By our choice of H, phi
%lifts to a map phiH colon C to MH. We claim that phiH is homotopic to an
%embedding. Since C is compact and phi is an immersion, it suffices to show we
%can nicely homotope phiH to be injective.
%
%Since pH*(pi1MH) = H = phi*(pi1C), phiH must induce an isomorphism on pi1 in
%order to have pH circ phiH = phi. Since C is an immersed 1-complex, without
%loss of generality the only failures to be embedded we need to consider are
%double points.  Let x1,x2 in C such that phiH(x1)=phiH(x2)=y.  Let alpha cin
%C be a path from x1 to x2.  phiH(alpha) is a closed curve in MH.  If
%phiH(alpha) is homotopically trivial, we'll apply the loop theorem and homotope
%locally to remove the intersection at y.
%
%To be precise, we have two cases to consider. First, if phiH(alpha) is a simple
%closed curve, that is except for its two endpoints alpha embeds in MH. Then we
%can apply the loop theorem directly to obtain a disc D cin MH with boundary
%phiH(alpha). By homotoping across D we can ensure that phiH(C) cap int(D)
%= empyset.

% FIXME Ok HOLD ON this isn't right. Dimensions don't add up. If dim C = 1 and
% dim M = 3 it's trivial! Because just homotope into the extra dimension
% (general position) to fix -> in fact, I think this can be done generically.
%
% So really what I'm concerned with is (1) dim C = 1, dim M = 2. And also (2)
% dim C = 2, dim M = 3.
%
% I hope I don't have to consider these two cases separately...

\end{proof}

We use the above condition repeatedly to show that we can construct
a finite-sheeted cover of our book of $I$-bundles that satisfies certain nice
properties, by starting with an infinite-sheeted cover and then pushing down.

%%%%%%%%%%%%%%%%%%%%%%%%%%%%%%%%%%%%%%%%%%%%%%%%%
\section{Some generalizations}
%%%%%%%%%%%%%%%%%%%%%%%%%%%%%%%%%%%%%%%%%%%%%%%%%

We use the following general notation for books of $I$-bundles. Let
$M_{c1},...,M_{cm}$ be the spines, all solid tori. Let $M_1,...,M_n$ be the
pages, all thickened surfaces. Each $M_i$ has boundary components
$A_{i1},...,A_{ik_i}$ all annuli. We glue each annulus to a boundary annulus on
some spine. For each spine, all these gluing annuli must be disjoint parallel
incompressible. If $M$ is nonelementary, We can assume without loss of
generality that all pages have negative Euler characteristic, and all spines
have at least 3 pages glued to them (otherwise, just consolidate into fewer
pages / spines).

Given such a book of $I$-bundles $M$, let $G=(V,E)$ be the graph associated to
the embedded surface $A = \bigcup_{i,j} A_{ij} \cin M$ which is the union of
all the page gluing annuli. Combinatorially, $G$ has a vertex for each spine or
page, and an edge for each gluing of a page to a spine along an annulus. We
call $G$ the gluing graph of the book of $I$-bundles.

We say a book of $I$-bundles is \emph{tree-shaped} if its gluing graph is
a tree.

As above, if the pared locus $P$ contains any components that fit inside
a single $I$-bundle, then we can ``cross off that page.'' Just like in our
earlier lemma, any $\pi_1$-injective surface that passed through that page
would have would have to contain the component of the pared locus. By
repeatedly crossing off pages and consolidating into fewer pages / spines, we
can reduce any pared book of $I$-bundles to one with no such components.

\begin{conj}

Let $M$ be a tree-shaped book of $I$-bundles. Suppose $P$ has no components
$P_0\cin \bd_\pm M_i$ for any $i$. Then there exists a surface satisfying
\eqref{E:qf}.

\end{conj}

We're quite certain that this is true. It should just be a hairy inductive
adaptation of the above argument. In particular, it is totally unclear how one
might construct a counterexample. However, we haven't completed a full proof
yet.

\begin{thm}

Let $M$ be a tree-shaped book of $I$-bundles. Suppose $P$ has no components
$P_0 \cin \bd_\pm M_i$ for any $i$. Suppose further that every
component of $P$ crosses at least one gluing circle at a spine where at least
4 pages are attached. Then there exists a surface satisfying \eqref{E:qf}.

In particular, note that if all spines are at least valence 4, the technical
crossing condition must be satisfied.

\end{thm}
\begin{proof}

Our construction of $S$ is very simple. In fact, in this case, $S$ embeds in
our book of $I$-bundles. Construct $S$ by starting with a single page's core
surface, and traversing the tree-shaped gluing as follows. For each boundary
component of $S$, look at the valence of the associated spine. If it's valence
3, choose an arbitrary one of the other two pages. If it's valence at least 4,
choose a page that's non-boundary-adjacent to $S$, in that neither boundary
component of the incoming page attaches to a boundary component of the newly
chosen page when we glue the book of $I$-bundles. This is always possible for
valence at least 4, because at a spine, each incoming boundary component
connects to a single outgoing boundary component, and there are only 2 incoming
boundary components but at least 3 new pages to choose from. Attach the core
surface of the chosen page to $S$ with an annulus across the core. Since $M$ is
tree-shaped, we can build $S$ inductively without it running into itself.

We claim $S$ satisfies \eqref{E:qf}. $S$ is immediately properly immersed (in
fact, embedded) and $\pi_1$-injective, as it's a union of page cores and spine
annuli. It suffices to check the pared locus. Suppose some $P_k$ had $\pi_1S
\cap \pi_1P_k neq 1$, ie some multiple of its core curve was homotopic into
$S$. By the crossing condition, $P_k$ must traverse a gluing circle on a spine
of valence at least 4.  But at such a spine, $P_k$, which is contained in
a boundary component of $M$, must connect two boundary-adjacent pages. But we
chose $S$ so it would connect at such a spine to a non-boundary-adjacent page.
Since $M$ is tree-shaped, applying van Kampen shows that these pages correspond
to different pieces of the amalgamated free product for $\pi_1M$, so it's
impossble for this curve to be in $\pi_1S$. This completes the proof.

\end{proof}

Finally, we have one more technical intermediate result. We hoped this would
extend to more cases, but we haven't been able to do anything with it yet.

Let $M$ be a book of $I$-bundles. Associated to every surface we construct in
an analogous way to what we've been doing (taking covers of the page core
surfaces and gluing them together), there's a set of fairly complicated graphs
we can construct as follows. For each $(M_i,A_{ij},\widetilde{S_i})$, where
$M_i$ is a page, $A_{ij}$ gluing annulus of that page, and $\widetilde{S_i}$
finite-sheeted cover of the page core surface, we can draw a \emph{local
pared-arc graph} $G=G(M_i,A_{ij},\widetilde{S_i})$.

The vertices are boundary components of $\widetilde{S_i}$ which sit above
$A_{ij}$. Two vertices are connected by an edge if there is an arc in the pared
locus which lifts to connect those two boundary components. For instance,
a double cover of the punctured torus page, with 3 disjoint non-parallel arcs
on each boundary, has a graph, over the single boundary annulus downstairs,
with 2 vertices and 3 edges (one between the two vertices, and one loop at each
vertex). This graph is a generalization of the cis-trans terminology we used in
that example.

We say a cover $\widetilde{S_i}$ is \emph{pared-bipartite} if each of its local
pared-arc graphs is bipartite. Note that none of the covers we constructed in
our first example are pared-bipartite. This is a very specialized condition.
Note also that if the pared locus does not intersect a page, all covers of that
page are pared-bipartite.

\begin{thm}

Let $M$ be a tree-shaped book of $I$-bundles. Suppose that $P$ has no
components in $\bd_\pm M_i$ for any $i$. Suppose that every page of $M$
has a pared-bipartite cover of its core surface. Then $M$ contains a surface
satisfying \eqref{E:qf}.

\end{thm}
\begin{proof}

First observe that every cover of a pared-bipartite cover is itself
pared-bipartite. Simply lift the partition of the vertices to obtain a new
partition of each local pared-arc graph. The arcs cannot break this
partition because otherwise they wouldn't cover arcs in the downstairs cover.

Choose a sufficiently large cover of a pared-bipartite cover of each page such
that at each spine, the number of incoming cover boundary components above each
incident page equal for that spine. We can do this taking a ``least common
multiple cover,'' unless pages are glued to the same spine multiple times. In
that case, choose a finite-sheeted cover of the book of $I$-bundles itself that
doesn't have this issue, lift the pared locus, and construct a surface there.
Afterwards, we can push it down and it will still satisfy \eqref{E:qf}.

% careful! finite subgp separability covers of tree-shaped things aren't
% necessarily tree-shaped! might need another condition / argument here.

Now, we construct $S$ by gluing these large covers. For each $\widetilde{S_i}$
and downstairs gluing annulus $A_{ij}$ there will be two adjacent annuli in the
associated spine. For each $\widetilde{S_i}$, attach half the boundary
components above $A_{ij}$ to the boundary components coming from each of the
``neighboring'' $\widetilde{S_i}$, via annuli. Use the pared-bipartite
structure to split the set of boundary components above $A_{ij}$ in half. Since
the number of incoming boundary components match at each spine, as constructed
above, the boundary components will match up. We have a ``cover boundary
component bipartition'' above each $A_{ij}$. Note that once we have the
bipartition, the details of how we glue the pieces inside the bipartition are
irrelevant.

We claim the resulting $S$ satisfies \eqref{E:qf}. We already know it's proper
immersed $\pi_1$-injective by the same arguments. To show it doesn't overlap
with the pared locus, observe that by the definition of local pared-arc graphs,
every core curve in the pared structure must correspond to a sequence of arcs
in covers, where each arc either connects boundary components above different
annuli $A_{ij}$, $A_{ij'}$, or it connects 2 boundary components above the same
annulus $A_{ij}$. But it cannot connect two boundary components above the same
annulus, as we glued those according to the pared bipartition in such a way
that the pared arc component, which is locally restricted to a single boundary
component of the $I$-bundles, can't follow. This is a generalization of the
first argument in our simplest case.

Note that if the arc connects boundary components above different annuli, these
annuli must attach to different spines (by our earlier simplifying assumption).
But now because $M$ is tree-shaped, no arc that's been redirected toward
different spines can possibly form a closed loop. This completes the proof.

\end{proof}

We tried to do this proof in the non-tree-shaped case, but it doesn't work!
There are counterexamples of pared-bipartite covers where the obvious fix
relies on messy combinatorics inside the boundary component matching.


%%%%%%%%%%%%%%%%%%%%%%%%%%%%%%%%%%%%%%%%%%%%%%%%%
\section{Next steps}
%%%%%%%%%%%%%%%%%%%%%%%%%%%%%%%%%%%%%%%%%%%%%%%%%

$I$-bundles. In fact, combinatorially, it seems that all books of $I$-bundles
(satisfying an appropriate condition on components of $P$) should admit these
surfaces. But we haven't proved anything yet. The general case is much messier
(combinatorially) than the tree-shaped case, because in addition to cis arcs
we'll also need to consider trans arcs that ``wrap around'' a loop in the
gluing graph. So even ensuring that every closed curve in the pared structure
is broken into arcs, at least one of which is trans, is insufficient.

We chose to study books of $I$-bundles first. Ian thinks the other infinite
volume cases ought to be easier. What about acylindrical manifolds?  Books of
$I$-bundles are somehow the opposite end of the spectrum of acylindrical
- they're glued together along a bunch of cylinders, and the pieces are all
very simple ($I$-bundles).  If we can address the acylindrical and book of
$I$-bundle cases we ought to be able to put these together to solve the entire
problem. Ian thinks maybe the acylindrical case can be addressed by a variant
of the Baker-Cooper argument.

}%tiny

% TODO rewrite
\textbf{ TODO Rewrite next steps after seeing how far we get.}
