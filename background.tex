%%%%%%%%%%%%%%%%%%%%%%%%%%%%%%%%%%%%%%%%%%%%%%%%%
\section{Quasi-Fuchsian surface subgroups}
%%%%%%%%%%%%%%%%%%%%%%%%%%%%%%%%%%%%%%%%%%%%%%%%%

We assume familiarity with standard terminology and results of $3$-manifold
topology and Kleinian groups.  For $3$-manifold topology, standard references
are books by Hempel \cite{He}, Jaco \cite{Ja}, and Thurston
\cite{Thurstonbook}, as well as Thurston's notes \cite{Thurstonnotes}.  For
a gentler introduction, see Schultens \cite{Sch} or notes by Lackenby
\cite{Lackenbynotes}, Hatcher \cite{Hatchernotes}, and Calegari
\cite{Calegarinotes2}. For general hyperbolic geometry, see Ratcliffe
\cite{Ratcliffe}, Benedetti--Petronio \cite{BenedettiPetronio}, or Hubbard
\cite{Hubbard}.  And for Kleinian groups, references are Maskit \cite{Maskit},
Marden \cite{Marden}, Morgan \cite{Mo}, Kapovich \cite{Kapovich}, and
Thurston's notes \cite{Thurstonnotes}.  For a gentler introduction, see
Matsuzaki--Taniguchi \cite{MatsuzakiTaniguchi} or Calegari's notes
\cite{Calegarinotes}.

All Kleinian groups in this paper are assumed to be finitely generated and
torsion-free. Unless stated otherwise, all hyperbolic manifolds are complete.

We first recall the following basic definitions.

\begin{defn}

A \emph{surface group} is a group isomorphic to the fundamental group of
a closed surface of negative Euler characteristic. Such a group can be written
with the finite presentation
%
$\pi_1 \Si_g = \langle a_1,b_1,\dots,a_g,b_g | a_1b_1a_1^{-1}b_1^{-1} \dots
a_gb_ga_g^{-1}b_g^{-1} = 1\rangle$
%
where $g \geq 2$.

\end{defn}

Note that surface groups are not free. The fundamental group of a compact
surface with boundary of negative Euler characteristic is $\Si_{g,b}$ is the
free group on $2g + b - 1$ generators. We wish to consider surface subgroups of
Kleinian groups which are quasi-Fuchsian.

\begin{defn}

A {quasi-Fuchsian group} is a Kleinian group $\Ga < \PSL_2\bC$ for which there
exists a quasi-conformal homeomorphism $f \colon S^2_\infty \to S^2_\infty$
which conjugates $\Ga$ to a Fuchsian group, that is, a discrete subgroup of
$\PSL_2\bR$.

\end{defn}

As discussed in the introduction, our general problem statement is quite
simple.

\begin{prob}

Let $\Ga$ be a Kleinian group of infinite covolume. Does $\Ga$ contain
a surface subgroup which is quasi-Fuchsian?

\end{prob}

We know that being quasi-Fuchsian is equivalent to a few other conditions.
These will be the conditions we actually use in this paper.

\begin{prop}

Let $S$ be a compact surface (possibly with boundary), and $\phi \colon \pi_1S
\to \PSL_2\bC$ be a discrete faithful representation. The following are
equivalent.

\begin{enumerate}

\item $\phi(\pi_1S)$ is quasi-Fuchsian.

\item $\phi(\pi_1S)$ is geometrically finite, and the conjugates of peripheral
subgroups of $\pi_1S$ are precisely those elements of $\pi_1S$ which are
parabolic after applying $\phi$.

\item $\bH^3/\phi(\pi_1S)$ has convex core homeomorphic to $S \times I$.

$\phi(\pi_1S)$ has limit set a Jordan curve, and no element of $\phi(\pi_1S)$
interchanges the complementary components.

\end{enumerate}

\end{prop}

\begin{proof}

See Morgan \cite[Proposition 9.2]{Mo}.% Proposition 9.2

\end{proof}

As discussed in the introduction, we wish to identify quasi-Fuchsian surface
groups --- that is, quasi-Fuchsian groups as above where $S$ is a closed
surface.  Condition (2) above tells us that such a subgroup must be
geometrically finite.  Furthermore, it must not contain any parabolic elements
at all.

We now make another observation. By condition (3), any quasi-Fuchsian surface
group $\phi(\pi_1S)$ will contain an embedded copy of the surface $S$ inside
its convex core.  This embedding $i \colon S \to \bH^3/\phi(\pi_1S)$ is an
isomorphism on $\pi_1$.  Given a Kleinian group $\Ga$, a quasi-Fuchsian surface
subgroup $\phi(\pi_1S) < \Ga$ corresponds to a covering map $p \colon
\bH^3/\phi(\pi_1S) \to \bH^3/\Ga$. This map is $\pi_1$-injective.  Projecting
$i(S)$ downward into $\bH^3/\Ga$, we obtain a closed surface $p(i(S))$. $p
\circ i \colon S to \bH^3/\Ga$ is $\pi_1$-injective. In fact, $(p \circ
i)_\star(\pi_1S))) = \phi(\pi_1S)$.

We can therefore re-state our general problem as follows.

\begin{prob}

Let $N$ be an infinite volume hyperbolic $3$-manifold. That is, $\pi_1N$ is
a Kleinian group of infinite covolume. Does there exist a $\pi_1$-injective map
$f \colon S \to N$, where $S$ is a closed surface, such that $f_\star(\pi_1S)$
is geometrically finite and contains no parabolic elements?

\end{prob}

In the remainder of this paper, we'll refer to surface maps $(S,f)$ which
satisfy the problem condition as \emph{quasi-Fuchsian surfaces} or \emph{(QF)
surfaces}. By abuse of notation, we'll often simply refer to the image $f(S)$
as a (QF) surface if the choice of $S$ and $f$ is clear.

However, even though we are now looking for a topological surface map, solving
this question as written still requires knowledge of the geometry of $N$. We
need to know which elements of $\pi_1N$ are parabolic, and whether or not
a given subgroup is geometrically finite. We now discuss how to reduce these
conditions to a purely topological conditions on $S$ and $f$.

%%%%%%%%%%%%%%%%%%%%%%%%%%%%%%%%%%%%%%%%%%%%%%%%%
\section{Pared $3$-manifolds and hyperbolization}
%%%%%%%%%%%%%%%%%%%%%%%%%%%%%%%%%%%%%%%%%%%%%%%%%

\begin{defn}

A \emph{pared $3$-manifold} $(M,P)$ is a compact orientable irreducible
$3$-manifold $M$, together with a submanifold $P\cin\bd M$, such that the
following conditions hold:

\begin{enumerate}
\item Every component of $P$ is a torus or annulus, incompressible in $M$.

\item Every noncyclic abelian subgroup of $\pi_1M$ is peripheral with respect
to $P$ --- ie, conjugate to the fundamental group of a component of $P$.

\item $(M,P)$ is ``$A^2$-incompressible'': every $\pi_1$-injective map $(A^2,
\bd A^2) \to (M,P)$ is homotopic (as a map of pairs) to a map into $P$.

\end{enumerate}

We call $P$ the \emph{pared locus} or \emph{parabolic locus} of the pared
$3$-manifold $(M,P)$.

\end{defn}

Note that there are a few pared manifolds that are special cases, like with
elementary Kleinian groups. In fact, these are precisely the pared manifolds
that correspond to elementary Kleinian groups when we construct pared
$3$-manifolds from Kleinian groups below. This is a standard definition --- for
instance, see Canary--McCullough \cite[pp88]{CMc}.

%(Canary-McCullough p88)
\begin{defn}

A pared manifold $(M,P)$ is \emph{elementary} if it is homeomorphic (as a pair)
to one of the following: $(T^2\x I,T^2\x 0)$, $(A^2\x I,A^2\x 0)$, or $(A^2\x
I,\emptyset)$, or $(S3,\emptyset)$.

\end{defn}

Pared manifolds arise naturally in the study of Kleinian groups. Intuitively,
given a geometrically finite Kleinian group $\Ga$, the pared $3$-manifold is
a compact core, that is a compact submanifold $M \cin \bH^3/\Ga$ with $\pi_1M
= \Ga$.  The pared locus $P$ is the union of boundary components corresponding
to cusps of $\bH^3/\Ga$, that is, parabolic subgroups of $\Ga$ (up to
conjugacy).  Reversing this correspondence hyperbolizes a pared $3$-manifold,
giving it a geometrically finite geometric structure where the desired boundary
regions become cusps.

To be precise, given a geometrically finite Kleinian group $\Ga$, we construct
a pared $3$-manifold $(M,P)=(\cM(\Ga),\cP(\Ga))$ as follows.

We know that the convex core $C(\Ga) \cin \bH^3/\Ga$ has finite volume. For
epsilon sufficiently small, The thick part $C(\Ga)[\epsilon,\infty)$ is
compact, and the thin part $C(\Ga)(0,\epsilon]$ is a union of small
neighborhoods of the cusps.  As long as epsilon is sufficiently small, the
exact choice of $\epsilon$ is arbitrary, as the resulting thick and thin parts
are homeomorphic.  Now let $\cM(\Ga) = C(\Ga)[\epsilon,\infty)$, and $\cP(\Ga)
= C(\Ga)[\epsilon] = \bd C(\Ga)[\epsilon,\infty) \cap \bd C(\Ga)(0,\epsilon]$.
That is, $\cP(\Ga)$ is the boundary locus along which we truncated to remove
the thin part. We can rebuild the convex core from $(\cM(\Ga),\cP(\Ga))$ by
gluing cusp neighborhoods onto each component of $\cP(\Ga)$.

\begin{prop}

$(\cM(\Ga),\cP(\Ga))$ is a pared $3$-manifold. $\cM(\Ga)- \cP(\Ga)$ is
homeomorphic to the convex core of $\Ga$, $C(\Ga)$. The parabolic elements of
$\Ga$ precisely correspond to conjugates of the peripheral subgroups
$\pi_1\cP(\Ga) \cin \pi_1\cM(\Ga)$ (really, the conjugates of the peripheral
subgroups associated to each component of $\cP(\Ga)$).

\end{prop}

\begin{proof}

See Morgan \cite[Corollary 6.10]{Mo}. % Corollary 6.10

\end{proof}

We refer to $(\cM(\Ga),\cP(\Ga))$ as the \emph{pared $3$-manifold associated
to} $\Ga$.  Similarly, if $N$ is a geometrically finite hyperbolic
$3$-manifold, we refer to $(\cM(\pi_1N),\cP(\pi_1N))$ as the \emph{pared
$3$-manifold associated to} $N$.

Conversely, there is Thurston's famous hyperbolization theorem:

\begin{thm}

Let $(M,P)$ be a pared $3$-manifold with $M$ Haken. Then there exists
a geometrically finite hyperbolic $3$-manifold $N$ such that
$\cM(\pi_1N),\cP(\pi_1N)$ is homeomorphic to $(M,P)$.

\end{thm}

\begin{proof}

See surveys by Morgan \cite{Mo} and Scott \cite{ThurstonviaScott}, or work of
Kapovich \cite{Kapovich} for a broad overview of the proof. Thurston first
announced the theorem in \cite{Thurston0}. Key details of the argument are
presented in Thurston's 3-part paper (\cite{ThurstonI}, \cite{ThurstonII}, and
\cite{ThurstonIII}).

\end{proof}

For more details and examples of pared $3$-manifolds, see Canary--McCullough
\cite{CMc}, Morgan \cite{Mo}, or Thurston's paper \cite{ThurstonI}.

We now return to our original problem. We can convert a given geometrically
finite Kleinian group $\Ga$ to a pared $3$-manifold $(M,P)$. Intuitively, since
$P$ corresponds to cusps, ie, parabolics, a surface subgroup of $\Ga$ avoids
parabolic elements if and only if it avoids the peripheral subgroups associated
to components of $P$.

First, suppose we have a geometrically finite Kleinian group.  We use the
following theorem of Thurston.

\begin{thm}

Let $N$ be a geometrically finite hyperbolic manifold with convex core $C(N)$.
Suppose that $N$ has infinite volume. Then every covering space $N'$ of $N$
with finitely generated nonelementary fundamental group is also geometrically
finite. In fact, it suffices to assume $N$ is geometrically finite and $\bd
C(N)$ is nonempty.

%In our case, we have a geometrically finite hyperbolic manifold of infinite
%volume, so letting E be an (infinite volume) relative end, C(N) cap E must
%yield a nonempty piece of bd C(N). This suffices to use the theorem.

\end{thm}

\begin{proof}

See Morgan \cite[Proposition 7.1]{Mo}. % Proposition 7.1.

\end{proof}

% This argument (the next few paragraphs) follows Morgan. Proposition 7.3

Let $(M,P)$ be the pared $3$-manifold structure associated to the hyperbolic
$3$-manifold $N$. The only elements of $\pi_1M$ which are parabolic are the
conjugates of elements in $\pi_1P_0$, where $P_0$ is a component of $P$.

Let $\Ga' < \pi_1N$ be a surface subgroup which is quasi-Fuchsian. Let $N'$ be
the cover of $N$ associated to $\Ga'$. $N'$ and hence its convex core $C(N')$
have the fundamental group of a closed surface. In fact, because $\Ga'$ is
quasi-Fuchsian, we know that the convex core $C(N')$ is homeomorphic to $S
\times I$, where $S$ is a closed surface such that $\pi_1S = \Ga'$.  Map $S \to
S \times \{1/2\} \cin S \times I$. This map is an isomorphism on $\pi_1$.
Composing this with the covering map $N' \to N$, we obtain a $\pi_1$-injective
map $\phi \colon S \to N$.  This corresponds to a $\pi_1$-injective map to
$(M,P)$ by retracting.  Applying the theorem above, no element of
$\Ga'=\phi(\pi_1S)$ is parabolic.  Because $(M,P)$ is a pared structure for
$N$, this implies that no loop in $S$ is freely homotopic to a loop in $P$.

Conversely, suppose we are given a $\pi_1$-injective map $S \to M$ where $S$ is
a closed surface.  Suppose further that no loop in $S$ is freely homotopic to
a loop in $P$.  This implies that the induced subgroup $\pi_1S < \pi_1M$ is
disjoint from all conjugates of $\pi_1P$.  Because $(M,P)$ is a pared structure
for the hyperbolic $3$-manifold $N$, this implies that no elements of $\pi_1S$
are parabolic under the representation $\rho \colon \pi_1S \to \PSL_2\bC$
induced by $\pi_1S < \pi_1M=\pi_1N$.  Furthermore $\rho(\pi_1S)$ is
geometrically finite (by the first theorem), and $S$ is a surface without
cusps.  Applying the theorem, $\pi_1S$ induces a quasi-Fuchsian surface
subgroup of $\pi_1N$.

We summarize our discussion of the geometrically finite (infinite volume) case
as follows.

\begin{prop}

Let $N$ be a geometrically finite hyperbolic $3$-manifold of infinite volume,
and $(M,P)$ a pared $3$-manifold structure for $N$. Then quasi-Fuchsian surface
subgroups of $\pi_1N$ precisely correspond to $\pi_1$-injective maps of
a closed surface $S \to M$ such that no loops in $P$ are freely homotopic to
images of loops in $S$.

\end{prop}

This allows us to reduce the main problem to the following:

\begin{prob}

Let $(M,P)$ be a pared $3$-manifold (which has an infinite volume
hyperbolization).  Does there exist a closed surface $S$ and
a $\pi_1$-injective map $f \colon S \to M$ such that no loops in $P$ are freely
homotopic to images of loops in $S$?

\end{prob}

Similarly, we will refer to surfaces $(S,f)$ which satisfy this condition for
a pared $3$-manifold $(M,P)$ as \emph{quasi-Fuchsian surfaces} or \emph{(QF)
surfaces}.  Again, by abuse of notation. We will simply refer to $f(S)$ as (QF)
if $f$ and $S$ are clear from context.

We now briefly discuss the geometrically infinite case. Because the arguments
that follow in later sections are purely topological arguments applied to the
pared structure, we can always use Thurston's uniformization theorem to produce
a geometrically finite Kleinian group to which our arguments apply. However, we
would like to be able to address the geometrically infinite case also.

First note that any finitely generated Kleinian group, whether geometrically
finite or geometrically infinite, is \emph{tame}, that is, homeomorphic to the
interior of a compact $3$-manifold. This is a deep result due to Agol
\cite{Agoltameness} and independently, Calegari and Gabai \cite{CalegariGabai}.

We now begin with the following theorem of Canary.

\begin{thm}

Let $N = \bH^3/\Ga$ be an infinite volume tame hyperbolic $3$-manifold, and let
$\Ga' < \Ga$ be a finitely generated subgroup. Then either

\begin{enumerate}

\item $N' = \bH^3 / \Ga'$ is geometrically finite, or

\item $\Ga'$ contains a (conjugate of a) finite index subgroup of
a geometrically infinite peripheral subgroup of $\Ga$.

\end{enumerate}

Above, a \emph{geometrically infinite peripheral subgroup} is a subgroup
corresponding to the subsurface of a relative compact core boundary which cuts
off a geometrically infinite relative end of N. Equivalently, since all
geometrically infinite ends of tame manifolds are simply degenerate,
a geometrically infinite peripheral subgroup is the subgroup associated to the
boundary in the interior of $N$ of a geometrically infinite end.

\end{thm}

\begin{proof}

See work of Canary \cite{Canary}. Also see previous related work by Canary
\cite{Canary2}, Bonahon \cite{Bonahon}, and Thurston \cite{Thurstonnotes}.

\end{proof}

The above proposition in the geometrically finite case still works in the
forward direction. The reverse direction will carry through if we can show that
$N'$ (ie $\Ga'$) is geometrically finite. That is, we want to avoid condition
(2) in Canary's theorem above. This requires an additional condition, which we
incorporate in the following proposition. (2) holds precisely when our surface
is freely homotopic to a finite-sheeted cover of the corresponding
geometrically infinite end boundary surface. Since such an end is simply
degenerate, that is, homeomorphic to a thickened surface, we can also apply the
covering lemma (see \ref{L:covering}) to obtain the following proposition.

\begin{prop}

Let $N$ be a hyperbolic $3$-manifold of infinite volume, possibly geometrically
infinite, and $(M,P)$ a pared $3$-manifold structure for $N$. Then
quasi-Fuchsian surface subgroups of $N$ precisely correspond to
$\pi_1$-injective maps of a closed surface $S \to M$ such that the following
conditions hold.

\begin{enumerate}

\item No loops in $P$ are freely homotopic to images of loops in $S$.

\item $S$ is not freely homotopic into a geometrically infinite end of $M$, or,
equivalently, not freely homotopic to a finite-sheeted cover of a geometrically
infinite end boundary surface.

\end{enumerate}

\end{prop}

We will not elaborate on the geometrically infinite ends case in this paper.
However, analogous constructions should provide a proof of results similar to
our main result.

