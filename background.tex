{\tiny

%%%%%%%%%%%%%%%%%%%%%%%%%%%%%%%%%%%%%%%%%%%%%%%%%
\section{Background on pared 3-manifolds and uniformization}
%%%%%%%%%%%%%%%%%%%%%%%%%%%%%%%%%%%%%%%%%%%%%%%%%

Material in this section is from \cite{Mo} and \cite{CMc}. Look there for
further details.

%Definition and important facts about pared manifolds. After Morgan, The Smith
%Conjecture, V ("Uniformization Theorem for Three-Dimensional Manifolds"),
%p 58-60. Or Canary-McCullough, Homotopy Equivalences of 3-Manifolds and
%Deformation Theory of Kleinian Groups, Ch 5 p. 87-92. Also Ch 7 p.105-107.
%Note that Canary-McCullough is much more recent.
%
%(Morgan p58) (Canary-McCullough p87)
\begin{defn}

A \emph{pared manifold} $(M,P)$ is a compact orientable irreducible 3-manifold
$M$, together with $P\cin\bd M$, such that the following conditions hold:

\begin{enumerate}
\item Every component of $P$ is a torus or annulus, incompressible in $M$.

\item Every noncyclic abelian subgroup of $\pi_1M$ is peripheral with respect
to $P$ -- ie, conjugate to the fundamental group of a component of $P$.

\item $(M,P)$ is ``annulus-incompressible'': every $\pi_1$-injective map $(A^2,
\bd A^2) \to (M,P)$ is homotopic (as a map of pairs) to a map into $P$.

\end{enumerate}

We call $P$ the \emph{pared locus} or \emph{parabolic locus} of the pared
manifold $(M,P)$.

\end{defn}

Note that there are a few pared manifolds that are special cases, like with
elementary Kleinian groups. In fact, these are precisely the pared manifolds
that correspond to elementary Kleinian groups when we construct pared
3-manifolds from Kleinian groups below.

%(Canary-McCullough p88)
\begin{defn}

A pared manifold $(M,P)$ is \emph{elementary} if it is homeomorphic (as a pair)
to one of the following: $(T^2\x I,T^2\x 0)$, $(A^2\x I,A^2\x 0)$, or $(A^2\x
I,\emptyset)$.
%(or (Morgan only) (S3,empty)).

\end{defn}

%(Morgan p59, Canary-McCullough p88)
\begin{prop}

Let $(M,P)$ nonelementary. The following facts hold:

%In fact, Canary-McCullough makes these 3 statements. Morgan only has the 4th.
\begin{enumerate}
\item M is not a $T^2\x I$, $K^2$ $I$-bundle, or $A^2\x I$ (solid torus).
\item M does not contain an embedded $K^2$
\item For every component $Q$ of $P$, $\pi_1Q$ is a maximal abelian subgroup of
$\pi_1M$.
\item Every component of $\bd M-P$ has negative Euler characteristic.
% Yes this is correct. But the only extra info we get is that there's no
% parallel annuli. Canary-McCullough includes the statement:
% "Every toroidal component of dM is contained in P"
% but this is redundant with item 4 above (from Morgan)
\end{enumerate}

\end{prop}
\begin{proof}
See \cite{CMc} or \cite{Mo}.
\end{proof}

Pared manifolds arise naturally in the study of Kleinian groups. Given
a geometrically finite torsion-free Kleinian group $G$, we construct $(M,P)$ by
truncating small neighborhoods of the cusps of the quotient manifold $M(G)
= \left(\mathbb{H}^3\cup \Om(G)\right)/G$.  Then we let $M$ be the resulting
compact manifold with boundary, and $P$ the boundary locus along which we
truncated.  That is, we can rebuild the quotient manifold from $(M,P)$ by
gluing cusp neighborhoods onto each component of $P$. It follows from basic
Kleinian group theory that the resulting $(M,P)$ will satisfy conditions 1-3
above, making it a pared manifold.  The parabolic elements of $G$ precisely
correspond to (conjugacy classes of) the peripheral subgroups $\pi_1P$ in
$\pi_1M$.

Conversely, Thurston's famous uniformization theorem states that for every
pared manifold $(M,P)$ with $M$ Haken, there exists a finite volume (in
particular, geometrically finite) hyperbolic 3-manifold structure on the
interior satisfying the appropriate conditions for a pared 3-manifold! To be
precise:

%(see Morgan for details of this construction)

%(Morgan p60, but his statement is confusing / less modern definitions. He does
%state the more general theorem for Haken pared manifolds though.)
%(Canary-McCullough p105-106)

\begin{thm}

If $(M,P)$ is an oriented pared 3-manifold with nonempty boundary, then
there exists a geometrically finite uniformization of $(M,P)$, that is, a map
$\rho: \pi_1M \to \mbox{\textrm{PSL}}_2\mathbb{C}$ such that $\rho(g)$ is
parabolic if $g \in \pi_1(P)$, and an orientation-preserving homeomorphism $M-P
\to \left(\mathbb{H}^3 \cup \Om(\rho(\pi_1M))\right)/\rho(\pi_1M)$.

\end{thm}
\begin{proof}

See \cite{CMc}.
Also see discussion on (CMc p106, bottom of page) for why this definition of
geometrically finite is equivalent to the standard one.

\end{proof}

% Example. Maybe? TODO

\textbf{ TODO Include an example.}

%%%%%%%%%%%%%%%%%%%%%%%%%%%%%%%%%%%%%%%%%%%%%%%%%
\section{Background on quasifuchsian surface groups}
%%%%%%%%%%%%%%%%%%%%%%%%%%%%%%%%%%%%%%%%%%%%%%%%%

We study the problem of finding quasifuchsian surface subgroups of finitely
generated Kleinian groups from the topological perspective. We convert our
given Kleinian group to a pared 3-manifold, and look for surface subgroups of
this Kleinian group that avoid the peripheral pared structure. We can reduce
the problem of finding a quasifuchsian surface subgroup of a Kleinian group as
follows.

We first recall the following basic definitions.

defn

A \emph{surface group} is (a group isomorphic to) the fundamental group of
a closed surface of negative Euler characteristic.

enddefn

defn

A {quasifuchsian group} is a Kleinian group Gamma < PSL2C for which there
exists a quasiconformal homeomorphism f colon S2infty to S2infty which
conjugates Gamma to a Fuchsian group, that is, a subgroup of PSL2R.

enddefn

% TODO
\textbf{TODO perhaps I should elaborate on quasiconformal background a little.}

First, suppose we have a geometrically finite Kleinian group. We use the
following theorem of Thurston.

thm

Let N be a geometrically finite hyperbolic manifold with convex core C(N).
Suppose that bd C(N) is nonempty. Then every covering space N' of N with
finitely generated nonelementary fundamental group is also geometrically
finite.

endthm

proof

See \cite{Mo}, Proposition 7.1.

endproof

In our case, we have a geometrically finite hyperbolic manifold of infinite
volume, so letting E be an (infinite volume) relative end, C(N) cap E must
yield a nonempty piece of bd C(N). This suffices to use the theorem.

% TODO
\textbf{TODO this explanation is a bit vague. Might want to say more details
about convex core etc.}

We now apply the following result in \cite{Mo}, Corollary 9.2.

thm

Let S be a complete hyperbolic surface of finite area, and rho colon pi1S to
PSL2C a discrete faithful representation. Then rho(pi1S) is quasifuchsian if
and only if rho(pi1S) is geometrically finite, and the conjugacy classes in
pi1S represented by cusps in S are exactly the conjugacy classes that are
parabolic under rho.

endthm
% also look at (Ohshika Lemma 4.66 - cited in AFW p.81).  FIXME look up exact
% statement in Ohshika. I have a feeling AFW are introducing some bs here...

proof

See \cite{Mo}.

% WHY is this true? See Morgan Proposition 7.2, Corollary 7.3, and Proposition
% 9.1 for some detailed arguments.

endproof

%TODO

\textbf{TODO The following paragraph's "associated to" requires proof! Convex
core cutting argument. See Morgan Corollary 6.10.}

Now we argue as in \cite{Mo}, Corollary 7.3. Let (M,P) be a pared 3-manifold
structure associated to the hyperbolic 3-manifold N. The only elements of pi1M
which are parabolic are the conjugates of elements in pi1P.

Let Gamma' < pi1N be a surface subgroup which is quasifuchsian. Let N' be the
cover of N associated to Gamma'. N' and hence its convex core C(N') have the
fundamental group of a closed surface. In fact, because Gamma' is
quasifuchsian, we know that the convex core C(N') is homeomorphic to S times I,
where S is a closed surface such that pi1S = Gamma'.\cite{Mo}, same Corollary.
Map S to S times 1/2 cin times I. This map is an isomorphism on pi1. Composing
this with the covering map N' to N, we obtain a pi1-injective map phi colon
S to N.  This corresponds to a pi1-injective map to (M,P) by retracting.
Applying the theorem above, no element of Gamma'=phi(pi1S) is parabolic.
Because (M,P) is a pared structure for N, this implies that no loop in S is
freely homotopic to a loop in P.

Conversely, suppose we are given a closed pi1-injective surface
S looprightarrow M.  Suppose further that no loop in S is freely homotopic to
a loop in P. This implies that the induced subgroup pi1S < pi1M is disjoint
from all conjugates of pi1P. Because (M,P) is a pared structure for the
hyperbolic 3-manifold N, this implies that no elements of pi1S are parabolic
under the representation rho colon pi1S to PSL2C induced by pi1S < pi1M=pi1N.
Furthermore rho(pi1S) is geometrically finite (by the first theorem), and S is
a surface without cusps.  Applying the theorem, pi1S induces a quasifuchsian
surface subgroup of pi1N.

We summarize our discussion of the geometrically finite case as follows.

prop

Let N be a geometrically finite hyperbolic 3-manifold of infinite volume, and
(M,P) a pared 3-manifold structure for N. Then quasifuchsian surface subgroups
of N precisely correspond to pi1-injective maps of a closed surface S to M such
that no loops in P are freely homotopic into the image of S.

% FIXME need to mumble something about "up to homotopy" to make this a real
% correspondence...

endprop

We will refer to surfaces in (M,P) which satisfy the proposition as
\emph{quasifuchsian surfaces} or \emph{(QF) surfaces}.

We now briefly discuss the geometrically infinite case. Because the arguments
that follow in later sections are purely topological arguments applied to the
pared structure, we can always use Thurston's uniformization theorem to produce
a geometrically finite Kleinian group to which our arguments apply. However, we
would like to be able to address the geometrically infinite case also.

First note that any finitely generated Kleinian group, whether geometrically
finite or geometrically infinite, is \emph{tame}, that is, homeomorphic to the
interior of a compact 3-manifold. This is a deep result due to Agol and
Calegari-Gabai.

We begin with the following theorem of Canary.

thm

Let N = H3/Gamma be an infinite volume tame hyperbolic 3-manifold, and let
Gamma' < Gamma be a finitely generated subgroup. Then either

(a) N' = H3 / Gamma' is geometrically finite, or

(b) Gamma' contains a (conjugate of a) finite index subgroup of a geometrically
infinite peripheral subgroup of Gamma

where a \emph{geometrically infinite peripheral subgroup} is a subgroup
corresponding to the subsurface of a relative compact core boundary which cuts
off a geometrically infinite relative end of N. Equivalently, since all
geometrically infinite ends of tame manifolds are simply degenerate,
a geometrically infinite peripheral subgroup is the subgroup associated to the
boundary in the interior of N of a geometrically infinite end.

endthm

The above proposition in the geometrically finite case still works in the
forward direction. The reverse direction will carry through if we can show that
N' (ie Gamma') is geometrically finite. That is, we want to avoid condition (b)
in Canary's theorem above. This requires an additional condition, which we
incorporate in the following proposition. (b) holds precisely when our surface
is freely homotopic to a finite-sheeted cover of the corresponding
geometrically infinite end boundary surface. Since such an end is simply
degenerate, that is, homeomorphic to a thickened surface, we can also apply the
covering lemma (from a FUTURE SECTION WOOO) to obtain the following
proposition.

prop

Let N be a hyperbolic 3-manifold of infinite volume, possibly geometrically
infinite, and (M,P) a pared 3-manifold structure for N. Then quasifuchsian
surface subgroups of N precisely correspond to pi1-injective maps of a closed
surface S to M such that

(1) no loops in P are freely homotopic into the image of S, and

(2) S is not freely homotopic into a geometrically infinite end of M, or,
equivalently, not freely homotopic to a finite-sheeted cover of a geometrically
infinite end boundary surface.

endprop

\textbf{ TODO put some explanatory examples in here too! The Menasco-Reid
article in Topology '90 should have some examples of surface groups that are
not QF. Is it possible to do an example of uniformization? Maybe follow along
Morgan's stuff? That sounds very long. }

%%%%%%%%%%%%%%%%%%%%%%%%%%%%%%%%%%%%%%%%%%%%%%%%%
\section{Background on decomposing things topologically}
%%%%%%%%%%%%%%%%%%%%%%%%%%%%%%%%%%%%%%%%%%%%%%%%%

Using the above propositions, we've reduced our question to a purely
topological one. However, the topology of infinite volume hyperbolic
3-manifolds may be quite complex. It is not clear how to find (QF) surfaces
directly.

% TODO
\textbf{ TODO maybe here I can discuss how Baker-Cooper or Masters-Zhang did it
topologically for finite volume? Or even on how Cooper-Long-Reid solved the
topological problem without parabolics. Seems like as good a place as any...}

% TODO

\textbf{TODO finish this - Thurston decomp, acylindrical etc}

}
