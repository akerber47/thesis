%%%%%%%%%%%%%%%%%%%%%%%%%%%%%%%%%%%%%%%%%%%%%%%%%
\section{Quasi-Fuchsian surface subgroups}
%%%%%%%%%%%%%%%%%%%%%%%%%%%%%%%%%%%%%%%%%%%%%%%%%

We assume familiarity with standard terminology and results of 3-manifold
topology and Kleinian groups.  For 3-manifold topology, see Hempel, Jaco,
Schultens, Thurston, Thurston, Lackenby notes. For general hyperbolic geometry,
see Ratcliffe, Benedetti-Petronio. And for Kleinian groups, see Maskit, Marden,
Matsuzaki-Taniguchi, Thurston, Calegari notes, Kapovich.

All Kleinian groups in this paper are assumed to be finitely generated and
torsion-free. Unless stated otherwise, all hyperbolic manifolds are complete.

We first recall the following basic definitions.

defn

A \emph{surface group} is a group isomorphic to the fundamental group of
a closed surface of negative Euler characteristic. Such a group can be written
with the finite presentation

pi1 Si g = <a1,b1,dots,ag,bg | a1b1a1-1b1-1 dots agbgag-1bg-1 = 1>

where g >= 2.

enddefn

Note that surface groups are not free. The fundamental group of a compact
surface with boundary of negative Euler characteristic is Si g,b is the free
group on 2g + b - 1 generators. We wish to consider surface subgroups of
Kleinian groups which are quasi-Fuchsian.

defn

A {quasi-Fuchsian group} is a Kleinian group Gamma < PSL2C for which there
exists a quasi-conformal homeomorphism f colon S2infty to S2infty which
conjugates Gamma to a Fuchsian group, that is, a discrete subgroup of PSL2R.

enddefn

As discussed in the introduction, our general problem statement is quite
simple.

begin prob

Let Gamma be a Kleinian group of infinite covolume. Does Gamma contain
a surface subgroup which is quasi-Fuchsian?

end prob


prop

Let S be a compact surface (possibly with boundary), and phi colon pi1S to
PSL2C be a discrete faithful representation. The following are equivalent.

phi(pi1S) is quasi-Fuchsian.

phi(pi1S) is geometrically finite, and the conjugates of peripheral subgroups
of pi1S are precisely those elements of pi1S which are parabolic after applying
phi.

H3/phi(pi1S) has convex core homeomorphic to S times I.

phi(pi1S) has limit set a Jordan curve, and no element of phi(pi1S)
interchanges the complementary components.

end prop

proof

See Morgan.% Proposition 9.2

end proof

As discussed in the introduction, we wish to identify quasi-Fuchsian surface
groups --- that is, quasi-Fuchsian groups as above where S is a closed surface.
Condition (2) above tells us that such a subgroup must be geometrically finite.
Furthermore, it must not contain any parabolic elements at all.

We now make another observation. By condition (3), any quasi-Fuchsian surface
group phi(pi1S) will contain an embedded copy of the surface S inside its
convex core.  This embedding i colon S hookto H3/phi(pi1S) is an isomorphism on
pi1.  Given a Kleinian group Ga, a quasi-Fuchsian surface subgroup phi(pi1S)
< Ga corresponds to a covering map p colon H3/phi(pi1S) to H3/Ga. This map is
pi1-injective.  Projecting i(S) downward into H3/Ga, we obtain a closed surface
p(i(S)). p circ i colon S to H3/Ga is pi1-injective. In fact, (p circ
i)star(pi1S))) = phi(pi1S).

We can therefore re-state our general problem as follows.

prob

Let N be an infinite volume hyperbolic 3-manifold. That is, pi1N is a Kleinian
group of infinite covolume. Does there exist a pi1-injective map f colon S to
N, where S is a closed surface, such that fstar(pi1S) is geometrically finite
and contains no parabolic elements?

end prob

In the remainder of this paper, we'll refer to surface maps (S,f) which satisfy
the problem condition as \emph{quasi-Fuchsian surfaces} or \emph{(QF)
surfaces}. By abuse of notation, we'll often simply refer to the image f(S) as
a (QF) surface if the choice of S and f is clear.

However, even though we are now looking for a topological surface map, solving
this question as written still requires knowledge of the geometry of N. We need
to know which elements of pi1N are parabolic, and whether or not a given
subgroup is geometrically finite. We now discuss how to reduce these conditions
to a purely topological conditions on S and f.

%%%%%%%%%%%%%%%%%%%%%%%%%%%%%%%%%%%%%%%%%%%%%%%%%
\section{Pared 3-manifolds and hyperbolization}
%%%%%%%%%%%%%%%%%%%%%%%%%%%%%%%%%%%%%%%%%%%%%%%%%

Material in this section is from \cite{Mo} and \cite{CMc}. Look there for
further details.

%Definition and important facts about pared manifolds. After Morgan, The Smith
%Conjecture, V ("Uniformization Theorem for Three-Dimensional Manifolds"),
%p 58-60. Or Canary-McCullough, Homotopy Equivalences of 3-Manifolds and
%Deformation Theory of Kleinian Groups, Ch 5 p. 87-92. Also Ch 7 p.105-107.
%Note that Canary-McCullough is much more recent.
%
%(Morgan p58) (Canary-McCullough p87)
\begin{defn}

A \emph{pared 3-manifold} $(M,P)$ is a compact orientable irreducible
3-manifold $M$, together with a submanifold $P\cin\bd M$, such that the
following conditions hold:

\begin{enumerate}
\item Every component of $P$ is a torus or annulus, incompressible in $M$.

\item Every noncyclic abelian subgroup of $\pi_1M$ is peripheral with respect
to $P$ -- ie, conjugate to the fundamental group of a component of $P$.

\item $(M,P)$ is ``$A^2$-incompressible'': every $\pi_1$-injective map $(A^2,
\bd A^2) \to (M,P)$ is homotopic (as a map of pairs) to a map into $P$.

\end{enumerate}

We call $P$ the \emph{pared locus} or \emph{parabolic locus} of the pared
3-manifold $(M,P)$.

\end{defn}

Note that there are a few pared manifolds that are special cases, like with
elementary Kleinian groups. In fact, these are precisely the pared manifolds
that correspond to elementary Kleinian groups when we construct pared
3-manifolds from Kleinian groups below.

%(Canary-McCullough p88)
\begin{defn}

A pared manifold $(M,P)$ is \emph{elementary} if it is homeomorphic (as a pair)
to one of the following: $(T^2\x I,T^2\x 0)$, $(A^2\x I,A^2\x 0)$, or $(A^2\x
I,\emptyset)$, or $(S3,\emptyset)$.

\end{defn}

Pared manifolds arise naturally in the study of Kleinian groups. Intuitively,
given a geometrically finite Kleinian group Gamma, the pared 3-manifold is
a compact core, that is a compact submanifold M cin H3/Gamma with pi1M = Gamma.
The pared locus P is the union of boundary components corresponding to cusps of
H3/Gamma, that is, parabolic subgroups of Gamma (up to conjugacy). Reversing
this correspondence hyperbolizes a pared 3-manifold, giving it a geometrically
finite geometric structure where the desired boundary regions become cusps.

To be precise, given a geometrically finite Kleinian group $\Ga$, we construct
a pared 3-manifold $(M,P)=(\cM(\Ga),\cP(\Ga))$ as follows.

We know that the convex core C(Ga) cin H3/Ga has finite volume. For epsilon
sufficiently small, The thick part C(Ga)[epsilon,infty) is compact, and the
thin part C(Ga)(0,epsilon] is a union of small neighborhoods of the cusps.  As
long as epsilon is sufficiently small, the exact choice of epsilon is
arbitrary, as the resulting thick and thin parts are homeomorphic. Now let
cM(Ga) = C(Ga)[epsilon,infty), and cP(Ga) = C(Ga)[epsilon] = bd
C(Ga)[epsilon,infty) cap bd C(Ga)(0,epsilon]. That is, cP(Ga) is the boundary
locus along which we truncated to remove the thin part. We can rebuild the
convex core from cM(Ga),cP(Ga) by gluing cusp neighborhoods onto each component
of cP(Ga).

prop

(cM(Ga),cP(Ga)) is a pared 3-manifold. cM(Ga)- cP(Ga) is homeomorphic to the
convex core of Ga, C(Ga). The parabolic elements of Ga precisely correspond to
conjugates of the peripheral subgroups pi1cP(Ga) cin pi1cM(Ga).

end prop

proof

See Morgan % Corollary 6.10

end proof

We refer to (cM(Ga),cP(Ga)) as the \emph{pared 3-manifold associated to} Ga.
Similarly, if N is a geometrically finite hyperbolic 3-manifold, we refer to
(cM(pi1N),cP(pi1N)) as the \emph{pared 3-manifold associated to} N.

Conversely, there is Thurston's famous hyperbolization theorem:

thm

Let $(M,P)$ be a pared 3-manifold with $M$ Haken. Then there exists
a geometrically finite hyperbolic 3-manifold N such that cM(N),cP(N) is
homeomorphic to M,P.

end thm

proof

See Morgan or Thurston I-III.

end proof

For details and examples of pared 3-manifolds, see Canary-McCullough, Morgan,
or Thurston (I and III? Maybe just I) for details and examples.

We now return to our original problem. We can convert a given geometrically
finite Kleinian group Ga to a pared 3-manifold M,P. Intuitively, since
P corresponds to cusps, ie, parabolics, a surface subgroup of Ga avoids
parabolic elements if and only if it avoids the peripheral subgroups associated
to components of P.

First, suppose we have a geometrically finite Kleinian group.  We use the
following theorem of Thurston.

thm

Let N be a geometrically finite hyperbolic manifold with convex core C(N).
Suppose that N has infinite volume. Then every covering space N' of N with
finitely generated nonelementary fundamental group is also geometrically
finite. In fact, it suffices to assume N is geometrically finite and bd C(N) is
nonempty.

%In our case, we have a geometrically finite hyperbolic manifold of infinite
%volume, so letting E be an (infinite volume) relative end, C(N) cap E must
%yield a nonempty piece of bd C(N). This suffices to use the theorem.

endthm

proof

See \cite{Mo}. % Proposition 7.1.

endproof

% This argument follows Morgan. Proposition 7.3

Let (M,P) be the pared 3-manifold structure associated to the hyperbolic
3-manifold N. The only elements of pi1M which are parabolic are the conjugates
of elements in pi1P0, where P0 is a component of P.

Let Gamma' < pi1N be a surface subgroup which is quasifuchsian. Let N' be the
cover of N associated to Gamma'. N' and hence its convex core C(N') have the
fundamental group of a closed surface. In fact, because Gamma' is
quasifuchsian, we know that the convex core C(N') is homeomorphic to S times I,
where S is a closed surface such that pi1S = Gamma'.\cite{Mo}, same Corollary.
Map S to S times 1/2 cin times I. This map is an isomorphism on pi1. Composing
this with the covering map N' to N, we obtain a pi1-injective map phi colon
S to N.  This corresponds to a pi1-injective map to (M,P) by retracting.
Applying the theorem above, no element of Gamma'=phi(pi1S) is parabolic.
Because (M,P) is a pared structure for N, this implies that no loop in S is
freely homotopic to a loop in P.

Conversely, suppose we are given a closed pi1-injective surface
S looprightarrow M.  Suppose further that no loop in S is freely homotopic to
a loop in P. This implies that the induced subgroup pi1S < pi1M is disjoint
from all conjugates of pi1P. Because (M,P) is a pared structure for the
hyperbolic 3-manifold N, this implies that no elements of pi1S are parabolic
under the representation rho colon pi1S to PSL2C induced by pi1S < pi1M=pi1N.
Furthermore rho(pi1S) is geometrically finite (by the first theorem), and S is
a surface without cusps.  Applying the theorem, pi1S induces a quasifuchsian
surface subgroup of pi1N.

We summarize our discussion of the geometrically finite (infinite volume) case
as follows.

prop

Let N be a geometrically finite hyperbolic 3-manifold of infinite volume, and
(M,P) a pared 3-manifold structure for N. Then quasi-Fuchsian surface subgroups
of pi1N precisely correspond to pi1-injective maps of a closed surface S to
M such that no loops in P are freely homotopic to images of loops in S.

endprop

This allows us to reduce the main problem to the following:

prob

Let (M,P) be a pared 3-manifold (which has an infinite volume hyperbolization).
Does there exist a closed surface S and a pi1-injective map f colon S to M such
that no loops in P are freely homotopic to images of loops in S?

end prob

Similarly, we will refer to surfaces (S,f) which satisfy this condition for
a pared 3-manifold (M,P) as \emph{quasi-Fuchsian surfaces} or \emph{(QF)
surfaces}.  Again, by abuse of notation. We will simply refer to f(S) as (QF)
if f and S are clear from context.

We now briefly discuss the geometrically infinite case. Because the arguments
that follow in later sections are purely topological arguments applied to the
pared structure, we can always use Thurston's uniformization theorem to produce
a geometrically finite Kleinian group to which our arguments apply. However, we
would like to be able to address the geometrically infinite case also.

First note that any finitely generated Kleinian group, whether geometrically
finite or geometrically infinite, is \emph{tame}, that is, homeomorphic to the
interior of a compact 3-manifold. This is a deep result due to Agol and
Calegari-Gabai.

We begin with the following theorem of Canary.

thm

Let N = H3/Gamma be an infinite volume tame hyperbolic 3-manifold, and let
Gamma' < Gamma be a finitely generated subgroup. Then either

(a) N' = H3 / Gamma' is geometrically finite, or

(b) Gamma' contains a (conjugate of a) finite index subgroup of a geometrically
infinite peripheral subgroup of Gamma

where a \emph{geometrically infinite peripheral subgroup} is a subgroup
corresponding to the subsurface of a relative compact core boundary which cuts
off a geometrically infinite relative end of N. Equivalently, since all
geometrically infinite ends of tame manifolds are simply degenerate,
a geometrically infinite peripheral subgroup is the subgroup associated to the
boundary in the interior of N of a geometrically infinite end.

endthm

The above proposition in the geometrically finite case still works in the
forward direction. The reverse direction will carry through if we can show that
N' (ie Gamma') is geometrically finite. That is, we want to avoid condition (b)
in Canary's theorem above. This requires an additional condition, which we
incorporate in the following proposition. (b) holds precisely when our surface
is freely homotopic to a finite-sheeted cover of the corresponding
geometrically infinite end boundary surface. Since such an end is simply
degenerate, that is, homeomorphic to a thickened surface, we can also apply the
covering lemma (see \ref{L:covering}) to obtain the following proposition..

prop

Let N be a hyperbolic 3-manifold of infinite volume, possibly geometrically
infinite, and (M,P) a pared 3-manifold structure for N. Then quasifuchsian
surface subgroups of N precisely correspond to pi1-injective maps of a closed
surface S to M such that

(1) no loops in P are freely homotopic into the image of S, and

(2) S is not freely homotopic into a geometrically infinite end of M, or,
equivalently, not freely homotopic to a finite-sheeted cover of a geometrically
infinite end boundary surface.

endprop

We will not elaborate on the geometrically infinite ends case in this paper.
However, analogous constructions should provide an effective proof.

