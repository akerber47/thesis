%%%%%%%%%%%%%%%%%%%%%%%%%%%%%%%%%%%%%%%%%%%%%%%%%
\section{Our first example}
%%%%%%%%%%%%%%%%%%%%%%%%%%%%%%%%%%%%%%%%%%%%%%%%%

We construct a book of $I$-bundles $M$ as follows. In the following we take
$\Si_{1,1}$ to be the compact surface of genus 1 with a single boundary
component. Let $B_1$,$B_2$,$B_3$ be 3 trivial $I$-bundles over $\Si_{1,1}$.
$B_1=B_2=B_3=\Si_{1,1}\x I$. Each $B_i$ has a single binding boundary and two
side boundaries. Denote the side boundaries by $\bd_+B_i$ and $\bd_-B_i$. Each
side boundary component of a trivial $I$-bundle is a homeomorphic copy of the
base surface $\Si_{1,1}$.

Let $C = S^1\times D^2$ be a solid torus. Attach the $B_i$ to $C$ by gluing the
binding boundaries to parallel annuli in $\bd C$, each of which intersects
a meridian disk exactly once. The result is a book of $I$-bundles $M$. Let
$A_1$, $A_2$, and $A_3$ be the respective gluing annuli. Then $M=\cB\cup_\cA
\cC$ where $\cB=B_1 \cup B_2 \cup B_3$, $\cC=C$, and $\cA = A_1 \cup A_2 \cup
A_3$.

Now $\bd M \cap C$ is a union of 3 annuli. Let these annuli be $A_{12}'$,
$A_{23}'$, and $A_{31}'$, where each boundary annulus is labeled based on its
adjacent gluing annuli. We choose orientations and a cyclic order for the
gluing such that the 3 boundary components of M are precisely $bd_+B_1 \cup
A_{12}' \cup \bd_-B_2)$, $\bd_+B_2 \cup A_{23}' \cup \bd_-B_3$, and $\bd_+B_3
\cup A_{31}' \cup \bd_-B_1$. We'll refer to these as $\bd_{12}M$, $\bd_{23}M$,
and $\bd_{31}M$, respectively.  Each boundary component of $M$ consists of two
copies of $\Si_{1,1}$ glued along an annulus. Topologically, this forms a genus
two surface.

We know from Proposition~\ref{P:annuli} that any pared structure $P$ on $M$
consists entirely of annuli. Furthermore, we can apply the Pared Decomposition
Lemma to homotope $P$ to minimal position with respect to $\cA$. We then have
the following consequences of the pared lifting criterion. (For all of the
following, we assume $P$ is in minimal position).

\begin{prop}

Suppose $P$ has a component $P_0$ such that $P_0 \cin C$. Then $(M,P)$ does not
contain a (QF) surface.

\end{prop}

\begin{proof}

Suppose it does. Let $S$ be a (QF) surface, and homotope $S$ to minimal
position with the surface decomposition lemma. Applying the pared lifting
criterion, any annulus component $S \cap C$ will cover the binding annuli at
$C$ and allow $P_0$ to lift to a closed curve in the pared lifting pattern.
This would show $S$ is not (QF), by the pared lifting criterion. So $S$ must be
disjoint from $C$.  The pages $B_i$ have free fundamental group, so none of
them admits a $\pi_1$-injective map from a closed surface.  Therefore such an
$S$ cannot exist.

\end{proof}

\begin{prop}

Suppose $P$ has no components contained in $C$, but does have a component $P_0$
such that $P_0 \cin B_i$, for some $i$.  Without loss of generality we can let
$P_0 \cin \bd B_1$.  Then there exists a (QF) surface if and only if $P$
contains no components that are contained in $\bd_-B_2 \sqcup \bd_{23}M \sqcup
\bd_+B_3$.

\end{prop}

\begin{proof}

For the forward direction, take $S=\bd_{23}M$. Intuitively, it's easy to see
that the conditions on $P$ force all its components to lie on surfaces where
they can't be homotoped into $S$.

Algebraically, $\pi_1S$ is the subgroup generated by $a_2,b_2,a_3,b_3 \in
\pi_1M$. We know that
\[
\pi_1S = \langle a_2,b_2,a_3,b_3 \mid [a_2,b_2]=[a_3,b_3]\rangle.
\]
$P$ contains no components in $\bd_{23}M$, so an arbitrary component $P_k$ of
$P$ must be contained in $\bd_{12}M$ or $\bd_{31}M$, which have fundamental
groups generated by $a_1,b_1,a_2,b_2$ and $a_3,b_3,a_1,b_1$ respectively.
$\pi_1P_k$ is cyclic, so let $g$ be its generator. No matter if it's contained
in $\bd_{12}M$ or $\bd_{31}M$, we can see that in order for $\pi_1P_k$ to
overlap with $\pi_1S$, it must have generator some word in $a_2,b_2$ (if $P_k
in \bd_{12}M$), or some word in $a_3,b_3$ (if $P_k in \bd_{31}M$). In either
case, this is precisely the condition for such a word to correspond to a curve
contained in $\bd B_2$ or $\bd B_3$, respectively, contradicting our assumption
on $P$.

($\Longrightarrow$) Let $S$ be a surface satisfying \eqref{E:qf}. Cut $S$ into
components in $B_1,B_2,B_3$ (after homotoping to minimal intersections with the
annuli).  Applying Lemma~\ref{L:sc}, we can see that $S \cap B_1 = \emptyset$,
as otherwise since it's a finite-sheeted cover it would have to contain
a multiple of $P_0$.  So $S \cin B_2 \cup B_3 \cup C$, which is homeomorphic
to $S_2 \x I$.  Deformation retracting this to a surface $S_2$ and applying
a covering argument like in Lemma~\ref{L:sc}, we can see that $S$ is a cover of
$S_2$.  Since it's a finite-sheeted cover, it will have to have $\pi_1$
intersecting any component $P_k$ that violates the conditions stated above.
This completes the proof.

\end{proof}

The proof above obviously generalizes to more complex books of $I$-bundles.
Later we'll use it to simplify any pared structures containing annuli that fit
in a single $I$-bundle.

\begin{thm}\label{T:ex1}

Let $P$ be a pared structure containing no components $P_0$ as in the above
proposition. Then there exists a surface satisfying \eqref{E:qf}.

\end{thm}

This is the main theorem we prove for this example. A few preliminary facts are
required. Note that we can first isotope P such that it's in minimal position
on each boundary component with respect to the gluing circle.

\begin{lemma}

Under the hypotheses of the theorem, each ``$I$-bundle piece'' boundary
component $S=\bd_\pm B_i$ intersects $P$ in a thickened set of disjoint
essential arcs, each arc connecting $\bd S$ to itself. The arcs form at most
3 ``bands'', where all the arcs in each band are parallel.  Furthermore, if we
choose a representative arc from each band (yielding a set of at most
3 disjoint non-parallel arcs in $\Si_{1,1}$), there exists an automorphism of
$\Si_{1,1}$ taking these arcs to a standard set of 3 disjoint non-parallel
arcs.  See diagram for an illustration of the standard set.

\end{lemma}
\begin{proof}

We first need a preliminary definition. After cutting a surface with boundary
along arcs, we'll obtain one or more connected surfaces, each with one or more
boundary components. Each boundary component after cutting will have pieces
from the original boundary as well as pieces from the arcs that we cut along.
Given labels $\gamma_1,...,\gamma_k$ for the boundary components and
$\alpha_1,...,\alpha_l$ for the arcs (on the original surface with boundary),
we can describe each boundary component of the cut surfaces as a union of arcs,
each labeled with $\gamma_1,...,\gamma_k,\alpha_1,...,\alpha_l$. We call this
an \emph{arc pattern} for that boundary component of the new surface.

We first show that $\Si_{1,1}$ admitts at most 3 disjoint essential
non-parallel arcs, and the possible surfaces and arc patterns obtained by
cutting along these arcs are very restricted.

Label the boundary component of $\Si_{1,1}$ by $\gamma$. Consider a proper
essential arc $\alpha_1 \cin \Si_{1,1}$. Fix an orientation for $\alpha_1$.
Since $\Si_{1,1}$ is orientable, we can look at local neighborhoods of
$\alpha_1$ and see that $\alpha_1$ has a well-defined ``left side'' and ``right
side'' as we travel along it. Looking at the endpoints of $\alpha_1$ along
$\gamma$, we are forced to connect certain endpoints in the cut-up $\gamma$
with $\alpha_1$, in order to preserve the parity. This tells us the (possibly
disconnected) cut surface $S_1$ will have two boundary components. Each will
have an arc pattern consisting of two arcs, one labeled $\gamma$ and one
labeled $\alpha_1$.

Since we cut along a properly embedded arc, the Euler characteristic increases
by one. $\chi(S_1)=0$ and $S_1$ has two boundary components. By classification
of surfaces, $S_1 = \Si_{0,2}$ or $\Si_{0,1} \sqcup \Si_{1,1}$. But if $S_1$
contained a disk with the arc pattern described above, embedding that disk back
in $\Si_{1,1}$ would describe a homotopy of $\alpha_1$ into the boundary. So
$S_1 = \Si_{0,2}$.

Now suppose we had a second proper essential arc $\alpha_2 \cin \Si_{1,1}$,
disjoint from and non-parallel to $\alpha_1$. Since it's disjoint from
$\alpha_1$, $\alpha_2$ induces a proper arc in $S_1$ which connects two regions
in the arc pattern labeled $\gamma$.  $\alpha_2$ must have one endpoint on each
boundary component of $S_1$. If both are on the same side, it's either
homotopic to the boundary of $\Si_{1,1}$ or parallel to $\alpha_1$. Cutting
along $\alpha_2$ yields a new surface $S_2$. Topologically $S_2=D^2$, with arc
pattern consisting of 8 components in the cyclic order
$(\gamma,\alpha_1,\gamma,\alpha_2,\gamma,\alpha_1,\gamma,\alpha_2)$.

Finally, adding our 3rd proper essential arc $\alpha_3$, disjoint and
non-parallel to the first two arcs, a similar argument shows that $\alpha_3$
must connect opposite $\gamma$ pieces in the arc pattern. Cutting along
$\alpha_3$ yields two disks with the same arc pattern. Depending on the choice
of $\alpha_3$, the arc pattern on these disks is either
$(\gamma,\alpha_1,\gamma,\alpha_2,\gamma,\alpha_3)$ or
$(\gamma,\alpha_1,\gamma,\alpha_3,\gamma,\alpha_2)$. So up to relabeling
$\alpha_1,\alpha_2,\alpha_3$, cutting along 3 proper essential disjoint
non-parallel arcs has only one possible choice of cut surfaces and arc
patterns.

Observe that it is not possible to add any more disjoint non-parallel arcs. In
particular, any arc we draw between $\gamma$ components of the arc pattern on
either disk is homotopic to the boundary or parallel to an existing arc.
Furthermore, if we add new disjoint arcs and allow them to be parallel, we can
see that they must form ``bands'' around the existing 3 arcs in order to remain
disjoint. That is, we can homotope all the arcs parallel to a given arc into
a small neighborhood of that arc in the disk, without intersecting any of the
non-parallel arcs.

We claim there exists an automorphism of $\Si_{1,1}$ taking any set of 3 such
arcs to any other set (in particular, to the standard set, as illustrated).
Since there is only one topological result of cutting along the arcs, choose
a homeomorphism of the cut surfaces. Up to relabeling the arcs, we can choose
a homeomorphism that identifies matching arcs in the arc patterns (as shown
above, there is only one possible arc pattern up to relabeling). Glue both
sides along the arcs to obtain the desired automorphism of $\Si_{1,1}$.

We now consider $P \cap \bd_S \cin S \cong \Si_{1,1}$. As shown, $P$ is a union
of annuli. By the assumptions of the theorem, no annulus of $P$ is contained in
$S$, so each annulus intersects $S$ in a union of rectangles, where two sides
of the rectangle are embedded in the boundary $\bd S$.  Since we homotoped $P$
to have minimal intersections, all the core arcs of the rectangles (pieces of
the core curve of the annulus) must be essential. They are disjoint by
definition of $P$. The statement of the lemma follows from applying the above
argument to these core arcs.

\end{proof}

We consider connected double covers $\widetilde{S} \to S \cong \Si_{1,1}$.
These covers have 2 boundary components. Every core arc of $P \cap S$ lifts to
2 arcs in $\widetilde{S}$.  We say an arc in $S$ is \emph{cis} for a given
$\widetilde{S}$ if any (ie all) lifts of that arc have both endpoints in the
same boundary component of $\widetilde{S}$.  Otherwise, we say it's
\emph{trans} for that cover.

\begin{lemma}

Given any 3 disjoint non-parallel proper arcs in $\Si_{1,1}$, and any double
cover of $\Si_{1,1}$, 2 of the 3 arcs are trans, and the 3rd is cis.
Furthermore, we can choose any two of the three we wish to be trans with an
appropriate cover.

\end{lemma}
\begin{proof}

As in earlier lemma, there exists an automorphism of $\Si_{1,1}$ taking these
arcs to the standard set of 3 disjoint non-parallel proper arcs.  Now it
suffices to observe, by looking at relative first homology or just by
construction, that each of the three standard connected double covers
(corresponding to nontrivial maps $\mathbb{Z}^2 \to \mathbb{Z}/2$) makes two of
the three arcs trans and the third cis.

\end{proof}

\begin{proof}[Proof of Theorem~\ref{T:ex1}]

Since $P$ has no components $P_0$, every $P \cap \bd_+B_i$ is a union of
thickened arcs (that is, there are no full annulus components in any $\bd_+B_i$
or $\bd_-B_i$). Apply the first lemma to break these into bands. The problem is
most constrained when there are 3 bands, so we'll consider that case (if there
are fewer than 3, just draw some more arcs on that component arbitrarily, and
the proof still works).

We will build our surface $S$ by taking a double cover of the core $\Si_{1,1}$
surface for each $I$-bundle page $B_1,B_2,B_3$. Call these covers
$S_1,S_2,S_3$.  Each of these has two boundary components. We'll then attach
the boundary components such that each of $S_1,S_2,S_3$ has exactly one
boundary component connected to each of the others. See diagram.

Choose an arbitrary connected double cover for $S_1$. We can view this as
a cover of $\bd_+B_1$ or $\bd_-B_1$, deformation retracting either way. By the
lemma, two of the
3 bands on $\bd_+B_1$ are trans, and the other is cis. The same holds for
  $\bd_-B_1$.

We cannot choose $S_2$ arbitrarily, as a cis arc for $S_1$ in $\bd_+B_1$ may
connect (in $S$) to a cis arc for $S_2$ in $\bd_-B_2$. These together would
form a closed curve that lifts to $S$, which once $S$ is immersed in $M$ will
yield a violation of condition \eqref{E:qf}. Instead, observe that there is at
most one band of the three in $\bd_-B_2$ containing arcs that, when glued
across the core circle, match up to arcs in the cis band of $\bd_+B_1$ to form
closed curves containing only arcs in those two bands. This is because once one
band has that behavior, by endpoint parity the vertices can't also match up for
a different band, if the arcs they have to match with on the other side are
parallel. Looking at the boundary circle, non-parallel arcs have endpoints in
cyclic order, but parallel arcs don't. See diagram. It suffices to check this
for our standard set of non-parallel arcs, by the same automorphism argument.

Since there is only one such band on $\bd_-B_2$, choose $S_2$ such that this
band is not cis.  Finally, for $S_3$ we have two connecting constraints. We
have a cis band on $\bd_-B_1$ from our choice of $S_1$, and a cis band on
$\bd_+B_2$ from our choice of $S_2$.  Applying the same argument, there is at
most one band on $\bd_-B_3$ and one band on $\bd_+B_3$ that will connect to
form closed curves. But we can choose $S_3$ such that both of these bands are
trans. (This requires a slight modification to the lemma - since these bands
are on different boundary surfaces, they may not be disjoint non-parallel. If
they aren't disjoint, we can tweak them by local modifications so they are, and
then ``untweak'' them in the cover. If they are parallel, just ignore one set
of bands)

Glue $S_1,S_2,S_3$ as described to obtain $S$. We claim that $S$ satisfies
condition \eqref{E:qf}. $S$ is closed and immersed by construction (as in
Lemma~\ref{L:sc}). It is $\pi_1$-injective inside each $I$-bundle by
Lemma~\ref{L:sc}, and inside the core $C$ because it's just a union of
incompressible gluing annuli there. It suffices to show that $\pi_1S \cap
\pi_1P_k = 1$ for each component $P_k$ of the pared locus. Looking at the core
curve of the annulus $P_k$, this implies that some multiple of that core curve
is homotopic into $S$.

But this is impossibly by the above construction. Every such curve contains at
least two bands, on two different components $\bd_\pm B_i$. By the construction
of $S$, of any two bands which connect up when the boundary components are
glued across circles, at least one must be trans. This means that when we try
to homotope the core curve multiple into $S$, in order to follow along $S$
locally (in each page, where $S$ is locally a cover of the core surface, it
must be a lift from that core surface) it would have to traverse between all
three pieces of the cover, as that's how we connected up the $Si$. Every trans
arc lifts to an arc that connects two different boundary components of an $Si$,
which are ``pointed in different directions.'' But this is obviously
impossible, as our $P_k$ is restricted to be in a single boundary component, so
up to homotopy it must be generated by those two $I$-bundle pages only.

\end{proof}

