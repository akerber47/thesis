%%%%%%%%%%%%%%%%%%%%%%%%%%%%%%%%%%%%%%%%%%%%%%%%%
\section{Our first example}
%%%%%%%%%%%%%%%%%%%%%%%%%%%%%%%%%%%%%%%%%%%%%%%%%

\textbf{ TODO shorten this section. Line it up with general result.}

We construct a book of $I$-bundles $M$ as follows. In the following we take
$\Si_{1,1}$ to be the compact surface of genus 1 with a single boundary
component, a circle. Let $M_1$,$M_2$,$M_3$ be 3 punctured torus $I$-bundles,
$M_1=M_2=M_3=\Si_{1,1}\x I$. For each $M_i$, write \[ \bd M_i = \bd \Si_{1,1}\x
I \cup \Si_{1,1}\x0 \cup \Si_{1,1} \x 1 \] and label these boundary pieces as
\[ \bd M_i = A_i \cup \bd_- M_i \cup \bd_+M_i \].

Let $M_c = S^1xD^2$ be a solid torus. Attach the $M_i$ to $M_c$ by gluing the
$A_i$ to parallel annuli in $\bd M_c$, each with longitudinal core curve. The
result is a compact 3-manifold with boundary. This is our $M$. We choose
orientations and a cyclic order for the gluing such that the 3 boundary
components of M are precisely
\[
\bd M = (\bd_+M_1 \cup_{S^1} \bd_-M_2) \sqcup (\bd_+M_2 \cup_{S^1} \bd_-M_3)
\sqcup (\bd_+M_3 \cup_{S^1} \bd_-M_1)
\]
and label these
\[
\bd M=\bd_{12}M \sqcup \bd_{23}M \sqcup \bd_{31}M
\].
Each boundary
component of $M$ consists of two punctured tori glued along their boundary
circles, topologically a genus two surface.

We consider possible pared structures $P$ on $M$. Our goal is to find
a surface $S \cin M$ such that
\begin{equation}\label{E:qf}
S \text{ is closed immersed $\pi_1$-injective, and  $\pi_1P_k \cap \pi_1S
= 1$
for each component $P_k$ of $P$} \tag{QF}
\end{equation}
%%% Equation (*) (star)

% FIXME straighten this out
%Note that since we
%haven't fixed a basepoint, these subgroups are really only defined up to
%conjugacy - what we're saying is they fail to intersect for an arbitrary choice
%of conjugacy class for each subgroup. (ask Ian about this? seems fuzzy)
%

Note that since every hyperbolic 3-manifold is aspherical, by elementary
obstruction theory any injective map pi1S to pi1M is induced by a pi1-injective
map S to M.
% cite Long-Reid, p11

Furthermore, by minimal surface theory, we can guarantee that any pi1-injective
surface in a hyperbolic 3-manifold is homotopic to an immersed surface.
A surface which is not immersed will contradict minimality.
% cite Neumann p1

In what follows we'll speak only of pi1-injective surfaces, or possibly surface
subgroups. But it is important to note that in this situation those are the
same thing, and can be chosen to be immersed as well.

lemma (lifting to embeddings)

Let M be a complete hyperbolic 3-manifold, and phi: S to M be a pi1-injective
map such that H=phi*pi1S is quasiFuchsian. Then phi is homotopic to a map which
lifts to an embedding in a cover of M. Furthermore, we can choose such a cover
to be finite-sheeted.

% TODO
%This belongs in background somewhere. Reduction to finite-sheeted cover means
%this is an application of separability, so maybe put with topological facts
%there?

lemma (The Covering Lemma, as I actually use it)

Let phi: S to S' be a proper pi1-injective map of compact connected oriented
surfaces with boundary, and suppose that S is not a 2-sphere.  Then phi is
homotopic to a finite-sheeted covering map.  Note that the converse holds more
generally - that is, any finite-sheeted covering is a proper pi1-injective map.

proof

For the forward direction, we can apply minimal surface theory to perturb phi
via homotopy rel boundary to an immersion. It's straightforward to ensure that
phi is a local homeomorphism on boundary curves. Now think of int(S') as being
embedded as a totally geodesic core subsurface of a Fuchsian group quotient of
hyperbolic 3-space.  Then as we pull phi(int(S)) cin int(S')  to minimality its
image will necessarily remain in int(S'), as it's totally geodesic.  Since it
is connected and not a sphere, it can't shrink down to nothing. Since it's
properly embedded, the boundary components of S correspond to local maps of
cusps into cusp tubes of the thickened int(S') and must remain fixed on the
same cusp as we pull phi(int(S)) to a minimal surface. Hence the boundary
components of S do not move under this homotopy. The result is an immersion
produced via a homotopy rel boundary.

Then since phi is a proper immersion between manifolds of the same dimension,
it's a local homeomorphism on boundary and interior neighborhoods, and
therefore a covering map. It must be finite-sheeted because preimages of points
are discrete sets because it's a covering, and S is compact, so it follows that
point fibers are finite sets.

The converse follows from elementary covering space theory. As in the above
note, we can use minimal surface theory to make that proper pi1-injective map
into an immersion.

\begin{lemma}\label{L:sc}

Let $S$ be a surface satisfying \eqref{E:qf}. Homotope $S$ to have minimal
intersection with each $A_i$, that is, so there are no ``bumps''.  Then for
each $M_i$, each component of $S \cap M_i$ is a finite-sheeted covering of the
core surface.  That is, given such a component $S' \cin M_i$, the map $S' \to
M_i = \Si_{1,1}\x I \to \Si_{1,1}\x{1/2}$ is homotopic to a finite-sheeted
covering map. Conversely, given any finite-sheeted covering $\widetilde{S} \to
\Si_{1,1}$, there exists a corresponding proper immersed $\pi_1$-injective
surface $S \cin \Si_{1,1}\x I$.

\end{lemma}
\begin{proof}

$S$ is compact, so every such component $S'$ is a compact surface with
boundary, properly immersed in $M_i$. Since $S$ is $\pi_1$-injective in $M$,
$S'$ must be $\pi_1$-injective in $M_i$. Otherwise, we'd have a nontrivial
element of $\pi_1S'$ which is trivial in $\pi_1M_i$, hence in $\pi_1M$, hence
in $\pi_1S$, contradicting the minimal position homotopy above. So the map
$\phi : S'\to\Si_{1,1}\x{1,2}$ is also $\pi_1$-injective, since $M_i$
deformation retracts to its core. Let $H = \phi_*(\pi_1S')$, and let
$\Si_{1,1}^H$ be the cover of $\Si_{1,1}$ associated to $H<\pi_1\Si_{1,1}$.
$\phi$ lifts to $\widetilde{\phi}\colon S'\to \Si_{1,1}^H$. This is a proper
map of compact surfaces which is an isomorphism on $\pi_1$.  By the
classification of surfaces, it must be homotopic (as a proper map) to
a homeomorphism. So $\phi$ is homotopic to a covering map. It must be
finite-sheeted as $S'$ is compact (by classification of surfaces again).

Conversely, given a finite-sheeted cover $\widetilde{S}\to \Si_{1,1}$, compose
with the embedding $\Si_{1,1} = \Si_{1,1}\x{1/2} \cin \Si_{1,1}\x I$. This is
proper and $\pi_1$-injective.  Perturb locally to obtain an immersion.

\end{proof}

\begin{lemma}

Let $P$ be a pared structure on $M$. $P$ cannot contain any tori. That is,
$P$ consists entirely of annuli.

\end{lemma}
\begin{proof}

We can compute $\pi_1M$ directly from the van Kampen theorem. Let $a_i,b_i$ be
the generators of $\pi_1(M_i)$, and $c$ be the generator of $\pi_1(M_c)$. Our
gluing yields
\[
\pi_1(M) = \langle a_1,b_1,a_2,b_2,a_3,b_3,c \mid
[a_1,b_1]=[a_2,b_2]=[a_3,b_3]=c \rangle
\]
It immediately follows that $\pi_1M$ contains no rank 2 abelian subgroups, as
no two elements commute. Since every component of $P$ is $\pi_1$-injective,
they must all be annuli.

Alternatively, simply observe that all the boundary components of $M$ have
negative Euler characteristic, so they don't admit $\pi_1$-injective maps from
a torus.

\end{proof}

\begin{prop}

Suppose $P$ contains a component $P_0$ that, up to homotopy, is contained
entirely within one ``$I$-bundle half'' of a boundary component. That is, $P_0
\cin \bd M_i$, for some i. Without loss of generality we can let $P_0 \cin \bd
M_1$.  Then there exists a surface satisfying \eqref{E:qf} if and only if $P$
contains no components that are (up to homotopy) contained in $\bd M_2 \cup \bd
M_3 = \bd_-M_2 \sqcup \bd_{23}M \sqcup \bd_+M_3$.

\end{prop}
\begin{proof}

($\Longleftarrow$) Take $S=\bd_{23}M$. Intuitively, it's easy to see that the
conditions on $P$ force all its components to lie on surfaces where they can't
be homotoped into $S$.

Algebraically, $\pi_1S$ is the subgroup generated by $a_2,b_2,a_3,b_3 \in
\pi_1M$. We know that
\[
\pi_1S = \langle a_2,b_2,a_3,b_3 \mid [a_2,b_2]=[a_3,b_3]\rangle.
\]
$P$ contains no components in $\bd_{23}M$, so an arbitrary component $P_k$ of
$P$ must be contained in $\bd_{12}M$ or $\bd_{31}M$, which have fundamental
groups generated by $a_1,b_1,a_2,b_2$ and $a_3,b_3,a_1,b_1$ respectively.
$\pi_1P_k$ is cyclic, so let $g$ be its generator. No matter if it's contained
in $\bd_{12}M$ or $\bd_{31}M$, we can see that in order for $\pi_1P_k$ to
overlap with $\pi_1S$, it must have generator some word in $a_2,b_2$ (if $P_k
in \bd_{12}M$), or some word in $a_3,b_3$ (if $P_k in \bd_{31}M$). In either
case, this is precisely the condition for such a word to correspond to a curve
contained in $\bd M_2$ or $\bd M_3$, respectively, contradicting our assumption
on $P$.

($\Longrightarrow$) Let $S$ be a surface satisfying \eqref{E:qf}. Cut $S$ into
components in $M_1,M_2,M_3$ (after homotoping to minimal intersections with the
annuli).  Applying Lemma~\ref{L:sc}, we can see that $S \cap M_1 = \emptyset$,
as otherwise since it's a finite-sheeted cover it would have to contain
a multiple of $P_0$.  So $S \cin M_2 \cup M_3 \cup M_c$, which is homeomorphic
to $S_2 \x I$.  Deformation retracting this to a surface $S_2$ and applying
a covering argument like in Lemma~\ref{L:sc}, we can see that $S$ is a cover of
$S_2$.  Since it's a finite-sheeted cover, it will have to have $\pi_1$
intersecting any component $P_k$ that violates the conditions stated above.
This completes the proof.

\end{proof}

The proof above obviously generalizes to more complex books of $I$-bundles.
Later we'll use it to simplify any pared structures containing annuli that fit
in a single $I$-bundle.

\begin{thm}\label{T:ex1}

Let $P$ be a pared structure containing no components $P_0$ as in the above
proposition. Then there exists a surface satisfying \eqref{E:qf}.

\end{thm}

This is the main theorem we prove for this example. A few preliminary facts are
required. Note that we can first isotope P such that it's in minimal position
on each boundary component with respect to the gluing circle.

\begin{lemma}

Under the hypotheses of the theorem, each ``$I$-bundle piece'' boundary
component $S=\bd_\pm M_i$ intersects $P$ in a thickened set of disjoint
essential arcs, each arc connecting $\bd S$ to itself. The arcs form at most
3 ``bands'', where all the arcs in each band are parallel.  Furthermore, if we
choose a representative arc from each band (yielding a set of at most
3 disjoint non-parallel arcs in $\Si_{1,1}$), there exists an automorphism of
$\Si_{1,1}$ taking these arcs to a standard set of 3 disjoint non-parallel
arcs.  See diagram for an illustration of the standard set.

\end{lemma}
\begin{proof}

We first need a preliminary definition. After cutting a surface with boundary
along arcs, we'll obtain one or more connected surfaces, each with one or more
boundary components. Each boundary component after cutting will have pieces
from the original boundary as well as pieces from the arcs that we cut along.
Given labels $\gamma_1,...,\gamma_k$ for the boundary components and
$\alpha_1,...,\alpha_l$ for the arcs (on the original surface with boundary),
we can describe each boundary component of the cut surfaces as a union of arcs,
each labeled with $\gamma_1,...,\gamma_k,\alpha_1,...,\alpha_l$. We call this
an \emph{arc pattern} for that boundary component of the new surface.

We first show that $\Si_{1,1}$ admitts at most 3 disjoint essential
non-parallel arcs, and the possible surfaces and arc patterns obtained by
cutting along these arcs are very restricted.

Label the boundary component of $\Si_{1,1}$ by $\gamma$. Consider a proper
essential arc $\alpha_1 \cin \Si_{1,1}$. Fix an orientation for $\alpha_1$.
Since $\Si_{1,1}$ is orientable, we can look at local neighborhoods of
$\alpha_1$ and see that $\alpha_1$ has a well-defined ``left side'' and ``right
side'' as we travel along it. Looking at the endpoints of $\alpha_1$ along
$\gamma$, we are forced to connect certain endpoints in the cut-up $\gamma$
with $\alpha_1$, in order to preserve the parity. This tells us the (possibly
disconnected) cut surface $S_1$ will have two boundary components. Each will
have an arc pattern consisting of two arcs, one labeled $\gamma$ and one
labeled $\alpha_1$.

Since we cut along a properly embedded arc, the Euler characteristic increases
by one. $\chi(S_1)=0$ and $S_1$ has two boundary components. By classification
of surfaces, $S_1 = \Si_{0,2}$ or $\Si_{0,1} \sqcup \Si_{1,1}$. But if $S_1$
contained a disk with the arc pattern described above, embedding that disk back
in $\Si_{1,1}$ would describe a homotopy of $\alpha_1$ into the boundary. So
$S_1 = \Si_{0,2}$.

Now suppose we had a second proper essential arc $\alpha_2 \cin \Si_{1,1}$,
disjoint from and non-parallel to $\alpha_1$. Since it's disjoint from
$\alpha_1$, $\alpha_2$ induces a proper arc in $S_1$ which connects two regions
in the arc pattern labeled $\gamma$.  $\alpha_2$ must have one endpoint on each
boundary component of $S_1$. If both are on the same side, it's either
homotopic to the boundary of $\Si_{1,1}$ or parallel to $\alpha_1$. Cutting
along $\alpha_2$ yields a new surface $S_2$. Topologically $S_2=D^2$, with arc
pattern consisting of 8 components in the cyclic order
$(\gamma,\alpha_1,\gamma,\alpha_2,\gamma,\alpha_1,\gamma,\alpha_2)$.

Finally, adding our 3rd proper essential arc $\alpha_3$, disjoint and
non-parallel to the first two arcs, a similar argument shows that $\alpha_3$
must connect opposite $\gamma$ pieces in the arc pattern. Cutting along
$\alpha_3$ yields two disks with the same arc pattern. Depending on the choice
of $\alpha_3$, the arc pattern on these disks is either
$(\gamma,\alpha_1,\gamma,\alpha_2,\gamma,\alpha_3)$ or
$(\gamma,\alpha_1,\gamma,\alpha_3,\gamma,\alpha_2)$. So up to relabeling
$\alpha_1,\alpha_2,\alpha_3$, cutting along 3 proper essential disjoint
non-parallel arcs has only one possible choice of cut surfaces and arc
patterns.

Observe that it is not possible to add any more disjoint non-parallel arcs. In
particular, any arc we draw between $\gamma$ components of the arc pattern on
either disk is homotopic to the boundary or parallel to an existing arc.
Furthermore, if we add new disjoint arcs and allow them to be parallel, we can
see that they must form ``bands'' around the existing 3 arcs in order to remain
disjoint. That is, we can homotope all the arcs parallel to a given arc into
a small neighborhood of that arc in the disk, without intersecting any of the
non-parallel arcs.

We claim there exists an automorphism of $\Si_{1,1}$ taking any set of 3 such
arcs to any other set (in particular, to the standard set, as illustrated).
Since there is only one topological result of cutting along the arcs, choose
a homeomorphism of the cut surfaces. Up to relabeling the arcs, we can choose
a homeomorphism that identifies matching arcs in the arc patterns (as shown
above, there is only one possible arc pattern up to relabeling). Glue both
sides along the arcs to obtain the desired automorphism of $\Si_{1,1}$.

We now consider $P \cap \bd_S \cin S \cong \Si_{1,1}$. As shown, $P$ is a union
of annuli. By the assumptions of the theorem, no annulus of $P$ is contained in
$S$, so each annulus intersects $S$ in a union of rectangles, where two sides
of the rectangle are embedded in the boundary $\bd S$.  Since we homotoped $P$
to have minimal intersections, all the core arcs of the rectangles (pieces of
the core curve of the annulus) must be essential. They are disjoint by
definition of $P$. The statement of the lemma follows from applying the above
argument to these core arcs.

\end{proof}

We consider connected double covers $\widetilde{S} \to S \cong \Si_{1,1}$.
These covers have 2 boundary components. Every core arc of $P \cap S$ lifts to
2 arcs in $\widetilde{S}$.  We say an arc in $S$ is \emph{cis} for a given
$\widetilde{S}$ if any (ie all) lifts of that arc have both endpoints in the
same boundary component of $\widetilde{S}$.  Otherwise, we say it's
\emph{trans} for that cover.

\begin{lemma}

Given any 3 disjoint non-parallel proper arcs in $\Si_{1,1}$, and any double
cover of $\Si_{1,1}$, 2 of the 3 arcs are trans, and the 3rd is cis.
Furthermore, we can choose any two of the three we wish to be trans with an
appropriate cover.

\end{lemma}
\begin{proof}

As in earlier lemma, there exists an automorphism of $\Si_{1,1}$ taking these
arcs to the standard set of 3 disjoint non-parallel proper arcs.  Now it
suffices to observe, by looking at relative first homology or just by
construction, that each of the three standard connected double covers
(corresponding to nontrivial maps $\mathbb{Z}^2 \to \mathbb{Z}/2$) makes two of
the three arcs trans and the third cis.

\end{proof}

\begin{proof}[Proof of Theorem~\ref{T:ex1}]

Since $P$ has no components $P_0$, every $P \cap \bd_+M_i$ is a union of
thickened arcs (that is, there are no full annulus components in any $\bd_+M_i$
or $\bd_-M_i$). Apply the first lemma to break these into bands. The problem is
most constrained when there are 3 bands, so we'll consider that case (if there
are fewer than 3, just draw some more arcs on that component arbitrarily, and
the proof still works).

We will build our surface $S$ by taking a double cover of the core $\Si_{1,1}$
surface for each $I$-bundle page $M_1,M_2,M_3$. Call these covers
$S_1,S_2,S_3$.  Each of these has two boundary components. We'll then attach
the boundary components such that each of $S_1,S_2,S_3$ has exactly one
boundary component connected to each of the others. See diagram.

Choose an arbitrary connected double cover for $S_1$. We can view this as
a cover of $\bd_+M_1$ or $\bd_-M_1$, deformation retracting either way. By the
lemma, two of the
3 bands on $\bd_+M_1$ are trans, and the other is cis. The same holds for
  $\bd_-M_1$.

We cannot choose $S_2$ arbitrarily, as a cis arc for $S_1$ in $\bd_+M_1$ may
connect (in $S$) to a cis arc for $S_2$ in $\bd_-M_2$. These together would
form a closed curve that lifts to $S$, which once $S$ is immersed in $M$ will
yield a violation of condition \eqref{E:qf}. Instead, observe that there is at
most one band of the three in $\bd_-M_2$ containing arcs that, when glued
across the core circle, match up to arcs in the cis band of $\bd_+M_1$ to form
closed curves containing only arcs in those two bands. This is because once one
band has that behavior, by endpoint parity the vertices can't also match up for
a different band, if the arcs they have to match with on the other side are
parallel. Looking at the boundary circle, non-parallel arcs have endpoints in
cyclic order, but parallel arcs don't. See diagram. It suffices to check this
for our standard set of non-parallel arcs, by the same automorphism argument.

Since there is only one such band on $\bd_-M_2$, choose $S_2$ such that this
band is not cis.  Finally, for $S_3$ we have two connecting constraints. We
have a cis band on $\bd_-M_1$ from our choice of $S_1$, and a cis band on
$\bd_+M_2$ from our choice of $S_2$.  Applying the same argument, there is at
most one band on $\bd_-M_3$ and one band on $\bd_+M_3$ that will connect to
form closed curves. But we can choose $S_3$ such that both of these bands are
trans. (This requires a slight modification to the lemma - since these bands
are on different boundary surfaces, they may not be disjoint non-parallel. If
they aren't disjoint, we can tweak them by local modifications so they are, and
then ``untweak'' them in the cover. If they are parallel, just ignore one set
of bands)

Glue $S_1,S_2,S_3$ as described to obtain $S$. We claim that $S$ satisfies
condition \eqref{E:qf}. $S$ is closed and immersed by construction (as in
Lemma~\ref{L:sc}). It is $\pi_1$-injective inside each $I$-bundle by
Lemma~\ref{L:sc}, and inside the core $M_c$ because it's just a union of
incompressible gluing annuli there. It suffices to show that $\pi_1S \cap
\pi_1P_k = 1$ for each component $P_k$ of the pared locus. Looking at the core
curve of the annulus $P_k$, this implies that some multiple of that core curve
is homotopic into $S$.

But this is impossibly by the above construction. Every such curve contains at
least two bands, on two different components $\bd_\pm M_i$. By the construction
of $S$, of any two bands which connect up when the boundary components are
glued across circles, at least one must be trans. This means that when we try
to homotope the core curve multiple into $S$, in order to follow along $S$
locally (in each page, where $S$ is locally a cover of the core surface, it
must be a lift from that core surface) it would have to traverse between all
three pieces of the cover, as that's how we connected up the $Si$. Every trans
arc lifts to an arc that connects two different boundary components of an $Si$,
which are ``pointed in different directions.'' But this is obviously
impossible, as our $P_k$ is restricted to be in a single boundary component, so
up to homotopy it must be generated by those two $I$-bundle pages only.

\end{proof}

