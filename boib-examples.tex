%%%%%%%%%%%%%%%%%%%%%%%%%%%%%%%%%%%%%%%%%%%%%%%%%
\section{Examples}
%%%%%%%%%%%%%%%%%%%%%%%%%%%%%%%%%%%%%%%%%%%%%%%%%

We construct a book of $I$-bundles $M$ as follows. In the following we take
$\Si_{1,1}$ to be the compact surface of genus 1 with a single boundary
component. Let $B_1$,$B_2$,$B_3$ be 3 trivial $I$-bundles over $\Si_{1,1}$.
$B_1=B_2=B_3=\Si_{1,1}\x I$. Each $B_i$ has a single binding boundary and two
side boundaries. Denote the side boundaries by $\bd_+B_i$ and $\bd_-B_i$.
Recall that each side boundary component of a trivial $I$-bundle is
a homeomorphic copy of the base surface $\Si_{1,1}$.

Let $C = S^1\times D^2$ be a solid torus. Attach the $B_i$ to $C$ by gluing the
binding boundaries to parallel annuli in $\bd C$, each of which intersects
a meridian disk exactly once. The result is a book of $I$-bundles $M$. Let
$A_1$, $A_2$, and $A_3$ be the respective gluing annuli. Then $M=\cB\cup_\cA
\cC$ where $\cB=B_1 \sqcup B_2 \sqcup B_3$, $\cC=C$, and $\cA = A_1 \sqcup A_2
\sqcup A_3$.

Now $\bd M \cap C$ is a union of 3 annuli. Let these annuli be $A_{12}'$,
$A_{23}'$, and $A_{31}'$, where each boundary annulus is labeled based on its
adjacent gluing annuli. We choose orientations and a cyclic order for the
gluing such that the 3 boundary components of M are precisely $\bd_+B_1 \cup
A_{12}' \cup \bd_-B_2$, $\bd_+B_2 \cup A_{23}' \cup \bd_-B_3$, and $\bd_+B_3
\cup A_{31}' \cup \bd_-B_1$. We'll refer to these as $\bd_{12}M$, $\bd_{23}M$,
and $\bd_{31}M$, respectively.  Each boundary component of $M$ consists of two
copies of $\Si_{1,1}$ glued along an annulus. Topologically, this forms a genus
two surface.

We know from Proposition~\ref{P:annuli} that any pared structure $P$ on $M$
consists entirely of annuli. Furthermore, we can apply the Pared Decomposition
Lemma to homotope $P$ to minimal position with respect to $\cA$. We then have
the following consequences of the pared lifting criterion. (For all of the
following, we assume $P$ is in minimal position).

\begin{prop}

Suppose $P$ has a component $P_0$ such that $P_0 \cin C$. Then $(M,P)$ does not
contain a (QF) surface.

\end{prop}

\begin{proof}

Suppose it does. Let $S$ be a (QF) surface, and homotope $S$ to minimal
position with the surface decomposition lemma. Applying the pared lifting
criterion, any annulus component $S \cap C$ will cover the binding annuli at
$C$ and allow $P_0$ to lift to a closed curve in the pared lifting pattern.
This would show $S$ is not (QF), by the pared lifting criterion. So $S$ must be
disjoint from $C$.  The pages $B_i$ have free fundamental group, so none of
them admits a $\pi_1$-injective map from a closed surface.  Therefore such an
$S$ cannot exist.

\end{proof}

\begin{prop}

Suppose $P$ has no components contained in $C$, but does have a component $P_0$
such that $P_0 \cin B_i$, for some $i$.  Without loss of generality we can let
$P_0 \cin \bd B_1$.  Then $(M,P)$ contains a (QF) surface if and only if $P$
has no components that are contained in $\bd_-B_2 \sqcup \bd_{23}M \sqcup
\bd_+B_3$. Furthermore, in this case the only (QF) surfaces are covers of
$\bd_{23}M$.

\end{prop}

\begin{proof}

For the forward direction, first place $S$ in minimal position. We claim that
$S\cap B_1=\emptyset$. Suppose not. Applying the pared lifting criterion to
$S\cap B_1$, the annulus $P_0\cin P \cap B_1$ will yield a closed curve in
$S\cap B_1$, which is a finite-sheeted cover of the core curve downstairs. Thus
the surface $S$ won't be (QF) in this case.

So $S$ cannot intersect $B_1$. Using the surface decomposition theorem, we can
see that the components of $S \cap B_2$ and $S \cap B_3$ are covers of the base
surfaces copies of $\Si_{1,1}$, and the components of $S \cap C$ are annuli.
Consider the annulus pieces in $S \cap C$.  Since $S$ has no decomposition
pieces in $B_1$, every annulus in $S \cap C$ must have one boundary component
in $A_2$ and one boundary component in $A_3$ (otherwise it connects to $A_1$,
which is impossible as $S$ is closed). Notice that these annuli locally cover
$\bd_{23} M \cap C = A_{23}'$, and are attached at their boundaries to surfaces
covering $\bd_{23} M \cap B_2$ and $\bd_{23} M \cap B_3$ (since these are both
homeomorphic to the respective base surfaces by projecting down). Hence $S$
covers $\bd_{23} M$.

It follows that $P$ cannot contain any components in $\bd- B_2$, $\bd+ B_3$, or
$\bd_{23} M$.  Any such component corresponds to a closed curve in the pared
lifting pattern for $\bd_{23} M \cin M$, which therefore lifts to a closed
curve in the pared lifting pattern for $S$.

Conversely, take $S=\bd_{23}M$.  $S$ meets $C$ and two of the spines, $B_2$ and
$B_3$.  By assumption, none of these intersects $P$ (in minimal position).
Therefore, the pared lifting pattern is empty, and $S$ is (QF).

\end{proof}

The proof above obviously generalizes to more complex books of $I$-bundles.
Later we'll use it to generally simplify any pared structures containing annuli
that fit in a single $I$-bundle.

The hardest case to consider (and the one we will devote much of this paper to
generalizing) is when none of these simplifying assumptions hold:

\begin{thm}\label{T:ex1}

Suppose $P$ has no components contained in $C$, or in any of the $B_i$. Then
$(M,P)$ contains a (QF) surface.

\end{thm}

Instead of using the full power of a general theorem, we'll solve this example
explicitly. This will give an idea of the flavor of the general construction.
Choose an arbitrary page boundary component $F=\bd_\pm B_i \cong \Si_{1,1}$.
Since $P$ is in minimal position, $F \cap P$ is a thickened set of disjoint
proper essential arcs in $F$ (there are no closed curves or non-essential
arcs).  We now have the following elementary fact about curves on surfaces.
The proof provided is new, but uses standard techniques. See \cite{FM} or
\cite{FLP}.

\begin{lemma}\label{L:ex1.1}

These arcs form at most 3 ``bands'' of parallel arcs. Furthermore, if we choose
a representative arc from each band , there exists an automorphism of
$\Si_{1,1}$ taking these arcs to a standard set of 3 disjoint non-parallel
arcs. We can construct the standard set by drawing closed curves of slope 0, 1,
and infinity through a point on the torus, and then deleting a small open
neighborhood of that point.

\end{lemma}
\begin{proof}

We first need a preliminary definition. After cutting a surface with boundary
along arcs, we'll obtain one or more connected surfaces, each with one or more
boundary components. Each boundary component after cutting will have pieces
from the original boundary as well as pieces from the arcs that we cut along.
Given labels $\gamma_1,...,\gamma_k$ for the boundary components and
$\alpha_1,...,\alpha_l$ for the arcs (on the original surface with boundary),
we can describe each boundary component of the cut surfaces as a union of arcs,
each labeled with $\gamma_1,...,\gamma_k,\alpha_1,...,\alpha_l$. We call this
an \emph{arc pattern} for that boundary component of the new surface.

We first show that $\Si_{1,1}$ admitts at most 3 disjoint essential
non-parallel arcs, and the possible surfaces and arc patterns obtained by
cutting along these arcs are very restricted.

Label the boundary component of $\Si_{1,1}$ by $\gamma$. Consider a proper
essential arc $\alpha_1 \cin \Si_{1,1}$. Fix an orientation for $\alpha_1$.
Since $\Si_{1,1}$ is orientable, we can look at local neighborhoods of
$\alpha_1$ and see that $\alpha_1$ has a well-defined ``left side'' and ``right
side'' as we travel along it. Looking at the endpoints of $\alpha_1$ along
$\gamma$, we are forced to connect certain endpoints in the cut-up $\gamma$
with $\alpha_1$, in order to preserve the parity. This tells us the (possibly
disconnected) cut surface $S_1$ will have two boundary components. Each will
have an arc pattern consisting of two arcs, one labeled $\gamma$ and one
labeled $\alpha_1$.

Since we cut along a properly embedded arc, the Euler characteristic increases
by one. $\chi(S_1)=0$ and $S_1$ has two boundary components. By classification
of surfaces, $S_1 = \Si_{0,2}$ or $\Si_{0,1} \sqcup \Si_{1,1}$. But if $S_1$
contained a disk with the arc pattern described above, embedding that disk back
in $\Si_{1,1}$ would describe a homotopy of $\alpha_1$ into the boundary. So
$S_1 = \Si_{0,2}$.

Now suppose we had a second proper essential arc $\alpha_2 \cin \Si_{1,1}$,
disjoint from and non-parallel to $\alpha_1$. Since it's disjoint from
$\alpha_1$, $\alpha_2$ induces a proper arc in $S_1$ which connects two regions
in the arc pattern labeled $\gamma$.  $\alpha_2$ must have one endpoint on each
boundary component of $S_1$. If both are on the same side, it's either
homotopic to the boundary of $\Si_{1,1}$ or parallel to $\alpha_1$. Cutting
along $\alpha_2$ yields a new surface $S_2$. Topologically $S_2=D^2$, with arc
pattern consisting of 8 components in the cyclic order
$(\gamma,\alpha_1,\gamma,\alpha_2,\gamma,\alpha_1,\gamma,\alpha_2)$.

Finally, adding our 3rd proper essential arc $\alpha_3$, disjoint and
non-parallel to the first two arcs, a similar argument shows that $\alpha_3$
must connect opposite $\gamma$ pieces in the arc pattern. Cutting along
$\alpha_3$ yields two disks with the same arc pattern. Depending on the choice
of $\alpha_3$, the arc pattern on these disks is either
$(\gamma,\alpha_1,\gamma,\alpha_2,\gamma,\alpha_3)$ or
$(\gamma,\alpha_1,\gamma,\alpha_3,\gamma,\alpha_2)$. So up to relabeling
$\alpha_1,\alpha_2,\alpha_3$, cutting along 3 proper essential disjoint
non-parallel arcs has only one possible choice of cut surfaces and arc
patterns.

Observe that it is not possible to add any more disjoint non-parallel arcs. In
particular, any arc we draw between $\gamma$ components of the arc pattern on
either disk is homotopic to the boundary or parallel to an existing arc.
Furthermore, if we add new disjoint arcs and allow them to be parallel, we can
see that they must form ``bands'' around the existing 3 arcs in order to remain
disjoint. That is, we can homotope all the arcs parallel to a given arc into
a small neighborhood of that arc in the disk, without intersecting any of the
non-parallel arcs.

We claim there exists an automorphism of $\Si_{1,1}$ taking any set of 3 such
arcs to any other set (in particular, to the standard set, as illustrated).
Since there is only one topological result of cutting along the arcs, choose
a homeomorphism of the cut surfaces. Up to relabeling the arcs, we can choose
a homeomorphism that identifies matching arcs in the arc patterns (as shown
above, there is only one possible arc pattern up to relabeling). Glue both
sides along the arcs to obtain the desired automorphism of $\Si_{1,1}$.

\end{proof}

We consider connected double covers $\widetilde{F} \to F \cong \Si_{1,1}$.
These covers have 2 boundary components. Any proper arc in $P \cap F$ lifts to
2 arcs in $\widetilde{F}$.  We say an arc in $F$ is \emph{cis} for a given
$\widetilde{F}$ if any (ie all) lifts of that arc have both endpoints in the
same boundary component of $\widetilde{F}$.  Otherwise, we say it's
\emph{trans} for that cover.

\begin{lemma}\label{L:ex1.2}

Given any 3 disjoint non-parallel proper essential arcs in $\Si_{1,1}$, and any
double cover of $\Si_{1,1}$, 2 of the 3 arcs are trans, and the 3rd is cis.
Furthermore, we can choose any two of the three we wish to be trans with an
appropriate cover.

\end{lemma}
\begin{proof}

By Lemma~\ref{L:ex1.1}, there exists an automorphism of $\Si_{1,1}$ taking
these arcs to the standard set of 3 disjoint non-parallel proper arcs. Now it
suffices to observe, by looking at relative first homology or just by direct
construction, that each of the three standard connected double covers
(corresponding to nontrivial maps $\mathbb{Z}^2 \to \mathbb{Z}/2$) makes two of
the three arcs trans and the third cis.

\end{proof}

\begin{proof}[Proof of Theorem~\ref{T:ex1}]

We begin with $P$ in minimal position. By assumption, each $P \cap \bd_\pm B_i$
is a union of thickened arcs (that is, there are no annulus components in any
$\bd_\pm B_i$). By Lemma~\ref{L:ex1.1}, these can be grouped into bands. The
problem is most constrained when there are exactly 3 bands, so we'll consider
that case (if there are fewer than 3, just draw some more arcs on that
component arbitrarily, and carry out the proof).

We build our surface $S$ as follows. Begin by taking copies of  $A_{12}'$,
$A_{23}'$, and $A_{31}'$ in $C$. Isotope each as a properly embedded
submanifold (in $(M,\cA)$) to bring the images into the interior of $C$ and
make the boundary components on $\cA$ disjoint. We'll refer to the homeomorphic
copies we obtain from this as $F_{12}'$, $F_{23}'$, and $F_{31}'$. Let $S\cap
C = F_{12}' \cup F_{23}' \cup F_{31}'$. Observe that $S\cap C$ intersects each
binding annulus $A_i$ in two parallel embedded curves.

Choose each $S \cap B_i$ to be a closed immersed $\pi_1$-injective surface
corresponding (via the Covering Lemma) to a connected double cover of the base
surface (we describe the exact choice of cover below). Call these covers $F_1$,
$F_2$, and $F_3$.  Each of these has two boundary components. Construct $S$ by
attaching the two boundary components of each $F_i$ to the two curves in the
binding annulus $A_i$ described above. (Note that since every double cover we
are considering is regular, it doesn't matter which boundary component of the
two we connect to which curve). By the surface decomposition theorem and the
covering lemma, $S$ is a closed immersed $\pi_1$-injective surface in $M$.  We
claim that we can choose the $F_i$ so that $S$ satisfies the pared lifting
criterion.

First observe that $S \cap C$ consists of a homeomorphic copies $F'_{12}$,
$F'_{23}$, and $F'_{31}$ of each of $A'_{12}$, $A'_{23}$, and $A'_{31}$. By
assumption, there are no annuli in $S \cap C$, so the pared lifting pattern on
each $F'_{ij}$ is simply a homeomorphic copy of (the core of) $P \cap A'_{ij}$.
This is because each is topologically a homeomorphic copy or 1-fold cover of
the appropriate piece of the boundary, so the ``lifting'' is actually trivial.
Each of these pared lifting patterns is a union of disjoint transverse arcs.

Choose an arbitrary connected double cover for $F_1$. By definition, the pared
lifting pattern on $F_1$ consists of lifts of (thickened) arcs on $\bd_+ B_1$
and $\bd_- B_1$. The arcs on each form at most 3 bands, by Lemma~\ref{L:ex1.1}.
$F_1$ is a double cover of each.  By Lemma~\ref{L:ex1.2}, two of the bands on
$\bd_+B_1$ are trans, and the other is cis. The same holds for $\bd_-B_1$.

We cannot choose $F_2$ arbitrarily and necessarily satisfy the pared lifting
criterion. We may accidentally connect a cis arc in $F_1$ to two arcs in
$F'_{12}$ which connect to a cis arc in $F_2$ in $\bd_-B_2$. This would yield
a closed curve in the pared lifting pattern. But by definition of the pared
lifting pattern, in order for this to occur, the cis arc in $F_1$ would have to
be lifted from $\bd_+ B_1$, and the cis arc in $F_2$ would have to be lifted
from $\bd_- B_2$. This is because arcs that connect in the pared lifting
pattern must come from pieces of the same component of $P$, and any component
of $P$ with pared lifting pieces in $F_1$ and $F_2$ has to lie on $\bd_{12}
M = \bd_+ B_1 \cup A'_{12} \cup \bd_- B_2$.

Instead, observe that there is at most one band of the three in $\bd_-B_2$
containing arcs that, when lifted to $F_2$ and glued across $F'_{12}$, match up
to arcs in the cis band of $\bd_+B_1$. This is because once one band has this
matching behavior, by endpoint parity the vertices can't also match up for
a different band. The arcs they have to match with on the other side are
parallel. Looking at the boundary circle of $\Si_{1,1}$, non-parallel arcs have
endpoints in cyclic order, but parallel arcs have endpoints adjacent.

So, since there is only one such band on $\bd_-B_2$, we can choose $F_2$ such
that this band is not cis.  Finally, for $F_3$ we have two such constraints.
We have a cis band on $\bd_-B_1$ from our choice of $F_1$, and a cis band on
$\bd_+B_2$ from our choice of $F_2$.  Applying the same argument, there is at
most one band on $\bd_-B_3$ and one band on $\bd_+B_3$ that may connect across
$F'_{23}$ or $F'_{31}$, respectively, to form closed curves. Choose $S_3$ such
that both of these bands are trans. Note that this requires a slight
modification to Lemma~\ref{L:ex1.2} - since these bands are on different
boundary surfaces, they may be parallel or non-disjoint.  If they aren't
disjoint, we can tweak them by local modifications so they are, and then
untweak them in the cover. If they are parallel, just ignore one set of bands.

Glue the $F_i$ and $F'_{ij}$ as described to obtain $S$. We claim that $S$
satisfies the pared lifting criterion. Suppose not, and let $\alpha$ be
a closed curve in the pared lifting pattern.  By assumption, $\alpha$ must
intersect at least two pages of $M$. Choose two such pieces of $\alpha$ that
are adjacent, that is, $\alpha_i\cin F_i$, $\alpha_j \cin F_j$ that are
directly joined in the pared lifting pattern by a single arc $\alpha_{ij} \cin
F'_{ij}$.  By the construction of $S$, at least one of $\alpha_i$, $\alpha_j$
is trans. We guaranteed above that two cis arcs were never directly connected.

Without loss of generality, let $\alpha_i$ be trans. This means that the other
endpoint of $\alpha_i$ (the one not connected to $\alpha_{ij}$) lies in the
other boundary component of $F_i$. By construction of $S$, this boundary
component connects to a different annulus $F'_{ik}$ in $S\cap C$. This annulus
cannot contain any lifts of components of $\alpha$, as its two boundary
components are parallel to a different boundary annulus $F_{ik}$ of $C\cin M$.
This means that $\alpha$ cannot continue past this gluing in the pared lifting
pattern, and therefore cannot be a closed curve. Hence $S$ satisfies the pared
lifting criterion, and is (QF).

\end{proof}

We briefly summarize the salient features of this proof which we will
generalize.  We begin by describing which cases would be simplified or
eliminated by certain closed loops, and solving those. Then, for a more general
pared structure, we construct covers of the page base surfaces which behave
nicely with respect to arcs in the pared lifting pattern.  These properties
will become more elaborate in the general case, but are analogous --- we want
arcs to connect different boundary components as the trans arcs do.  We then
derive a method for gluing that will avoid creating closed loops, by taking
advantage of the ability to branch between different annuli at each spine. This
is the trickiest part of the general proof.
