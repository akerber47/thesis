%%%%%%%%%%%%%%%%%%%%%%%%%%%%%%%%%%%%%%%%%%%%%%%%%
\section{The case of no quasi-Fuchsian surface}
%%%%%%%%%%%%%%%%%%%%%%%%%%%%%%%%%%%%%%%%%%%%%%%%%

We prove a generalization of the above example. Basically, we want to first
identify the cases that obviously can't have a surface that is (QF). Then we'll
prove by construction that every other book of $I$-bundles does in fact contain
such a surface.

We introduce an additional definition.

\begin{defn}

Let $(M,P)$ be a pared book of $I$-bundles. We say $(M,P)$ is \emph{reduced} if
it satisfies the following.

\begin{enumerate}

\item A meridian disk of each spine intersects the union of all gluing annuli
at least three times.

\item All page boundary components are glued (there are no ``free boundary
components'').

\item Each component of $P$ (once placed in minimal position with respect to
the gluing annuli) intersects at least one page $B$. Furthermore, if it
intersects exactly one page, it traverses at least one spine $C$ attached to
that page. In this case, either $C$ is of valence one, or it's of valence
greater than one and $B$ is attached to $C$ along more than one boundary
component.  If valence one, we require that $B$ is attached to $C$ along
a gluing of degree at least 3.

Note that any spines of valence one and degree one are eliminated below.  Also
note that since it lies on a single boundary component of $M$, a parabolic
cannot enter and exit the same page gluing annulus at a spine unless that spine
has valence one. Therefore, we can divide the cases with a parabolic that
intersects a spine but only hits a single page into valence one spines, and
multiple attached gluing annuli from same page. The above criterion restricts
the possible valence one spine cases to those which have degree at least 3.

\end{enumerate}

\end{defn}

This definition is motivated by the following important idea. If $(M,P)$ is not
reduced, we can attempt to reduce it inductively. At each step, we remove pages
or spines from our pared books of $I$-bundles to produce new pared books of
$I$-bundles.  Because we're deleting pieces, we may end up with multiple
connected components. Each component is either a pared book of $I$-bundles or
an isolated page with no remaining boundary components - that is, a thickened
closed surface.

\begin{defn}

We \emph{reduce} a book of $I$-bundles as follows. We can repeat this
inductively on connected components of the result of a reduction.

\begin{enumerate}

\item[(A)] If $(M,P)$ has a spine $C$ which violates (1):

\begin{enumerate}

\item[(A1)] If $C$ has 0 gluing annuli, delete $C$.

\item[(A2)] If $C$ has 1 gluing annulus, note that that annulus may intersect
a meridian disk once or twice. If it intersects once, delete $C$. Note that the
attached page will soon be deleted by criterion (2).

\item[(A3)] If it has 1 gluing annulus which intersects twice, the book of
$I$-bundles is homeomorphic to one obtained from the following.  Delete $C$ and
glue the attached page to a twisted $I$-bundle over a Mobius strip. This
corresponds to attaching a Mobius strip to the page's base surface.

\item[(A4)] If $C$ has 2 gluing annuli, note that they must be parallel and
each intersect a meridian disk only once. The book of $I$-bundles is
homeomorphic to one obtained from the following. Delete $C$ and glue two pages
(which are possibly two gluing loci of the same page) along the now-free gluing
annuli.  Notice that this also matches up parabolics.

Note: this is the only step that can produce a thickened closed surface
(instead of a true book of $I$-bundles).

\end{enumerate}

\item[(B)] If $(M,P)$ has a page $B$ which violates (2), delete $B$.

\item[(C)] If $(M,P)$ has a component $P_0$ of $P$ which violates (3), $P_0$
intersects at most one page:

\begin{enumerate}

\item[(C1)] If $P_0$ intersects a single page $B$, delete $B$.

\item[(C2)] If $P_0$ intersects no pages, it lies entirely inside a spine $C$.
Delete $C$.

\end{enumerate}

\end{enumerate}

In all cases, modify $P$ by removing all components which have nonempty
intersection with the deleted pages or spines.

\end{defn}

Notice that in all cases we have a pared embedding of our new manifold into the
original. See Figure~\ref{F:reductionex} for an example of this reduction
algorithm. Note that in the figure, all pages are trivial $I$-bundles, and all
binding annuli intersect a meridian disk of the spine exactly once. To simplify
the picture, the annuli are omitted and the spines are drawn as their base
surfaces.

\myrotfig{fig-reductionex}{F:reductionex}{Example of the reduction algorithm}

\begin{thm}(Reduction theorem)

Let $(M,P)$ be a pared admissible book of $I$-bundles. Repeatedly reducing
terminates after finitely many reductions in some $(M_r,P_r)$, with an
associated pared embedding $\iota\colon (M_r,P_r) \to (M,P)$. Each resulting
connected component is either a reduced book of $I$-bundles or a thickened
closed surface with empty pared locus.

Furthermore, any (QF) surface in $(M,P)$ is contained in $\iota(M_r,P_r)$. In
particular, if $(M_r,P_r)=\emptyset$, $(M,P)$ does not contain a (QF) surface.

\end{thm}

\begin{proof}

Write $\cM_0 = (M,P)$ and $\cM_i$ for the (possibly disconnected) manifold we
obtain after reducing i times.

Let $\operatorname{size}(\cM_i) = \Sigma_{M_{ij} \text{ component of } \cM_i}
\operatorname{pages}(M_{ij}) + \operatorname{spines}(M_{ij})$. We claim that
any reduction must decrease the size. If we consider a thickened closed surface
to be made out of one page and no spines, it's immediately clear that
$size(\cM_i)>0$ if $\cM_i \neq \emptyset$.  So this will imply that reducing
repeatedly must terminate after finitely many reductions. Furthermore, to show
that (QF) surfaces are included, we must show that any reduction deletes a set
disjoint from all (QF) surfaces.

We consider each possible reduction.

Reduction (A1) removes spines without attached pages. No such spine can contain
a $\pi_1$-injective closed surface, so it's clearly disjoint from (QF)
surfaces.

Reduction (A2) removes spines attached to a single page along a gluing
intersecting a meridian disk only once. Again, no $\pi_1$-injective closed
surface can cross this spine, as by Proposition~\ref{P:1ptcap} the boundary
admits a compression which our surface can be made disjoint from.

Reductions (A3) and (A4) do not actually delete anything. They produce
a homeomorphic book of $I$-bundles (or possibly a thickened closed surface)
with a homeomorphically corresponding pared structure. Therefore there's
nothing to prove in this case.

Reduction (B) removes a page with a boundary component that's not glued to
a spine. By the surface decomposition lemma, any $\pi_1$-injective surface that
intersects the page at all must be a cover of the that page's core surface.
Therefore it must have boundary components covering all boundary components of
the page under consideration. This includes the free boundary component, but
then by the surface decomposition lemma again, these have nothing they can glue
to. So it is impossible to produce a closed $\pi_1$-injective surface (without
boundary) that intersects this page.

Alternatively, apply Proposition~\ref{P:1ptcap} to obtain a compression of the
boundary, which we can make any $\pi_1$-injective surface disjoint from. By the
surface decomposition lemma, this forces our surface to be disjoint from that
page.

Reduction (C1) removes a page $B$ which intersects a parabolic $P_0$. There are
two cases to consider here.

\begin{enumerate}

\item If $P_0$ is contained entirely inside $B$, that is, it doesn't intersect
any spines at all, it's easy to see why no (QF) surface can intersect $B$. By
the surface decomposition lemma, any surface intersecting $B$ would have to be
a cover of $B$. But then $P_0$ will lift to any such cover, violating the
parabolic lifting criterion for (QF).

\item If $P_0$ intersects a spine, in order to satisfy the (C1) reduction
criterion, it must be a valence one spine $C$ with $B$ attached by degree 2.
$C$ cannot be valence 1 degree 1 because otherwise we'd violate (A2). Now it
suffices to consider a surface that intersects the page $B$. Such a surface
must be a cover of $B$. By the surface decomposition lemma, it must be glued to
itself by annuli along the spine $C$. But since the gluing of $B$ to $C$ is
degree 2, there can only be one topological choice of gluing through $C$, and
this is the same choice that the parabolic $P_0$ makes. This is easy to see if
we look at a double cover near where $B$ is glued to $C$. Since our parabolic
only traverses spines of this form, our surface will necessarily remain
parallel to the parabolic through each spine, and the resulting surface will
violate the parabolic lifting criterion for the parabolic $P_0$. So no (QF)
surface can intersect a page which satisfies the (C1) reduction criterion.

\end{enumerate}

Reduction (C2) removes a spine $C$ which intersects a parabolic $P_0$. Again,
by the surface decomposition lemma, any surface which traverses that spine will
have to contain an annulus connecting two page gluings - that is, contain
a multiple of that parabolic. This violates the lifting criterion for (QF).

Finally, observe that in all cases, the parabolic components we remove from
$(M,P)$ as we decompose - ie those that intersect the deleted pieces - cannot
possibly affect whether a surface outside the deleted pieces is (QF), as any
surface that was otherwise (QF) but contained a multiple of any deleted
parabolic would have to intersect a deleted piece. So as we inductively delete
pieces of our book of $I$-bundles, we preserve the property that all (QF)
surfaces and parabolics we need to determine which surfaces are (QF) remain
outside the deleted parts of the book of $I$-bundles. Since our induction is
guaranteed to terminate by the size measure above, and we know that every
termination is a reduced book of $I$-bundles, a thickened closed surface with
no parabolics, or empty, this completes the proof.

\end{proof}

As motivation, also observe that our specific example in the earlier section is
one of the simplest possible topological structures for a reduced book of
$I$-bundles.

Note that the surface case is very easy.

\begin{prop}

Let $(M,P) = (\Si \times I, \emptyset)$ be a thickened closed surface with
empty pared locus. Then every cover of Sigma induces a (QF) surface in.$(M,P)$.

\end{prop}

\begin{proof}

$P = \emptyset$, so this follows directly from the covering lemma. Every closed
$\pi_1$-injective surface is (QF).

\end{proof}
