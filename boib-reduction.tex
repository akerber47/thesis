%%%%%%%%%%%%%%%%%%%%%%%%%%%%%%%%%%%%%%%%%%%%%%%%%
\section{General case - preliminaries}
%%%%%%%%%%%%%%%%%%%%%%%%%%%%%%%%%%%%%%%%%%%%%%%%%

We prove a generalization of the above example. Basically, we want to first
identify the cases that obviously can't have a surface satisfying \eqref{E:qf}.
Then we'll prove by construction that every other book of $I$-bundles does in
fact contain such a surface.

We use the following general description for books of $I$-bundles. Let
$M_{c1},...,M_{cr}$ be the spines, all solid tori. Let $M_{p1},...,M_{ps}$ be
the pages, all thickened surfaces with boundary. Each page has boundary
components $A_{ij}$, all annuli. We glue each annulus by a homeomorphism
$\phi_{ij}$ to a boundary annulus $A_{ij'}\cin \bd M_{ck}$. Note that $Aij'$
might not be a simple longitude of, that is, a page can wrap multiple times
around a spine.  However, for each spine, all the $A_{ij'}$ must be disjoint
and parallel, or the result of gluing will not be a manifold.

We first make the following simple observation.

prop

Let $M$ be a book of $I$-bundles, and $M'$ to $M$ be a finite-sheeted covering
space.  Then $M'$ is also a book of $I$-bundles.

proof

This is straightforward. Lift the spines and pages of $M$ to obtain spines and
pages of $M'$. Lift the gluing annuli of $M$ to gluing annuli of $M'$.

We want to consider books of $I$-bundles with some basic topological
properties, to make sure they are Kleinian manifolds.

\begin{defn}

A book of $I$-bundles $M$ is \emph{admissible} if it satisfies the following
conditions.

% TODO
\textbf{TODO as discussed with Ian, this should just be "a book of I-bundles."
See definitions in Morgan or Thurston or Culler-Shalen and fix accordingly.}

\begin{enumerate}

\item The underlying manifold is a compact connected orientable hyperbolic
3-manifold.

\item Each page is an I-bundle over a surface with boundary of negative Euler
characteristic.  That is, no base surface is a disk, Mobius strip, or annulus.

\end{enumerate}

\end{defn}

Henceforth we will only consider admissible books of $I$-bundles. Let's briefly
explain why.

Obviously we want M to be compact connected orientable hyperbolic, as this is
the general case we're studying (Kleinian manifolds). If M were nonorientable
we could reduce to the orientable double cover.

Each page's base surface must have at least one boundary component to glue to
in order for M to be connected. We don't want to consider base surfaces which
are disks.  Observe that each page which is a copy of $S0,1$ means that that
page and its attached spine form a 3-manifold with a finite-sheeted cover by
a ball (we can arrange things so that in the cover, the $S0,1$ attaching map
only traverses the longitude once, and we get a thickened disk).  This means
that we'll have finite order summands in our group, which correspond to
elliptic pieces, for instance lens spaces, in the JSJ decomposition. These
cases are not hyperbolic.

%TODO
\textbf{TODO explain Mobius strips / annuli with "big bad loops" idea which
breaks our book of I-bundles reduction theorem.}

\begin{prop}

Let M be an admissible book of I-bundles. Then M must have incompressible
boundary.

\end{prop}

% TODO
\textbf{TODO figure out/throw in Ian's explanation of incompressible boundary.}

Now consider a pared structure P on a book of I-bundles. We'll say (M,P) is
admissible if the book of I-bundles M is admissible. Note that the first
requirement forces all boundary components of M to have negative Euler
characteristic. Because components of P are pi1-injective, P therefore must be
a union of annuli in any reduced book of I-bundles. We introduce an additional
definition.

\begin{defn}

Let (M,P) be a pared admissible book of I-bundles. We say (M,P) is
\emph{reduced} if it satisfies the following.

\begin{enumerate}

\item A meridian disk of each spine intersects the union of all gluing annuli
at least 2 times. Furthermore, if it intersects exactly two times, that spine
has exactly one page attached (via a map which "wraps around twice").

\item All page boundary components are glued (there are no "free boundary
components").

\item Each component of P (once placed in minimal position with respect to the
gluing annuli) intersects at least two pages.

\end{enumerate}

\end{defn}

This definition is motivated by the following important idea. If (M,P) is not
reduced, we can attempt to reduce it inductively. At each step, we remove pages
or spines from our pared books of I-bundles to produce new pared books of
I-bundles.  Because we're deleting pieces, we may end up with multiple
connected components. Each component is either a pared book of I-bundles or an
isolated page with no remaining boundary components - that is, a thickened
closed surface.

\begin{defn}

We \emph{reduce} a book of I-bundles as follows. We can repeat this inductively
on connected components of the result of a reduction.

% FIXME change enumeration to letters so can refer below
\begin{itemize}

\item If (M,P) has a spine C which violates (1):

\begin{itemize}

\item If C has 0 gluing annuli, delete C.

\item If C has 1 gluing annulus, delete C and the attached page.

\item If C has 2 gluing annuli, delete C and glue the two pages along the
annuli (or glue two annuli already on the same page) to produce a single page.

\end{itemize}

\item If (M,P) has a page B which violates (2), delete B.

\item If (M,P) has a page B which violates (3), delete B.

\item In all cases, modify P by removing all components which have nonempty
intersection with the deleted pages or spines.

\end{itemize}

\end{defn}

Notice that in all cases we have a pared embedding of our new manifold into the
original.

\begin{thm}(Reduction Theorem)

Let (M,P) be a pared admissible book of I-bundles. Repeatedly reducing
terminates after finitely many reductions in some (Mr,Pr), with an associated
pared embedding iota colon (Mr,Pr) to (M,P). Each resulting connected component
is either a reduced book of I-bundles or a thickened closed surface with empty
pared locus.

Furthermore, any (QF) surface in (M,P) is contained in iota(Mr,Pr). In
particular, if (Mr,Pr)=emptyset, (M,P) does not contain a (QF) surface.

\end{thm}

\begin{proof}

Write sM0 = (M,P) and sMi for the (possibly disconnected) manifold we obtain
after reducing i times.

Let size(sMi) = Sigma{(Mij,Pij) component of sMi} (num{pages of Mij})
+ (num{spines of Mij}). We claim that any reduction must decrease the size. If
we consider a thickened closed surface to be made out of one page and no
spines, it's immediately clear that size(sMi)>0 if sMi neq emptyset. So this
will imply that reducing repeatedly must terminate after finitely many
reductions. Furthermore, to show that (QF) surfaces are included, we must show
that any reduction deletes a set disjoint from all (QF) surfaces.

We consider each possible reduction.

%TODO
\textbf{TODO finish this}

\end{proof}

As motivation, also observe that our specific example in the earlier section is
one of the simplest possible topological structures for a reduced book of
I-bundles.

Note that the surface case is very easy.

\begin{prop}

Let (M,P) = (Sigma times I, emptyset) be a thickened closed surface with empty
pared locus. Then every cover of Sigma induces a (QF) surface in.(M,P).

\end{prop}

\begin{proof}

P = emptyset, so this follows directly from the covering lemma.

\end{proof}

